% arara: xetex: {shell: on}

\input texosquery

\def\querydata#1{% empty #1 means use default locale
  \TeXOSQueryLocaleData{\result}{#1}% 
  \ifx\result\empty
    Query failed.%
  \else
    \expandafter\parseresult\result
  \fi
}

\long\def\parseresult#1{%
  % first block is the language and region information
  \parselocale#1% lose outer braces
  \par
  \parsedateblock
}

\def\parselocale#1#2#3#4#5#6#7{%
  Tag: {\tt #1}.
  Language: #2 % language name in default locale's language
  (#3)% language name in this locale's language
  . Region: #4 % region name in default locale's language
  (#5)% region name in this locale's language
  . Variant: #6 % variant name in default locale's language
  (#7). % region name in this locale's language
}

\long\def\parsedateblock#1{%
  \parsedate#1% lose outer braces
  \par
  \parsedatefmtblock
}

\def\parsedate#1#2#3#4#5{%
  Full date: #1.
  Long date: #2.
  Medium date: #3.
  Short date: #4.
  First day of week: #5. % index 0 = Monday, etc
}

\long\def\parsedatefmtblock#1{%
  \parsedatefmt#1% lose outer braces
  \par
  \parsetimeblock
}

\def\parsedatefmt#1#2#3#4{%
  Full date format: #1.
  Long date format: #2.
  Medium date format: #3.
  Short date format: #4.
}

\long\def\parsetimeblock#1{%
  \parsetime#1% lose outer braces
  \par
  \parsetimefmtblock
}

\def\parsetime#1#2#3#4{%
  Full time: #1.
  Long time: #2.
  Medium time: #3.
  Short time: #4.
}

\long\def\parsetimefmtblock#1{%
  \parsetimefmt#1% lose outer braces
  \par
  \parsedatetimeblock
}

\def\parsetimefmt#1#2#3#4{%
  Full time format: #1.
  Long time format: #2.
  Medium time format: #3.
  Short time format: #4.
}

\long\def\parsedatetimeblock#1{%
  \parsedatetime#1% lose outer braces
  \par
  \parsedatetimefmtblock
}

\def\parsedatetime#1#2#3#4{%
  Full date time: #1.
  Long date time: #2.
  Medium date time: #3.
  Short date time: #4.
}

\long\def\parsedatetimefmtblock#1{%
  \parsedatetimefmt#1% lose outer braces
  \par
  \parsedayblock
}

\def\parsedatetimefmt#1#2#3#4{%
  Full date time format: #1.
  Long date time format: #2.
  Medium date time format: #3.
  Short date time format: #4.
}

\long\def\parsedayblock#1{%
  \parsedaynames#1% lose outer block
  \par
  \parseshortdayblock
}

\def\parsedaynames#1#2#3#4#5#6#7{%
  Days of the week: #1, #2, #3, #4, #5, #6, #7.
}

\def\parseshortdayblock#1{%
  \parseshortdaynames#1% lose outer block
  \par
  \parsemonthblock
}

\def\parseshortdaynames#1#2#3#4#5#6#7{%
  Abbreviated days of the week: #1, #2, #3, #4, #5, #6, #7.
}

\long\def\parsemonthblock#1{%
  \parsemonthnames#1% lose outer block
  \par
  \parseshortmonthblock
}

\def\parsemonthnames#1#2#3#4#5#6#7#8#9{%
  Months: #1, #2, #3, #4, #5, #6, #7, #8, #9,
  \parseendmonthnames
}

\def\parseendmonthnames#1#2#3{%
  #1, #2, #3.
}

\long\def\parseshortmonthblock#1{%
  \parseshortmonthnames#1% lose outer block
  \par
  \parsestandalonedayblock
}

\def\parseshortmonthnames#1#2#3#4#5#6#7#8#9{%
  Abbreviated months: #1, #2, #3, #4, #5, #6, #7, #8, #9,
  \parseendshortmonthnames
}

\def\parseendshortmonthnames#1#2#3{%
  #1, #2, #3.
}

\long\def\parsestandalonedayblock#1{%
  \parsestandalonedaynames#1% lose outer block
  \par
  \parsestandaloneshortdayblock
}

\def\parsestandalonedaynames#1#2#3#4#5#6#7{%
  Standalone days of the week: #1, #2, #3, #4, #5, #6, #7.
}

\long\def\parsestandaloneshortdayblock#1{%
  \parsestandaloneshortdaynames#1% lose outer block
  \par
  \parsestandalonemonthblock
}

\def\parsestandaloneshortdaynames#1#2#3#4#5#6#7{%
  Standalone abbreviated days of the week: #1, #2, #3, #4, #5, #6, #7.
}

\long\def\parsestandalonemonthblock#1{%
  \parsestandalonemonthnames#1% lose outer block
  \par
  \parsestandaloneshortmonthblock
}

\def\parsestandalonemonthnames#1#2#3#4#5#6#7#8#9{%
  Standalone months: #1, #2, #3, #4, #5, #6, #7, #8, #9,
  \parseendstandalonemonthnames
}

\def\parseendstandalonemonthnames#1#2#3{%
  #1, #2, #3.
}

\long\def\parsestandaloneshortmonthblock#1{%
  \parsestandaloneshortmonthnames#1% lose outer block
  \par
  \parsenumericblock
}

\def\parsestandaloneshortmonthnames#1#2#3#4#5#6#7#8#9{%
  Standalone abbreviated months: #1, #2, #3, #4, #5, #6, #7, #8, #9,
  \parseendstandaloneshortmonthnames
}

\def\parseendstandaloneshortmonthnames#1#2#3{%
  #1, #2, #3.
}

\def\parsenumericblock#1{%
  \parsenumericdata#1% lose outer block
  \parsenumericfmtdatablock
}

\def\parsenumericdata#1#2#3#4#5#6#7#8#9{%
  Numeric data: #1 (number group separator)
  #2 (decimal separator)
  #3 (exponent symbol)
  #9 (monetary decimal separator).
  Uses number grouping: \ifnum#4=1 true\else false\fi.
  Currency ISO~4217 code: #5
  (#6).
  % Support may be needed for any non-ASCII characters in #7
  Currency symbol: #7 (#8).
  % The final two arguments are the percent and permill symbols.
  \parsepercentmill
}

\def\parsepercentmill#1#2{%
  Percent: #1.
  Per mill: #2.
}

\def\parsenumericfmtdatablock#1{%
  \parsenumericfmtdata#1% discard outer group
}

\def\parsenumericfmtdata#1#2#3#4{%
  Numeric data formats:\par
  Decimals: #1\par
  Integers: #2\par
  Currency: #3\par
  Percentages: #4
}


{\bf en-GB}

\querydata{en-GB}

{\bf en-IM}

\querydata{en-IM}

{\bf pt-BR}

\querydata{pt-BR}

{\bf fr-BE}

\querydata{fr-BE}

{\bf de-DE-1996}

\querydata{de-DE-1996}

\bye
