%\iffalse
%<*package>
%% \CharacterTable
%%  {Upper-case    \A\B\C\D\E\F\G\H\I\J\K\L\M\N\O\P\Q\R\S\T\U\V\W\X\Y\Z
%%   Lower-case    \a\b\c\d\e\f\g\h\i\j\k\l\m\n\o\p\q\r\s\t\u\v\w\x\y\z
%%   Digits        \0\1\2\3\4\5\6\7\8\9
%%   Exclamation   \!     Double quote  \"     Hash (number) \#
%%   Dollar        \$     Percent       \%     Ampersand     \&
%%   Acute accent  \'     Left paren    \(     Right paren   \)
%%   Asterisk      \*     Plus          \+     Comma         \,
%%   Minus         \-     Point         \.     Solidus       \/
%%   Colon         \:     Semicolon     \;     Less than     \<
%%   Equals        \=     Greater than  \>     Question mark \?
%%   Commercial at \@     Left bracket  \[     Backslash     \\
%%   Right bracket \]     Circumflex    \^     Underscore    \_
%%   Grave accent  \`     Left brace    \{     Vertical bar  \|
%%   Right brace   \}     Tilde         \~}
%</package>
%\fi
% \iffalse
% Doc-Source file
% Copyright (C) 2016 Nicola Talbot, all rights reserved.
% \fi
% \iffalse
%<*driver>
\documentclass{ltxdoc}

\usepackage[T1]{fontenc}
\usepackage{tgtermes}
\usepackage{metalogo}
\usepackage[colorlinks,hyperindex=false]{hyperref}

\CheckSum{273}

\RecordChanges
\PageIndex
\CodelineNumbered
\IndexPrologue{}

\renewcommand*{\usage}[1]{\hyperpage{#1}}
\renewcommand*{\main}[1]{\hyperpage{#1}}

\newcommand*{\shortargfmt}[1]{\texttt{-#1}}
\newcommand*{\longargfmt}[1]{\texttt{-{}-#1}}

\newcommand*{\shortarg}[1]{\shortargfmt{#1}%
 \index{texosquery options=\texttt{texosquery} options>#1=\protect\shortargfmt{-#1}|hyperpage}}
\newcommand*{\longarg}[1]{\longargfmt{#1}%
 \index{texosquery options=\texttt{texosquery} options>#1=\protect\longargfmt{#1}|hyperpage}}

\newenvironment{important}
 {\begin{quote}\textbf{Important Note:}}
 {\end{quote}}

\begin{document}
\DocInput{texosquery.dtx}
\end{document}
%</driver>
%\fi
%
%\DeleteShortVerb{|}
%
%\title{texosquery: query OS information from \TeX}
%\author{Nicola L. C. Talbot\\
%\href{http://www.dickimaw-books.com/}{\nolinkurl{dickimaw-books.com}}
%\and Paulo Cereda}
%\date{2016-07-14 (v1.1)}
%\maketitle
%
%\begin{abstract}
%\texttt{texosquery.jar} is a cross-platform Java application
%to query certain OS information designed for use in \TeX's
%shell escape mechanism.
%The accompanying \TeX\ code provides a simple interface to the
%\texttt{texosquery} Java application, which may be skipped
%if you want to explicitly use \cs{input} (but take
%care of any special characters appearing in the result).
%\end{abstract}
%
%\begin{important}
%You will need \TeX's shell escape enabled,
%and you will also need the Java Runtime Environment (JRE) 
%installed to use \texttt{texosquery}.
%\end{important}
%
%If you want to rebuild the application, instructions for
%compiling the source code (including the code for this document)
%are in the accompanying \texttt{README.md} file.
%
%\tableofcontents
%
%\section{texosquery.jar: the Java application}
%The \texttt{texosquery} Java command line application looks up
%certain system information that may be of use in \TeX\ 
%documents. All this information can easily be obtained using 
%native commands, but the Java application allows an 
%OS-independent approach with results that can easily be
%captured by \TeX's shell-escape without having to strip 
%formatting information.
%
%\begin{important}
%\texttt{texosquery} provides read-only actions, and I don't 
%intend adding any actions that modify system settings or files.
%\end{important}
%
%Since the application is designed to work with \TeX, each
%function will display the result on a single line without
%formatting. (For multiple results, each line is grouped 
%from v1.1.) A blank line (or empty group) will be displayed
%if the information isn't available. A forward slash is always 
%used as a directory divider, regardless of the operating 
%system, so the result can be used, for example, in \cs{input} or
%\cs{includegraphics}.
%
%If an input file name is required (for example, with the
%\longargfmt{pdfdate} argument described below) then the file may be in
%the current working directory, relative to the current directory
%(with forward slash \texttt{/} as the directory divider), an absolute path 
%(again with forward slash) or on \TeX's path (in
%which case, \texttt{kpsewhich} is used to locate it).
%
%Command line invocation:
%\begin{flushleft}\ttfamily
%texosquery \meta{action} ...
%\end{flushleft}
%Available actions (at least one required):
%\begin{description}
%\item[\shortarg{L} or \longarg{locale}] Display the locale
%information in the form
%\begin{flushleft}\ttfamily
%\meta{lang}-\meta{region}.\meta{codeset}@\meta{modifier}
%\end{flushleft}
%where \meta{lang} is the ISO code for the language (e.g.\ 
%\texttt{en}),
%\meta{region} is the ISO code for the region (e.g.\ \texttt{GB}),
%\meta{codeset} is the default code set (e.g.\ \texttt{UTF-8})
% and \meta{modifier} is the modifier. Elements may be omitted
% if unavailable. For example, \texttt{en-GB.UTF-8} has the
% \meta{modifier} omitted, and \texttt{en} has all but the language
% omitted.
%\item[\shortarg{l} or \longarg{locale-lcs}] As the above but the
%codeset (if present) is converted to lower case and any hyphens are
%stripped. For example, if \longargfmt{locale} returns
%\texttt{en-GB.UTF-8}, then \longargfmt{locale-lcs} would return
%\texttt{en-GB.utf8}.
%\item[\shortarg{b} or \longarg{bcp47}] (New to version 1.2.)
%Displays the BCP~47 language tag.
%
%\item[\shortarg{c} or \longarg{cwd}] Displays the current working
%directory.
%\item[\shortarg{m} or \longarg{userhome}] Displays the user's home
%directory.
%\item[\shortarg{t} or \longarg{tmpdir}] Displays the temporary
%directory.
%\item[\shortarg{o} or \longarg{osname}] Displays the operating
%system name.
%\item[\shortarg{r} or \longarg{osversion}] Displays the operating
%system version.
%\item[\shortarg{a} or \longarg{osarch}] Displays the operating
%system architecture.
%\item[\shortarg{n} or \longarg{pdfnow}] Displays the current
% date and time in PDF format. For example 
%\begin{verbatim}
%D:20160704131006+01'00'
%\end{verbatim}
%Note that some, but not all, \TeX\ formats provide
%\cs{pdfcreationdate}, which is more efficient than using the shell
%escape, but this can be used
%as a fallback method for those that don't (for example, \XeTeX).
%\item[\shortarg{d} \meta{file} or \longarg{pdfdate} \meta{file}]
% Displays the last modified time stamp of the given file in PDF format or a blank line
% if the file doesn't exist or the file permissions prohibit this
% action.
%Again some, but not all, \TeX\ formats provide
%\cs{pdffilemoddate}\marg{file}, which is more efficient than using
%the shell escape.
%\item[\shortarg{s} \meta{file} or \longarg{filesize} \meta{file}]
% Displays the size in bytes of the given file or a blank line
% if the file doesn't exist or the file permissions prohibit this
% action.
%Some, but not all, \TeX\ formats provide
%\cs{pdffilesize}\marg{file}, which is more efficient than using
%the shell escape.
%\item[\shortarg{i} \meta{sep} \meta{dir} or \longarg{list}
%\meta{sep} \meta{dir}]
%List all files in the given directory with the output on a single
%line using \meta{sep} as the separator between entries. Note that
%the list doesn't include the full path, just the file names.
%\item[\shortarg{f} \meta{sep} \meta{regex} \meta{dir} or
%\longarg{filterlist} \meta{sep} \meta{regex} \meta{dir}]
%Like the above but only lists those files whose name matches the
%regular expression given in \meta{regex}. Note that this tests for a
%complete match on the file name (not including path). For example,
%if \meta{regexp} is \texttt{foo.*}, it will only match files whose
%name starts with \texttt{foo} (for example, \texttt{foobar} will
%match but \texttt{barfoo} won't).
%\item[\shortarg{u} \meta{file} or \longarg{uri} \meta{file}]
%Displays the URI of the given file or a blank line if the file
%doesn't exist or the file permissions prohibit this action.
%Note that this may include in a percent character in the result. The \TeX\ 
%command \cs{TeXOSQuery} protects against this by changing the
%category code, but if you explicitly call \texttt{texosquery}
%using the shell escape, you'll need to be careful of this.
%\item[\shortarg{p} \meta{file} or \longarg{path} \meta{file}]
%Displays the canonical path of the given file or a blank line if
%the file doesn't exist or the file permissions prohibit this
%action.
%\item[\shortarg{e} \meta{file} or \longarg{dirname} \meta{file}]
%(New to v1.1.)
%Displays the canonical path of the given file's parent (that is, the
%directory containing \meta{file}) or a blank line if
%the file doesn't exist or the file permissions prohibit this
%action. Note that this is different to the Unix-like 
%\texttt{dirname} command, which will return a relative path if
%\meta{file} isn't an absolute path.
%\item[\shortarg{h} or \longarg{help}] Displays help message and
%exits.
%\item[\shortarg{v} or \longarg{version}] Displays version
%information and exits.
%\end{description}
%
%If multiple options are given, they will be processed in the
%order specified in the command line invocation. Each result will
%be displayed on a separate line. As from v1.1, if there are
%multiple actions, each result will be grouped. This makes it easier
%to process the results in \TeX. For example:
%\begin{verbatim}
%texosquery -l
%\end{verbatim}
%This just produces (for me):
%\begin{verbatim}
%en-GB.utf8
%\end{verbatim}
%whereas
%\begin{verbatim}
%texosquery -l -n
%\end{verbatim}
%produces:
%\begin{verbatim}
%{en-GB.utf8}
%{D:20160714112732+01'00'}
%\end{verbatim}
%Note that unavailable information will produce an empty group.
%For example (assuming \texttt{nofile} doesn't exist):
%\begin{verbatim}
%texosquery -l -d nofile -n
%\end{verbatim}
%produces:
%\begin{verbatim}
%{en-GB.utf8}
%{}
%{D:20160714112732+01'00'}
%\end{verbatim}
%whereas 
%\begin{verbatim}
%texosquery -d nofile
%\end{verbatim}
%just displays an empty line.
%
%\section{texosquery.tex: generic \TeX\ code}
%You can run \texttt{texosquery} directly from \TeX's shell escape.
%For example:
%\begin{verbatim}
%\input|"texosquery --locale"
%\end{verbatim}
%However, \texttt{texosquery.tex} provides generic \TeX\ code to do
%this for you and store the result in a control sequence.
%
%Plain \TeX\ users can input this file through the usual \cs{input}
%method:
%\begin{verbatim}
%\input texosquery
%\end{verbatim}
%\LaTeX\ users may also simply input this file:
%\begin{verbatim}
%%\iffalse
%<*package>
%% \CharacterTable
%%  {Upper-case    \A\B\C\D\E\F\G\H\I\J\K\L\M\N\O\P\Q\R\S\T\U\V\W\X\Y\Z
%%   Lower-case    \a\b\c\d\e\f\g\h\i\j\k\l\m\n\o\p\q\r\s\t\u\v\w\x\y\z
%%   Digits        \0\1\2\3\4\5\6\7\8\9
%%   Exclamation   \!     Double quote  \"     Hash (number) \#
%%   Dollar        \$     Percent       \%     Ampersand     \&
%%   Acute accent  \'     Left paren    \(     Right paren   \)
%%   Asterisk      \*     Plus          \+     Comma         \,
%%   Minus         \-     Point         \.     Solidus       \/
%%   Colon         \:     Semicolon     \;     Less than     \<
%%   Equals        \=     Greater than  \>     Question mark \?
%%   Commercial at \@     Left bracket  \[     Backslash     \\
%%   Right bracket \]     Circumflex    \^     Underscore    \_
%%   Grave accent  \`     Left brace    \{     Vertical bar  \|
%%   Right brace   \}     Tilde         \~}
%</package>
%\fi
% \iffalse
% Doc-Source file
% Copyright (C) 2016 Nicola Talbot, all rights reserved.
% \fi
% \iffalse
%<*driver>
\documentclass{ltxdoc}

\usepackage[T1]{fontenc}
\usepackage{tgtermes}
\usepackage{metalogo}
\usepackage[colorlinks,hyperindex=false]{hyperref}

\CheckSum{273}

\RecordChanges
\PageIndex
\CodelineNumbered
\IndexPrologue{}

\renewcommand*{\usage}[1]{\hyperpage{#1}}
\renewcommand*{\main}[1]{\hyperpage{#1}}

\newcommand*{\shortargfmt}[1]{\texttt{-#1}}
\newcommand*{\longargfmt}[1]{\texttt{-{}-#1}}

\newcommand*{\shortarg}[1]{\shortargfmt{#1}%
 \index{texosquery options=\texttt{texosquery} options>#1=\protect\shortargfmt{-#1}|hyperpage}}
\newcommand*{\longarg}[1]{\longargfmt{#1}%
 \index{texosquery options=\texttt{texosquery} options>#1=\protect\longargfmt{#1}|hyperpage}}

\newenvironment{important}
 {\begin{quote}\textbf{Important Note:}}
 {\end{quote}}

\begin{document}
\DocInput{texosquery.dtx}
\end{document}
%</driver>
%\fi
%
%\DeleteShortVerb{|}
%
%\title{texosquery: query OS information from \TeX}
%\author{Nicola L. C. Talbot\\
%\href{http://www.dickimaw-books.com/}{\nolinkurl{dickimaw-books.com}}
%\and Paulo Cereda}
%\date{2016-07-14 (v1.1)}
%\maketitle
%
%\begin{abstract}
%\texttt{texosquery.jar} is a cross-platform Java application
%to query certain OS information designed for use in \TeX's
%shell escape mechanism.
%The accompanying \TeX\ code provides a simple interface to the
%\texttt{texosquery} Java application, which may be skipped
%if you want to explicitly use \cs{input} (but take
%care of any special characters appearing in the result).
%\end{abstract}
%
%\begin{important}
%You will need \TeX's shell escape enabled,
%and you will also need the Java Runtime Environment (JRE) 
%installed to use \texttt{texosquery}.
%\end{important}
%
%If you want to rebuild the application, instructions for
%compiling the source code (including the code for this document)
%are in the accompanying \texttt{README.md} file.
%
%\tableofcontents
%
%\section{texosquery.jar: the Java application}
%The \texttt{texosquery} Java command line application looks up
%certain system information that may be of use in \TeX\ 
%documents. All this information can easily be obtained using 
%native commands, but the Java application allows an 
%OS-independent approach with results that can easily be
%captured by \TeX's shell-escape without having to strip 
%formatting information.
%
%\begin{important}
%\texttt{texosquery} provides read-only actions, and I don't 
%intend adding any actions that modify system settings or files.
%\end{important}
%
%Since the application is designed to work with \TeX, each
%function will display the result on a single line without
%formatting. (For multiple results, each line is grouped 
%from v1.1.) A blank line (or empty group) will be displayed
%if the information isn't available. A forward slash is always 
%used as a directory divider, regardless of the operating 
%system, so the result can be used, for example, in \cs{input} or
%\cs{includegraphics}.
%
%If an input file name is required (for example, with the
%\longargfmt{pdfdate} argument described below) then the file may be in
%the current working directory, relative to the current directory
%(with forward slash \texttt{/} as the directory divider), an absolute path 
%(again with forward slash) or on \TeX's path (in
%which case, \texttt{kpsewhich} is used to locate it).
%
%Command line invocation:
%\begin{flushleft}\ttfamily
%texosquery \meta{action} ...
%\end{flushleft}
%Available actions (at least one required):
%\begin{description}
%\item[\shortarg{L} or \longarg{locale}] Display the locale
%information in the form
%\begin{flushleft}\ttfamily
%\meta{lang}-\meta{region}.\meta{codeset}@\meta{modifier}
%\end{flushleft}
%where \meta{lang} is the ISO code for the language (e.g.\ 
%\texttt{en}),
%\meta{region} is the ISO code for the region (e.g.\ \texttt{GB}),
%\meta{codeset} is the default code set (e.g.\ \texttt{UTF-8})
% and \meta{modifier} is the modifier. Elements may be omitted
% if unavailable. For example, \texttt{en-GB.UTF-8} has the
% \meta{modifier} omitted, and \texttt{en} has all but the language
% omitted.
%\item[\shortarg{l} or \longarg{locale-lcs}] As the above but the
%codeset (if present) is converted to lower case and any hyphens are
%stripped. For example, if \longargfmt{locale} returns
%\texttt{en-GB.UTF-8}, then \longargfmt{locale-lcs} would return
%\texttt{en-GB.utf8}.
%\item[\shortarg{b} or \longarg{bcp47}] (New to version 1.2.)
%Displays the BCP~47 language tag.
%
%\item[\shortarg{c} or \longarg{cwd}] Displays the current working
%directory.
%\item[\shortarg{m} or \longarg{userhome}] Displays the user's home
%directory.
%\item[\shortarg{t} or \longarg{tmpdir}] Displays the temporary
%directory.
%\item[\shortarg{o} or \longarg{osname}] Displays the operating
%system name.
%\item[\shortarg{r} or \longarg{osversion}] Displays the operating
%system version.
%\item[\shortarg{a} or \longarg{osarch}] Displays the operating
%system architecture.
%\item[\shortarg{n} or \longarg{pdfnow}] Displays the current
% date and time in PDF format. For example 
%\begin{verbatim}
%D:20160704131006+01'00'
%\end{verbatim}
%Note that some, but not all, \TeX\ formats provide
%\cs{pdfcreationdate}, which is more efficient than using the shell
%escape, but this can be used
%as a fallback method for those that don't (for example, \XeTeX).
%\item[\shortarg{d} \meta{file} or \longarg{pdfdate} \meta{file}]
% Displays the last modified time stamp of the given file in PDF format or a blank line
% if the file doesn't exist or the file permissions prohibit this
% action.
%Again some, but not all, \TeX\ formats provide
%\cs{pdffilemoddate}\marg{file}, which is more efficient than using
%the shell escape.
%\item[\shortarg{s} \meta{file} or \longarg{filesize} \meta{file}]
% Displays the size in bytes of the given file or a blank line
% if the file doesn't exist or the file permissions prohibit this
% action.
%Some, but not all, \TeX\ formats provide
%\cs{pdffilesize}\marg{file}, which is more efficient than using
%the shell escape.
%\item[\shortarg{i} \meta{sep} \meta{dir} or \longarg{list}
%\meta{sep} \meta{dir}]
%List all files in the given directory with the output on a single
%line using \meta{sep} as the separator between entries. Note that
%the list doesn't include the full path, just the file names.
%\item[\shortarg{f} \meta{sep} \meta{regex} \meta{dir} or
%\longarg{filterlist} \meta{sep} \meta{regex} \meta{dir}]
%Like the above but only lists those files whose name matches the
%regular expression given in \meta{regex}. Note that this tests for a
%complete match on the file name (not including path). For example,
%if \meta{regexp} is \texttt{foo.*}, it will only match files whose
%name starts with \texttt{foo} (for example, \texttt{foobar} will
%match but \texttt{barfoo} won't).
%\item[\shortarg{u} \meta{file} or \longarg{uri} \meta{file}]
%Displays the URI of the given file or a blank line if the file
%doesn't exist or the file permissions prohibit this action.
%Note that this may include in a percent character in the result. The \TeX\ 
%command \cs{TeXOSQuery} protects against this by changing the
%category code, but if you explicitly call \texttt{texosquery}
%using the shell escape, you'll need to be careful of this.
%\item[\shortarg{p} \meta{file} or \longarg{path} \meta{file}]
%Displays the canonical path of the given file or a blank line if
%the file doesn't exist or the file permissions prohibit this
%action.
%\item[\shortarg{e} \meta{file} or \longarg{dirname} \meta{file}]
%(New to v1.1.)
%Displays the canonical path of the given file's parent (that is, the
%directory containing \meta{file}) or a blank line if
%the file doesn't exist or the file permissions prohibit this
%action. Note that this is different to the Unix-like 
%\texttt{dirname} command, which will return a relative path if
%\meta{file} isn't an absolute path.
%\item[\shortarg{h} or \longarg{help}] Displays help message and
%exits.
%\item[\shortarg{v} or \longarg{version}] Displays version
%information and exits.
%\end{description}
%
%If multiple options are given, they will be processed in the
%order specified in the command line invocation. Each result will
%be displayed on a separate line. As from v1.1, if there are
%multiple actions, each result will be grouped. This makes it easier
%to process the results in \TeX. For example:
%\begin{verbatim}
%texosquery -l
%\end{verbatim}
%This just produces (for me):
%\begin{verbatim}
%en-GB.utf8
%\end{verbatim}
%whereas
%\begin{verbatim}
%texosquery -l -n
%\end{verbatim}
%produces:
%\begin{verbatim}
%{en-GB.utf8}
%{D:20160714112732+01'00'}
%\end{verbatim}
%Note that unavailable information will produce an empty group.
%For example (assuming \texttt{nofile} doesn't exist):
%\begin{verbatim}
%texosquery -l -d nofile -n
%\end{verbatim}
%produces:
%\begin{verbatim}
%{en-GB.utf8}
%{}
%{D:20160714112732+01'00'}
%\end{verbatim}
%whereas 
%\begin{verbatim}
%texosquery -d nofile
%\end{verbatim}
%just displays an empty line.
%
%\section{texosquery.tex: generic \TeX\ code}
%You can run \texttt{texosquery} directly from \TeX's shell escape.
%For example:
%\begin{verbatim}
%\input|"texosquery --locale"
%\end{verbatim}
%However, \texttt{texosquery.tex} provides generic \TeX\ code to do
%this for you and store the result in a control sequence.
%
%Plain \TeX\ users can input this file through the usual \cs{input}
%method:
%\begin{verbatim}
%\input texosquery
%\end{verbatim}
%\LaTeX\ users may also simply input this file:
%\begin{verbatim}
%%\iffalse
%<*package>
%% \CharacterTable
%%  {Upper-case    \A\B\C\D\E\F\G\H\I\J\K\L\M\N\O\P\Q\R\S\T\U\V\W\X\Y\Z
%%   Lower-case    \a\b\c\d\e\f\g\h\i\j\k\l\m\n\o\p\q\r\s\t\u\v\w\x\y\z
%%   Digits        \0\1\2\3\4\5\6\7\8\9
%%   Exclamation   \!     Double quote  \"     Hash (number) \#
%%   Dollar        \$     Percent       \%     Ampersand     \&
%%   Acute accent  \'     Left paren    \(     Right paren   \)
%%   Asterisk      \*     Plus          \+     Comma         \,
%%   Minus         \-     Point         \.     Solidus       \/
%%   Colon         \:     Semicolon     \;     Less than     \<
%%   Equals        \=     Greater than  \>     Question mark \?
%%   Commercial at \@     Left bracket  \[     Backslash     \\
%%   Right bracket \]     Circumflex    \^     Underscore    \_
%%   Grave accent  \`     Left brace    \{     Vertical bar  \|
%%   Right brace   \}     Tilde         \~}
%</package>
%\fi
% \iffalse
% Doc-Source file
% Copyright (C) 2016 Nicola Talbot, all rights reserved.
% \fi
% \iffalse
%<*driver>
\documentclass{ltxdoc}

\usepackage[T1]{fontenc}
\usepackage{tgtermes}
\usepackage{metalogo}
\usepackage[colorlinks,hyperindex=false]{hyperref}

\CheckSum{273}

\RecordChanges
\PageIndex
\CodelineNumbered
\IndexPrologue{}

\renewcommand*{\usage}[1]{\hyperpage{#1}}
\renewcommand*{\main}[1]{\hyperpage{#1}}

\newcommand*{\shortargfmt}[1]{\texttt{-#1}}
\newcommand*{\longargfmt}[1]{\texttt{-{}-#1}}

\newcommand*{\shortarg}[1]{\shortargfmt{#1}%
 \index{texosquery options=\texttt{texosquery} options>#1=\protect\shortargfmt{-#1}|hyperpage}}
\newcommand*{\longarg}[1]{\longargfmt{#1}%
 \index{texosquery options=\texttt{texosquery} options>#1=\protect\longargfmt{#1}|hyperpage}}

\newenvironment{important}
 {\begin{quote}\textbf{Important Note:}}
 {\end{quote}}

\begin{document}
\DocInput{texosquery.dtx}
\end{document}
%</driver>
%\fi
%
%\DeleteShortVerb{|}
%
%\title{texosquery: query OS information from \TeX}
%\author{Nicola L. C. Talbot\\
%\href{http://www.dickimaw-books.com/}{\nolinkurl{dickimaw-books.com}}
%\and Paulo Cereda}
%\date{2016-07-14 (v1.1)}
%\maketitle
%
%\begin{abstract}
%\texttt{texosquery.jar} is a cross-platform Java application
%to query certain OS information designed for use in \TeX's
%shell escape mechanism.
%The accompanying \TeX\ code provides a simple interface to the
%\texttt{texosquery} Java application, which may be skipped
%if you want to explicitly use \cs{input} (but take
%care of any special characters appearing in the result).
%\end{abstract}
%
%\begin{important}
%You will need \TeX's shell escape enabled,
%and you will also need the Java Runtime Environment (JRE) 
%installed to use \texttt{texosquery}.
%\end{important}
%
%If you want to rebuild the application, instructions for
%compiling the source code (including the code for this document)
%are in the accompanying \texttt{README.md} file.
%
%\tableofcontents
%
%\section{texosquery.jar: the Java application}
%The \texttt{texosquery} Java command line application looks up
%certain system information that may be of use in \TeX\ 
%documents. All this information can easily be obtained using 
%native commands, but the Java application allows an 
%OS-independent approach with results that can easily be
%captured by \TeX's shell-escape without having to strip 
%formatting information.
%
%\begin{important}
%\texttt{texosquery} provides read-only actions, and I don't 
%intend adding any actions that modify system settings or files.
%\end{important}
%
%Since the application is designed to work with \TeX, each
%function will display the result on a single line without
%formatting. (For multiple results, each line is grouped 
%from v1.1.) A blank line (or empty group) will be displayed
%if the information isn't available. A forward slash is always 
%used as a directory divider, regardless of the operating 
%system, so the result can be used, for example, in \cs{input} or
%\cs{includegraphics}.
%
%If an input file name is required (for example, with the
%\longargfmt{pdfdate} argument described below) then the file may be in
%the current working directory, relative to the current directory
%(with forward slash \texttt{/} as the directory divider), an absolute path 
%(again with forward slash) or on \TeX's path (in
%which case, \texttt{kpsewhich} is used to locate it).
%
%Command line invocation:
%\begin{flushleft}\ttfamily
%texosquery \meta{action} ...
%\end{flushleft}
%Available actions (at least one required):
%\begin{description}
%\item[\shortarg{L} or \longarg{locale}] Display the locale
%information in the form
%\begin{flushleft}\ttfamily
%\meta{lang}-\meta{region}.\meta{codeset}@\meta{modifier}
%\end{flushleft}
%where \meta{lang} is the ISO code for the language (e.g.\ 
%\texttt{en}),
%\meta{region} is the ISO code for the region (e.g.\ \texttt{GB}),
%\meta{codeset} is the default code set (e.g.\ \texttt{UTF-8})
% and \meta{modifier} is the modifier. Elements may be omitted
% if unavailable. For example, \texttt{en-GB.UTF-8} has the
% \meta{modifier} omitted, and \texttt{en} has all but the language
% omitted.
%\item[\shortarg{l} or \longarg{locale-lcs}] As the above but the
%codeset (if present) is converted to lower case and any hyphens are
%stripped. For example, if \longargfmt{locale} returns
%\texttt{en-GB.UTF-8}, then \longargfmt{locale-lcs} would return
%\texttt{en-GB.utf8}.
%\item[\shortarg{b} or \longarg{bcp47}] (New to version 1.2.)
%Displays the BCP~47 language tag.
%
%\item[\shortarg{c} or \longarg{cwd}] Displays the current working
%directory.
%\item[\shortarg{m} or \longarg{userhome}] Displays the user's home
%directory.
%\item[\shortarg{t} or \longarg{tmpdir}] Displays the temporary
%directory.
%\item[\shortarg{o} or \longarg{osname}] Displays the operating
%system name.
%\item[\shortarg{r} or \longarg{osversion}] Displays the operating
%system version.
%\item[\shortarg{a} or \longarg{osarch}] Displays the operating
%system architecture.
%\item[\shortarg{n} or \longarg{pdfnow}] Displays the current
% date and time in PDF format. For example 
%\begin{verbatim}
%D:20160704131006+01'00'
%\end{verbatim}
%Note that some, but not all, \TeX\ formats provide
%\cs{pdfcreationdate}, which is more efficient than using the shell
%escape, but this can be used
%as a fallback method for those that don't (for example, \XeTeX).
%\item[\shortarg{d} \meta{file} or \longarg{pdfdate} \meta{file}]
% Displays the last modified time stamp of the given file in PDF format or a blank line
% if the file doesn't exist or the file permissions prohibit this
% action.
%Again some, but not all, \TeX\ formats provide
%\cs{pdffilemoddate}\marg{file}, which is more efficient than using
%the shell escape.
%\item[\shortarg{s} \meta{file} or \longarg{filesize} \meta{file}]
% Displays the size in bytes of the given file or a blank line
% if the file doesn't exist or the file permissions prohibit this
% action.
%Some, but not all, \TeX\ formats provide
%\cs{pdffilesize}\marg{file}, which is more efficient than using
%the shell escape.
%\item[\shortarg{i} \meta{sep} \meta{dir} or \longarg{list}
%\meta{sep} \meta{dir}]
%List all files in the given directory with the output on a single
%line using \meta{sep} as the separator between entries. Note that
%the list doesn't include the full path, just the file names.
%\item[\shortarg{f} \meta{sep} \meta{regex} \meta{dir} or
%\longarg{filterlist} \meta{sep} \meta{regex} \meta{dir}]
%Like the above but only lists those files whose name matches the
%regular expression given in \meta{regex}. Note that this tests for a
%complete match on the file name (not including path). For example,
%if \meta{regexp} is \texttt{foo.*}, it will only match files whose
%name starts with \texttt{foo} (for example, \texttt{foobar} will
%match but \texttt{barfoo} won't).
%\item[\shortarg{u} \meta{file} or \longarg{uri} \meta{file}]
%Displays the URI of the given file or a blank line if the file
%doesn't exist or the file permissions prohibit this action.
%Note that this may include in a percent character in the result. The \TeX\ 
%command \cs{TeXOSQuery} protects against this by changing the
%category code, but if you explicitly call \texttt{texosquery}
%using the shell escape, you'll need to be careful of this.
%\item[\shortarg{p} \meta{file} or \longarg{path} \meta{file}]
%Displays the canonical path of the given file or a blank line if
%the file doesn't exist or the file permissions prohibit this
%action.
%\item[\shortarg{e} \meta{file} or \longarg{dirname} \meta{file}]
%(New to v1.1.)
%Displays the canonical path of the given file's parent (that is, the
%directory containing \meta{file}) or a blank line if
%the file doesn't exist or the file permissions prohibit this
%action. Note that this is different to the Unix-like 
%\texttt{dirname} command, which will return a relative path if
%\meta{file} isn't an absolute path.
%\item[\shortarg{h} or \longarg{help}] Displays help message and
%exits.
%\item[\shortarg{v} or \longarg{version}] Displays version
%information and exits.
%\end{description}
%
%If multiple options are given, they will be processed in the
%order specified in the command line invocation. Each result will
%be displayed on a separate line. As from v1.1, if there are
%multiple actions, each result will be grouped. This makes it easier
%to process the results in \TeX. For example:
%\begin{verbatim}
%texosquery -l
%\end{verbatim}
%This just produces (for me):
%\begin{verbatim}
%en-GB.utf8
%\end{verbatim}
%whereas
%\begin{verbatim}
%texosquery -l -n
%\end{verbatim}
%produces:
%\begin{verbatim}
%{en-GB.utf8}
%{D:20160714112732+01'00'}
%\end{verbatim}
%Note that unavailable information will produce an empty group.
%For example (assuming \texttt{nofile} doesn't exist):
%\begin{verbatim}
%texosquery -l -d nofile -n
%\end{verbatim}
%produces:
%\begin{verbatim}
%{en-GB.utf8}
%{}
%{D:20160714112732+01'00'}
%\end{verbatim}
%whereas 
%\begin{verbatim}
%texosquery -d nofile
%\end{verbatim}
%just displays an empty line.
%
%\section{texosquery.tex: generic \TeX\ code}
%You can run \texttt{texosquery} directly from \TeX's shell escape.
%For example:
%\begin{verbatim}
%\input|"texosquery --locale"
%\end{verbatim}
%However, \texttt{texosquery.tex} provides generic \TeX\ code to do
%this for you and store the result in a control sequence.
%
%Plain \TeX\ users can input this file through the usual \cs{input}
%method:
%\begin{verbatim}
%\input texosquery
%\end{verbatim}
%\LaTeX\ users may also simply input this file:
%\begin{verbatim}
%%\iffalse
%<*package>
%% \CharacterTable
%%  {Upper-case    \A\B\C\D\E\F\G\H\I\J\K\L\M\N\O\P\Q\R\S\T\U\V\W\X\Y\Z
%%   Lower-case    \a\b\c\d\e\f\g\h\i\j\k\l\m\n\o\p\q\r\s\t\u\v\w\x\y\z
%%   Digits        \0\1\2\3\4\5\6\7\8\9
%%   Exclamation   \!     Double quote  \"     Hash (number) \#
%%   Dollar        \$     Percent       \%     Ampersand     \&
%%   Acute accent  \'     Left paren    \(     Right paren   \)
%%   Asterisk      \*     Plus          \+     Comma         \,
%%   Minus         \-     Point         \.     Solidus       \/
%%   Colon         \:     Semicolon     \;     Less than     \<
%%   Equals        \=     Greater than  \>     Question mark \?
%%   Commercial at \@     Left bracket  \[     Backslash     \\
%%   Right bracket \]     Circumflex    \^     Underscore    \_
%%   Grave accent  \`     Left brace    \{     Vertical bar  \|
%%   Right brace   \}     Tilde         \~}
%</package>
%\fi
% \iffalse
% Doc-Source file
% Copyright (C) 2016 Nicola Talbot, all rights reserved.
% \fi
% \iffalse
%<*driver>
\documentclass{ltxdoc}

\usepackage[T1]{fontenc}
\usepackage{tgtermes}
\usepackage{metalogo}
\usepackage[colorlinks,hyperindex=false]{hyperref}

\CheckSum{273}

\RecordChanges
\PageIndex
\CodelineNumbered
\IndexPrologue{}

\renewcommand*{\usage}[1]{\hyperpage{#1}}
\renewcommand*{\main}[1]{\hyperpage{#1}}

\newcommand*{\shortargfmt}[1]{\texttt{-#1}}
\newcommand*{\longargfmt}[1]{\texttt{-{}-#1}}

\newcommand*{\shortarg}[1]{\shortargfmt{#1}%
 \index{texosquery options=\texttt{texosquery} options>#1=\protect\shortargfmt{-#1}|hyperpage}}
\newcommand*{\longarg}[1]{\longargfmt{#1}%
 \index{texosquery options=\texttt{texosquery} options>#1=\protect\longargfmt{#1}|hyperpage}}

\newenvironment{important}
 {\begin{quote}\textbf{Important Note:}}
 {\end{quote}}

\begin{document}
\DocInput{texosquery.dtx}
\end{document}
%</driver>
%\fi
%
%\DeleteShortVerb{|}
%
%\title{texosquery: query OS information from \TeX}
%\author{Nicola L. C. Talbot\\
%\href{http://www.dickimaw-books.com/}{\nolinkurl{dickimaw-books.com}}
%\and Paulo Cereda}
%\date{2016-07-14 (v1.1)}
%\maketitle
%
%\begin{abstract}
%\texttt{texosquery.jar} is a cross-platform Java application
%to query certain OS information designed for use in \TeX's
%shell escape mechanism.
%The accompanying \TeX\ code provides a simple interface to the
%\texttt{texosquery} Java application, which may be skipped
%if you want to explicitly use \cs{input} (but take
%care of any special characters appearing in the result).
%\end{abstract}
%
%\begin{important}
%You will need \TeX's shell escape enabled,
%and you will also need the Java Runtime Environment (JRE) 
%installed to use \texttt{texosquery}.
%\end{important}
%
%If you want to rebuild the application, instructions for
%compiling the source code (including the code for this document)
%are in the accompanying \texttt{README.md} file.
%
%\tableofcontents
%
%\section{texosquery.jar: the Java application}
%The \texttt{texosquery} Java command line application looks up
%certain system information that may be of use in \TeX\ 
%documents. All this information can easily be obtained using 
%native commands, but the Java application allows an 
%OS-independent approach with results that can easily be
%captured by \TeX's shell-escape without having to strip 
%formatting information.
%
%\begin{important}
%\texttt{texosquery} provides read-only actions, and I don't 
%intend adding any actions that modify system settings or files.
%\end{important}
%
%Since the application is designed to work with \TeX, each
%function will display the result on a single line without
%formatting. (For multiple results, each line is grouped 
%from v1.1.) A blank line (or empty group) will be displayed
%if the information isn't available. A forward slash is always 
%used as a directory divider, regardless of the operating 
%system, so the result can be used, for example, in \cs{input} or
%\cs{includegraphics}.
%
%If an input file name is required (for example, with the
%\longargfmt{pdfdate} argument described below) then the file may be in
%the current working directory, relative to the current directory
%(with forward slash \texttt{/} as the directory divider), an absolute path 
%(again with forward slash) or on \TeX's path (in
%which case, \texttt{kpsewhich} is used to locate it).
%
%Command line invocation:
%\begin{flushleft}\ttfamily
%texosquery \meta{action} ...
%\end{flushleft}
%Available actions (at least one required):
%\begin{description}
%\item[\shortarg{L} or \longarg{locale}] Display the locale
%information in the form
%\begin{flushleft}\ttfamily
%\meta{lang}-\meta{region}.\meta{codeset}@\meta{modifier}
%\end{flushleft}
%where \meta{lang} is the ISO code for the language (e.g.\ 
%\texttt{en}),
%\meta{region} is the ISO code for the region (e.g.\ \texttt{GB}),
%\meta{codeset} is the default code set (e.g.\ \texttt{UTF-8})
% and \meta{modifier} is the modifier. Elements may be omitted
% if unavailable. For example, \texttt{en-GB.UTF-8} has the
% \meta{modifier} omitted, and \texttt{en} has all but the language
% omitted.
%\item[\shortarg{l} or \longarg{locale-lcs}] As the above but the
%codeset (if present) is converted to lower case and any hyphens are
%stripped. For example, if \longargfmt{locale} returns
%\texttt{en-GB.UTF-8}, then \longargfmt{locale-lcs} would return
%\texttt{en-GB.utf8}.
%\item[\shortarg{b} or \longarg{bcp47}] (New to version 1.2.)
%Displays the BCP~47 language tag.
%
%\item[\shortarg{c} or \longarg{cwd}] Displays the current working
%directory.
%\item[\shortarg{m} or \longarg{userhome}] Displays the user's home
%directory.
%\item[\shortarg{t} or \longarg{tmpdir}] Displays the temporary
%directory.
%\item[\shortarg{o} or \longarg{osname}] Displays the operating
%system name.
%\item[\shortarg{r} or \longarg{osversion}] Displays the operating
%system version.
%\item[\shortarg{a} or \longarg{osarch}] Displays the operating
%system architecture.
%\item[\shortarg{n} or \longarg{pdfnow}] Displays the current
% date and time in PDF format. For example 
%\begin{verbatim}
%D:20160704131006+01'00'
%\end{verbatim}
%Note that some, but not all, \TeX\ formats provide
%\cs{pdfcreationdate}, which is more efficient than using the shell
%escape, but this can be used
%as a fallback method for those that don't (for example, \XeTeX).
%\item[\shortarg{d} \meta{file} or \longarg{pdfdate} \meta{file}]
% Displays the last modified time stamp of the given file in PDF format or a blank line
% if the file doesn't exist or the file permissions prohibit this
% action.
%Again some, but not all, \TeX\ formats provide
%\cs{pdffilemoddate}\marg{file}, which is more efficient than using
%the shell escape.
%\item[\shortarg{s} \meta{file} or \longarg{filesize} \meta{file}]
% Displays the size in bytes of the given file or a blank line
% if the file doesn't exist or the file permissions prohibit this
% action.
%Some, but not all, \TeX\ formats provide
%\cs{pdffilesize}\marg{file}, which is more efficient than using
%the shell escape.
%\item[\shortarg{i} \meta{sep} \meta{dir} or \longarg{list}
%\meta{sep} \meta{dir}]
%List all files in the given directory with the output on a single
%line using \meta{sep} as the separator between entries. Note that
%the list doesn't include the full path, just the file names.
%\item[\shortarg{f} \meta{sep} \meta{regex} \meta{dir} or
%\longarg{filterlist} \meta{sep} \meta{regex} \meta{dir}]
%Like the above but only lists those files whose name matches the
%regular expression given in \meta{regex}. Note that this tests for a
%complete match on the file name (not including path). For example,
%if \meta{regexp} is \texttt{foo.*}, it will only match files whose
%name starts with \texttt{foo} (for example, \texttt{foobar} will
%match but \texttt{barfoo} won't).
%\item[\shortarg{u} \meta{file} or \longarg{uri} \meta{file}]
%Displays the URI of the given file or a blank line if the file
%doesn't exist or the file permissions prohibit this action.
%Note that this may include in a percent character in the result. The \TeX\ 
%command \cs{TeXOSQuery} protects against this by changing the
%category code, but if you explicitly call \texttt{texosquery}
%using the shell escape, you'll need to be careful of this.
%\item[\shortarg{p} \meta{file} or \longarg{path} \meta{file}]
%Displays the canonical path of the given file or a blank line if
%the file doesn't exist or the file permissions prohibit this
%action.
%\item[\shortarg{e} \meta{file} or \longarg{dirname} \meta{file}]
%(New to v1.1.)
%Displays the canonical path of the given file's parent (that is, the
%directory containing \meta{file}) or a blank line if
%the file doesn't exist or the file permissions prohibit this
%action. Note that this is different to the Unix-like 
%\texttt{dirname} command, which will return a relative path if
%\meta{file} isn't an absolute path.
%\item[\shortarg{h} or \longarg{help}] Displays help message and
%exits.
%\item[\shortarg{v} or \longarg{version}] Displays version
%information and exits.
%\end{description}
%
%If multiple options are given, they will be processed in the
%order specified in the command line invocation. Each result will
%be displayed on a separate line. As from v1.1, if there are
%multiple actions, each result will be grouped. This makes it easier
%to process the results in \TeX. For example:
%\begin{verbatim}
%texosquery -l
%\end{verbatim}
%This just produces (for me):
%\begin{verbatim}
%en-GB.utf8
%\end{verbatim}
%whereas
%\begin{verbatim}
%texosquery -l -n
%\end{verbatim}
%produces:
%\begin{verbatim}
%{en-GB.utf8}
%{D:20160714112732+01'00'}
%\end{verbatim}
%Note that unavailable information will produce an empty group.
%For example (assuming \texttt{nofile} doesn't exist):
%\begin{verbatim}
%texosquery -l -d nofile -n
%\end{verbatim}
%produces:
%\begin{verbatim}
%{en-GB.utf8}
%{}
%{D:20160714112732+01'00'}
%\end{verbatim}
%whereas 
%\begin{verbatim}
%texosquery -d nofile
%\end{verbatim}
%just displays an empty line.
%
%\section{texosquery.tex: generic \TeX\ code}
%You can run \texttt{texosquery} directly from \TeX's shell escape.
%For example:
%\begin{verbatim}
%\input|"texosquery --locale"
%\end{verbatim}
%However, \texttt{texosquery.tex} provides generic \TeX\ code to do
%this for you and store the result in a control sequence.
%
%Plain \TeX\ users can input this file through the usual \cs{input}
%method:
%\begin{verbatim}
%\input texosquery
%\end{verbatim}
%\LaTeX\ users may also simply input this file:
%\begin{verbatim}
%\input{texosquery}
%\end{verbatim}
%but may prefer the standard package approach:
%\begin{verbatim}
%\usepackage{texosquery}
%\end{verbatim}
%
%\begin{important}
%The commands described below are all fragile.
%\end{important}
%
%The basic command to run \texttt{texosquery} and capture its output
%in a control sequence is:
%\DescribeMacro\TeXOSQuery
%\begin{flushleft}\ttfamily
%\cs{TeXOSQuery}\marg{cs}\marg{args}
%\end{flushleft}
%where \meta{cs} is the control sequence in which to store the
%result and \meta{args} are the command line arguments to pass to 
%\texttt{texosquery}. This first locally changes the category code
%of some problematic characters that may appear in the result.
%The only special characters that aren't changed are the backslash
%\verb|\|, curly braces \verb|{| and \verb|}|, and hash \verb|#|.
%(\texttt{texosquery.jar} will replace \verb|#| with \verb|\#| in places
%where it might possibly occur in the result, but in general it's
%best to avoid these characters in file names.)
%
%There are some short cut commands for convenience, described below.
%If any of these commands cause an error message in the form:
%\begin{verbatim}
%I can't find file `|texosquery'.
%\end{verbatim}
%then check that you have the shell escape on. If the error 
%persists with the shell escape enabled
%(and not restricted), check that \texttt{texosquery} is on 
%your system's path. To do this, open a command prompt or terminal and
%type \texttt{texosquery}. If it isn't installed correctly, there will be a
%message like:
%\begin{verbatim}
%`texosquery' is not recognised
%\end{verbatim}
%or
%\begin{verbatim}
%texosquery: command not found
%\end{verbatim}
%If this happens, check the installation. (Instructions are
%in the accompanying \texttt{README.md} file.)
%
%Dry run mode is determined by the conditional
%\DescribeMacro\ifTeXOSQueryDryRun
%\begin{verbatim}
%\ifTeXOSQueryDryRun
%\end{verbatim}
%If true, the shell escape won't be used and the requested command
%invocation will be printed in the transcript file prefixed with
%\begin{verbatim}
%TeXOSQuery:
%\end{verbatim}
%(the control sequence \meta{cs} will be set to empty).
%This conditional will automatically be switched on unless
%\cs{shellescape} or \cs{pdfshellescape} is 1. (If
%\texttt{texosquery.jar} is later allowed on the restricted list,
%newer versions may change this default.)
%
%If multiple queries are required, it's more efficient to
%perform them all in one go. For example:
%\begin{verbatim}
%\TeXOSQuery{\result}{-l -n -o}
%
%\def\parseresult#1#2#3{%
%  Locale: #1. Now: #2. OS: #3.%
%}
%
%\ifx\result\empty
%  Query failed.
%\else
%  \expandafter\parseresult\result
%\fi
%\end{verbatim}
%(Make sure you have at least v1.1 for this to work correctly.)
%
%The locale (\shortarg{l} or \longarg{locale-lcs}) information can be
%obtained using:
%\DescribeMacro\TeXOSQueryLocale
%\begin{flushleft}\ttfamily
%\cs{TeXOSQueryLocale}\marg{cs}
%\end{flushleft}
%Note that this uses the lower case codeset form, which has a better
%chance of matching the encoding names used by the \texttt{inputenc}
%package. If you want the unprocessed codeset name, you can do:
%\begin{flushleft}\ttfamily
%\cs{TeXOSQuery}\marg{cs}\{\shortarg{L}\}
%\end{flushleft}
%
%The BCP 47 language tag (\shortarg{b} or \longarg{bcp47}) can be obtained using:
%\DescribeMacro\TeXOSQueryLangTag
%\begin{flushleft}\ttfamily
%\cs{TeXOSQueryLangTag}\marg{cs}
%\end{flushleft}
%
%The current working directory (\shortarg{c} or \longarg{cwd}) can
%be obtained using:
%\DescribeMacro\TeXOSQueryCwd
%\begin{flushleft}\ttfamily
%\cs{TeXOSQueryCwd}\marg{cs}
%\end{flushleft}
%
%The home directory (\shortarg{m} or \longarg{userhome}) can
%be obtained using:
%\DescribeMacro\TeXOSQueryHome
%\begin{flushleft}\ttfamily
%\cs{TeXOSQueryHome}\marg{cs}
%\end{flushleft}
%
%The temporary directory (\shortarg{t} or \longarg{tmpdir}) can
%be obtained using:
%\DescribeMacro\TeXOSQueryTmpDir
%\begin{flushleft}\ttfamily
%\cs{TeXOSQueryTmpDir}\marg{cs}
%\end{flushleft}
%
%The OS name (\shortarg{o} or \longarg{osname}) can
%be obtained using:
%\DescribeMacro\TeXOSQueryName
%\begin{flushleft}\ttfamily
%\cs{TeXOSQueryName}\marg{cs}
%\end{flushleft}
%
%The OS version (\shortarg{r} or \longarg{osversion}) can
%be obtained using:
%\DescribeMacro\TeXOSQueryVersion
%\begin{flushleft}\ttfamily
%\cs{TeXOSQueryVersion}\marg{cs}
%\end{flushleft}
%
%The OS architecture (\shortarg{a} or \longarg{osarch}) can
%be obtained using:
%\DescribeMacro\TeXOSQueryArch
%\begin{flushleft}\ttfamily
%\cs{TeXOSQueryArch}\marg{cs}
%\end{flushleft}
%
%The current date-time stamp in PDF format (\shortarg{n} or
%\longarg{pdfnow}) can be obtained using:
%\DescribeMacro\TeXOSQueryNow
%\begin{flushleft}\ttfamily
%\cs{TeXOSQueryNow}\marg{cs}
%\end{flushleft}
%This is provided for the benefit of users who don't have
%\cs{pdfcreationdate} defined by their \TeX\ format (for example,
%\XeTeX). As from v1.1, this ensures that the initial \texttt{D} has
%category code 12 (which won't happen if \cs{TeXOSQuery} is used
%explicitly).
%
%\begin{important}
%The remaining commands all require extra arguments after the
%relevant switch. These are automatically enclosed in single-quotes
%to protect any spaces. If the argument actually contains any single-quote
%characters, make sure you use \verb|\string\'| to prevent
%interference. \emph{However, in general it's a very bad idea to use
%quotes as part of a file name (rather than using them as
%delimiters).}
%Since the file name may need to be obtained from \cs{jobname},
%which sometimes includes double-quotes, the first double-quote pair
%found is stripped in file name arguments. Any other double-quotes
%will need to be protected in the same manner as single-quotes
%(but, again, this shouldn't be an issue if you use a safe
%file naming scheme). Any paths should use a
%forward slash for the directory divider.
%\end{important}
%
%The modification date-time stamp in PDF format for a file
%(\shortarg{d} or \longarg{pdfdate}) can be obtained using:
%\DescribeMacro\TeXOSQueryFileDate
%\begin{flushleft}\ttfamily
%\cs{TeXOSQueryFileDate}\marg{cs}\marg{filename}
%\end{flushleft}
%where \meta{filename} is the name of the file.
%This is provided for the benefit of users who don't have
%\cs{pdffilemoddate} defined by their \TeX\ format. As from v1.1,
%this ensures that the initial \texttt{D} has category code 12 (which 
%won't happen if \cs{TeXOSQuery} is used explicitly).
%
%The size in bytes of a file
%(\shortarg{s} or \longarg{filesize}) can be obtained using:
%\DescribeMacro\TeXOSQueryFileSize
%\begin{flushleft}\ttfamily
%\cs{TeXOSQueryFileSize}\marg{cs}\marg{filename}
%\end{flushleft}
%where \meta{filename} is the name of the file.
%This is provided for the benefit of users who don't have
%\cs{pdffilesize} defined by their \TeX\ format.
%
%The list of files in a given directory
%(\shortarg{i} or \longarg{list}) can be obtained using:
%\DescribeMacro\TeXOSQueryFileList
%\begin{flushleft}\ttfamily
%\cs{TeXOSQueryFileList}\marg{cs}\marg{sep}\marg{dir}
%\end{flushleft}
%where \meta{sep} is the separator and \meta{dir} is the directory name.
%For example:
%\begin{verbatim}
%\TeXOSQueryFileList{\result}{,}{.}
%\end{verbatim}
%will store a comma-separated list of all the files contained in the
%current directory in the control sequence \cs{result}.
%
%A filtered list of files in a given directory
%(\shortarg{f} or \longarg{filterlist}) can be obtained using:
%\DescribeMacro\TeXOSQueryFilterFileList
%\begin{flushleft}\ttfamily
%\cs{TeXOSQueryFilterFileList}\marg{cs}\marg{sep}\marg{regex}\marg{dir}
%\end{flushleft}
%where \meta{regex} is a regular expression. \emph{Take care of any backslashes
%in the regular expression!}
%For example, to list only those files that have an extension:
%\begin{verbatim}
%\TeXOSQueryFilterFileList{\result}{,}{.+\string\..*}{.}
%\end{verbatim}
%Note the use of \verb|\string\.| to ensure that \cs{.} isn't
%interpreted as a command. Another example, list only \texttt{.png}
%and \texttt{.jpg} files in the directory called \texttt{images}:
%\begin{verbatim}
%\TeXOSQueryFilterFileList{\result}{,}{.+\string\.(jpg|png)}{images}
%\end{verbatim}
%
%The URI of a file
%(\shortarg{u} or \longarg{uri}) can be obtained using:
%\DescribeMacro\TeXOSQueryFileURI
%\begin{flushleft}\ttfamily
%\cs{TeXOSQueryFileURI}\marg{cs}\marg{filename}
%\end{flushleft}
%where \meta{filename} is the name of the file.
%(Any percent symbols \verb|%| contained in the URI will have
%their category code set to 12.)
%
%The canonical path of a file
%(\shortarg{p} or \longarg{path}) can be obtained using:
%\DescribeMacro\TeXOSQueryFilePath
%\begin{flushleft}\ttfamily
%\cs{TeXOSQueryFilePath}\marg{cs}\marg{filename}
%\end{flushleft}
%where \meta{filename} is the name of the file.
%
%The canonical path of a file's parent
%(\shortarg{e} or \longarg{dirname}) can be obtained using:
%\DescribeMacro\TeXOSQueryDirName
%\begin{flushleft}\ttfamily
%\cs{TeXOSQueryDirName}\marg{cs}\marg{filename}
%\end{flushleft}
%where \meta{filename} is the name of the file.
%
%\StopEventually{\phantomsection
%\addcontentsline{toc}{section}{Change History}%
%\PrintChanges
%\addcontentsline{toc}{section}{\indexname}%
%\PrintIndex
%}
%
%\section{The Code}
%\changes{1.0}{2016-07-08}{Initial release}
%\subsection{Generic \TeX\ Code}
%\iffalse
%    \begin{macrocode}
%<*texosquery.tex>
%    \end{macrocode}
%\fi
%Change category code of \texttt{@} if necessary.
%    \begin{macrocode}
\ifnum\catcode`\@=11\relax
  \def\@texosquery@restore@at{}%
\else
  \expandafter\edef\csname @texosquery@restore@at\endcsname{%
    \noexpand\catcode`\noexpand\@=\number\catcode`\@\relax
  }%
 \catcode`\@=11\relax
\fi
%    \end{macrocode}
% Check if already loaded.
%    \begin{macrocode}
\ifx\TeXOSQuery\undefined
\else
  \@texosquery@restore@at
  \expandafter\endinput
\fi
%    \end{macrocode}
%\begin{macro}{\TeXOSInvokerName}
%The name of the \texttt{texosquery} application.
%    \begin{macrocode}
\def\TeXOSInvokerName{texosquery}
%    \end{macrocode}
%\end{macro}
%
%\begin{macro}{\TeXOSQueryInvoker}
%If we're using \LaTeX, we'll need to use \cs{@@input} rather
%than \cs{input}.
%    \begin{macrocode}
\ifx\@@input\undefined
  \def\TeXOSQueryInvoker#1{\input|"\TeXOSInvokerName\space#1" }
\else
  \def\TeXOSQueryInvoker#1{\@@input|"\TeXOSInvokerName\space#1" }
\fi
%    \end{macrocode}
%\end{macro}
%
%\begin{macro}{\ifTeXOSQueryDryRun}
%Provide a dry-run mode.
%\changes{1.1}{2016-07-14}{dry run mode only false by default if with
%unrestricted mode}
%    \begin{macrocode}
\newif\ifTeXOSQueryDryRun
\TeXOSQueryDryRuntrue
%    \end{macrocode}
%\end{macro}
%
%If shell escape is unrestricted, automatically switch off dry-run mode.
%    \begin{macrocode}
\ifx\shellescape\undefined
  \ifx\pdfshellescape\undefined
  \else
    \ifnum\pdfshellescape=1\relax
      \TeXOSQueryDryRunfalse
    \fi
  \fi
\else
  \ifnum\shellescape=1\relax
    \TeXOSQueryDryRunfalse
  \fi
\fi
%    \end{macrocode}
%
%\begin{macro}{\TeXOSQuery}
%Use \texttt{texosquery} with the option given in the second
%argument and store the result in control sequence given in the
%first argument.
%    \begin{macrocode}
\def\TeXOSQuery#1#2{%
  \ifTeXOSQueryDryRun
    \begingroup
      \newlinechar=`\^^J
      \message{^^JTeXOSQuery: \TeXOSInvokerName\space#2^^J}%
    \endgroup
    \def#1{}%
  \else
    \begingroup
    \endlinechar=-1\relax
%    \end{macrocode}
%Just in case the result contains any awkward characters that have
%some special meaning to \TeX. (Can't really protect against hash,
%backslash or curly braces, but they're unlikely to occur unless
%the user has a very wacky and inappropriate file naming scheme.)
%    \begin{macrocode}
    \catcode`\-=12\relax
    \catcode`\_=12\relax
    \catcode`\^=12\relax
    \catcode`\~=12\relax
    \catcode`\$=12\relax
    \catcode`\&=12\relax
    \catcode`\"=12\relax
    \catcode`\'=12\relax
    \catcode`\.=12\relax
    \catcode`\/=12\relax
    \catcode`\:=12\relax
    \catcode`\;=12\relax
    \catcode`\%=12\relax
    \everyeof{\noexpand}\relax
    \edef\x{\endgroup\def\noexpand#1{\TeXOSQueryInvoker{#2}}}\x
  \fi
}
%    \end{macrocode}
%\end{macro}
%
%Now for some convenient shortcuts so the user doesn't have to
%remember the command line options. \cs{string} is used in
%case the hyphen character has been made active.
%\begin{macro}{\TeXOSQueryLocale}
%Query the locale and store the result in the control sequence
%provided in the argument.
%    \begin{macrocode}
\def\TeXOSQueryLocale#1{\TeXOSQuery{#1}{\string-l}}
%    \end{macrocode}
%\end{macro}
%
%\begin{macro}{\TeXOSQueryLangTag}
%Query the language tag and store the result in the control sequence
%provided in the argument.
%\changes{1.2}{??}{new}
%    \begin{macrocode}
\def\TeXOSQueryLangTag#1{\TeXOSQuery{#1}{\string-b}}
%    \end{macrocode}
%\end{macro}
%
%\begin{macro}{\TeXOSQueryCwd}
%Query the current working directory.
%    \begin{macrocode}
\def\TeXOSQueryCwd#1{\TeXOSQuery{#1}{\string-c}}
%    \end{macrocode}
%\end{macro}
%
%\begin{macro}{\TeXOSQueryHome}
%Query the user's home directory.
%    \begin{macrocode}
\def\TeXOSQueryHome#1{\TeXOSQuery{#1}{\string-m}}
%    \end{macrocode}
%\end{macro}
%
%\begin{macro}{\TeXOSQueryTmpDir}
%Query the temporary directory.
%    \begin{macrocode}
\def\TeXOSQueryTmpDir#1{\TeXOSQuery{#1}{\string-t}}
%    \end{macrocode}
%\end{macro}
%
%\begin{macro}{\TeXOSQueryVersion}
%Query the operating system version.
%    \begin{macrocode}
\def\TeXOSQueryVersion#1{\TeXOSQuery{#1}{\string-r}}
%    \end{macrocode}
%\end{macro}
%
%\begin{macro}{\TeXOSQueryArch}
%Query the operating system architecture.
%    \begin{macrocode}
\def\TeXOSQueryArch#1{\TeXOSQuery{#1}{\string-a}}
%    \end{macrocode}
%\end{macro}
%
%\begin{macro}{\TeXOSQueryName}
%Query the operating system name.
%    \begin{macrocode}
\def\TeXOSQueryName#1{\TeXOSQuery{#1}{\string-o}}
%    \end{macrocode}
%\end{macro}
%
%\begin{macro}{\TeXOSQueryNow}
%\changes{1.1}{2016-07-14}{changed catcode of D to 12}
%Query the current time stamp.
%    \begin{macrocode}
\def\TeXOSQueryNow#1{%
%    \end{macrocode}
% The \texttt{D} needs category code 12. (Don't need to worry about
% \texttt{Z} as \texttt{texosquery.jar} uses \texttt{+00'00'} for
% UTC+0.) This change can't be done with the other catcode changes
% in \cs{TeXOSQuery}, as this is only appropriate for the PDF dates.
% Save and restore the catcode rather than fiddle around with
% scoping.
%    \begin{macrocode}
  \edef\@texosquery@restore@D{%
    \noexpand\catcode`\noexpand\D=\the\catcode`\D\relax}%
  \catcode`\D=12\relax
  \TeXOSQuery{#1}{\string-n}%
  \@texosquery@restore@D
}
%    \end{macrocode}
%\end{macro}
%
%If the file name is supplied using \cs{jobname} it may have
%double-quotes which will interfere with things.
%\begin{macro}{\texosquerystripquotes}
%    \begin{macrocode}
\def\texosquerystripquotes#1{%
  \@texosquery@stripquotes#1\@mid@texosquery@stripquotes
    "\relax"\relax\@end@texosquery@stripquotes
}
\def\@texosquery@stripquotes#1"#2"{%
  \@@texosquery@stripquotes#1#2%
}
\def\@@texosquery@stripquotes#1\@mid@texosquery@stripquotes#2\@end@texosquery@stripquotes{%
  #1%
}
%    \end{macrocode}
%\end{macro}
%
%\begin{macro}{\TeXOSQueryFileDate}
%\changes{1.1}{2016-07-14}{changed catcode of D to 12}
%Query the time stamp of the file given in the second argument.
%    \begin{macrocode}
\def\TeXOSQueryFileDate#1#2{%
  \edef\@texosquery@restore@D{%
    \noexpand\catcode`\noexpand\D=\the\catcode`\D\relax}%
  \catcode`\D=12\relax
  \TeXOSQuery{#1}{\string-d \string'\texosquerystripquotes{#2}\string'}%
  \@texosquery@restore@D
}
%    \end{macrocode}
%\end{macro}
%
%\begin{macro}{\TeXOSQueryFileSize}
%Query the size of the file given in the second argument.
%    \begin{macrocode}
\def\TeXOSQueryFileSize#1#2{\TeXOSQuery{#1}{\string-s 
 \string'\texosquerystripquotes{#2}\string'}}
%    \end{macrocode}
%\end{macro}
%
%\begin{macro}{\TeXOSQueryFileList}
%List all files in the directory given in the third argument,
%separated by the second argument.
%    \begin{macrocode}
\def\TeXOSQueryFileList#1#2#3{\TeXOSQuery{#1}{%
 \string-i \string'#2\string' 
 \string'\texosquerystripquotes{#3}\string'}}
%    \end{macrocode}
%\end{macro}
%
%\begin{macro}{\TeXOSQueryFilterFileList}
%Filtered list files in the directory given in the fourth argument,
%separated by the second argument. The third argument is the regular
%expression used to filter the list. \emph{Take care of backslashes
%in the regular expression!}
%    \begin{macrocode}
\def\TeXOSQueryFilterFileList#1#2#3#4{\TeXOSQuery{#1}{%
 \string-f \string'#2\string' \string'#3\string' 
 \string'\texosquerystripquotes{#4}\string'}}
%    \end{macrocode}
%\end{macro}
%
%\begin{macro}{\TeXOSQueryFileURI}
%Get the URI of the file given in the second argument.
%    \begin{macrocode}
\def\TeXOSQueryFileURI#1#2{\TeXOSQuery{#1}{\string-u 
 \string'\texosquerystripquotes{#2}\string'}}
%    \end{macrocode}
%\end{macro}
%
%\begin{macro}{\TeXOSQueryFilePath}
%Get the canonical path of the file given in the second argument.
%    \begin{macrocode}
\def\TeXOSQueryFilePath#1#2{\TeXOSQuery{#1}{\string-p 
 \string'\texosquerystripquotes{#2}\string'}}
%    \end{macrocode}
%\end{macro}
%
%\begin{macro}{\TeXOSQueryDirName}
%\changes{1.1}{2016-07-14}{new}
%Get the canonical path of the directory containing the file given 
%in the second argument.
%    \begin{macrocode}
\def\TeXOSQueryDirName#1#2{\TeXOSQuery{#1}{\string-e 
 \string'\texosquerystripquotes{#2}\string'}}
%    \end{macrocode}
%\end{macro}
%
%All done.
%Restore the category code of \texttt{@}:
%    \begin{macrocode}
\@texosquery@restore@at
%    \end{macrocode}
%\iffalse
%    \begin{macrocode}
%</texosquery.tex>
%    \end{macrocode}
%\fi
%
%\subsection{\LaTeX\ Code}
%This is just a simple wrapper for \texttt{texosquery.tex}
%so that it can be loaded using \LaTeX's standard \cs{usepackage}
%method.
%\iffalse
%    \begin{macrocode}
%<*texosquery.sty>
%    \end{macrocode}
%\fi
%Identify package:
%    \begin{macrocode}
\NeedsTeXFormat{LaTeX2e}
\ProvidesPackage{texosquery}[2016/07/14 v1.1 (NLCT)]
%    \end{macrocode}
%Load \texttt{texosquery.tex}:
%    \begin{macrocode}
\input{texosquery}
%    \end{macrocode}
%That's it!
%\iffalse
%    \begin{macrocode}
%</texosquery.sty>
%    \end{macrocode}
%\fi
%
%\Finale
\endinput

%\end{verbatim}
%but may prefer the standard package approach:
%\begin{verbatim}
%\usepackage{texosquery}
%\end{verbatim}
%
%\begin{important}
%The commands described below are all fragile.
%\end{important}
%
%The basic command to run \texttt{texosquery} and capture its output
%in a control sequence is:
%\DescribeMacro\TeXOSQuery
%\begin{flushleft}\ttfamily
%\cs{TeXOSQuery}\marg{cs}\marg{args}
%\end{flushleft}
%where \meta{cs} is the control sequence in which to store the
%result and \meta{args} are the command line arguments to pass to 
%\texttt{texosquery}. This first locally changes the category code
%of some problematic characters that may appear in the result.
%The only special characters that aren't changed are the backslash
%\verb|\|, curly braces \verb|{| and \verb|}|, and hash \verb|#|.
%(\texttt{texosquery.jar} will replace \verb|#| with \verb|\#| in places
%where it might possibly occur in the result, but in general it's
%best to avoid these characters in file names.)
%
%There are some short cut commands for convenience, described below.
%If any of these commands cause an error message in the form:
%\begin{verbatim}
%I can't find file `|texosquery'.
%\end{verbatim}
%then check that you have the shell escape on. If the error 
%persists with the shell escape enabled
%(and not restricted), check that \texttt{texosquery} is on 
%your system's path. To do this, open a command prompt or terminal and
%type \texttt{texosquery}. If it isn't installed correctly, there will be a
%message like:
%\begin{verbatim}
%`texosquery' is not recognised
%\end{verbatim}
%or
%\begin{verbatim}
%texosquery: command not found
%\end{verbatim}
%If this happens, check the installation. (Instructions are
%in the accompanying \texttt{README.md} file.)
%
%Dry run mode is determined by the conditional
%\DescribeMacro\ifTeXOSQueryDryRun
%\begin{verbatim}
%\ifTeXOSQueryDryRun
%\end{verbatim}
%If true, the shell escape won't be used and the requested command
%invocation will be printed in the transcript file prefixed with
%\begin{verbatim}
%TeXOSQuery:
%\end{verbatim}
%(the control sequence \meta{cs} will be set to empty).
%This conditional will automatically be switched on unless
%\cs{shellescape} or \cs{pdfshellescape} is 1. (If
%\texttt{texosquery.jar} is later allowed on the restricted list,
%newer versions may change this default.)
%
%If multiple queries are required, it's more efficient to
%perform them all in one go. For example:
%\begin{verbatim}
%\TeXOSQuery{\result}{-l -n -o}
%
%\def\parseresult#1#2#3{%
%  Locale: #1. Now: #2. OS: #3.%
%}
%
%\ifx\result\empty
%  Query failed.
%\else
%  \expandafter\parseresult\result
%\fi
%\end{verbatim}
%(Make sure you have at least v1.1 for this to work correctly.)
%
%The locale (\shortarg{l} or \longarg{locale-lcs}) information can be
%obtained using:
%\DescribeMacro\TeXOSQueryLocale
%\begin{flushleft}\ttfamily
%\cs{TeXOSQueryLocale}\marg{cs}
%\end{flushleft}
%Note that this uses the lower case codeset form, which has a better
%chance of matching the encoding names used by the \texttt{inputenc}
%package. If you want the unprocessed codeset name, you can do:
%\begin{flushleft}\ttfamily
%\cs{TeXOSQuery}\marg{cs}\{\shortarg{L}\}
%\end{flushleft}
%
%The BCP 47 language tag (\shortarg{b} or \longarg{bcp47}) can be obtained using:
%\DescribeMacro\TeXOSQueryLangTag
%\begin{flushleft}\ttfamily
%\cs{TeXOSQueryLangTag}\marg{cs}
%\end{flushleft}
%
%The current working directory (\shortarg{c} or \longarg{cwd}) can
%be obtained using:
%\DescribeMacro\TeXOSQueryCwd
%\begin{flushleft}\ttfamily
%\cs{TeXOSQueryCwd}\marg{cs}
%\end{flushleft}
%
%The home directory (\shortarg{m} or \longarg{userhome}) can
%be obtained using:
%\DescribeMacro\TeXOSQueryHome
%\begin{flushleft}\ttfamily
%\cs{TeXOSQueryHome}\marg{cs}
%\end{flushleft}
%
%The temporary directory (\shortarg{t} or \longarg{tmpdir}) can
%be obtained using:
%\DescribeMacro\TeXOSQueryTmpDir
%\begin{flushleft}\ttfamily
%\cs{TeXOSQueryTmpDir}\marg{cs}
%\end{flushleft}
%
%The OS name (\shortarg{o} or \longarg{osname}) can
%be obtained using:
%\DescribeMacro\TeXOSQueryName
%\begin{flushleft}\ttfamily
%\cs{TeXOSQueryName}\marg{cs}
%\end{flushleft}
%
%The OS version (\shortarg{r} or \longarg{osversion}) can
%be obtained using:
%\DescribeMacro\TeXOSQueryVersion
%\begin{flushleft}\ttfamily
%\cs{TeXOSQueryVersion}\marg{cs}
%\end{flushleft}
%
%The OS architecture (\shortarg{a} or \longarg{osarch}) can
%be obtained using:
%\DescribeMacro\TeXOSQueryArch
%\begin{flushleft}\ttfamily
%\cs{TeXOSQueryArch}\marg{cs}
%\end{flushleft}
%
%The current date-time stamp in PDF format (\shortarg{n} or
%\longarg{pdfnow}) can be obtained using:
%\DescribeMacro\TeXOSQueryNow
%\begin{flushleft}\ttfamily
%\cs{TeXOSQueryNow}\marg{cs}
%\end{flushleft}
%This is provided for the benefit of users who don't have
%\cs{pdfcreationdate} defined by their \TeX\ format (for example,
%\XeTeX). As from v1.1, this ensures that the initial \texttt{D} has
%category code 12 (which won't happen if \cs{TeXOSQuery} is used
%explicitly).
%
%\begin{important}
%The remaining commands all require extra arguments after the
%relevant switch. These are automatically enclosed in single-quotes
%to protect any spaces. If the argument actually contains any single-quote
%characters, make sure you use \verb|\string\'| to prevent
%interference. \emph{However, in general it's a very bad idea to use
%quotes as part of a file name (rather than using them as
%delimiters).}
%Since the file name may need to be obtained from \cs{jobname},
%which sometimes includes double-quotes, the first double-quote pair
%found is stripped in file name arguments. Any other double-quotes
%will need to be protected in the same manner as single-quotes
%(but, again, this shouldn't be an issue if you use a safe
%file naming scheme). Any paths should use a
%forward slash for the directory divider.
%\end{important}
%
%The modification date-time stamp in PDF format for a file
%(\shortarg{d} or \longarg{pdfdate}) can be obtained using:
%\DescribeMacro\TeXOSQueryFileDate
%\begin{flushleft}\ttfamily
%\cs{TeXOSQueryFileDate}\marg{cs}\marg{filename}
%\end{flushleft}
%where \meta{filename} is the name of the file.
%This is provided for the benefit of users who don't have
%\cs{pdffilemoddate} defined by their \TeX\ format. As from v1.1,
%this ensures that the initial \texttt{D} has category code 12 (which 
%won't happen if \cs{TeXOSQuery} is used explicitly).
%
%The size in bytes of a file
%(\shortarg{s} or \longarg{filesize}) can be obtained using:
%\DescribeMacro\TeXOSQueryFileSize
%\begin{flushleft}\ttfamily
%\cs{TeXOSQueryFileSize}\marg{cs}\marg{filename}
%\end{flushleft}
%where \meta{filename} is the name of the file.
%This is provided for the benefit of users who don't have
%\cs{pdffilesize} defined by their \TeX\ format.
%
%The list of files in a given directory
%(\shortarg{i} or \longarg{list}) can be obtained using:
%\DescribeMacro\TeXOSQueryFileList
%\begin{flushleft}\ttfamily
%\cs{TeXOSQueryFileList}\marg{cs}\marg{sep}\marg{dir}
%\end{flushleft}
%where \meta{sep} is the separator and \meta{dir} is the directory name.
%For example:
%\begin{verbatim}
%\TeXOSQueryFileList{\result}{,}{.}
%\end{verbatim}
%will store a comma-separated list of all the files contained in the
%current directory in the control sequence \cs{result}.
%
%A filtered list of files in a given directory
%(\shortarg{f} or \longarg{filterlist}) can be obtained using:
%\DescribeMacro\TeXOSQueryFilterFileList
%\begin{flushleft}\ttfamily
%\cs{TeXOSQueryFilterFileList}\marg{cs}\marg{sep}\marg{regex}\marg{dir}
%\end{flushleft}
%where \meta{regex} is a regular expression. \emph{Take care of any backslashes
%in the regular expression!}
%For example, to list only those files that have an extension:
%\begin{verbatim}
%\TeXOSQueryFilterFileList{\result}{,}{.+\string\..*}{.}
%\end{verbatim}
%Note the use of \verb|\string\.| to ensure that \cs{.} isn't
%interpreted as a command. Another example, list only \texttt{.png}
%and \texttt{.jpg} files in the directory called \texttt{images}:
%\begin{verbatim}
%\TeXOSQueryFilterFileList{\result}{,}{.+\string\.(jpg|png)}{images}
%\end{verbatim}
%
%The URI of a file
%(\shortarg{u} or \longarg{uri}) can be obtained using:
%\DescribeMacro\TeXOSQueryFileURI
%\begin{flushleft}\ttfamily
%\cs{TeXOSQueryFileURI}\marg{cs}\marg{filename}
%\end{flushleft}
%where \meta{filename} is the name of the file.
%(Any percent symbols \verb|%| contained in the URI will have
%their category code set to 12.)
%
%The canonical path of a file
%(\shortarg{p} or \longarg{path}) can be obtained using:
%\DescribeMacro\TeXOSQueryFilePath
%\begin{flushleft}\ttfamily
%\cs{TeXOSQueryFilePath}\marg{cs}\marg{filename}
%\end{flushleft}
%where \meta{filename} is the name of the file.
%
%The canonical path of a file's parent
%(\shortarg{e} or \longarg{dirname}) can be obtained using:
%\DescribeMacro\TeXOSQueryDirName
%\begin{flushleft}\ttfamily
%\cs{TeXOSQueryDirName}\marg{cs}\marg{filename}
%\end{flushleft}
%where \meta{filename} is the name of the file.
%
%\StopEventually{\phantomsection
%\addcontentsline{toc}{section}{Change History}%
%\PrintChanges
%\addcontentsline{toc}{section}{\indexname}%
%\PrintIndex
%}
%
%\section{The Code}
%\changes{1.0}{2016-07-08}{Initial release}
%\subsection{Generic \TeX\ Code}
%\iffalse
%    \begin{macrocode}
%<*texosquery.tex>
%    \end{macrocode}
%\fi
%Change category code of \texttt{@} if necessary.
%    \begin{macrocode}
\ifnum\catcode`\@=11\relax
  \def\@texosquery@restore@at{}%
\else
  \expandafter\edef\csname @texosquery@restore@at\endcsname{%
    \noexpand\catcode`\noexpand\@=\number\catcode`\@\relax
  }%
 \catcode`\@=11\relax
\fi
%    \end{macrocode}
% Check if already loaded.
%    \begin{macrocode}
\ifx\TeXOSQuery\undefined
\else
  \@texosquery@restore@at
  \expandafter\endinput
\fi
%    \end{macrocode}
%\begin{macro}{\TeXOSInvokerName}
%The name of the \texttt{texosquery} application.
%    \begin{macrocode}
\def\TeXOSInvokerName{texosquery}
%    \end{macrocode}
%\end{macro}
%
%\begin{macro}{\TeXOSQueryInvoker}
%If we're using \LaTeX, we'll need to use \cs{@@input} rather
%than \cs{input}.
%    \begin{macrocode}
\ifx\@@input\undefined
  \def\TeXOSQueryInvoker#1{\input|"\TeXOSInvokerName\space#1" }
\else
  \def\TeXOSQueryInvoker#1{\@@input|"\TeXOSInvokerName\space#1" }
\fi
%    \end{macrocode}
%\end{macro}
%
%\begin{macro}{\ifTeXOSQueryDryRun}
%Provide a dry-run mode.
%\changes{1.1}{2016-07-14}{dry run mode only false by default if with
%unrestricted mode}
%    \begin{macrocode}
\newif\ifTeXOSQueryDryRun
\TeXOSQueryDryRuntrue
%    \end{macrocode}
%\end{macro}
%
%If shell escape is unrestricted, automatically switch off dry-run mode.
%    \begin{macrocode}
\ifx\shellescape\undefined
  \ifx\pdfshellescape\undefined
  \else
    \ifnum\pdfshellescape=1\relax
      \TeXOSQueryDryRunfalse
    \fi
  \fi
\else
  \ifnum\shellescape=1\relax
    \TeXOSQueryDryRunfalse
  \fi
\fi
%    \end{macrocode}
%
%\begin{macro}{\TeXOSQuery}
%Use \texttt{texosquery} with the option given in the second
%argument and store the result in control sequence given in the
%first argument.
%    \begin{macrocode}
\def\TeXOSQuery#1#2{%
  \ifTeXOSQueryDryRun
    \begingroup
      \newlinechar=`\^^J
      \message{^^JTeXOSQuery: \TeXOSInvokerName\space#2^^J}%
    \endgroup
    \def#1{}%
  \else
    \begingroup
    \endlinechar=-1\relax
%    \end{macrocode}
%Just in case the result contains any awkward characters that have
%some special meaning to \TeX. (Can't really protect against hash,
%backslash or curly braces, but they're unlikely to occur unless
%the user has a very wacky and inappropriate file naming scheme.)
%    \begin{macrocode}
    \catcode`\-=12\relax
    \catcode`\_=12\relax
    \catcode`\^=12\relax
    \catcode`\~=12\relax
    \catcode`\$=12\relax
    \catcode`\&=12\relax
    \catcode`\"=12\relax
    \catcode`\'=12\relax
    \catcode`\.=12\relax
    \catcode`\/=12\relax
    \catcode`\:=12\relax
    \catcode`\;=12\relax
    \catcode`\%=12\relax
    \everyeof{\noexpand}\relax
    \edef\x{\endgroup\def\noexpand#1{\TeXOSQueryInvoker{#2}}}\x
  \fi
}
%    \end{macrocode}
%\end{macro}
%
%Now for some convenient shortcuts so the user doesn't have to
%remember the command line options. \cs{string} is used in
%case the hyphen character has been made active.
%\begin{macro}{\TeXOSQueryLocale}
%Query the locale and store the result in the control sequence
%provided in the argument.
%    \begin{macrocode}
\def\TeXOSQueryLocale#1{\TeXOSQuery{#1}{\string-l}}
%    \end{macrocode}
%\end{macro}
%
%\begin{macro}{\TeXOSQueryLangTag}
%Query the language tag and store the result in the control sequence
%provided in the argument.
%\changes{1.2}{??}{new}
%    \begin{macrocode}
\def\TeXOSQueryLangTag#1{\TeXOSQuery{#1}{\string-b}}
%    \end{macrocode}
%\end{macro}
%
%\begin{macro}{\TeXOSQueryCwd}
%Query the current working directory.
%    \begin{macrocode}
\def\TeXOSQueryCwd#1{\TeXOSQuery{#1}{\string-c}}
%    \end{macrocode}
%\end{macro}
%
%\begin{macro}{\TeXOSQueryHome}
%Query the user's home directory.
%    \begin{macrocode}
\def\TeXOSQueryHome#1{\TeXOSQuery{#1}{\string-m}}
%    \end{macrocode}
%\end{macro}
%
%\begin{macro}{\TeXOSQueryTmpDir}
%Query the temporary directory.
%    \begin{macrocode}
\def\TeXOSQueryTmpDir#1{\TeXOSQuery{#1}{\string-t}}
%    \end{macrocode}
%\end{macro}
%
%\begin{macro}{\TeXOSQueryVersion}
%Query the operating system version.
%    \begin{macrocode}
\def\TeXOSQueryVersion#1{\TeXOSQuery{#1}{\string-r}}
%    \end{macrocode}
%\end{macro}
%
%\begin{macro}{\TeXOSQueryArch}
%Query the operating system architecture.
%    \begin{macrocode}
\def\TeXOSQueryArch#1{\TeXOSQuery{#1}{\string-a}}
%    \end{macrocode}
%\end{macro}
%
%\begin{macro}{\TeXOSQueryName}
%Query the operating system name.
%    \begin{macrocode}
\def\TeXOSQueryName#1{\TeXOSQuery{#1}{\string-o}}
%    \end{macrocode}
%\end{macro}
%
%\begin{macro}{\TeXOSQueryNow}
%\changes{1.1}{2016-07-14}{changed catcode of D to 12}
%Query the current time stamp.
%    \begin{macrocode}
\def\TeXOSQueryNow#1{%
%    \end{macrocode}
% The \texttt{D} needs category code 12. (Don't need to worry about
% \texttt{Z} as \texttt{texosquery.jar} uses \texttt{+00'00'} for
% UTC+0.) This change can't be done with the other catcode changes
% in \cs{TeXOSQuery}, as this is only appropriate for the PDF dates.
% Save and restore the catcode rather than fiddle around with
% scoping.
%    \begin{macrocode}
  \edef\@texosquery@restore@D{%
    \noexpand\catcode`\noexpand\D=\the\catcode`\D\relax}%
  \catcode`\D=12\relax
  \TeXOSQuery{#1}{\string-n}%
  \@texosquery@restore@D
}
%    \end{macrocode}
%\end{macro}
%
%If the file name is supplied using \cs{jobname} it may have
%double-quotes which will interfere with things.
%\begin{macro}{\texosquerystripquotes}
%    \begin{macrocode}
\def\texosquerystripquotes#1{%
  \@texosquery@stripquotes#1\@mid@texosquery@stripquotes
    "\relax"\relax\@end@texosquery@stripquotes
}
\def\@texosquery@stripquotes#1"#2"{%
  \@@texosquery@stripquotes#1#2%
}
\def\@@texosquery@stripquotes#1\@mid@texosquery@stripquotes#2\@end@texosquery@stripquotes{%
  #1%
}
%    \end{macrocode}
%\end{macro}
%
%\begin{macro}{\TeXOSQueryFileDate}
%\changes{1.1}{2016-07-14}{changed catcode of D to 12}
%Query the time stamp of the file given in the second argument.
%    \begin{macrocode}
\def\TeXOSQueryFileDate#1#2{%
  \edef\@texosquery@restore@D{%
    \noexpand\catcode`\noexpand\D=\the\catcode`\D\relax}%
  \catcode`\D=12\relax
  \TeXOSQuery{#1}{\string-d \string'\texosquerystripquotes{#2}\string'}%
  \@texosquery@restore@D
}
%    \end{macrocode}
%\end{macro}
%
%\begin{macro}{\TeXOSQueryFileSize}
%Query the size of the file given in the second argument.
%    \begin{macrocode}
\def\TeXOSQueryFileSize#1#2{\TeXOSQuery{#1}{\string-s 
 \string'\texosquerystripquotes{#2}\string'}}
%    \end{macrocode}
%\end{macro}
%
%\begin{macro}{\TeXOSQueryFileList}
%List all files in the directory given in the third argument,
%separated by the second argument.
%    \begin{macrocode}
\def\TeXOSQueryFileList#1#2#3{\TeXOSQuery{#1}{%
 \string-i \string'#2\string' 
 \string'\texosquerystripquotes{#3}\string'}}
%    \end{macrocode}
%\end{macro}
%
%\begin{macro}{\TeXOSQueryFilterFileList}
%Filtered list files in the directory given in the fourth argument,
%separated by the second argument. The third argument is the regular
%expression used to filter the list. \emph{Take care of backslashes
%in the regular expression!}
%    \begin{macrocode}
\def\TeXOSQueryFilterFileList#1#2#3#4{\TeXOSQuery{#1}{%
 \string-f \string'#2\string' \string'#3\string' 
 \string'\texosquerystripquotes{#4}\string'}}
%    \end{macrocode}
%\end{macro}
%
%\begin{macro}{\TeXOSQueryFileURI}
%Get the URI of the file given in the second argument.
%    \begin{macrocode}
\def\TeXOSQueryFileURI#1#2{\TeXOSQuery{#1}{\string-u 
 \string'\texosquerystripquotes{#2}\string'}}
%    \end{macrocode}
%\end{macro}
%
%\begin{macro}{\TeXOSQueryFilePath}
%Get the canonical path of the file given in the second argument.
%    \begin{macrocode}
\def\TeXOSQueryFilePath#1#2{\TeXOSQuery{#1}{\string-p 
 \string'\texosquerystripquotes{#2}\string'}}
%    \end{macrocode}
%\end{macro}
%
%\begin{macro}{\TeXOSQueryDirName}
%\changes{1.1}{2016-07-14}{new}
%Get the canonical path of the directory containing the file given 
%in the second argument.
%    \begin{macrocode}
\def\TeXOSQueryDirName#1#2{\TeXOSQuery{#1}{\string-e 
 \string'\texosquerystripquotes{#2}\string'}}
%    \end{macrocode}
%\end{macro}
%
%All done.
%Restore the category code of \texttt{@}:
%    \begin{macrocode}
\@texosquery@restore@at
%    \end{macrocode}
%\iffalse
%    \begin{macrocode}
%</texosquery.tex>
%    \end{macrocode}
%\fi
%
%\subsection{\LaTeX\ Code}
%This is just a simple wrapper for \texttt{texosquery.tex}
%so that it can be loaded using \LaTeX's standard \cs{usepackage}
%method.
%\iffalse
%    \begin{macrocode}
%<*texosquery.sty>
%    \end{macrocode}
%\fi
%Identify package:
%    \begin{macrocode}
\NeedsTeXFormat{LaTeX2e}
\ProvidesPackage{texosquery}[2016/07/14 v1.1 (NLCT)]
%    \end{macrocode}
%Load \texttt{texosquery.tex}:
%    \begin{macrocode}
%\iffalse
%<*package>
%% \CharacterTable
%%  {Upper-case    \A\B\C\D\E\F\G\H\I\J\K\L\M\N\O\P\Q\R\S\T\U\V\W\X\Y\Z
%%   Lower-case    \a\b\c\d\e\f\g\h\i\j\k\l\m\n\o\p\q\r\s\t\u\v\w\x\y\z
%%   Digits        \0\1\2\3\4\5\6\7\8\9
%%   Exclamation   \!     Double quote  \"     Hash (number) \#
%%   Dollar        \$     Percent       \%     Ampersand     \&
%%   Acute accent  \'     Left paren    \(     Right paren   \)
%%   Asterisk      \*     Plus          \+     Comma         \,
%%   Minus         \-     Point         \.     Solidus       \/
%%   Colon         \:     Semicolon     \;     Less than     \<
%%   Equals        \=     Greater than  \>     Question mark \?
%%   Commercial at \@     Left bracket  \[     Backslash     \\
%%   Right bracket \]     Circumflex    \^     Underscore    \_
%%   Grave accent  \`     Left brace    \{     Vertical bar  \|
%%   Right brace   \}     Tilde         \~}
%</package>
%\fi
% \iffalse
% Doc-Source file
% Copyright (C) 2016 Nicola Talbot, all rights reserved.
% \fi
% \iffalse
%<*driver>
\documentclass{ltxdoc}

\usepackage[T1]{fontenc}
\usepackage{tgtermes}
\usepackage{metalogo}
\usepackage[colorlinks,hyperindex=false]{hyperref}

\CheckSum{273}

\RecordChanges
\PageIndex
\CodelineNumbered
\IndexPrologue{}

\renewcommand*{\usage}[1]{\hyperpage{#1}}
\renewcommand*{\main}[1]{\hyperpage{#1}}

\newcommand*{\shortargfmt}[1]{\texttt{-#1}}
\newcommand*{\longargfmt}[1]{\texttt{-{}-#1}}

\newcommand*{\shortarg}[1]{\shortargfmt{#1}%
 \index{texosquery options=\texttt{texosquery} options>#1=\protect\shortargfmt{-#1}|hyperpage}}
\newcommand*{\longarg}[1]{\longargfmt{#1}%
 \index{texosquery options=\texttt{texosquery} options>#1=\protect\longargfmt{#1}|hyperpage}}

\newenvironment{important}
 {\begin{quote}\textbf{Important Note:}}
 {\end{quote}}

\begin{document}
\DocInput{texosquery.dtx}
\end{document}
%</driver>
%\fi
%
%\DeleteShortVerb{|}
%
%\title{texosquery: query OS information from \TeX}
%\author{Nicola L. C. Talbot\\
%\href{http://www.dickimaw-books.com/}{\nolinkurl{dickimaw-books.com}}
%\and Paulo Cereda}
%\date{2016-07-14 (v1.1)}
%\maketitle
%
%\begin{abstract}
%\texttt{texosquery.jar} is a cross-platform Java application
%to query certain OS information designed for use in \TeX's
%shell escape mechanism.
%The accompanying \TeX\ code provides a simple interface to the
%\texttt{texosquery} Java application, which may be skipped
%if you want to explicitly use \cs{input} (but take
%care of any special characters appearing in the result).
%\end{abstract}
%
%\begin{important}
%You will need \TeX's shell escape enabled,
%and you will also need the Java Runtime Environment (JRE) 
%installed to use \texttt{texosquery}.
%\end{important}
%
%If you want to rebuild the application, instructions for
%compiling the source code (including the code for this document)
%are in the accompanying \texttt{README.md} file.
%
%\tableofcontents
%
%\section{texosquery.jar: the Java application}
%The \texttt{texosquery} Java command line application looks up
%certain system information that may be of use in \TeX\ 
%documents. All this information can easily be obtained using 
%native commands, but the Java application allows an 
%OS-independent approach with results that can easily be
%captured by \TeX's shell-escape without having to strip 
%formatting information.
%
%\begin{important}
%\texttt{texosquery} provides read-only actions, and I don't 
%intend adding any actions that modify system settings or files.
%\end{important}
%
%Since the application is designed to work with \TeX, each
%function will display the result on a single line without
%formatting. (For multiple results, each line is grouped 
%from v1.1.) A blank line (or empty group) will be displayed
%if the information isn't available. A forward slash is always 
%used as a directory divider, regardless of the operating 
%system, so the result can be used, for example, in \cs{input} or
%\cs{includegraphics}.
%
%If an input file name is required (for example, with the
%\longargfmt{pdfdate} argument described below) then the file may be in
%the current working directory, relative to the current directory
%(with forward slash \texttt{/} as the directory divider), an absolute path 
%(again with forward slash) or on \TeX's path (in
%which case, \texttt{kpsewhich} is used to locate it).
%
%Command line invocation:
%\begin{flushleft}\ttfamily
%texosquery \meta{action} ...
%\end{flushleft}
%Available actions (at least one required):
%\begin{description}
%\item[\shortarg{L} or \longarg{locale}] Display the locale
%information in the form
%\begin{flushleft}\ttfamily
%\meta{lang}-\meta{region}.\meta{codeset}@\meta{modifier}
%\end{flushleft}
%where \meta{lang} is the ISO code for the language (e.g.\ 
%\texttt{en}),
%\meta{region} is the ISO code for the region (e.g.\ \texttt{GB}),
%\meta{codeset} is the default code set (e.g.\ \texttt{UTF-8})
% and \meta{modifier} is the modifier. Elements may be omitted
% if unavailable. For example, \texttt{en-GB.UTF-8} has the
% \meta{modifier} omitted, and \texttt{en} has all but the language
% omitted.
%\item[\shortarg{l} or \longarg{locale-lcs}] As the above but the
%codeset (if present) is converted to lower case and any hyphens are
%stripped. For example, if \longargfmt{locale} returns
%\texttt{en-GB.UTF-8}, then \longargfmt{locale-lcs} would return
%\texttt{en-GB.utf8}.
%\item[\shortarg{b} or \longarg{bcp47}] (New to version 1.2.)
%Displays the BCP~47 language tag.
%
%\item[\shortarg{c} or \longarg{cwd}] Displays the current working
%directory.
%\item[\shortarg{m} or \longarg{userhome}] Displays the user's home
%directory.
%\item[\shortarg{t} or \longarg{tmpdir}] Displays the temporary
%directory.
%\item[\shortarg{o} or \longarg{osname}] Displays the operating
%system name.
%\item[\shortarg{r} or \longarg{osversion}] Displays the operating
%system version.
%\item[\shortarg{a} or \longarg{osarch}] Displays the operating
%system architecture.
%\item[\shortarg{n} or \longarg{pdfnow}] Displays the current
% date and time in PDF format. For example 
%\begin{verbatim}
%D:20160704131006+01'00'
%\end{verbatim}
%Note that some, but not all, \TeX\ formats provide
%\cs{pdfcreationdate}, which is more efficient than using the shell
%escape, but this can be used
%as a fallback method for those that don't (for example, \XeTeX).
%\item[\shortarg{d} \meta{file} or \longarg{pdfdate} \meta{file}]
% Displays the last modified time stamp of the given file in PDF format or a blank line
% if the file doesn't exist or the file permissions prohibit this
% action.
%Again some, but not all, \TeX\ formats provide
%\cs{pdffilemoddate}\marg{file}, which is more efficient than using
%the shell escape.
%\item[\shortarg{s} \meta{file} or \longarg{filesize} \meta{file}]
% Displays the size in bytes of the given file or a blank line
% if the file doesn't exist or the file permissions prohibit this
% action.
%Some, but not all, \TeX\ formats provide
%\cs{pdffilesize}\marg{file}, which is more efficient than using
%the shell escape.
%\item[\shortarg{i} \meta{sep} \meta{dir} or \longarg{list}
%\meta{sep} \meta{dir}]
%List all files in the given directory with the output on a single
%line using \meta{sep} as the separator between entries. Note that
%the list doesn't include the full path, just the file names.
%\item[\shortarg{f} \meta{sep} \meta{regex} \meta{dir} or
%\longarg{filterlist} \meta{sep} \meta{regex} \meta{dir}]
%Like the above but only lists those files whose name matches the
%regular expression given in \meta{regex}. Note that this tests for a
%complete match on the file name (not including path). For example,
%if \meta{regexp} is \texttt{foo.*}, it will only match files whose
%name starts with \texttt{foo} (for example, \texttt{foobar} will
%match but \texttt{barfoo} won't).
%\item[\shortarg{u} \meta{file} or \longarg{uri} \meta{file}]
%Displays the URI of the given file or a blank line if the file
%doesn't exist or the file permissions prohibit this action.
%Note that this may include in a percent character in the result. The \TeX\ 
%command \cs{TeXOSQuery} protects against this by changing the
%category code, but if you explicitly call \texttt{texosquery}
%using the shell escape, you'll need to be careful of this.
%\item[\shortarg{p} \meta{file} or \longarg{path} \meta{file}]
%Displays the canonical path of the given file or a blank line if
%the file doesn't exist or the file permissions prohibit this
%action.
%\item[\shortarg{e} \meta{file} or \longarg{dirname} \meta{file}]
%(New to v1.1.)
%Displays the canonical path of the given file's parent (that is, the
%directory containing \meta{file}) or a blank line if
%the file doesn't exist or the file permissions prohibit this
%action. Note that this is different to the Unix-like 
%\texttt{dirname} command, which will return a relative path if
%\meta{file} isn't an absolute path.
%\item[\shortarg{h} or \longarg{help}] Displays help message and
%exits.
%\item[\shortarg{v} or \longarg{version}] Displays version
%information and exits.
%\end{description}
%
%If multiple options are given, they will be processed in the
%order specified in the command line invocation. Each result will
%be displayed on a separate line. As from v1.1, if there are
%multiple actions, each result will be grouped. This makes it easier
%to process the results in \TeX. For example:
%\begin{verbatim}
%texosquery -l
%\end{verbatim}
%This just produces (for me):
%\begin{verbatim}
%en-GB.utf8
%\end{verbatim}
%whereas
%\begin{verbatim}
%texosquery -l -n
%\end{verbatim}
%produces:
%\begin{verbatim}
%{en-GB.utf8}
%{D:20160714112732+01'00'}
%\end{verbatim}
%Note that unavailable information will produce an empty group.
%For example (assuming \texttt{nofile} doesn't exist):
%\begin{verbatim}
%texosquery -l -d nofile -n
%\end{verbatim}
%produces:
%\begin{verbatim}
%{en-GB.utf8}
%{}
%{D:20160714112732+01'00'}
%\end{verbatim}
%whereas 
%\begin{verbatim}
%texosquery -d nofile
%\end{verbatim}
%just displays an empty line.
%
%\section{texosquery.tex: generic \TeX\ code}
%You can run \texttt{texosquery} directly from \TeX's shell escape.
%For example:
%\begin{verbatim}
%\input|"texosquery --locale"
%\end{verbatim}
%However, \texttt{texosquery.tex} provides generic \TeX\ code to do
%this for you and store the result in a control sequence.
%
%Plain \TeX\ users can input this file through the usual \cs{input}
%method:
%\begin{verbatim}
%\input texosquery
%\end{verbatim}
%\LaTeX\ users may also simply input this file:
%\begin{verbatim}
%\input{texosquery}
%\end{verbatim}
%but may prefer the standard package approach:
%\begin{verbatim}
%\usepackage{texosquery}
%\end{verbatim}
%
%\begin{important}
%The commands described below are all fragile.
%\end{important}
%
%The basic command to run \texttt{texosquery} and capture its output
%in a control sequence is:
%\DescribeMacro\TeXOSQuery
%\begin{flushleft}\ttfamily
%\cs{TeXOSQuery}\marg{cs}\marg{args}
%\end{flushleft}
%where \meta{cs} is the control sequence in which to store the
%result and \meta{args} are the command line arguments to pass to 
%\texttt{texosquery}. This first locally changes the category code
%of some problematic characters that may appear in the result.
%The only special characters that aren't changed are the backslash
%\verb|\|, curly braces \verb|{| and \verb|}|, and hash \verb|#|.
%(\texttt{texosquery.jar} will replace \verb|#| with \verb|\#| in places
%where it might possibly occur in the result, but in general it's
%best to avoid these characters in file names.)
%
%There are some short cut commands for convenience, described below.
%If any of these commands cause an error message in the form:
%\begin{verbatim}
%I can't find file `|texosquery'.
%\end{verbatim}
%then check that you have the shell escape on. If the error 
%persists with the shell escape enabled
%(and not restricted), check that \texttt{texosquery} is on 
%your system's path. To do this, open a command prompt or terminal and
%type \texttt{texosquery}. If it isn't installed correctly, there will be a
%message like:
%\begin{verbatim}
%`texosquery' is not recognised
%\end{verbatim}
%or
%\begin{verbatim}
%texosquery: command not found
%\end{verbatim}
%If this happens, check the installation. (Instructions are
%in the accompanying \texttt{README.md} file.)
%
%Dry run mode is determined by the conditional
%\DescribeMacro\ifTeXOSQueryDryRun
%\begin{verbatim}
%\ifTeXOSQueryDryRun
%\end{verbatim}
%If true, the shell escape won't be used and the requested command
%invocation will be printed in the transcript file prefixed with
%\begin{verbatim}
%TeXOSQuery:
%\end{verbatim}
%(the control sequence \meta{cs} will be set to empty).
%This conditional will automatically be switched on unless
%\cs{shellescape} or \cs{pdfshellescape} is 1. (If
%\texttt{texosquery.jar} is later allowed on the restricted list,
%newer versions may change this default.)
%
%If multiple queries are required, it's more efficient to
%perform them all in one go. For example:
%\begin{verbatim}
%\TeXOSQuery{\result}{-l -n -o}
%
%\def\parseresult#1#2#3{%
%  Locale: #1. Now: #2. OS: #3.%
%}
%
%\ifx\result\empty
%  Query failed.
%\else
%  \expandafter\parseresult\result
%\fi
%\end{verbatim}
%(Make sure you have at least v1.1 for this to work correctly.)
%
%The locale (\shortarg{l} or \longarg{locale-lcs}) information can be
%obtained using:
%\DescribeMacro\TeXOSQueryLocale
%\begin{flushleft}\ttfamily
%\cs{TeXOSQueryLocale}\marg{cs}
%\end{flushleft}
%Note that this uses the lower case codeset form, which has a better
%chance of matching the encoding names used by the \texttt{inputenc}
%package. If you want the unprocessed codeset name, you can do:
%\begin{flushleft}\ttfamily
%\cs{TeXOSQuery}\marg{cs}\{\shortarg{L}\}
%\end{flushleft}
%
%The BCP 47 language tag (\shortarg{b} or \longarg{bcp47}) can be obtained using:
%\DescribeMacro\TeXOSQueryLangTag
%\begin{flushleft}\ttfamily
%\cs{TeXOSQueryLangTag}\marg{cs}
%\end{flushleft}
%
%The current working directory (\shortarg{c} or \longarg{cwd}) can
%be obtained using:
%\DescribeMacro\TeXOSQueryCwd
%\begin{flushleft}\ttfamily
%\cs{TeXOSQueryCwd}\marg{cs}
%\end{flushleft}
%
%The home directory (\shortarg{m} or \longarg{userhome}) can
%be obtained using:
%\DescribeMacro\TeXOSQueryHome
%\begin{flushleft}\ttfamily
%\cs{TeXOSQueryHome}\marg{cs}
%\end{flushleft}
%
%The temporary directory (\shortarg{t} or \longarg{tmpdir}) can
%be obtained using:
%\DescribeMacro\TeXOSQueryTmpDir
%\begin{flushleft}\ttfamily
%\cs{TeXOSQueryTmpDir}\marg{cs}
%\end{flushleft}
%
%The OS name (\shortarg{o} or \longarg{osname}) can
%be obtained using:
%\DescribeMacro\TeXOSQueryName
%\begin{flushleft}\ttfamily
%\cs{TeXOSQueryName}\marg{cs}
%\end{flushleft}
%
%The OS version (\shortarg{r} or \longarg{osversion}) can
%be obtained using:
%\DescribeMacro\TeXOSQueryVersion
%\begin{flushleft}\ttfamily
%\cs{TeXOSQueryVersion}\marg{cs}
%\end{flushleft}
%
%The OS architecture (\shortarg{a} or \longarg{osarch}) can
%be obtained using:
%\DescribeMacro\TeXOSQueryArch
%\begin{flushleft}\ttfamily
%\cs{TeXOSQueryArch}\marg{cs}
%\end{flushleft}
%
%The current date-time stamp in PDF format (\shortarg{n} or
%\longarg{pdfnow}) can be obtained using:
%\DescribeMacro\TeXOSQueryNow
%\begin{flushleft}\ttfamily
%\cs{TeXOSQueryNow}\marg{cs}
%\end{flushleft}
%This is provided for the benefit of users who don't have
%\cs{pdfcreationdate} defined by their \TeX\ format (for example,
%\XeTeX). As from v1.1, this ensures that the initial \texttt{D} has
%category code 12 (which won't happen if \cs{TeXOSQuery} is used
%explicitly).
%
%\begin{important}
%The remaining commands all require extra arguments after the
%relevant switch. These are automatically enclosed in single-quotes
%to protect any spaces. If the argument actually contains any single-quote
%characters, make sure you use \verb|\string\'| to prevent
%interference. \emph{However, in general it's a very bad idea to use
%quotes as part of a file name (rather than using them as
%delimiters).}
%Since the file name may need to be obtained from \cs{jobname},
%which sometimes includes double-quotes, the first double-quote pair
%found is stripped in file name arguments. Any other double-quotes
%will need to be protected in the same manner as single-quotes
%(but, again, this shouldn't be an issue if you use a safe
%file naming scheme). Any paths should use a
%forward slash for the directory divider.
%\end{important}
%
%The modification date-time stamp in PDF format for a file
%(\shortarg{d} or \longarg{pdfdate}) can be obtained using:
%\DescribeMacro\TeXOSQueryFileDate
%\begin{flushleft}\ttfamily
%\cs{TeXOSQueryFileDate}\marg{cs}\marg{filename}
%\end{flushleft}
%where \meta{filename} is the name of the file.
%This is provided for the benefit of users who don't have
%\cs{pdffilemoddate} defined by their \TeX\ format. As from v1.1,
%this ensures that the initial \texttt{D} has category code 12 (which 
%won't happen if \cs{TeXOSQuery} is used explicitly).
%
%The size in bytes of a file
%(\shortarg{s} or \longarg{filesize}) can be obtained using:
%\DescribeMacro\TeXOSQueryFileSize
%\begin{flushleft}\ttfamily
%\cs{TeXOSQueryFileSize}\marg{cs}\marg{filename}
%\end{flushleft}
%where \meta{filename} is the name of the file.
%This is provided for the benefit of users who don't have
%\cs{pdffilesize} defined by their \TeX\ format.
%
%The list of files in a given directory
%(\shortarg{i} or \longarg{list}) can be obtained using:
%\DescribeMacro\TeXOSQueryFileList
%\begin{flushleft}\ttfamily
%\cs{TeXOSQueryFileList}\marg{cs}\marg{sep}\marg{dir}
%\end{flushleft}
%where \meta{sep} is the separator and \meta{dir} is the directory name.
%For example:
%\begin{verbatim}
%\TeXOSQueryFileList{\result}{,}{.}
%\end{verbatim}
%will store a comma-separated list of all the files contained in the
%current directory in the control sequence \cs{result}.
%
%A filtered list of files in a given directory
%(\shortarg{f} or \longarg{filterlist}) can be obtained using:
%\DescribeMacro\TeXOSQueryFilterFileList
%\begin{flushleft}\ttfamily
%\cs{TeXOSQueryFilterFileList}\marg{cs}\marg{sep}\marg{regex}\marg{dir}
%\end{flushleft}
%where \meta{regex} is a regular expression. \emph{Take care of any backslashes
%in the regular expression!}
%For example, to list only those files that have an extension:
%\begin{verbatim}
%\TeXOSQueryFilterFileList{\result}{,}{.+\string\..*}{.}
%\end{verbatim}
%Note the use of \verb|\string\.| to ensure that \cs{.} isn't
%interpreted as a command. Another example, list only \texttt{.png}
%and \texttt{.jpg} files in the directory called \texttt{images}:
%\begin{verbatim}
%\TeXOSQueryFilterFileList{\result}{,}{.+\string\.(jpg|png)}{images}
%\end{verbatim}
%
%The URI of a file
%(\shortarg{u} or \longarg{uri}) can be obtained using:
%\DescribeMacro\TeXOSQueryFileURI
%\begin{flushleft}\ttfamily
%\cs{TeXOSQueryFileURI}\marg{cs}\marg{filename}
%\end{flushleft}
%where \meta{filename} is the name of the file.
%(Any percent symbols \verb|%| contained in the URI will have
%their category code set to 12.)
%
%The canonical path of a file
%(\shortarg{p} or \longarg{path}) can be obtained using:
%\DescribeMacro\TeXOSQueryFilePath
%\begin{flushleft}\ttfamily
%\cs{TeXOSQueryFilePath}\marg{cs}\marg{filename}
%\end{flushleft}
%where \meta{filename} is the name of the file.
%
%The canonical path of a file's parent
%(\shortarg{e} or \longarg{dirname}) can be obtained using:
%\DescribeMacro\TeXOSQueryDirName
%\begin{flushleft}\ttfamily
%\cs{TeXOSQueryDirName}\marg{cs}\marg{filename}
%\end{flushleft}
%where \meta{filename} is the name of the file.
%
%\StopEventually{\phantomsection
%\addcontentsline{toc}{section}{Change History}%
%\PrintChanges
%\addcontentsline{toc}{section}{\indexname}%
%\PrintIndex
%}
%
%\section{The Code}
%\changes{1.0}{2016-07-08}{Initial release}
%\subsection{Generic \TeX\ Code}
%\iffalse
%    \begin{macrocode}
%<*texosquery.tex>
%    \end{macrocode}
%\fi
%Change category code of \texttt{@} if necessary.
%    \begin{macrocode}
\ifnum\catcode`\@=11\relax
  \def\@texosquery@restore@at{}%
\else
  \expandafter\edef\csname @texosquery@restore@at\endcsname{%
    \noexpand\catcode`\noexpand\@=\number\catcode`\@\relax
  }%
 \catcode`\@=11\relax
\fi
%    \end{macrocode}
% Check if already loaded.
%    \begin{macrocode}
\ifx\TeXOSQuery\undefined
\else
  \@texosquery@restore@at
  \expandafter\endinput
\fi
%    \end{macrocode}
%\begin{macro}{\TeXOSInvokerName}
%The name of the \texttt{texosquery} application.
%    \begin{macrocode}
\def\TeXOSInvokerName{texosquery}
%    \end{macrocode}
%\end{macro}
%
%\begin{macro}{\TeXOSQueryInvoker}
%If we're using \LaTeX, we'll need to use \cs{@@input} rather
%than \cs{input}.
%    \begin{macrocode}
\ifx\@@input\undefined
  \def\TeXOSQueryInvoker#1{\input|"\TeXOSInvokerName\space#1" }
\else
  \def\TeXOSQueryInvoker#1{\@@input|"\TeXOSInvokerName\space#1" }
\fi
%    \end{macrocode}
%\end{macro}
%
%\begin{macro}{\ifTeXOSQueryDryRun}
%Provide a dry-run mode.
%\changes{1.1}{2016-07-14}{dry run mode only false by default if with
%unrestricted mode}
%    \begin{macrocode}
\newif\ifTeXOSQueryDryRun
\TeXOSQueryDryRuntrue
%    \end{macrocode}
%\end{macro}
%
%If shell escape is unrestricted, automatically switch off dry-run mode.
%    \begin{macrocode}
\ifx\shellescape\undefined
  \ifx\pdfshellescape\undefined
  \else
    \ifnum\pdfshellescape=1\relax
      \TeXOSQueryDryRunfalse
    \fi
  \fi
\else
  \ifnum\shellescape=1\relax
    \TeXOSQueryDryRunfalse
  \fi
\fi
%    \end{macrocode}
%
%\begin{macro}{\TeXOSQuery}
%Use \texttt{texosquery} with the option given in the second
%argument and store the result in control sequence given in the
%first argument.
%    \begin{macrocode}
\def\TeXOSQuery#1#2{%
  \ifTeXOSQueryDryRun
    \begingroup
      \newlinechar=`\^^J
      \message{^^JTeXOSQuery: \TeXOSInvokerName\space#2^^J}%
    \endgroup
    \def#1{}%
  \else
    \begingroup
    \endlinechar=-1\relax
%    \end{macrocode}
%Just in case the result contains any awkward characters that have
%some special meaning to \TeX. (Can't really protect against hash,
%backslash or curly braces, but they're unlikely to occur unless
%the user has a very wacky and inappropriate file naming scheme.)
%    \begin{macrocode}
    \catcode`\-=12\relax
    \catcode`\_=12\relax
    \catcode`\^=12\relax
    \catcode`\~=12\relax
    \catcode`\$=12\relax
    \catcode`\&=12\relax
    \catcode`\"=12\relax
    \catcode`\'=12\relax
    \catcode`\.=12\relax
    \catcode`\/=12\relax
    \catcode`\:=12\relax
    \catcode`\;=12\relax
    \catcode`\%=12\relax
    \everyeof{\noexpand}\relax
    \edef\x{\endgroup\def\noexpand#1{\TeXOSQueryInvoker{#2}}}\x
  \fi
}
%    \end{macrocode}
%\end{macro}
%
%Now for some convenient shortcuts so the user doesn't have to
%remember the command line options. \cs{string} is used in
%case the hyphen character has been made active.
%\begin{macro}{\TeXOSQueryLocale}
%Query the locale and store the result in the control sequence
%provided in the argument.
%    \begin{macrocode}
\def\TeXOSQueryLocale#1{\TeXOSQuery{#1}{\string-l}}
%    \end{macrocode}
%\end{macro}
%
%\begin{macro}{\TeXOSQueryLangTag}
%Query the language tag and store the result in the control sequence
%provided in the argument.
%\changes{1.2}{??}{new}
%    \begin{macrocode}
\def\TeXOSQueryLangTag#1{\TeXOSQuery{#1}{\string-b}}
%    \end{macrocode}
%\end{macro}
%
%\begin{macro}{\TeXOSQueryCwd}
%Query the current working directory.
%    \begin{macrocode}
\def\TeXOSQueryCwd#1{\TeXOSQuery{#1}{\string-c}}
%    \end{macrocode}
%\end{macro}
%
%\begin{macro}{\TeXOSQueryHome}
%Query the user's home directory.
%    \begin{macrocode}
\def\TeXOSQueryHome#1{\TeXOSQuery{#1}{\string-m}}
%    \end{macrocode}
%\end{macro}
%
%\begin{macro}{\TeXOSQueryTmpDir}
%Query the temporary directory.
%    \begin{macrocode}
\def\TeXOSQueryTmpDir#1{\TeXOSQuery{#1}{\string-t}}
%    \end{macrocode}
%\end{macro}
%
%\begin{macro}{\TeXOSQueryVersion}
%Query the operating system version.
%    \begin{macrocode}
\def\TeXOSQueryVersion#1{\TeXOSQuery{#1}{\string-r}}
%    \end{macrocode}
%\end{macro}
%
%\begin{macro}{\TeXOSQueryArch}
%Query the operating system architecture.
%    \begin{macrocode}
\def\TeXOSQueryArch#1{\TeXOSQuery{#1}{\string-a}}
%    \end{macrocode}
%\end{macro}
%
%\begin{macro}{\TeXOSQueryName}
%Query the operating system name.
%    \begin{macrocode}
\def\TeXOSQueryName#1{\TeXOSQuery{#1}{\string-o}}
%    \end{macrocode}
%\end{macro}
%
%\begin{macro}{\TeXOSQueryNow}
%\changes{1.1}{2016-07-14}{changed catcode of D to 12}
%Query the current time stamp.
%    \begin{macrocode}
\def\TeXOSQueryNow#1{%
%    \end{macrocode}
% The \texttt{D} needs category code 12. (Don't need to worry about
% \texttt{Z} as \texttt{texosquery.jar} uses \texttt{+00'00'} for
% UTC+0.) This change can't be done with the other catcode changes
% in \cs{TeXOSQuery}, as this is only appropriate for the PDF dates.
% Save and restore the catcode rather than fiddle around with
% scoping.
%    \begin{macrocode}
  \edef\@texosquery@restore@D{%
    \noexpand\catcode`\noexpand\D=\the\catcode`\D\relax}%
  \catcode`\D=12\relax
  \TeXOSQuery{#1}{\string-n}%
  \@texosquery@restore@D
}
%    \end{macrocode}
%\end{macro}
%
%If the file name is supplied using \cs{jobname} it may have
%double-quotes which will interfere with things.
%\begin{macro}{\texosquerystripquotes}
%    \begin{macrocode}
\def\texosquerystripquotes#1{%
  \@texosquery@stripquotes#1\@mid@texosquery@stripquotes
    "\relax"\relax\@end@texosquery@stripquotes
}
\def\@texosquery@stripquotes#1"#2"{%
  \@@texosquery@stripquotes#1#2%
}
\def\@@texosquery@stripquotes#1\@mid@texosquery@stripquotes#2\@end@texosquery@stripquotes{%
  #1%
}
%    \end{macrocode}
%\end{macro}
%
%\begin{macro}{\TeXOSQueryFileDate}
%\changes{1.1}{2016-07-14}{changed catcode of D to 12}
%Query the time stamp of the file given in the second argument.
%    \begin{macrocode}
\def\TeXOSQueryFileDate#1#2{%
  \edef\@texosquery@restore@D{%
    \noexpand\catcode`\noexpand\D=\the\catcode`\D\relax}%
  \catcode`\D=12\relax
  \TeXOSQuery{#1}{\string-d \string'\texosquerystripquotes{#2}\string'}%
  \@texosquery@restore@D
}
%    \end{macrocode}
%\end{macro}
%
%\begin{macro}{\TeXOSQueryFileSize}
%Query the size of the file given in the second argument.
%    \begin{macrocode}
\def\TeXOSQueryFileSize#1#2{\TeXOSQuery{#1}{\string-s 
 \string'\texosquerystripquotes{#2}\string'}}
%    \end{macrocode}
%\end{macro}
%
%\begin{macro}{\TeXOSQueryFileList}
%List all files in the directory given in the third argument,
%separated by the second argument.
%    \begin{macrocode}
\def\TeXOSQueryFileList#1#2#3{\TeXOSQuery{#1}{%
 \string-i \string'#2\string' 
 \string'\texosquerystripquotes{#3}\string'}}
%    \end{macrocode}
%\end{macro}
%
%\begin{macro}{\TeXOSQueryFilterFileList}
%Filtered list files in the directory given in the fourth argument,
%separated by the second argument. The third argument is the regular
%expression used to filter the list. \emph{Take care of backslashes
%in the regular expression!}
%    \begin{macrocode}
\def\TeXOSQueryFilterFileList#1#2#3#4{\TeXOSQuery{#1}{%
 \string-f \string'#2\string' \string'#3\string' 
 \string'\texosquerystripquotes{#4}\string'}}
%    \end{macrocode}
%\end{macro}
%
%\begin{macro}{\TeXOSQueryFileURI}
%Get the URI of the file given in the second argument.
%    \begin{macrocode}
\def\TeXOSQueryFileURI#1#2{\TeXOSQuery{#1}{\string-u 
 \string'\texosquerystripquotes{#2}\string'}}
%    \end{macrocode}
%\end{macro}
%
%\begin{macro}{\TeXOSQueryFilePath}
%Get the canonical path of the file given in the second argument.
%    \begin{macrocode}
\def\TeXOSQueryFilePath#1#2{\TeXOSQuery{#1}{\string-p 
 \string'\texosquerystripquotes{#2}\string'}}
%    \end{macrocode}
%\end{macro}
%
%\begin{macro}{\TeXOSQueryDirName}
%\changes{1.1}{2016-07-14}{new}
%Get the canonical path of the directory containing the file given 
%in the second argument.
%    \begin{macrocode}
\def\TeXOSQueryDirName#1#2{\TeXOSQuery{#1}{\string-e 
 \string'\texosquerystripquotes{#2}\string'}}
%    \end{macrocode}
%\end{macro}
%
%All done.
%Restore the category code of \texttt{@}:
%    \begin{macrocode}
\@texosquery@restore@at
%    \end{macrocode}
%\iffalse
%    \begin{macrocode}
%</texosquery.tex>
%    \end{macrocode}
%\fi
%
%\subsection{\LaTeX\ Code}
%This is just a simple wrapper for \texttt{texosquery.tex}
%so that it can be loaded using \LaTeX's standard \cs{usepackage}
%method.
%\iffalse
%    \begin{macrocode}
%<*texosquery.sty>
%    \end{macrocode}
%\fi
%Identify package:
%    \begin{macrocode}
\NeedsTeXFormat{LaTeX2e}
\ProvidesPackage{texosquery}[2016/07/14 v1.1 (NLCT)]
%    \end{macrocode}
%Load \texttt{texosquery.tex}:
%    \begin{macrocode}
\input{texosquery}
%    \end{macrocode}
%That's it!
%\iffalse
%    \begin{macrocode}
%</texosquery.sty>
%    \end{macrocode}
%\fi
%
%\Finale
\endinput

%    \end{macrocode}
%That's it!
%\iffalse
%    \begin{macrocode}
%</texosquery.sty>
%    \end{macrocode}
%\fi
%
%\Finale
\endinput

%\end{verbatim}
%but may prefer the standard package approach:
%\begin{verbatim}
%\usepackage{texosquery}
%\end{verbatim}
%
%\begin{important}
%The commands described below are all fragile.
%\end{important}
%
%The basic command to run \texttt{texosquery} and capture its output
%in a control sequence is:
%\DescribeMacro\TeXOSQuery
%\begin{flushleft}\ttfamily
%\cs{TeXOSQuery}\marg{cs}\marg{args}
%\end{flushleft}
%where \meta{cs} is the control sequence in which to store the
%result and \meta{args} are the command line arguments to pass to 
%\texttt{texosquery}. This first locally changes the category code
%of some problematic characters that may appear in the result.
%The only special characters that aren't changed are the backslash
%\verb|\|, curly braces \verb|{| and \verb|}|, and hash \verb|#|.
%(\texttt{texosquery.jar} will replace \verb|#| with \verb|\#| in places
%where it might possibly occur in the result, but in general it's
%best to avoid these characters in file names.)
%
%There are some short cut commands for convenience, described below.
%If any of these commands cause an error message in the form:
%\begin{verbatim}
%I can't find file `|texosquery'.
%\end{verbatim}
%then check that you have the shell escape on. If the error 
%persists with the shell escape enabled
%(and not restricted), check that \texttt{texosquery} is on 
%your system's path. To do this, open a command prompt or terminal and
%type \texttt{texosquery}. If it isn't installed correctly, there will be a
%message like:
%\begin{verbatim}
%`texosquery' is not recognised
%\end{verbatim}
%or
%\begin{verbatim}
%texosquery: command not found
%\end{verbatim}
%If this happens, check the installation. (Instructions are
%in the accompanying \texttt{README.md} file.)
%
%Dry run mode is determined by the conditional
%\DescribeMacro\ifTeXOSQueryDryRun
%\begin{verbatim}
%\ifTeXOSQueryDryRun
%\end{verbatim}
%If true, the shell escape won't be used and the requested command
%invocation will be printed in the transcript file prefixed with
%\begin{verbatim}
%TeXOSQuery:
%\end{verbatim}
%(the control sequence \meta{cs} will be set to empty).
%This conditional will automatically be switched on unless
%\cs{shellescape} or \cs{pdfshellescape} is 1. (If
%\texttt{texosquery.jar} is later allowed on the restricted list,
%newer versions may change this default.)
%
%If multiple queries are required, it's more efficient to
%perform them all in one go. For example:
%\begin{verbatim}
%\TeXOSQuery{\result}{-l -n -o}
%
%\def\parseresult#1#2#3{%
%  Locale: #1. Now: #2. OS: #3.%
%}
%
%\ifx\result\empty
%  Query failed.
%\else
%  \expandafter\parseresult\result
%\fi
%\end{verbatim}
%(Make sure you have at least v1.1 for this to work correctly.)
%
%The locale (\shortarg{l} or \longarg{locale-lcs}) information can be
%obtained using:
%\DescribeMacro\TeXOSQueryLocale
%\begin{flushleft}\ttfamily
%\cs{TeXOSQueryLocale}\marg{cs}
%\end{flushleft}
%Note that this uses the lower case codeset form, which has a better
%chance of matching the encoding names used by the \texttt{inputenc}
%package. If you want the unprocessed codeset name, you can do:
%\begin{flushleft}\ttfamily
%\cs{TeXOSQuery}\marg{cs}\{\shortarg{L}\}
%\end{flushleft}
%
%The BCP 47 language tag (\shortarg{b} or \longarg{bcp47}) can be obtained using:
%\DescribeMacro\TeXOSQueryLangTag
%\begin{flushleft}\ttfamily
%\cs{TeXOSQueryLangTag}\marg{cs}
%\end{flushleft}
%
%The current working directory (\shortarg{c} or \longarg{cwd}) can
%be obtained using:
%\DescribeMacro\TeXOSQueryCwd
%\begin{flushleft}\ttfamily
%\cs{TeXOSQueryCwd}\marg{cs}
%\end{flushleft}
%
%The home directory (\shortarg{m} or \longarg{userhome}) can
%be obtained using:
%\DescribeMacro\TeXOSQueryHome
%\begin{flushleft}\ttfamily
%\cs{TeXOSQueryHome}\marg{cs}
%\end{flushleft}
%
%The temporary directory (\shortarg{t} or \longarg{tmpdir}) can
%be obtained using:
%\DescribeMacro\TeXOSQueryTmpDir
%\begin{flushleft}\ttfamily
%\cs{TeXOSQueryTmpDir}\marg{cs}
%\end{flushleft}
%
%The OS name (\shortarg{o} or \longarg{osname}) can
%be obtained using:
%\DescribeMacro\TeXOSQueryName
%\begin{flushleft}\ttfamily
%\cs{TeXOSQueryName}\marg{cs}
%\end{flushleft}
%
%The OS version (\shortarg{r} or \longarg{osversion}) can
%be obtained using:
%\DescribeMacro\TeXOSQueryVersion
%\begin{flushleft}\ttfamily
%\cs{TeXOSQueryVersion}\marg{cs}
%\end{flushleft}
%
%The OS architecture (\shortarg{a} or \longarg{osarch}) can
%be obtained using:
%\DescribeMacro\TeXOSQueryArch
%\begin{flushleft}\ttfamily
%\cs{TeXOSQueryArch}\marg{cs}
%\end{flushleft}
%
%The current date-time stamp in PDF format (\shortarg{n} or
%\longarg{pdfnow}) can be obtained using:
%\DescribeMacro\TeXOSQueryNow
%\begin{flushleft}\ttfamily
%\cs{TeXOSQueryNow}\marg{cs}
%\end{flushleft}
%This is provided for the benefit of users who don't have
%\cs{pdfcreationdate} defined by their \TeX\ format (for example,
%\XeTeX). As from v1.1, this ensures that the initial \texttt{D} has
%category code 12 (which won't happen if \cs{TeXOSQuery} is used
%explicitly).
%
%\begin{important}
%The remaining commands all require extra arguments after the
%relevant switch. These are automatically enclosed in single-quotes
%to protect any spaces. If the argument actually contains any single-quote
%characters, make sure you use \verb|\string\'| to prevent
%interference. \emph{However, in general it's a very bad idea to use
%quotes as part of a file name (rather than using them as
%delimiters).}
%Since the file name may need to be obtained from \cs{jobname},
%which sometimes includes double-quotes, the first double-quote pair
%found is stripped in file name arguments. Any other double-quotes
%will need to be protected in the same manner as single-quotes
%(but, again, this shouldn't be an issue if you use a safe
%file naming scheme). Any paths should use a
%forward slash for the directory divider.
%\end{important}
%
%The modification date-time stamp in PDF format for a file
%(\shortarg{d} or \longarg{pdfdate}) can be obtained using:
%\DescribeMacro\TeXOSQueryFileDate
%\begin{flushleft}\ttfamily
%\cs{TeXOSQueryFileDate}\marg{cs}\marg{filename}
%\end{flushleft}
%where \meta{filename} is the name of the file.
%This is provided for the benefit of users who don't have
%\cs{pdffilemoddate} defined by their \TeX\ format. As from v1.1,
%this ensures that the initial \texttt{D} has category code 12 (which 
%won't happen if \cs{TeXOSQuery} is used explicitly).
%
%The size in bytes of a file
%(\shortarg{s} or \longarg{filesize}) can be obtained using:
%\DescribeMacro\TeXOSQueryFileSize
%\begin{flushleft}\ttfamily
%\cs{TeXOSQueryFileSize}\marg{cs}\marg{filename}
%\end{flushleft}
%where \meta{filename} is the name of the file.
%This is provided for the benefit of users who don't have
%\cs{pdffilesize} defined by their \TeX\ format.
%
%The list of files in a given directory
%(\shortarg{i} or \longarg{list}) can be obtained using:
%\DescribeMacro\TeXOSQueryFileList
%\begin{flushleft}\ttfamily
%\cs{TeXOSQueryFileList}\marg{cs}\marg{sep}\marg{dir}
%\end{flushleft}
%where \meta{sep} is the separator and \meta{dir} is the directory name.
%For example:
%\begin{verbatim}
%\TeXOSQueryFileList{\result}{,}{.}
%\end{verbatim}
%will store a comma-separated list of all the files contained in the
%current directory in the control sequence \cs{result}.
%
%A filtered list of files in a given directory
%(\shortarg{f} or \longarg{filterlist}) can be obtained using:
%\DescribeMacro\TeXOSQueryFilterFileList
%\begin{flushleft}\ttfamily
%\cs{TeXOSQueryFilterFileList}\marg{cs}\marg{sep}\marg{regex}\marg{dir}
%\end{flushleft}
%where \meta{regex} is a regular expression. \emph{Take care of any backslashes
%in the regular expression!}
%For example, to list only those files that have an extension:
%\begin{verbatim}
%\TeXOSQueryFilterFileList{\result}{,}{.+\string\..*}{.}
%\end{verbatim}
%Note the use of \verb|\string\.| to ensure that \cs{.} isn't
%interpreted as a command. Another example, list only \texttt{.png}
%and \texttt{.jpg} files in the directory called \texttt{images}:
%\begin{verbatim}
%\TeXOSQueryFilterFileList{\result}{,}{.+\string\.(jpg|png)}{images}
%\end{verbatim}
%
%The URI of a file
%(\shortarg{u} or \longarg{uri}) can be obtained using:
%\DescribeMacro\TeXOSQueryFileURI
%\begin{flushleft}\ttfamily
%\cs{TeXOSQueryFileURI}\marg{cs}\marg{filename}
%\end{flushleft}
%where \meta{filename} is the name of the file.
%(Any percent symbols \verb|%| contained in the URI will have
%their category code set to 12.)
%
%The canonical path of a file
%(\shortarg{p} or \longarg{path}) can be obtained using:
%\DescribeMacro\TeXOSQueryFilePath
%\begin{flushleft}\ttfamily
%\cs{TeXOSQueryFilePath}\marg{cs}\marg{filename}
%\end{flushleft}
%where \meta{filename} is the name of the file.
%
%The canonical path of a file's parent
%(\shortarg{e} or \longarg{dirname}) can be obtained using:
%\DescribeMacro\TeXOSQueryDirName
%\begin{flushleft}\ttfamily
%\cs{TeXOSQueryDirName}\marg{cs}\marg{filename}
%\end{flushleft}
%where \meta{filename} is the name of the file.
%
%\StopEventually{\phantomsection
%\addcontentsline{toc}{section}{Change History}%
%\PrintChanges
%\addcontentsline{toc}{section}{\indexname}%
%\PrintIndex
%}
%
%\section{The Code}
%\changes{1.0}{2016-07-08}{Initial release}
%\subsection{Generic \TeX\ Code}
%\iffalse
%    \begin{macrocode}
%<*texosquery.tex>
%    \end{macrocode}
%\fi
%Change category code of \texttt{@} if necessary.
%    \begin{macrocode}
\ifnum\catcode`\@=11\relax
  \def\@texosquery@restore@at{}%
\else
  \expandafter\edef\csname @texosquery@restore@at\endcsname{%
    \noexpand\catcode`\noexpand\@=\number\catcode`\@\relax
  }%
 \catcode`\@=11\relax
\fi
%    \end{macrocode}
% Check if already loaded.
%    \begin{macrocode}
\ifx\TeXOSQuery\undefined
\else
  \@texosquery@restore@at
  \expandafter\endinput
\fi
%    \end{macrocode}
%\begin{macro}{\TeXOSInvokerName}
%The name of the \texttt{texosquery} application.
%    \begin{macrocode}
\def\TeXOSInvokerName{texosquery}
%    \end{macrocode}
%\end{macro}
%
%\begin{macro}{\TeXOSQueryInvoker}
%If we're using \LaTeX, we'll need to use \cs{@@input} rather
%than \cs{input}.
%    \begin{macrocode}
\ifx\@@input\undefined
  \def\TeXOSQueryInvoker#1{\input|"\TeXOSInvokerName\space#1" }
\else
  \def\TeXOSQueryInvoker#1{\@@input|"\TeXOSInvokerName\space#1" }
\fi
%    \end{macrocode}
%\end{macro}
%
%\begin{macro}{\ifTeXOSQueryDryRun}
%Provide a dry-run mode.
%\changes{1.1}{2016-07-14}{dry run mode only false by default if with
%unrestricted mode}
%    \begin{macrocode}
\newif\ifTeXOSQueryDryRun
\TeXOSQueryDryRuntrue
%    \end{macrocode}
%\end{macro}
%
%If shell escape is unrestricted, automatically switch off dry-run mode.
%    \begin{macrocode}
\ifx\shellescape\undefined
  \ifx\pdfshellescape\undefined
  \else
    \ifnum\pdfshellescape=1\relax
      \TeXOSQueryDryRunfalse
    \fi
  \fi
\else
  \ifnum\shellescape=1\relax
    \TeXOSQueryDryRunfalse
  \fi
\fi
%    \end{macrocode}
%
%\begin{macro}{\TeXOSQuery}
%Use \texttt{texosquery} with the option given in the second
%argument and store the result in control sequence given in the
%first argument.
%    \begin{macrocode}
\def\TeXOSQuery#1#2{%
  \ifTeXOSQueryDryRun
    \begingroup
      \newlinechar=`\^^J
      \message{^^JTeXOSQuery: \TeXOSInvokerName\space#2^^J}%
    \endgroup
    \def#1{}%
  \else
    \begingroup
    \endlinechar=-1\relax
%    \end{macrocode}
%Just in case the result contains any awkward characters that have
%some special meaning to \TeX. (Can't really protect against hash,
%backslash or curly braces, but they're unlikely to occur unless
%the user has a very wacky and inappropriate file naming scheme.)
%    \begin{macrocode}
    \catcode`\-=12\relax
    \catcode`\_=12\relax
    \catcode`\^=12\relax
    \catcode`\~=12\relax
    \catcode`\$=12\relax
    \catcode`\&=12\relax
    \catcode`\"=12\relax
    \catcode`\'=12\relax
    \catcode`\.=12\relax
    \catcode`\/=12\relax
    \catcode`\:=12\relax
    \catcode`\;=12\relax
    \catcode`\%=12\relax
    \everyeof{\noexpand}\relax
    \edef\x{\endgroup\def\noexpand#1{\TeXOSQueryInvoker{#2}}}\x
  \fi
}
%    \end{macrocode}
%\end{macro}
%
%Now for some convenient shortcuts so the user doesn't have to
%remember the command line options. \cs{string} is used in
%case the hyphen character has been made active.
%\begin{macro}{\TeXOSQueryLocale}
%Query the locale and store the result in the control sequence
%provided in the argument.
%    \begin{macrocode}
\def\TeXOSQueryLocale#1{\TeXOSQuery{#1}{\string-l}}
%    \end{macrocode}
%\end{macro}
%
%\begin{macro}{\TeXOSQueryLangTag}
%Query the language tag and store the result in the control sequence
%provided in the argument.
%\changes{1.2}{??}{new}
%    \begin{macrocode}
\def\TeXOSQueryLangTag#1{\TeXOSQuery{#1}{\string-b}}
%    \end{macrocode}
%\end{macro}
%
%\begin{macro}{\TeXOSQueryCwd}
%Query the current working directory.
%    \begin{macrocode}
\def\TeXOSQueryCwd#1{\TeXOSQuery{#1}{\string-c}}
%    \end{macrocode}
%\end{macro}
%
%\begin{macro}{\TeXOSQueryHome}
%Query the user's home directory.
%    \begin{macrocode}
\def\TeXOSQueryHome#1{\TeXOSQuery{#1}{\string-m}}
%    \end{macrocode}
%\end{macro}
%
%\begin{macro}{\TeXOSQueryTmpDir}
%Query the temporary directory.
%    \begin{macrocode}
\def\TeXOSQueryTmpDir#1{\TeXOSQuery{#1}{\string-t}}
%    \end{macrocode}
%\end{macro}
%
%\begin{macro}{\TeXOSQueryVersion}
%Query the operating system version.
%    \begin{macrocode}
\def\TeXOSQueryVersion#1{\TeXOSQuery{#1}{\string-r}}
%    \end{macrocode}
%\end{macro}
%
%\begin{macro}{\TeXOSQueryArch}
%Query the operating system architecture.
%    \begin{macrocode}
\def\TeXOSQueryArch#1{\TeXOSQuery{#1}{\string-a}}
%    \end{macrocode}
%\end{macro}
%
%\begin{macro}{\TeXOSQueryName}
%Query the operating system name.
%    \begin{macrocode}
\def\TeXOSQueryName#1{\TeXOSQuery{#1}{\string-o}}
%    \end{macrocode}
%\end{macro}
%
%\begin{macro}{\TeXOSQueryNow}
%\changes{1.1}{2016-07-14}{changed catcode of D to 12}
%Query the current time stamp.
%    \begin{macrocode}
\def\TeXOSQueryNow#1{%
%    \end{macrocode}
% The \texttt{D} needs category code 12. (Don't need to worry about
% \texttt{Z} as \texttt{texosquery.jar} uses \texttt{+00'00'} for
% UTC+0.) This change can't be done with the other catcode changes
% in \cs{TeXOSQuery}, as this is only appropriate for the PDF dates.
% Save and restore the catcode rather than fiddle around with
% scoping.
%    \begin{macrocode}
  \edef\@texosquery@restore@D{%
    \noexpand\catcode`\noexpand\D=\the\catcode`\D\relax}%
  \catcode`\D=12\relax
  \TeXOSQuery{#1}{\string-n}%
  \@texosquery@restore@D
}
%    \end{macrocode}
%\end{macro}
%
%If the file name is supplied using \cs{jobname} it may have
%double-quotes which will interfere with things.
%\begin{macro}{\texosquerystripquotes}
%    \begin{macrocode}
\def\texosquerystripquotes#1{%
  \@texosquery@stripquotes#1\@mid@texosquery@stripquotes
    "\relax"\relax\@end@texosquery@stripquotes
}
\def\@texosquery@stripquotes#1"#2"{%
  \@@texosquery@stripquotes#1#2%
}
\def\@@texosquery@stripquotes#1\@mid@texosquery@stripquotes#2\@end@texosquery@stripquotes{%
  #1%
}
%    \end{macrocode}
%\end{macro}
%
%\begin{macro}{\TeXOSQueryFileDate}
%\changes{1.1}{2016-07-14}{changed catcode of D to 12}
%Query the time stamp of the file given in the second argument.
%    \begin{macrocode}
\def\TeXOSQueryFileDate#1#2{%
  \edef\@texosquery@restore@D{%
    \noexpand\catcode`\noexpand\D=\the\catcode`\D\relax}%
  \catcode`\D=12\relax
  \TeXOSQuery{#1}{\string-d \string'\texosquerystripquotes{#2}\string'}%
  \@texosquery@restore@D
}
%    \end{macrocode}
%\end{macro}
%
%\begin{macro}{\TeXOSQueryFileSize}
%Query the size of the file given in the second argument.
%    \begin{macrocode}
\def\TeXOSQueryFileSize#1#2{\TeXOSQuery{#1}{\string-s 
 \string'\texosquerystripquotes{#2}\string'}}
%    \end{macrocode}
%\end{macro}
%
%\begin{macro}{\TeXOSQueryFileList}
%List all files in the directory given in the third argument,
%separated by the second argument.
%    \begin{macrocode}
\def\TeXOSQueryFileList#1#2#3{\TeXOSQuery{#1}{%
 \string-i \string'#2\string' 
 \string'\texosquerystripquotes{#3}\string'}}
%    \end{macrocode}
%\end{macro}
%
%\begin{macro}{\TeXOSQueryFilterFileList}
%Filtered list files in the directory given in the fourth argument,
%separated by the second argument. The third argument is the regular
%expression used to filter the list. \emph{Take care of backslashes
%in the regular expression!}
%    \begin{macrocode}
\def\TeXOSQueryFilterFileList#1#2#3#4{\TeXOSQuery{#1}{%
 \string-f \string'#2\string' \string'#3\string' 
 \string'\texosquerystripquotes{#4}\string'}}
%    \end{macrocode}
%\end{macro}
%
%\begin{macro}{\TeXOSQueryFileURI}
%Get the URI of the file given in the second argument.
%    \begin{macrocode}
\def\TeXOSQueryFileURI#1#2{\TeXOSQuery{#1}{\string-u 
 \string'\texosquerystripquotes{#2}\string'}}
%    \end{macrocode}
%\end{macro}
%
%\begin{macro}{\TeXOSQueryFilePath}
%Get the canonical path of the file given in the second argument.
%    \begin{macrocode}
\def\TeXOSQueryFilePath#1#2{\TeXOSQuery{#1}{\string-p 
 \string'\texosquerystripquotes{#2}\string'}}
%    \end{macrocode}
%\end{macro}
%
%\begin{macro}{\TeXOSQueryDirName}
%\changes{1.1}{2016-07-14}{new}
%Get the canonical path of the directory containing the file given 
%in the second argument.
%    \begin{macrocode}
\def\TeXOSQueryDirName#1#2{\TeXOSQuery{#1}{\string-e 
 \string'\texosquerystripquotes{#2}\string'}}
%    \end{macrocode}
%\end{macro}
%
%All done.
%Restore the category code of \texttt{@}:
%    \begin{macrocode}
\@texosquery@restore@at
%    \end{macrocode}
%\iffalse
%    \begin{macrocode}
%</texosquery.tex>
%    \end{macrocode}
%\fi
%
%\subsection{\LaTeX\ Code}
%This is just a simple wrapper for \texttt{texosquery.tex}
%so that it can be loaded using \LaTeX's standard \cs{usepackage}
%method.
%\iffalse
%    \begin{macrocode}
%<*texosquery.sty>
%    \end{macrocode}
%\fi
%Identify package:
%    \begin{macrocode}
\NeedsTeXFormat{LaTeX2e}
\ProvidesPackage{texosquery}[2016/07/14 v1.1 (NLCT)]
%    \end{macrocode}
%Load \texttt{texosquery.tex}:
%    \begin{macrocode}
%\iffalse
%<*package>
%% \CharacterTable
%%  {Upper-case    \A\B\C\D\E\F\G\H\I\J\K\L\M\N\O\P\Q\R\S\T\U\V\W\X\Y\Z
%%   Lower-case    \a\b\c\d\e\f\g\h\i\j\k\l\m\n\o\p\q\r\s\t\u\v\w\x\y\z
%%   Digits        \0\1\2\3\4\5\6\7\8\9
%%   Exclamation   \!     Double quote  \"     Hash (number) \#
%%   Dollar        \$     Percent       \%     Ampersand     \&
%%   Acute accent  \'     Left paren    \(     Right paren   \)
%%   Asterisk      \*     Plus          \+     Comma         \,
%%   Minus         \-     Point         \.     Solidus       \/
%%   Colon         \:     Semicolon     \;     Less than     \<
%%   Equals        \=     Greater than  \>     Question mark \?
%%   Commercial at \@     Left bracket  \[     Backslash     \\
%%   Right bracket \]     Circumflex    \^     Underscore    \_
%%   Grave accent  \`     Left brace    \{     Vertical bar  \|
%%   Right brace   \}     Tilde         \~}
%</package>
%\fi
% \iffalse
% Doc-Source file
% Copyright (C) 2016 Nicola Talbot, all rights reserved.
% \fi
% \iffalse
%<*driver>
\documentclass{ltxdoc}

\usepackage[T1]{fontenc}
\usepackage{tgtermes}
\usepackage{metalogo}
\usepackage[colorlinks,hyperindex=false]{hyperref}

\CheckSum{273}

\RecordChanges
\PageIndex
\CodelineNumbered
\IndexPrologue{}

\renewcommand*{\usage}[1]{\hyperpage{#1}}
\renewcommand*{\main}[1]{\hyperpage{#1}}

\newcommand*{\shortargfmt}[1]{\texttt{-#1}}
\newcommand*{\longargfmt}[1]{\texttt{-{}-#1}}

\newcommand*{\shortarg}[1]{\shortargfmt{#1}%
 \index{texosquery options=\texttt{texosquery} options>#1=\protect\shortargfmt{-#1}|hyperpage}}
\newcommand*{\longarg}[1]{\longargfmt{#1}%
 \index{texosquery options=\texttt{texosquery} options>#1=\protect\longargfmt{#1}|hyperpage}}

\newenvironment{important}
 {\begin{quote}\textbf{Important Note:}}
 {\end{quote}}

\begin{document}
\DocInput{texosquery.dtx}
\end{document}
%</driver>
%\fi
%
%\DeleteShortVerb{|}
%
%\title{texosquery: query OS information from \TeX}
%\author{Nicola L. C. Talbot\\
%\href{http://www.dickimaw-books.com/}{\nolinkurl{dickimaw-books.com}}
%\and Paulo Cereda}
%\date{2016-07-14 (v1.1)}
%\maketitle
%
%\begin{abstract}
%\texttt{texosquery.jar} is a cross-platform Java application
%to query certain OS information designed for use in \TeX's
%shell escape mechanism.
%The accompanying \TeX\ code provides a simple interface to the
%\texttt{texosquery} Java application, which may be skipped
%if you want to explicitly use \cs{input} (but take
%care of any special characters appearing in the result).
%\end{abstract}
%
%\begin{important}
%You will need \TeX's shell escape enabled,
%and you will also need the Java Runtime Environment (JRE) 
%installed to use \texttt{texosquery}.
%\end{important}
%
%If you want to rebuild the application, instructions for
%compiling the source code (including the code for this document)
%are in the accompanying \texttt{README.md} file.
%
%\tableofcontents
%
%\section{texosquery.jar: the Java application}
%The \texttt{texosquery} Java command line application looks up
%certain system information that may be of use in \TeX\ 
%documents. All this information can easily be obtained using 
%native commands, but the Java application allows an 
%OS-independent approach with results that can easily be
%captured by \TeX's shell-escape without having to strip 
%formatting information.
%
%\begin{important}
%\texttt{texosquery} provides read-only actions, and I don't 
%intend adding any actions that modify system settings or files.
%\end{important}
%
%Since the application is designed to work with \TeX, each
%function will display the result on a single line without
%formatting. (For multiple results, each line is grouped 
%from v1.1.) A blank line (or empty group) will be displayed
%if the information isn't available. A forward slash is always 
%used as a directory divider, regardless of the operating 
%system, so the result can be used, for example, in \cs{input} or
%\cs{includegraphics}.
%
%If an input file name is required (for example, with the
%\longargfmt{pdfdate} argument described below) then the file may be in
%the current working directory, relative to the current directory
%(with forward slash \texttt{/} as the directory divider), an absolute path 
%(again with forward slash) or on \TeX's path (in
%which case, \texttt{kpsewhich} is used to locate it).
%
%Command line invocation:
%\begin{flushleft}\ttfamily
%texosquery \meta{action} ...
%\end{flushleft}
%Available actions (at least one required):
%\begin{description}
%\item[\shortarg{L} or \longarg{locale}] Display the locale
%information in the form
%\begin{flushleft}\ttfamily
%\meta{lang}-\meta{region}.\meta{codeset}@\meta{modifier}
%\end{flushleft}
%where \meta{lang} is the ISO code for the language (e.g.\ 
%\texttt{en}),
%\meta{region} is the ISO code for the region (e.g.\ \texttt{GB}),
%\meta{codeset} is the default code set (e.g.\ \texttt{UTF-8})
% and \meta{modifier} is the modifier. Elements may be omitted
% if unavailable. For example, \texttt{en-GB.UTF-8} has the
% \meta{modifier} omitted, and \texttt{en} has all but the language
% omitted.
%\item[\shortarg{l} or \longarg{locale-lcs}] As the above but the
%codeset (if present) is converted to lower case and any hyphens are
%stripped. For example, if \longargfmt{locale} returns
%\texttt{en-GB.UTF-8}, then \longargfmt{locale-lcs} would return
%\texttt{en-GB.utf8}.
%\item[\shortarg{b} or \longarg{bcp47}] (New to version 1.2.)
%Displays the BCP~47 language tag.
%
%\item[\shortarg{c} or \longarg{cwd}] Displays the current working
%directory.
%\item[\shortarg{m} or \longarg{userhome}] Displays the user's home
%directory.
%\item[\shortarg{t} or \longarg{tmpdir}] Displays the temporary
%directory.
%\item[\shortarg{o} or \longarg{osname}] Displays the operating
%system name.
%\item[\shortarg{r} or \longarg{osversion}] Displays the operating
%system version.
%\item[\shortarg{a} or \longarg{osarch}] Displays the operating
%system architecture.
%\item[\shortarg{n} or \longarg{pdfnow}] Displays the current
% date and time in PDF format. For example 
%\begin{verbatim}
%D:20160704131006+01'00'
%\end{verbatim}
%Note that some, but not all, \TeX\ formats provide
%\cs{pdfcreationdate}, which is more efficient than using the shell
%escape, but this can be used
%as a fallback method for those that don't (for example, \XeTeX).
%\item[\shortarg{d} \meta{file} or \longarg{pdfdate} \meta{file}]
% Displays the last modified time stamp of the given file in PDF format or a blank line
% if the file doesn't exist or the file permissions prohibit this
% action.
%Again some, but not all, \TeX\ formats provide
%\cs{pdffilemoddate}\marg{file}, which is more efficient than using
%the shell escape.
%\item[\shortarg{s} \meta{file} or \longarg{filesize} \meta{file}]
% Displays the size in bytes of the given file or a blank line
% if the file doesn't exist or the file permissions prohibit this
% action.
%Some, but not all, \TeX\ formats provide
%\cs{pdffilesize}\marg{file}, which is more efficient than using
%the shell escape.
%\item[\shortarg{i} \meta{sep} \meta{dir} or \longarg{list}
%\meta{sep} \meta{dir}]
%List all files in the given directory with the output on a single
%line using \meta{sep} as the separator between entries. Note that
%the list doesn't include the full path, just the file names.
%\item[\shortarg{f} \meta{sep} \meta{regex} \meta{dir} or
%\longarg{filterlist} \meta{sep} \meta{regex} \meta{dir}]
%Like the above but only lists those files whose name matches the
%regular expression given in \meta{regex}. Note that this tests for a
%complete match on the file name (not including path). For example,
%if \meta{regexp} is \texttt{foo.*}, it will only match files whose
%name starts with \texttt{foo} (for example, \texttt{foobar} will
%match but \texttt{barfoo} won't).
%\item[\shortarg{u} \meta{file} or \longarg{uri} \meta{file}]
%Displays the URI of the given file or a blank line if the file
%doesn't exist or the file permissions prohibit this action.
%Note that this may include in a percent character in the result. The \TeX\ 
%command \cs{TeXOSQuery} protects against this by changing the
%category code, but if you explicitly call \texttt{texosquery}
%using the shell escape, you'll need to be careful of this.
%\item[\shortarg{p} \meta{file} or \longarg{path} \meta{file}]
%Displays the canonical path of the given file or a blank line if
%the file doesn't exist or the file permissions prohibit this
%action.
%\item[\shortarg{e} \meta{file} or \longarg{dirname} \meta{file}]
%(New to v1.1.)
%Displays the canonical path of the given file's parent (that is, the
%directory containing \meta{file}) or a blank line if
%the file doesn't exist or the file permissions prohibit this
%action. Note that this is different to the Unix-like 
%\texttt{dirname} command, which will return a relative path if
%\meta{file} isn't an absolute path.
%\item[\shortarg{h} or \longarg{help}] Displays help message and
%exits.
%\item[\shortarg{v} or \longarg{version}] Displays version
%information and exits.
%\end{description}
%
%If multiple options are given, they will be processed in the
%order specified in the command line invocation. Each result will
%be displayed on a separate line. As from v1.1, if there are
%multiple actions, each result will be grouped. This makes it easier
%to process the results in \TeX. For example:
%\begin{verbatim}
%texosquery -l
%\end{verbatim}
%This just produces (for me):
%\begin{verbatim}
%en-GB.utf8
%\end{verbatim}
%whereas
%\begin{verbatim}
%texosquery -l -n
%\end{verbatim}
%produces:
%\begin{verbatim}
%{en-GB.utf8}
%{D:20160714112732+01'00'}
%\end{verbatim}
%Note that unavailable information will produce an empty group.
%For example (assuming \texttt{nofile} doesn't exist):
%\begin{verbatim}
%texosquery -l -d nofile -n
%\end{verbatim}
%produces:
%\begin{verbatim}
%{en-GB.utf8}
%{}
%{D:20160714112732+01'00'}
%\end{verbatim}
%whereas 
%\begin{verbatim}
%texosquery -d nofile
%\end{verbatim}
%just displays an empty line.
%
%\section{texosquery.tex: generic \TeX\ code}
%You can run \texttt{texosquery} directly from \TeX's shell escape.
%For example:
%\begin{verbatim}
%\input|"texosquery --locale"
%\end{verbatim}
%However, \texttt{texosquery.tex} provides generic \TeX\ code to do
%this for you and store the result in a control sequence.
%
%Plain \TeX\ users can input this file through the usual \cs{input}
%method:
%\begin{verbatim}
%\input texosquery
%\end{verbatim}
%\LaTeX\ users may also simply input this file:
%\begin{verbatim}
%%\iffalse
%<*package>
%% \CharacterTable
%%  {Upper-case    \A\B\C\D\E\F\G\H\I\J\K\L\M\N\O\P\Q\R\S\T\U\V\W\X\Y\Z
%%   Lower-case    \a\b\c\d\e\f\g\h\i\j\k\l\m\n\o\p\q\r\s\t\u\v\w\x\y\z
%%   Digits        \0\1\2\3\4\5\6\7\8\9
%%   Exclamation   \!     Double quote  \"     Hash (number) \#
%%   Dollar        \$     Percent       \%     Ampersand     \&
%%   Acute accent  \'     Left paren    \(     Right paren   \)
%%   Asterisk      \*     Plus          \+     Comma         \,
%%   Minus         \-     Point         \.     Solidus       \/
%%   Colon         \:     Semicolon     \;     Less than     \<
%%   Equals        \=     Greater than  \>     Question mark \?
%%   Commercial at \@     Left bracket  \[     Backslash     \\
%%   Right bracket \]     Circumflex    \^     Underscore    \_
%%   Grave accent  \`     Left brace    \{     Vertical bar  \|
%%   Right brace   \}     Tilde         \~}
%</package>
%\fi
% \iffalse
% Doc-Source file
% Copyright (C) 2016 Nicola Talbot, all rights reserved.
% \fi
% \iffalse
%<*driver>
\documentclass{ltxdoc}

\usepackage[T1]{fontenc}
\usepackage{tgtermes}
\usepackage{metalogo}
\usepackage[colorlinks,hyperindex=false]{hyperref}

\CheckSum{273}

\RecordChanges
\PageIndex
\CodelineNumbered
\IndexPrologue{}

\renewcommand*{\usage}[1]{\hyperpage{#1}}
\renewcommand*{\main}[1]{\hyperpage{#1}}

\newcommand*{\shortargfmt}[1]{\texttt{-#1}}
\newcommand*{\longargfmt}[1]{\texttt{-{}-#1}}

\newcommand*{\shortarg}[1]{\shortargfmt{#1}%
 \index{texosquery options=\texttt{texosquery} options>#1=\protect\shortargfmt{-#1}|hyperpage}}
\newcommand*{\longarg}[1]{\longargfmt{#1}%
 \index{texosquery options=\texttt{texosquery} options>#1=\protect\longargfmt{#1}|hyperpage}}

\newenvironment{important}
 {\begin{quote}\textbf{Important Note:}}
 {\end{quote}}

\begin{document}
\DocInput{texosquery.dtx}
\end{document}
%</driver>
%\fi
%
%\DeleteShortVerb{|}
%
%\title{texosquery: query OS information from \TeX}
%\author{Nicola L. C. Talbot\\
%\href{http://www.dickimaw-books.com/}{\nolinkurl{dickimaw-books.com}}
%\and Paulo Cereda}
%\date{2016-07-14 (v1.1)}
%\maketitle
%
%\begin{abstract}
%\texttt{texosquery.jar} is a cross-platform Java application
%to query certain OS information designed for use in \TeX's
%shell escape mechanism.
%The accompanying \TeX\ code provides a simple interface to the
%\texttt{texosquery} Java application, which may be skipped
%if you want to explicitly use \cs{input} (but take
%care of any special characters appearing in the result).
%\end{abstract}
%
%\begin{important}
%You will need \TeX's shell escape enabled,
%and you will also need the Java Runtime Environment (JRE) 
%installed to use \texttt{texosquery}.
%\end{important}
%
%If you want to rebuild the application, instructions for
%compiling the source code (including the code for this document)
%are in the accompanying \texttt{README.md} file.
%
%\tableofcontents
%
%\section{texosquery.jar: the Java application}
%The \texttt{texosquery} Java command line application looks up
%certain system information that may be of use in \TeX\ 
%documents. All this information can easily be obtained using 
%native commands, but the Java application allows an 
%OS-independent approach with results that can easily be
%captured by \TeX's shell-escape without having to strip 
%formatting information.
%
%\begin{important}
%\texttt{texosquery} provides read-only actions, and I don't 
%intend adding any actions that modify system settings or files.
%\end{important}
%
%Since the application is designed to work with \TeX, each
%function will display the result on a single line without
%formatting. (For multiple results, each line is grouped 
%from v1.1.) A blank line (or empty group) will be displayed
%if the information isn't available. A forward slash is always 
%used as a directory divider, regardless of the operating 
%system, so the result can be used, for example, in \cs{input} or
%\cs{includegraphics}.
%
%If an input file name is required (for example, with the
%\longargfmt{pdfdate} argument described below) then the file may be in
%the current working directory, relative to the current directory
%(with forward slash \texttt{/} as the directory divider), an absolute path 
%(again with forward slash) or on \TeX's path (in
%which case, \texttt{kpsewhich} is used to locate it).
%
%Command line invocation:
%\begin{flushleft}\ttfamily
%texosquery \meta{action} ...
%\end{flushleft}
%Available actions (at least one required):
%\begin{description}
%\item[\shortarg{L} or \longarg{locale}] Display the locale
%information in the form
%\begin{flushleft}\ttfamily
%\meta{lang}-\meta{region}.\meta{codeset}@\meta{modifier}
%\end{flushleft}
%where \meta{lang} is the ISO code for the language (e.g.\ 
%\texttt{en}),
%\meta{region} is the ISO code for the region (e.g.\ \texttt{GB}),
%\meta{codeset} is the default code set (e.g.\ \texttt{UTF-8})
% and \meta{modifier} is the modifier. Elements may be omitted
% if unavailable. For example, \texttt{en-GB.UTF-8} has the
% \meta{modifier} omitted, and \texttt{en} has all but the language
% omitted.
%\item[\shortarg{l} or \longarg{locale-lcs}] As the above but the
%codeset (if present) is converted to lower case and any hyphens are
%stripped. For example, if \longargfmt{locale} returns
%\texttt{en-GB.UTF-8}, then \longargfmt{locale-lcs} would return
%\texttt{en-GB.utf8}.
%\item[\shortarg{b} or \longarg{bcp47}] (New to version 1.2.)
%Displays the BCP~47 language tag.
%
%\item[\shortarg{c} or \longarg{cwd}] Displays the current working
%directory.
%\item[\shortarg{m} or \longarg{userhome}] Displays the user's home
%directory.
%\item[\shortarg{t} or \longarg{tmpdir}] Displays the temporary
%directory.
%\item[\shortarg{o} or \longarg{osname}] Displays the operating
%system name.
%\item[\shortarg{r} or \longarg{osversion}] Displays the operating
%system version.
%\item[\shortarg{a} or \longarg{osarch}] Displays the operating
%system architecture.
%\item[\shortarg{n} or \longarg{pdfnow}] Displays the current
% date and time in PDF format. For example 
%\begin{verbatim}
%D:20160704131006+01'00'
%\end{verbatim}
%Note that some, but not all, \TeX\ formats provide
%\cs{pdfcreationdate}, which is more efficient than using the shell
%escape, but this can be used
%as a fallback method for those that don't (for example, \XeTeX).
%\item[\shortarg{d} \meta{file} or \longarg{pdfdate} \meta{file}]
% Displays the last modified time stamp of the given file in PDF format or a blank line
% if the file doesn't exist or the file permissions prohibit this
% action.
%Again some, but not all, \TeX\ formats provide
%\cs{pdffilemoddate}\marg{file}, which is more efficient than using
%the shell escape.
%\item[\shortarg{s} \meta{file} or \longarg{filesize} \meta{file}]
% Displays the size in bytes of the given file or a blank line
% if the file doesn't exist or the file permissions prohibit this
% action.
%Some, but not all, \TeX\ formats provide
%\cs{pdffilesize}\marg{file}, which is more efficient than using
%the shell escape.
%\item[\shortarg{i} \meta{sep} \meta{dir} or \longarg{list}
%\meta{sep} \meta{dir}]
%List all files in the given directory with the output on a single
%line using \meta{sep} as the separator between entries. Note that
%the list doesn't include the full path, just the file names.
%\item[\shortarg{f} \meta{sep} \meta{regex} \meta{dir} or
%\longarg{filterlist} \meta{sep} \meta{regex} \meta{dir}]
%Like the above but only lists those files whose name matches the
%regular expression given in \meta{regex}. Note that this tests for a
%complete match on the file name (not including path). For example,
%if \meta{regexp} is \texttt{foo.*}, it will only match files whose
%name starts with \texttt{foo} (for example, \texttt{foobar} will
%match but \texttt{barfoo} won't).
%\item[\shortarg{u} \meta{file} or \longarg{uri} \meta{file}]
%Displays the URI of the given file or a blank line if the file
%doesn't exist or the file permissions prohibit this action.
%Note that this may include in a percent character in the result. The \TeX\ 
%command \cs{TeXOSQuery} protects against this by changing the
%category code, but if you explicitly call \texttt{texosquery}
%using the shell escape, you'll need to be careful of this.
%\item[\shortarg{p} \meta{file} or \longarg{path} \meta{file}]
%Displays the canonical path of the given file or a blank line if
%the file doesn't exist or the file permissions prohibit this
%action.
%\item[\shortarg{e} \meta{file} or \longarg{dirname} \meta{file}]
%(New to v1.1.)
%Displays the canonical path of the given file's parent (that is, the
%directory containing \meta{file}) or a blank line if
%the file doesn't exist or the file permissions prohibit this
%action. Note that this is different to the Unix-like 
%\texttt{dirname} command, which will return a relative path if
%\meta{file} isn't an absolute path.
%\item[\shortarg{h} or \longarg{help}] Displays help message and
%exits.
%\item[\shortarg{v} or \longarg{version}] Displays version
%information and exits.
%\end{description}
%
%If multiple options are given, they will be processed in the
%order specified in the command line invocation. Each result will
%be displayed on a separate line. As from v1.1, if there are
%multiple actions, each result will be grouped. This makes it easier
%to process the results in \TeX. For example:
%\begin{verbatim}
%texosquery -l
%\end{verbatim}
%This just produces (for me):
%\begin{verbatim}
%en-GB.utf8
%\end{verbatim}
%whereas
%\begin{verbatim}
%texosquery -l -n
%\end{verbatim}
%produces:
%\begin{verbatim}
%{en-GB.utf8}
%{D:20160714112732+01'00'}
%\end{verbatim}
%Note that unavailable information will produce an empty group.
%For example (assuming \texttt{nofile} doesn't exist):
%\begin{verbatim}
%texosquery -l -d nofile -n
%\end{verbatim}
%produces:
%\begin{verbatim}
%{en-GB.utf8}
%{}
%{D:20160714112732+01'00'}
%\end{verbatim}
%whereas 
%\begin{verbatim}
%texosquery -d nofile
%\end{verbatim}
%just displays an empty line.
%
%\section{texosquery.tex: generic \TeX\ code}
%You can run \texttt{texosquery} directly from \TeX's shell escape.
%For example:
%\begin{verbatim}
%\input|"texosquery --locale"
%\end{verbatim}
%However, \texttt{texosquery.tex} provides generic \TeX\ code to do
%this for you and store the result in a control sequence.
%
%Plain \TeX\ users can input this file through the usual \cs{input}
%method:
%\begin{verbatim}
%\input texosquery
%\end{verbatim}
%\LaTeX\ users may also simply input this file:
%\begin{verbatim}
%\input{texosquery}
%\end{verbatim}
%but may prefer the standard package approach:
%\begin{verbatim}
%\usepackage{texosquery}
%\end{verbatim}
%
%\begin{important}
%The commands described below are all fragile.
%\end{important}
%
%The basic command to run \texttt{texosquery} and capture its output
%in a control sequence is:
%\DescribeMacro\TeXOSQuery
%\begin{flushleft}\ttfamily
%\cs{TeXOSQuery}\marg{cs}\marg{args}
%\end{flushleft}
%where \meta{cs} is the control sequence in which to store the
%result and \meta{args} are the command line arguments to pass to 
%\texttt{texosquery}. This first locally changes the category code
%of some problematic characters that may appear in the result.
%The only special characters that aren't changed are the backslash
%\verb|\|, curly braces \verb|{| and \verb|}|, and hash \verb|#|.
%(\texttt{texosquery.jar} will replace \verb|#| with \verb|\#| in places
%where it might possibly occur in the result, but in general it's
%best to avoid these characters in file names.)
%
%There are some short cut commands for convenience, described below.
%If any of these commands cause an error message in the form:
%\begin{verbatim}
%I can't find file `|texosquery'.
%\end{verbatim}
%then check that you have the shell escape on. If the error 
%persists with the shell escape enabled
%(and not restricted), check that \texttt{texosquery} is on 
%your system's path. To do this, open a command prompt or terminal and
%type \texttt{texosquery}. If it isn't installed correctly, there will be a
%message like:
%\begin{verbatim}
%`texosquery' is not recognised
%\end{verbatim}
%or
%\begin{verbatim}
%texosquery: command not found
%\end{verbatim}
%If this happens, check the installation. (Instructions are
%in the accompanying \texttt{README.md} file.)
%
%Dry run mode is determined by the conditional
%\DescribeMacro\ifTeXOSQueryDryRun
%\begin{verbatim}
%\ifTeXOSQueryDryRun
%\end{verbatim}
%If true, the shell escape won't be used and the requested command
%invocation will be printed in the transcript file prefixed with
%\begin{verbatim}
%TeXOSQuery:
%\end{verbatim}
%(the control sequence \meta{cs} will be set to empty).
%This conditional will automatically be switched on unless
%\cs{shellescape} or \cs{pdfshellescape} is 1. (If
%\texttt{texosquery.jar} is later allowed on the restricted list,
%newer versions may change this default.)
%
%If multiple queries are required, it's more efficient to
%perform them all in one go. For example:
%\begin{verbatim}
%\TeXOSQuery{\result}{-l -n -o}
%
%\def\parseresult#1#2#3{%
%  Locale: #1. Now: #2. OS: #3.%
%}
%
%\ifx\result\empty
%  Query failed.
%\else
%  \expandafter\parseresult\result
%\fi
%\end{verbatim}
%(Make sure you have at least v1.1 for this to work correctly.)
%
%The locale (\shortarg{l} or \longarg{locale-lcs}) information can be
%obtained using:
%\DescribeMacro\TeXOSQueryLocale
%\begin{flushleft}\ttfamily
%\cs{TeXOSQueryLocale}\marg{cs}
%\end{flushleft}
%Note that this uses the lower case codeset form, which has a better
%chance of matching the encoding names used by the \texttt{inputenc}
%package. If you want the unprocessed codeset name, you can do:
%\begin{flushleft}\ttfamily
%\cs{TeXOSQuery}\marg{cs}\{\shortarg{L}\}
%\end{flushleft}
%
%The BCP 47 language tag (\shortarg{b} or \longarg{bcp47}) can be obtained using:
%\DescribeMacro\TeXOSQueryLangTag
%\begin{flushleft}\ttfamily
%\cs{TeXOSQueryLangTag}\marg{cs}
%\end{flushleft}
%
%The current working directory (\shortarg{c} or \longarg{cwd}) can
%be obtained using:
%\DescribeMacro\TeXOSQueryCwd
%\begin{flushleft}\ttfamily
%\cs{TeXOSQueryCwd}\marg{cs}
%\end{flushleft}
%
%The home directory (\shortarg{m} or \longarg{userhome}) can
%be obtained using:
%\DescribeMacro\TeXOSQueryHome
%\begin{flushleft}\ttfamily
%\cs{TeXOSQueryHome}\marg{cs}
%\end{flushleft}
%
%The temporary directory (\shortarg{t} or \longarg{tmpdir}) can
%be obtained using:
%\DescribeMacro\TeXOSQueryTmpDir
%\begin{flushleft}\ttfamily
%\cs{TeXOSQueryTmpDir}\marg{cs}
%\end{flushleft}
%
%The OS name (\shortarg{o} or \longarg{osname}) can
%be obtained using:
%\DescribeMacro\TeXOSQueryName
%\begin{flushleft}\ttfamily
%\cs{TeXOSQueryName}\marg{cs}
%\end{flushleft}
%
%The OS version (\shortarg{r} or \longarg{osversion}) can
%be obtained using:
%\DescribeMacro\TeXOSQueryVersion
%\begin{flushleft}\ttfamily
%\cs{TeXOSQueryVersion}\marg{cs}
%\end{flushleft}
%
%The OS architecture (\shortarg{a} or \longarg{osarch}) can
%be obtained using:
%\DescribeMacro\TeXOSQueryArch
%\begin{flushleft}\ttfamily
%\cs{TeXOSQueryArch}\marg{cs}
%\end{flushleft}
%
%The current date-time stamp in PDF format (\shortarg{n} or
%\longarg{pdfnow}) can be obtained using:
%\DescribeMacro\TeXOSQueryNow
%\begin{flushleft}\ttfamily
%\cs{TeXOSQueryNow}\marg{cs}
%\end{flushleft}
%This is provided for the benefit of users who don't have
%\cs{pdfcreationdate} defined by their \TeX\ format (for example,
%\XeTeX). As from v1.1, this ensures that the initial \texttt{D} has
%category code 12 (which won't happen if \cs{TeXOSQuery} is used
%explicitly).
%
%\begin{important}
%The remaining commands all require extra arguments after the
%relevant switch. These are automatically enclosed in single-quotes
%to protect any spaces. If the argument actually contains any single-quote
%characters, make sure you use \verb|\string\'| to prevent
%interference. \emph{However, in general it's a very bad idea to use
%quotes as part of a file name (rather than using them as
%delimiters).}
%Since the file name may need to be obtained from \cs{jobname},
%which sometimes includes double-quotes, the first double-quote pair
%found is stripped in file name arguments. Any other double-quotes
%will need to be protected in the same manner as single-quotes
%(but, again, this shouldn't be an issue if you use a safe
%file naming scheme). Any paths should use a
%forward slash for the directory divider.
%\end{important}
%
%The modification date-time stamp in PDF format for a file
%(\shortarg{d} or \longarg{pdfdate}) can be obtained using:
%\DescribeMacro\TeXOSQueryFileDate
%\begin{flushleft}\ttfamily
%\cs{TeXOSQueryFileDate}\marg{cs}\marg{filename}
%\end{flushleft}
%where \meta{filename} is the name of the file.
%This is provided for the benefit of users who don't have
%\cs{pdffilemoddate} defined by their \TeX\ format. As from v1.1,
%this ensures that the initial \texttt{D} has category code 12 (which 
%won't happen if \cs{TeXOSQuery} is used explicitly).
%
%The size in bytes of a file
%(\shortarg{s} or \longarg{filesize}) can be obtained using:
%\DescribeMacro\TeXOSQueryFileSize
%\begin{flushleft}\ttfamily
%\cs{TeXOSQueryFileSize}\marg{cs}\marg{filename}
%\end{flushleft}
%where \meta{filename} is the name of the file.
%This is provided for the benefit of users who don't have
%\cs{pdffilesize} defined by their \TeX\ format.
%
%The list of files in a given directory
%(\shortarg{i} or \longarg{list}) can be obtained using:
%\DescribeMacro\TeXOSQueryFileList
%\begin{flushleft}\ttfamily
%\cs{TeXOSQueryFileList}\marg{cs}\marg{sep}\marg{dir}
%\end{flushleft}
%where \meta{sep} is the separator and \meta{dir} is the directory name.
%For example:
%\begin{verbatim}
%\TeXOSQueryFileList{\result}{,}{.}
%\end{verbatim}
%will store a comma-separated list of all the files contained in the
%current directory in the control sequence \cs{result}.
%
%A filtered list of files in a given directory
%(\shortarg{f} or \longarg{filterlist}) can be obtained using:
%\DescribeMacro\TeXOSQueryFilterFileList
%\begin{flushleft}\ttfamily
%\cs{TeXOSQueryFilterFileList}\marg{cs}\marg{sep}\marg{regex}\marg{dir}
%\end{flushleft}
%where \meta{regex} is a regular expression. \emph{Take care of any backslashes
%in the regular expression!}
%For example, to list only those files that have an extension:
%\begin{verbatim}
%\TeXOSQueryFilterFileList{\result}{,}{.+\string\..*}{.}
%\end{verbatim}
%Note the use of \verb|\string\.| to ensure that \cs{.} isn't
%interpreted as a command. Another example, list only \texttt{.png}
%and \texttt{.jpg} files in the directory called \texttt{images}:
%\begin{verbatim}
%\TeXOSQueryFilterFileList{\result}{,}{.+\string\.(jpg|png)}{images}
%\end{verbatim}
%
%The URI of a file
%(\shortarg{u} or \longarg{uri}) can be obtained using:
%\DescribeMacro\TeXOSQueryFileURI
%\begin{flushleft}\ttfamily
%\cs{TeXOSQueryFileURI}\marg{cs}\marg{filename}
%\end{flushleft}
%where \meta{filename} is the name of the file.
%(Any percent symbols \verb|%| contained in the URI will have
%their category code set to 12.)
%
%The canonical path of a file
%(\shortarg{p} or \longarg{path}) can be obtained using:
%\DescribeMacro\TeXOSQueryFilePath
%\begin{flushleft}\ttfamily
%\cs{TeXOSQueryFilePath}\marg{cs}\marg{filename}
%\end{flushleft}
%where \meta{filename} is the name of the file.
%
%The canonical path of a file's parent
%(\shortarg{e} or \longarg{dirname}) can be obtained using:
%\DescribeMacro\TeXOSQueryDirName
%\begin{flushleft}\ttfamily
%\cs{TeXOSQueryDirName}\marg{cs}\marg{filename}
%\end{flushleft}
%where \meta{filename} is the name of the file.
%
%\StopEventually{\phantomsection
%\addcontentsline{toc}{section}{Change History}%
%\PrintChanges
%\addcontentsline{toc}{section}{\indexname}%
%\PrintIndex
%}
%
%\section{The Code}
%\changes{1.0}{2016-07-08}{Initial release}
%\subsection{Generic \TeX\ Code}
%\iffalse
%    \begin{macrocode}
%<*texosquery.tex>
%    \end{macrocode}
%\fi
%Change category code of \texttt{@} if necessary.
%    \begin{macrocode}
\ifnum\catcode`\@=11\relax
  \def\@texosquery@restore@at{}%
\else
  \expandafter\edef\csname @texosquery@restore@at\endcsname{%
    \noexpand\catcode`\noexpand\@=\number\catcode`\@\relax
  }%
 \catcode`\@=11\relax
\fi
%    \end{macrocode}
% Check if already loaded.
%    \begin{macrocode}
\ifx\TeXOSQuery\undefined
\else
  \@texosquery@restore@at
  \expandafter\endinput
\fi
%    \end{macrocode}
%\begin{macro}{\TeXOSInvokerName}
%The name of the \texttt{texosquery} application.
%    \begin{macrocode}
\def\TeXOSInvokerName{texosquery}
%    \end{macrocode}
%\end{macro}
%
%\begin{macro}{\TeXOSQueryInvoker}
%If we're using \LaTeX, we'll need to use \cs{@@input} rather
%than \cs{input}.
%    \begin{macrocode}
\ifx\@@input\undefined
  \def\TeXOSQueryInvoker#1{\input|"\TeXOSInvokerName\space#1" }
\else
  \def\TeXOSQueryInvoker#1{\@@input|"\TeXOSInvokerName\space#1" }
\fi
%    \end{macrocode}
%\end{macro}
%
%\begin{macro}{\ifTeXOSQueryDryRun}
%Provide a dry-run mode.
%\changes{1.1}{2016-07-14}{dry run mode only false by default if with
%unrestricted mode}
%    \begin{macrocode}
\newif\ifTeXOSQueryDryRun
\TeXOSQueryDryRuntrue
%    \end{macrocode}
%\end{macro}
%
%If shell escape is unrestricted, automatically switch off dry-run mode.
%    \begin{macrocode}
\ifx\shellescape\undefined
  \ifx\pdfshellescape\undefined
  \else
    \ifnum\pdfshellescape=1\relax
      \TeXOSQueryDryRunfalse
    \fi
  \fi
\else
  \ifnum\shellescape=1\relax
    \TeXOSQueryDryRunfalse
  \fi
\fi
%    \end{macrocode}
%
%\begin{macro}{\TeXOSQuery}
%Use \texttt{texosquery} with the option given in the second
%argument and store the result in control sequence given in the
%first argument.
%    \begin{macrocode}
\def\TeXOSQuery#1#2{%
  \ifTeXOSQueryDryRun
    \begingroup
      \newlinechar=`\^^J
      \message{^^JTeXOSQuery: \TeXOSInvokerName\space#2^^J}%
    \endgroup
    \def#1{}%
  \else
    \begingroup
    \endlinechar=-1\relax
%    \end{macrocode}
%Just in case the result contains any awkward characters that have
%some special meaning to \TeX. (Can't really protect against hash,
%backslash or curly braces, but they're unlikely to occur unless
%the user has a very wacky and inappropriate file naming scheme.)
%    \begin{macrocode}
    \catcode`\-=12\relax
    \catcode`\_=12\relax
    \catcode`\^=12\relax
    \catcode`\~=12\relax
    \catcode`\$=12\relax
    \catcode`\&=12\relax
    \catcode`\"=12\relax
    \catcode`\'=12\relax
    \catcode`\.=12\relax
    \catcode`\/=12\relax
    \catcode`\:=12\relax
    \catcode`\;=12\relax
    \catcode`\%=12\relax
    \everyeof{\noexpand}\relax
    \edef\x{\endgroup\def\noexpand#1{\TeXOSQueryInvoker{#2}}}\x
  \fi
}
%    \end{macrocode}
%\end{macro}
%
%Now for some convenient shortcuts so the user doesn't have to
%remember the command line options. \cs{string} is used in
%case the hyphen character has been made active.
%\begin{macro}{\TeXOSQueryLocale}
%Query the locale and store the result in the control sequence
%provided in the argument.
%    \begin{macrocode}
\def\TeXOSQueryLocale#1{\TeXOSQuery{#1}{\string-l}}
%    \end{macrocode}
%\end{macro}
%
%\begin{macro}{\TeXOSQueryLangTag}
%Query the language tag and store the result in the control sequence
%provided in the argument.
%\changes{1.2}{??}{new}
%    \begin{macrocode}
\def\TeXOSQueryLangTag#1{\TeXOSQuery{#1}{\string-b}}
%    \end{macrocode}
%\end{macro}
%
%\begin{macro}{\TeXOSQueryCwd}
%Query the current working directory.
%    \begin{macrocode}
\def\TeXOSQueryCwd#1{\TeXOSQuery{#1}{\string-c}}
%    \end{macrocode}
%\end{macro}
%
%\begin{macro}{\TeXOSQueryHome}
%Query the user's home directory.
%    \begin{macrocode}
\def\TeXOSQueryHome#1{\TeXOSQuery{#1}{\string-m}}
%    \end{macrocode}
%\end{macro}
%
%\begin{macro}{\TeXOSQueryTmpDir}
%Query the temporary directory.
%    \begin{macrocode}
\def\TeXOSQueryTmpDir#1{\TeXOSQuery{#1}{\string-t}}
%    \end{macrocode}
%\end{macro}
%
%\begin{macro}{\TeXOSQueryVersion}
%Query the operating system version.
%    \begin{macrocode}
\def\TeXOSQueryVersion#1{\TeXOSQuery{#1}{\string-r}}
%    \end{macrocode}
%\end{macro}
%
%\begin{macro}{\TeXOSQueryArch}
%Query the operating system architecture.
%    \begin{macrocode}
\def\TeXOSQueryArch#1{\TeXOSQuery{#1}{\string-a}}
%    \end{macrocode}
%\end{macro}
%
%\begin{macro}{\TeXOSQueryName}
%Query the operating system name.
%    \begin{macrocode}
\def\TeXOSQueryName#1{\TeXOSQuery{#1}{\string-o}}
%    \end{macrocode}
%\end{macro}
%
%\begin{macro}{\TeXOSQueryNow}
%\changes{1.1}{2016-07-14}{changed catcode of D to 12}
%Query the current time stamp.
%    \begin{macrocode}
\def\TeXOSQueryNow#1{%
%    \end{macrocode}
% The \texttt{D} needs category code 12. (Don't need to worry about
% \texttt{Z} as \texttt{texosquery.jar} uses \texttt{+00'00'} for
% UTC+0.) This change can't be done with the other catcode changes
% in \cs{TeXOSQuery}, as this is only appropriate for the PDF dates.
% Save and restore the catcode rather than fiddle around with
% scoping.
%    \begin{macrocode}
  \edef\@texosquery@restore@D{%
    \noexpand\catcode`\noexpand\D=\the\catcode`\D\relax}%
  \catcode`\D=12\relax
  \TeXOSQuery{#1}{\string-n}%
  \@texosquery@restore@D
}
%    \end{macrocode}
%\end{macro}
%
%If the file name is supplied using \cs{jobname} it may have
%double-quotes which will interfere with things.
%\begin{macro}{\texosquerystripquotes}
%    \begin{macrocode}
\def\texosquerystripquotes#1{%
  \@texosquery@stripquotes#1\@mid@texosquery@stripquotes
    "\relax"\relax\@end@texosquery@stripquotes
}
\def\@texosquery@stripquotes#1"#2"{%
  \@@texosquery@stripquotes#1#2%
}
\def\@@texosquery@stripquotes#1\@mid@texosquery@stripquotes#2\@end@texosquery@stripquotes{%
  #1%
}
%    \end{macrocode}
%\end{macro}
%
%\begin{macro}{\TeXOSQueryFileDate}
%\changes{1.1}{2016-07-14}{changed catcode of D to 12}
%Query the time stamp of the file given in the second argument.
%    \begin{macrocode}
\def\TeXOSQueryFileDate#1#2{%
  \edef\@texosquery@restore@D{%
    \noexpand\catcode`\noexpand\D=\the\catcode`\D\relax}%
  \catcode`\D=12\relax
  \TeXOSQuery{#1}{\string-d \string'\texosquerystripquotes{#2}\string'}%
  \@texosquery@restore@D
}
%    \end{macrocode}
%\end{macro}
%
%\begin{macro}{\TeXOSQueryFileSize}
%Query the size of the file given in the second argument.
%    \begin{macrocode}
\def\TeXOSQueryFileSize#1#2{\TeXOSQuery{#1}{\string-s 
 \string'\texosquerystripquotes{#2}\string'}}
%    \end{macrocode}
%\end{macro}
%
%\begin{macro}{\TeXOSQueryFileList}
%List all files in the directory given in the third argument,
%separated by the second argument.
%    \begin{macrocode}
\def\TeXOSQueryFileList#1#2#3{\TeXOSQuery{#1}{%
 \string-i \string'#2\string' 
 \string'\texosquerystripquotes{#3}\string'}}
%    \end{macrocode}
%\end{macro}
%
%\begin{macro}{\TeXOSQueryFilterFileList}
%Filtered list files in the directory given in the fourth argument,
%separated by the second argument. The third argument is the regular
%expression used to filter the list. \emph{Take care of backslashes
%in the regular expression!}
%    \begin{macrocode}
\def\TeXOSQueryFilterFileList#1#2#3#4{\TeXOSQuery{#1}{%
 \string-f \string'#2\string' \string'#3\string' 
 \string'\texosquerystripquotes{#4}\string'}}
%    \end{macrocode}
%\end{macro}
%
%\begin{macro}{\TeXOSQueryFileURI}
%Get the URI of the file given in the second argument.
%    \begin{macrocode}
\def\TeXOSQueryFileURI#1#2{\TeXOSQuery{#1}{\string-u 
 \string'\texosquerystripquotes{#2}\string'}}
%    \end{macrocode}
%\end{macro}
%
%\begin{macro}{\TeXOSQueryFilePath}
%Get the canonical path of the file given in the second argument.
%    \begin{macrocode}
\def\TeXOSQueryFilePath#1#2{\TeXOSQuery{#1}{\string-p 
 \string'\texosquerystripquotes{#2}\string'}}
%    \end{macrocode}
%\end{macro}
%
%\begin{macro}{\TeXOSQueryDirName}
%\changes{1.1}{2016-07-14}{new}
%Get the canonical path of the directory containing the file given 
%in the second argument.
%    \begin{macrocode}
\def\TeXOSQueryDirName#1#2{\TeXOSQuery{#1}{\string-e 
 \string'\texosquerystripquotes{#2}\string'}}
%    \end{macrocode}
%\end{macro}
%
%All done.
%Restore the category code of \texttt{@}:
%    \begin{macrocode}
\@texosquery@restore@at
%    \end{macrocode}
%\iffalse
%    \begin{macrocode}
%</texosquery.tex>
%    \end{macrocode}
%\fi
%
%\subsection{\LaTeX\ Code}
%This is just a simple wrapper for \texttt{texosquery.tex}
%so that it can be loaded using \LaTeX's standard \cs{usepackage}
%method.
%\iffalse
%    \begin{macrocode}
%<*texosquery.sty>
%    \end{macrocode}
%\fi
%Identify package:
%    \begin{macrocode}
\NeedsTeXFormat{LaTeX2e}
\ProvidesPackage{texosquery}[2016/07/14 v1.1 (NLCT)]
%    \end{macrocode}
%Load \texttt{texosquery.tex}:
%    \begin{macrocode}
\input{texosquery}
%    \end{macrocode}
%That's it!
%\iffalse
%    \begin{macrocode}
%</texosquery.sty>
%    \end{macrocode}
%\fi
%
%\Finale
\endinput

%\end{verbatim}
%but may prefer the standard package approach:
%\begin{verbatim}
%\usepackage{texosquery}
%\end{verbatim}
%
%\begin{important}
%The commands described below are all fragile.
%\end{important}
%
%The basic command to run \texttt{texosquery} and capture its output
%in a control sequence is:
%\DescribeMacro\TeXOSQuery
%\begin{flushleft}\ttfamily
%\cs{TeXOSQuery}\marg{cs}\marg{args}
%\end{flushleft}
%where \meta{cs} is the control sequence in which to store the
%result and \meta{args} are the command line arguments to pass to 
%\texttt{texosquery}. This first locally changes the category code
%of some problematic characters that may appear in the result.
%The only special characters that aren't changed are the backslash
%\verb|\|, curly braces \verb|{| and \verb|}|, and hash \verb|#|.
%(\texttt{texosquery.jar} will replace \verb|#| with \verb|\#| in places
%where it might possibly occur in the result, but in general it's
%best to avoid these characters in file names.)
%
%There are some short cut commands for convenience, described below.
%If any of these commands cause an error message in the form:
%\begin{verbatim}
%I can't find file `|texosquery'.
%\end{verbatim}
%then check that you have the shell escape on. If the error 
%persists with the shell escape enabled
%(and not restricted), check that \texttt{texosquery} is on 
%your system's path. To do this, open a command prompt or terminal and
%type \texttt{texosquery}. If it isn't installed correctly, there will be a
%message like:
%\begin{verbatim}
%`texosquery' is not recognised
%\end{verbatim}
%or
%\begin{verbatim}
%texosquery: command not found
%\end{verbatim}
%If this happens, check the installation. (Instructions are
%in the accompanying \texttt{README.md} file.)
%
%Dry run mode is determined by the conditional
%\DescribeMacro\ifTeXOSQueryDryRun
%\begin{verbatim}
%\ifTeXOSQueryDryRun
%\end{verbatim}
%If true, the shell escape won't be used and the requested command
%invocation will be printed in the transcript file prefixed with
%\begin{verbatim}
%TeXOSQuery:
%\end{verbatim}
%(the control sequence \meta{cs} will be set to empty).
%This conditional will automatically be switched on unless
%\cs{shellescape} or \cs{pdfshellescape} is 1. (If
%\texttt{texosquery.jar} is later allowed on the restricted list,
%newer versions may change this default.)
%
%If multiple queries are required, it's more efficient to
%perform them all in one go. For example:
%\begin{verbatim}
%\TeXOSQuery{\result}{-l -n -o}
%
%\def\parseresult#1#2#3{%
%  Locale: #1. Now: #2. OS: #3.%
%}
%
%\ifx\result\empty
%  Query failed.
%\else
%  \expandafter\parseresult\result
%\fi
%\end{verbatim}
%(Make sure you have at least v1.1 for this to work correctly.)
%
%The locale (\shortarg{l} or \longarg{locale-lcs}) information can be
%obtained using:
%\DescribeMacro\TeXOSQueryLocale
%\begin{flushleft}\ttfamily
%\cs{TeXOSQueryLocale}\marg{cs}
%\end{flushleft}
%Note that this uses the lower case codeset form, which has a better
%chance of matching the encoding names used by the \texttt{inputenc}
%package. If you want the unprocessed codeset name, you can do:
%\begin{flushleft}\ttfamily
%\cs{TeXOSQuery}\marg{cs}\{\shortarg{L}\}
%\end{flushleft}
%
%The BCP 47 language tag (\shortarg{b} or \longarg{bcp47}) can be obtained using:
%\DescribeMacro\TeXOSQueryLangTag
%\begin{flushleft}\ttfamily
%\cs{TeXOSQueryLangTag}\marg{cs}
%\end{flushleft}
%
%The current working directory (\shortarg{c} or \longarg{cwd}) can
%be obtained using:
%\DescribeMacro\TeXOSQueryCwd
%\begin{flushleft}\ttfamily
%\cs{TeXOSQueryCwd}\marg{cs}
%\end{flushleft}
%
%The home directory (\shortarg{m} or \longarg{userhome}) can
%be obtained using:
%\DescribeMacro\TeXOSQueryHome
%\begin{flushleft}\ttfamily
%\cs{TeXOSQueryHome}\marg{cs}
%\end{flushleft}
%
%The temporary directory (\shortarg{t} or \longarg{tmpdir}) can
%be obtained using:
%\DescribeMacro\TeXOSQueryTmpDir
%\begin{flushleft}\ttfamily
%\cs{TeXOSQueryTmpDir}\marg{cs}
%\end{flushleft}
%
%The OS name (\shortarg{o} or \longarg{osname}) can
%be obtained using:
%\DescribeMacro\TeXOSQueryName
%\begin{flushleft}\ttfamily
%\cs{TeXOSQueryName}\marg{cs}
%\end{flushleft}
%
%The OS version (\shortarg{r} or \longarg{osversion}) can
%be obtained using:
%\DescribeMacro\TeXOSQueryVersion
%\begin{flushleft}\ttfamily
%\cs{TeXOSQueryVersion}\marg{cs}
%\end{flushleft}
%
%The OS architecture (\shortarg{a} or \longarg{osarch}) can
%be obtained using:
%\DescribeMacro\TeXOSQueryArch
%\begin{flushleft}\ttfamily
%\cs{TeXOSQueryArch}\marg{cs}
%\end{flushleft}
%
%The current date-time stamp in PDF format (\shortarg{n} or
%\longarg{pdfnow}) can be obtained using:
%\DescribeMacro\TeXOSQueryNow
%\begin{flushleft}\ttfamily
%\cs{TeXOSQueryNow}\marg{cs}
%\end{flushleft}
%This is provided for the benefit of users who don't have
%\cs{pdfcreationdate} defined by their \TeX\ format (for example,
%\XeTeX). As from v1.1, this ensures that the initial \texttt{D} has
%category code 12 (which won't happen if \cs{TeXOSQuery} is used
%explicitly).
%
%\begin{important}
%The remaining commands all require extra arguments after the
%relevant switch. These are automatically enclosed in single-quotes
%to protect any spaces. If the argument actually contains any single-quote
%characters, make sure you use \verb|\string\'| to prevent
%interference. \emph{However, in general it's a very bad idea to use
%quotes as part of a file name (rather than using them as
%delimiters).}
%Since the file name may need to be obtained from \cs{jobname},
%which sometimes includes double-quotes, the first double-quote pair
%found is stripped in file name arguments. Any other double-quotes
%will need to be protected in the same manner as single-quotes
%(but, again, this shouldn't be an issue if you use a safe
%file naming scheme). Any paths should use a
%forward slash for the directory divider.
%\end{important}
%
%The modification date-time stamp in PDF format for a file
%(\shortarg{d} or \longarg{pdfdate}) can be obtained using:
%\DescribeMacro\TeXOSQueryFileDate
%\begin{flushleft}\ttfamily
%\cs{TeXOSQueryFileDate}\marg{cs}\marg{filename}
%\end{flushleft}
%where \meta{filename} is the name of the file.
%This is provided for the benefit of users who don't have
%\cs{pdffilemoddate} defined by their \TeX\ format. As from v1.1,
%this ensures that the initial \texttt{D} has category code 12 (which 
%won't happen if \cs{TeXOSQuery} is used explicitly).
%
%The size in bytes of a file
%(\shortarg{s} or \longarg{filesize}) can be obtained using:
%\DescribeMacro\TeXOSQueryFileSize
%\begin{flushleft}\ttfamily
%\cs{TeXOSQueryFileSize}\marg{cs}\marg{filename}
%\end{flushleft}
%where \meta{filename} is the name of the file.
%This is provided for the benefit of users who don't have
%\cs{pdffilesize} defined by their \TeX\ format.
%
%The list of files in a given directory
%(\shortarg{i} or \longarg{list}) can be obtained using:
%\DescribeMacro\TeXOSQueryFileList
%\begin{flushleft}\ttfamily
%\cs{TeXOSQueryFileList}\marg{cs}\marg{sep}\marg{dir}
%\end{flushleft}
%where \meta{sep} is the separator and \meta{dir} is the directory name.
%For example:
%\begin{verbatim}
%\TeXOSQueryFileList{\result}{,}{.}
%\end{verbatim}
%will store a comma-separated list of all the files contained in the
%current directory in the control sequence \cs{result}.
%
%A filtered list of files in a given directory
%(\shortarg{f} or \longarg{filterlist}) can be obtained using:
%\DescribeMacro\TeXOSQueryFilterFileList
%\begin{flushleft}\ttfamily
%\cs{TeXOSQueryFilterFileList}\marg{cs}\marg{sep}\marg{regex}\marg{dir}
%\end{flushleft}
%where \meta{regex} is a regular expression. \emph{Take care of any backslashes
%in the regular expression!}
%For example, to list only those files that have an extension:
%\begin{verbatim}
%\TeXOSQueryFilterFileList{\result}{,}{.+\string\..*}{.}
%\end{verbatim}
%Note the use of \verb|\string\.| to ensure that \cs{.} isn't
%interpreted as a command. Another example, list only \texttt{.png}
%and \texttt{.jpg} files in the directory called \texttt{images}:
%\begin{verbatim}
%\TeXOSQueryFilterFileList{\result}{,}{.+\string\.(jpg|png)}{images}
%\end{verbatim}
%
%The URI of a file
%(\shortarg{u} or \longarg{uri}) can be obtained using:
%\DescribeMacro\TeXOSQueryFileURI
%\begin{flushleft}\ttfamily
%\cs{TeXOSQueryFileURI}\marg{cs}\marg{filename}
%\end{flushleft}
%where \meta{filename} is the name of the file.
%(Any percent symbols \verb|%| contained in the URI will have
%their category code set to 12.)
%
%The canonical path of a file
%(\shortarg{p} or \longarg{path}) can be obtained using:
%\DescribeMacro\TeXOSQueryFilePath
%\begin{flushleft}\ttfamily
%\cs{TeXOSQueryFilePath}\marg{cs}\marg{filename}
%\end{flushleft}
%where \meta{filename} is the name of the file.
%
%The canonical path of a file's parent
%(\shortarg{e} or \longarg{dirname}) can be obtained using:
%\DescribeMacro\TeXOSQueryDirName
%\begin{flushleft}\ttfamily
%\cs{TeXOSQueryDirName}\marg{cs}\marg{filename}
%\end{flushleft}
%where \meta{filename} is the name of the file.
%
%\StopEventually{\phantomsection
%\addcontentsline{toc}{section}{Change History}%
%\PrintChanges
%\addcontentsline{toc}{section}{\indexname}%
%\PrintIndex
%}
%
%\section{The Code}
%\changes{1.0}{2016-07-08}{Initial release}
%\subsection{Generic \TeX\ Code}
%\iffalse
%    \begin{macrocode}
%<*texosquery.tex>
%    \end{macrocode}
%\fi
%Change category code of \texttt{@} if necessary.
%    \begin{macrocode}
\ifnum\catcode`\@=11\relax
  \def\@texosquery@restore@at{}%
\else
  \expandafter\edef\csname @texosquery@restore@at\endcsname{%
    \noexpand\catcode`\noexpand\@=\number\catcode`\@\relax
  }%
 \catcode`\@=11\relax
\fi
%    \end{macrocode}
% Check if already loaded.
%    \begin{macrocode}
\ifx\TeXOSQuery\undefined
\else
  \@texosquery@restore@at
  \expandafter\endinput
\fi
%    \end{macrocode}
%\begin{macro}{\TeXOSInvokerName}
%The name of the \texttt{texosquery} application.
%    \begin{macrocode}
\def\TeXOSInvokerName{texosquery}
%    \end{macrocode}
%\end{macro}
%
%\begin{macro}{\TeXOSQueryInvoker}
%If we're using \LaTeX, we'll need to use \cs{@@input} rather
%than \cs{input}.
%    \begin{macrocode}
\ifx\@@input\undefined
  \def\TeXOSQueryInvoker#1{\input|"\TeXOSInvokerName\space#1" }
\else
  \def\TeXOSQueryInvoker#1{\@@input|"\TeXOSInvokerName\space#1" }
\fi
%    \end{macrocode}
%\end{macro}
%
%\begin{macro}{\ifTeXOSQueryDryRun}
%Provide a dry-run mode.
%\changes{1.1}{2016-07-14}{dry run mode only false by default if with
%unrestricted mode}
%    \begin{macrocode}
\newif\ifTeXOSQueryDryRun
\TeXOSQueryDryRuntrue
%    \end{macrocode}
%\end{macro}
%
%If shell escape is unrestricted, automatically switch off dry-run mode.
%    \begin{macrocode}
\ifx\shellescape\undefined
  \ifx\pdfshellescape\undefined
  \else
    \ifnum\pdfshellescape=1\relax
      \TeXOSQueryDryRunfalse
    \fi
  \fi
\else
  \ifnum\shellescape=1\relax
    \TeXOSQueryDryRunfalse
  \fi
\fi
%    \end{macrocode}
%
%\begin{macro}{\TeXOSQuery}
%Use \texttt{texosquery} with the option given in the second
%argument and store the result in control sequence given in the
%first argument.
%    \begin{macrocode}
\def\TeXOSQuery#1#2{%
  \ifTeXOSQueryDryRun
    \begingroup
      \newlinechar=`\^^J
      \message{^^JTeXOSQuery: \TeXOSInvokerName\space#2^^J}%
    \endgroup
    \def#1{}%
  \else
    \begingroup
    \endlinechar=-1\relax
%    \end{macrocode}
%Just in case the result contains any awkward characters that have
%some special meaning to \TeX. (Can't really protect against hash,
%backslash or curly braces, but they're unlikely to occur unless
%the user has a very wacky and inappropriate file naming scheme.)
%    \begin{macrocode}
    \catcode`\-=12\relax
    \catcode`\_=12\relax
    \catcode`\^=12\relax
    \catcode`\~=12\relax
    \catcode`\$=12\relax
    \catcode`\&=12\relax
    \catcode`\"=12\relax
    \catcode`\'=12\relax
    \catcode`\.=12\relax
    \catcode`\/=12\relax
    \catcode`\:=12\relax
    \catcode`\;=12\relax
    \catcode`\%=12\relax
    \everyeof{\noexpand}\relax
    \edef\x{\endgroup\def\noexpand#1{\TeXOSQueryInvoker{#2}}}\x
  \fi
}
%    \end{macrocode}
%\end{macro}
%
%Now for some convenient shortcuts so the user doesn't have to
%remember the command line options. \cs{string} is used in
%case the hyphen character has been made active.
%\begin{macro}{\TeXOSQueryLocale}
%Query the locale and store the result in the control sequence
%provided in the argument.
%    \begin{macrocode}
\def\TeXOSQueryLocale#1{\TeXOSQuery{#1}{\string-l}}
%    \end{macrocode}
%\end{macro}
%
%\begin{macro}{\TeXOSQueryLangTag}
%Query the language tag and store the result in the control sequence
%provided in the argument.
%\changes{1.2}{??}{new}
%    \begin{macrocode}
\def\TeXOSQueryLangTag#1{\TeXOSQuery{#1}{\string-b}}
%    \end{macrocode}
%\end{macro}
%
%\begin{macro}{\TeXOSQueryCwd}
%Query the current working directory.
%    \begin{macrocode}
\def\TeXOSQueryCwd#1{\TeXOSQuery{#1}{\string-c}}
%    \end{macrocode}
%\end{macro}
%
%\begin{macro}{\TeXOSQueryHome}
%Query the user's home directory.
%    \begin{macrocode}
\def\TeXOSQueryHome#1{\TeXOSQuery{#1}{\string-m}}
%    \end{macrocode}
%\end{macro}
%
%\begin{macro}{\TeXOSQueryTmpDir}
%Query the temporary directory.
%    \begin{macrocode}
\def\TeXOSQueryTmpDir#1{\TeXOSQuery{#1}{\string-t}}
%    \end{macrocode}
%\end{macro}
%
%\begin{macro}{\TeXOSQueryVersion}
%Query the operating system version.
%    \begin{macrocode}
\def\TeXOSQueryVersion#1{\TeXOSQuery{#1}{\string-r}}
%    \end{macrocode}
%\end{macro}
%
%\begin{macro}{\TeXOSQueryArch}
%Query the operating system architecture.
%    \begin{macrocode}
\def\TeXOSQueryArch#1{\TeXOSQuery{#1}{\string-a}}
%    \end{macrocode}
%\end{macro}
%
%\begin{macro}{\TeXOSQueryName}
%Query the operating system name.
%    \begin{macrocode}
\def\TeXOSQueryName#1{\TeXOSQuery{#1}{\string-o}}
%    \end{macrocode}
%\end{macro}
%
%\begin{macro}{\TeXOSQueryNow}
%\changes{1.1}{2016-07-14}{changed catcode of D to 12}
%Query the current time stamp.
%    \begin{macrocode}
\def\TeXOSQueryNow#1{%
%    \end{macrocode}
% The \texttt{D} needs category code 12. (Don't need to worry about
% \texttt{Z} as \texttt{texosquery.jar} uses \texttt{+00'00'} for
% UTC+0.) This change can't be done with the other catcode changes
% in \cs{TeXOSQuery}, as this is only appropriate for the PDF dates.
% Save and restore the catcode rather than fiddle around with
% scoping.
%    \begin{macrocode}
  \edef\@texosquery@restore@D{%
    \noexpand\catcode`\noexpand\D=\the\catcode`\D\relax}%
  \catcode`\D=12\relax
  \TeXOSQuery{#1}{\string-n}%
  \@texosquery@restore@D
}
%    \end{macrocode}
%\end{macro}
%
%If the file name is supplied using \cs{jobname} it may have
%double-quotes which will interfere with things.
%\begin{macro}{\texosquerystripquotes}
%    \begin{macrocode}
\def\texosquerystripquotes#1{%
  \@texosquery@stripquotes#1\@mid@texosquery@stripquotes
    "\relax"\relax\@end@texosquery@stripquotes
}
\def\@texosquery@stripquotes#1"#2"{%
  \@@texosquery@stripquotes#1#2%
}
\def\@@texosquery@stripquotes#1\@mid@texosquery@stripquotes#2\@end@texosquery@stripquotes{%
  #1%
}
%    \end{macrocode}
%\end{macro}
%
%\begin{macro}{\TeXOSQueryFileDate}
%\changes{1.1}{2016-07-14}{changed catcode of D to 12}
%Query the time stamp of the file given in the second argument.
%    \begin{macrocode}
\def\TeXOSQueryFileDate#1#2{%
  \edef\@texosquery@restore@D{%
    \noexpand\catcode`\noexpand\D=\the\catcode`\D\relax}%
  \catcode`\D=12\relax
  \TeXOSQuery{#1}{\string-d \string'\texosquerystripquotes{#2}\string'}%
  \@texosquery@restore@D
}
%    \end{macrocode}
%\end{macro}
%
%\begin{macro}{\TeXOSQueryFileSize}
%Query the size of the file given in the second argument.
%    \begin{macrocode}
\def\TeXOSQueryFileSize#1#2{\TeXOSQuery{#1}{\string-s 
 \string'\texosquerystripquotes{#2}\string'}}
%    \end{macrocode}
%\end{macro}
%
%\begin{macro}{\TeXOSQueryFileList}
%List all files in the directory given in the third argument,
%separated by the second argument.
%    \begin{macrocode}
\def\TeXOSQueryFileList#1#2#3{\TeXOSQuery{#1}{%
 \string-i \string'#2\string' 
 \string'\texosquerystripquotes{#3}\string'}}
%    \end{macrocode}
%\end{macro}
%
%\begin{macro}{\TeXOSQueryFilterFileList}
%Filtered list files in the directory given in the fourth argument,
%separated by the second argument. The third argument is the regular
%expression used to filter the list. \emph{Take care of backslashes
%in the regular expression!}
%    \begin{macrocode}
\def\TeXOSQueryFilterFileList#1#2#3#4{\TeXOSQuery{#1}{%
 \string-f \string'#2\string' \string'#3\string' 
 \string'\texosquerystripquotes{#4}\string'}}
%    \end{macrocode}
%\end{macro}
%
%\begin{macro}{\TeXOSQueryFileURI}
%Get the URI of the file given in the second argument.
%    \begin{macrocode}
\def\TeXOSQueryFileURI#1#2{\TeXOSQuery{#1}{\string-u 
 \string'\texosquerystripquotes{#2}\string'}}
%    \end{macrocode}
%\end{macro}
%
%\begin{macro}{\TeXOSQueryFilePath}
%Get the canonical path of the file given in the second argument.
%    \begin{macrocode}
\def\TeXOSQueryFilePath#1#2{\TeXOSQuery{#1}{\string-p 
 \string'\texosquerystripquotes{#2}\string'}}
%    \end{macrocode}
%\end{macro}
%
%\begin{macro}{\TeXOSQueryDirName}
%\changes{1.1}{2016-07-14}{new}
%Get the canonical path of the directory containing the file given 
%in the second argument.
%    \begin{macrocode}
\def\TeXOSQueryDirName#1#2{\TeXOSQuery{#1}{\string-e 
 \string'\texosquerystripquotes{#2}\string'}}
%    \end{macrocode}
%\end{macro}
%
%All done.
%Restore the category code of \texttt{@}:
%    \begin{macrocode}
\@texosquery@restore@at
%    \end{macrocode}
%\iffalse
%    \begin{macrocode}
%</texosquery.tex>
%    \end{macrocode}
%\fi
%
%\subsection{\LaTeX\ Code}
%This is just a simple wrapper for \texttt{texosquery.tex}
%so that it can be loaded using \LaTeX's standard \cs{usepackage}
%method.
%\iffalse
%    \begin{macrocode}
%<*texosquery.sty>
%    \end{macrocode}
%\fi
%Identify package:
%    \begin{macrocode}
\NeedsTeXFormat{LaTeX2e}
\ProvidesPackage{texosquery}[2016/07/14 v1.1 (NLCT)]
%    \end{macrocode}
%Load \texttt{texosquery.tex}:
%    \begin{macrocode}
%\iffalse
%<*package>
%% \CharacterTable
%%  {Upper-case    \A\B\C\D\E\F\G\H\I\J\K\L\M\N\O\P\Q\R\S\T\U\V\W\X\Y\Z
%%   Lower-case    \a\b\c\d\e\f\g\h\i\j\k\l\m\n\o\p\q\r\s\t\u\v\w\x\y\z
%%   Digits        \0\1\2\3\4\5\6\7\8\9
%%   Exclamation   \!     Double quote  \"     Hash (number) \#
%%   Dollar        \$     Percent       \%     Ampersand     \&
%%   Acute accent  \'     Left paren    \(     Right paren   \)
%%   Asterisk      \*     Plus          \+     Comma         \,
%%   Minus         \-     Point         \.     Solidus       \/
%%   Colon         \:     Semicolon     \;     Less than     \<
%%   Equals        \=     Greater than  \>     Question mark \?
%%   Commercial at \@     Left bracket  \[     Backslash     \\
%%   Right bracket \]     Circumflex    \^     Underscore    \_
%%   Grave accent  \`     Left brace    \{     Vertical bar  \|
%%   Right brace   \}     Tilde         \~}
%</package>
%\fi
% \iffalse
% Doc-Source file
% Copyright (C) 2016 Nicola Talbot, all rights reserved.
% \fi
% \iffalse
%<*driver>
\documentclass{ltxdoc}

\usepackage[T1]{fontenc}
\usepackage{tgtermes}
\usepackage{metalogo}
\usepackage[colorlinks,hyperindex=false]{hyperref}

\CheckSum{273}

\RecordChanges
\PageIndex
\CodelineNumbered
\IndexPrologue{}

\renewcommand*{\usage}[1]{\hyperpage{#1}}
\renewcommand*{\main}[1]{\hyperpage{#1}}

\newcommand*{\shortargfmt}[1]{\texttt{-#1}}
\newcommand*{\longargfmt}[1]{\texttt{-{}-#1}}

\newcommand*{\shortarg}[1]{\shortargfmt{#1}%
 \index{texosquery options=\texttt{texosquery} options>#1=\protect\shortargfmt{-#1}|hyperpage}}
\newcommand*{\longarg}[1]{\longargfmt{#1}%
 \index{texosquery options=\texttt{texosquery} options>#1=\protect\longargfmt{#1}|hyperpage}}

\newenvironment{important}
 {\begin{quote}\textbf{Important Note:}}
 {\end{quote}}

\begin{document}
\DocInput{texosquery.dtx}
\end{document}
%</driver>
%\fi
%
%\DeleteShortVerb{|}
%
%\title{texosquery: query OS information from \TeX}
%\author{Nicola L. C. Talbot\\
%\href{http://www.dickimaw-books.com/}{\nolinkurl{dickimaw-books.com}}
%\and Paulo Cereda}
%\date{2016-07-14 (v1.1)}
%\maketitle
%
%\begin{abstract}
%\texttt{texosquery.jar} is a cross-platform Java application
%to query certain OS information designed for use in \TeX's
%shell escape mechanism.
%The accompanying \TeX\ code provides a simple interface to the
%\texttt{texosquery} Java application, which may be skipped
%if you want to explicitly use \cs{input} (but take
%care of any special characters appearing in the result).
%\end{abstract}
%
%\begin{important}
%You will need \TeX's shell escape enabled,
%and you will also need the Java Runtime Environment (JRE) 
%installed to use \texttt{texosquery}.
%\end{important}
%
%If you want to rebuild the application, instructions for
%compiling the source code (including the code for this document)
%are in the accompanying \texttt{README.md} file.
%
%\tableofcontents
%
%\section{texosquery.jar: the Java application}
%The \texttt{texosquery} Java command line application looks up
%certain system information that may be of use in \TeX\ 
%documents. All this information can easily be obtained using 
%native commands, but the Java application allows an 
%OS-independent approach with results that can easily be
%captured by \TeX's shell-escape without having to strip 
%formatting information.
%
%\begin{important}
%\texttt{texosquery} provides read-only actions, and I don't 
%intend adding any actions that modify system settings or files.
%\end{important}
%
%Since the application is designed to work with \TeX, each
%function will display the result on a single line without
%formatting. (For multiple results, each line is grouped 
%from v1.1.) A blank line (or empty group) will be displayed
%if the information isn't available. A forward slash is always 
%used as a directory divider, regardless of the operating 
%system, so the result can be used, for example, in \cs{input} or
%\cs{includegraphics}.
%
%If an input file name is required (for example, with the
%\longargfmt{pdfdate} argument described below) then the file may be in
%the current working directory, relative to the current directory
%(with forward slash \texttt{/} as the directory divider), an absolute path 
%(again with forward slash) or on \TeX's path (in
%which case, \texttt{kpsewhich} is used to locate it).
%
%Command line invocation:
%\begin{flushleft}\ttfamily
%texosquery \meta{action} ...
%\end{flushleft}
%Available actions (at least one required):
%\begin{description}
%\item[\shortarg{L} or \longarg{locale}] Display the locale
%information in the form
%\begin{flushleft}\ttfamily
%\meta{lang}-\meta{region}.\meta{codeset}@\meta{modifier}
%\end{flushleft}
%where \meta{lang} is the ISO code for the language (e.g.\ 
%\texttt{en}),
%\meta{region} is the ISO code for the region (e.g.\ \texttt{GB}),
%\meta{codeset} is the default code set (e.g.\ \texttt{UTF-8})
% and \meta{modifier} is the modifier. Elements may be omitted
% if unavailable. For example, \texttt{en-GB.UTF-8} has the
% \meta{modifier} omitted, and \texttt{en} has all but the language
% omitted.
%\item[\shortarg{l} or \longarg{locale-lcs}] As the above but the
%codeset (if present) is converted to lower case and any hyphens are
%stripped. For example, if \longargfmt{locale} returns
%\texttt{en-GB.UTF-8}, then \longargfmt{locale-lcs} would return
%\texttt{en-GB.utf8}.
%\item[\shortarg{b} or \longarg{bcp47}] (New to version 1.2.)
%Displays the BCP~47 language tag.
%
%\item[\shortarg{c} or \longarg{cwd}] Displays the current working
%directory.
%\item[\shortarg{m} or \longarg{userhome}] Displays the user's home
%directory.
%\item[\shortarg{t} or \longarg{tmpdir}] Displays the temporary
%directory.
%\item[\shortarg{o} or \longarg{osname}] Displays the operating
%system name.
%\item[\shortarg{r} or \longarg{osversion}] Displays the operating
%system version.
%\item[\shortarg{a} or \longarg{osarch}] Displays the operating
%system architecture.
%\item[\shortarg{n} or \longarg{pdfnow}] Displays the current
% date and time in PDF format. For example 
%\begin{verbatim}
%D:20160704131006+01'00'
%\end{verbatim}
%Note that some, but not all, \TeX\ formats provide
%\cs{pdfcreationdate}, which is more efficient than using the shell
%escape, but this can be used
%as a fallback method for those that don't (for example, \XeTeX).
%\item[\shortarg{d} \meta{file} or \longarg{pdfdate} \meta{file}]
% Displays the last modified time stamp of the given file in PDF format or a blank line
% if the file doesn't exist or the file permissions prohibit this
% action.
%Again some, but not all, \TeX\ formats provide
%\cs{pdffilemoddate}\marg{file}, which is more efficient than using
%the shell escape.
%\item[\shortarg{s} \meta{file} or \longarg{filesize} \meta{file}]
% Displays the size in bytes of the given file or a blank line
% if the file doesn't exist or the file permissions prohibit this
% action.
%Some, but not all, \TeX\ formats provide
%\cs{pdffilesize}\marg{file}, which is more efficient than using
%the shell escape.
%\item[\shortarg{i} \meta{sep} \meta{dir} or \longarg{list}
%\meta{sep} \meta{dir}]
%List all files in the given directory with the output on a single
%line using \meta{sep} as the separator between entries. Note that
%the list doesn't include the full path, just the file names.
%\item[\shortarg{f} \meta{sep} \meta{regex} \meta{dir} or
%\longarg{filterlist} \meta{sep} \meta{regex} \meta{dir}]
%Like the above but only lists those files whose name matches the
%regular expression given in \meta{regex}. Note that this tests for a
%complete match on the file name (not including path). For example,
%if \meta{regexp} is \texttt{foo.*}, it will only match files whose
%name starts with \texttt{foo} (for example, \texttt{foobar} will
%match but \texttt{barfoo} won't).
%\item[\shortarg{u} \meta{file} or \longarg{uri} \meta{file}]
%Displays the URI of the given file or a blank line if the file
%doesn't exist or the file permissions prohibit this action.
%Note that this may include in a percent character in the result. The \TeX\ 
%command \cs{TeXOSQuery} protects against this by changing the
%category code, but if you explicitly call \texttt{texosquery}
%using the shell escape, you'll need to be careful of this.
%\item[\shortarg{p} \meta{file} or \longarg{path} \meta{file}]
%Displays the canonical path of the given file or a blank line if
%the file doesn't exist or the file permissions prohibit this
%action.
%\item[\shortarg{e} \meta{file} or \longarg{dirname} \meta{file}]
%(New to v1.1.)
%Displays the canonical path of the given file's parent (that is, the
%directory containing \meta{file}) or a blank line if
%the file doesn't exist or the file permissions prohibit this
%action. Note that this is different to the Unix-like 
%\texttt{dirname} command, which will return a relative path if
%\meta{file} isn't an absolute path.
%\item[\shortarg{h} or \longarg{help}] Displays help message and
%exits.
%\item[\shortarg{v} or \longarg{version}] Displays version
%information and exits.
%\end{description}
%
%If multiple options are given, they will be processed in the
%order specified in the command line invocation. Each result will
%be displayed on a separate line. As from v1.1, if there are
%multiple actions, each result will be grouped. This makes it easier
%to process the results in \TeX. For example:
%\begin{verbatim}
%texosquery -l
%\end{verbatim}
%This just produces (for me):
%\begin{verbatim}
%en-GB.utf8
%\end{verbatim}
%whereas
%\begin{verbatim}
%texosquery -l -n
%\end{verbatim}
%produces:
%\begin{verbatim}
%{en-GB.utf8}
%{D:20160714112732+01'00'}
%\end{verbatim}
%Note that unavailable information will produce an empty group.
%For example (assuming \texttt{nofile} doesn't exist):
%\begin{verbatim}
%texosquery -l -d nofile -n
%\end{verbatim}
%produces:
%\begin{verbatim}
%{en-GB.utf8}
%{}
%{D:20160714112732+01'00'}
%\end{verbatim}
%whereas 
%\begin{verbatim}
%texosquery -d nofile
%\end{verbatim}
%just displays an empty line.
%
%\section{texosquery.tex: generic \TeX\ code}
%You can run \texttt{texosquery} directly from \TeX's shell escape.
%For example:
%\begin{verbatim}
%\input|"texosquery --locale"
%\end{verbatim}
%However, \texttt{texosquery.tex} provides generic \TeX\ code to do
%this for you and store the result in a control sequence.
%
%Plain \TeX\ users can input this file through the usual \cs{input}
%method:
%\begin{verbatim}
%\input texosquery
%\end{verbatim}
%\LaTeX\ users may also simply input this file:
%\begin{verbatim}
%\input{texosquery}
%\end{verbatim}
%but may prefer the standard package approach:
%\begin{verbatim}
%\usepackage{texosquery}
%\end{verbatim}
%
%\begin{important}
%The commands described below are all fragile.
%\end{important}
%
%The basic command to run \texttt{texosquery} and capture its output
%in a control sequence is:
%\DescribeMacro\TeXOSQuery
%\begin{flushleft}\ttfamily
%\cs{TeXOSQuery}\marg{cs}\marg{args}
%\end{flushleft}
%where \meta{cs} is the control sequence in which to store the
%result and \meta{args} are the command line arguments to pass to 
%\texttt{texosquery}. This first locally changes the category code
%of some problematic characters that may appear in the result.
%The only special characters that aren't changed are the backslash
%\verb|\|, curly braces \verb|{| and \verb|}|, and hash \verb|#|.
%(\texttt{texosquery.jar} will replace \verb|#| with \verb|\#| in places
%where it might possibly occur in the result, but in general it's
%best to avoid these characters in file names.)
%
%There are some short cut commands for convenience, described below.
%If any of these commands cause an error message in the form:
%\begin{verbatim}
%I can't find file `|texosquery'.
%\end{verbatim}
%then check that you have the shell escape on. If the error 
%persists with the shell escape enabled
%(and not restricted), check that \texttt{texosquery} is on 
%your system's path. To do this, open a command prompt or terminal and
%type \texttt{texosquery}. If it isn't installed correctly, there will be a
%message like:
%\begin{verbatim}
%`texosquery' is not recognised
%\end{verbatim}
%or
%\begin{verbatim}
%texosquery: command not found
%\end{verbatim}
%If this happens, check the installation. (Instructions are
%in the accompanying \texttt{README.md} file.)
%
%Dry run mode is determined by the conditional
%\DescribeMacro\ifTeXOSQueryDryRun
%\begin{verbatim}
%\ifTeXOSQueryDryRun
%\end{verbatim}
%If true, the shell escape won't be used and the requested command
%invocation will be printed in the transcript file prefixed with
%\begin{verbatim}
%TeXOSQuery:
%\end{verbatim}
%(the control sequence \meta{cs} will be set to empty).
%This conditional will automatically be switched on unless
%\cs{shellescape} or \cs{pdfshellescape} is 1. (If
%\texttt{texosquery.jar} is later allowed on the restricted list,
%newer versions may change this default.)
%
%If multiple queries are required, it's more efficient to
%perform them all in one go. For example:
%\begin{verbatim}
%\TeXOSQuery{\result}{-l -n -o}
%
%\def\parseresult#1#2#3{%
%  Locale: #1. Now: #2. OS: #3.%
%}
%
%\ifx\result\empty
%  Query failed.
%\else
%  \expandafter\parseresult\result
%\fi
%\end{verbatim}
%(Make sure you have at least v1.1 for this to work correctly.)
%
%The locale (\shortarg{l} or \longarg{locale-lcs}) information can be
%obtained using:
%\DescribeMacro\TeXOSQueryLocale
%\begin{flushleft}\ttfamily
%\cs{TeXOSQueryLocale}\marg{cs}
%\end{flushleft}
%Note that this uses the lower case codeset form, which has a better
%chance of matching the encoding names used by the \texttt{inputenc}
%package. If you want the unprocessed codeset name, you can do:
%\begin{flushleft}\ttfamily
%\cs{TeXOSQuery}\marg{cs}\{\shortarg{L}\}
%\end{flushleft}
%
%The BCP 47 language tag (\shortarg{b} or \longarg{bcp47}) can be obtained using:
%\DescribeMacro\TeXOSQueryLangTag
%\begin{flushleft}\ttfamily
%\cs{TeXOSQueryLangTag}\marg{cs}
%\end{flushleft}
%
%The current working directory (\shortarg{c} or \longarg{cwd}) can
%be obtained using:
%\DescribeMacro\TeXOSQueryCwd
%\begin{flushleft}\ttfamily
%\cs{TeXOSQueryCwd}\marg{cs}
%\end{flushleft}
%
%The home directory (\shortarg{m} or \longarg{userhome}) can
%be obtained using:
%\DescribeMacro\TeXOSQueryHome
%\begin{flushleft}\ttfamily
%\cs{TeXOSQueryHome}\marg{cs}
%\end{flushleft}
%
%The temporary directory (\shortarg{t} or \longarg{tmpdir}) can
%be obtained using:
%\DescribeMacro\TeXOSQueryTmpDir
%\begin{flushleft}\ttfamily
%\cs{TeXOSQueryTmpDir}\marg{cs}
%\end{flushleft}
%
%The OS name (\shortarg{o} or \longarg{osname}) can
%be obtained using:
%\DescribeMacro\TeXOSQueryName
%\begin{flushleft}\ttfamily
%\cs{TeXOSQueryName}\marg{cs}
%\end{flushleft}
%
%The OS version (\shortarg{r} or \longarg{osversion}) can
%be obtained using:
%\DescribeMacro\TeXOSQueryVersion
%\begin{flushleft}\ttfamily
%\cs{TeXOSQueryVersion}\marg{cs}
%\end{flushleft}
%
%The OS architecture (\shortarg{a} or \longarg{osarch}) can
%be obtained using:
%\DescribeMacro\TeXOSQueryArch
%\begin{flushleft}\ttfamily
%\cs{TeXOSQueryArch}\marg{cs}
%\end{flushleft}
%
%The current date-time stamp in PDF format (\shortarg{n} or
%\longarg{pdfnow}) can be obtained using:
%\DescribeMacro\TeXOSQueryNow
%\begin{flushleft}\ttfamily
%\cs{TeXOSQueryNow}\marg{cs}
%\end{flushleft}
%This is provided for the benefit of users who don't have
%\cs{pdfcreationdate} defined by their \TeX\ format (for example,
%\XeTeX). As from v1.1, this ensures that the initial \texttt{D} has
%category code 12 (which won't happen if \cs{TeXOSQuery} is used
%explicitly).
%
%\begin{important}
%The remaining commands all require extra arguments after the
%relevant switch. These are automatically enclosed in single-quotes
%to protect any spaces. If the argument actually contains any single-quote
%characters, make sure you use \verb|\string\'| to prevent
%interference. \emph{However, in general it's a very bad idea to use
%quotes as part of a file name (rather than using them as
%delimiters).}
%Since the file name may need to be obtained from \cs{jobname},
%which sometimes includes double-quotes, the first double-quote pair
%found is stripped in file name arguments. Any other double-quotes
%will need to be protected in the same manner as single-quotes
%(but, again, this shouldn't be an issue if you use a safe
%file naming scheme). Any paths should use a
%forward slash for the directory divider.
%\end{important}
%
%The modification date-time stamp in PDF format for a file
%(\shortarg{d} or \longarg{pdfdate}) can be obtained using:
%\DescribeMacro\TeXOSQueryFileDate
%\begin{flushleft}\ttfamily
%\cs{TeXOSQueryFileDate}\marg{cs}\marg{filename}
%\end{flushleft}
%where \meta{filename} is the name of the file.
%This is provided for the benefit of users who don't have
%\cs{pdffilemoddate} defined by their \TeX\ format. As from v1.1,
%this ensures that the initial \texttt{D} has category code 12 (which 
%won't happen if \cs{TeXOSQuery} is used explicitly).
%
%The size in bytes of a file
%(\shortarg{s} or \longarg{filesize}) can be obtained using:
%\DescribeMacro\TeXOSQueryFileSize
%\begin{flushleft}\ttfamily
%\cs{TeXOSQueryFileSize}\marg{cs}\marg{filename}
%\end{flushleft}
%where \meta{filename} is the name of the file.
%This is provided for the benefit of users who don't have
%\cs{pdffilesize} defined by their \TeX\ format.
%
%The list of files in a given directory
%(\shortarg{i} or \longarg{list}) can be obtained using:
%\DescribeMacro\TeXOSQueryFileList
%\begin{flushleft}\ttfamily
%\cs{TeXOSQueryFileList}\marg{cs}\marg{sep}\marg{dir}
%\end{flushleft}
%where \meta{sep} is the separator and \meta{dir} is the directory name.
%For example:
%\begin{verbatim}
%\TeXOSQueryFileList{\result}{,}{.}
%\end{verbatim}
%will store a comma-separated list of all the files contained in the
%current directory in the control sequence \cs{result}.
%
%A filtered list of files in a given directory
%(\shortarg{f} or \longarg{filterlist}) can be obtained using:
%\DescribeMacro\TeXOSQueryFilterFileList
%\begin{flushleft}\ttfamily
%\cs{TeXOSQueryFilterFileList}\marg{cs}\marg{sep}\marg{regex}\marg{dir}
%\end{flushleft}
%where \meta{regex} is a regular expression. \emph{Take care of any backslashes
%in the regular expression!}
%For example, to list only those files that have an extension:
%\begin{verbatim}
%\TeXOSQueryFilterFileList{\result}{,}{.+\string\..*}{.}
%\end{verbatim}
%Note the use of \verb|\string\.| to ensure that \cs{.} isn't
%interpreted as a command. Another example, list only \texttt{.png}
%and \texttt{.jpg} files in the directory called \texttt{images}:
%\begin{verbatim}
%\TeXOSQueryFilterFileList{\result}{,}{.+\string\.(jpg|png)}{images}
%\end{verbatim}
%
%The URI of a file
%(\shortarg{u} or \longarg{uri}) can be obtained using:
%\DescribeMacro\TeXOSQueryFileURI
%\begin{flushleft}\ttfamily
%\cs{TeXOSQueryFileURI}\marg{cs}\marg{filename}
%\end{flushleft}
%where \meta{filename} is the name of the file.
%(Any percent symbols \verb|%| contained in the URI will have
%their category code set to 12.)
%
%The canonical path of a file
%(\shortarg{p} or \longarg{path}) can be obtained using:
%\DescribeMacro\TeXOSQueryFilePath
%\begin{flushleft}\ttfamily
%\cs{TeXOSQueryFilePath}\marg{cs}\marg{filename}
%\end{flushleft}
%where \meta{filename} is the name of the file.
%
%The canonical path of a file's parent
%(\shortarg{e} or \longarg{dirname}) can be obtained using:
%\DescribeMacro\TeXOSQueryDirName
%\begin{flushleft}\ttfamily
%\cs{TeXOSQueryDirName}\marg{cs}\marg{filename}
%\end{flushleft}
%where \meta{filename} is the name of the file.
%
%\StopEventually{\phantomsection
%\addcontentsline{toc}{section}{Change History}%
%\PrintChanges
%\addcontentsline{toc}{section}{\indexname}%
%\PrintIndex
%}
%
%\section{The Code}
%\changes{1.0}{2016-07-08}{Initial release}
%\subsection{Generic \TeX\ Code}
%\iffalse
%    \begin{macrocode}
%<*texosquery.tex>
%    \end{macrocode}
%\fi
%Change category code of \texttt{@} if necessary.
%    \begin{macrocode}
\ifnum\catcode`\@=11\relax
  \def\@texosquery@restore@at{}%
\else
  \expandafter\edef\csname @texosquery@restore@at\endcsname{%
    \noexpand\catcode`\noexpand\@=\number\catcode`\@\relax
  }%
 \catcode`\@=11\relax
\fi
%    \end{macrocode}
% Check if already loaded.
%    \begin{macrocode}
\ifx\TeXOSQuery\undefined
\else
  \@texosquery@restore@at
  \expandafter\endinput
\fi
%    \end{macrocode}
%\begin{macro}{\TeXOSInvokerName}
%The name of the \texttt{texosquery} application.
%    \begin{macrocode}
\def\TeXOSInvokerName{texosquery}
%    \end{macrocode}
%\end{macro}
%
%\begin{macro}{\TeXOSQueryInvoker}
%If we're using \LaTeX, we'll need to use \cs{@@input} rather
%than \cs{input}.
%    \begin{macrocode}
\ifx\@@input\undefined
  \def\TeXOSQueryInvoker#1{\input|"\TeXOSInvokerName\space#1" }
\else
  \def\TeXOSQueryInvoker#1{\@@input|"\TeXOSInvokerName\space#1" }
\fi
%    \end{macrocode}
%\end{macro}
%
%\begin{macro}{\ifTeXOSQueryDryRun}
%Provide a dry-run mode.
%\changes{1.1}{2016-07-14}{dry run mode only false by default if with
%unrestricted mode}
%    \begin{macrocode}
\newif\ifTeXOSQueryDryRun
\TeXOSQueryDryRuntrue
%    \end{macrocode}
%\end{macro}
%
%If shell escape is unrestricted, automatically switch off dry-run mode.
%    \begin{macrocode}
\ifx\shellescape\undefined
  \ifx\pdfshellescape\undefined
  \else
    \ifnum\pdfshellescape=1\relax
      \TeXOSQueryDryRunfalse
    \fi
  \fi
\else
  \ifnum\shellescape=1\relax
    \TeXOSQueryDryRunfalse
  \fi
\fi
%    \end{macrocode}
%
%\begin{macro}{\TeXOSQuery}
%Use \texttt{texosquery} with the option given in the second
%argument and store the result in control sequence given in the
%first argument.
%    \begin{macrocode}
\def\TeXOSQuery#1#2{%
  \ifTeXOSQueryDryRun
    \begingroup
      \newlinechar=`\^^J
      \message{^^JTeXOSQuery: \TeXOSInvokerName\space#2^^J}%
    \endgroup
    \def#1{}%
  \else
    \begingroup
    \endlinechar=-1\relax
%    \end{macrocode}
%Just in case the result contains any awkward characters that have
%some special meaning to \TeX. (Can't really protect against hash,
%backslash or curly braces, but they're unlikely to occur unless
%the user has a very wacky and inappropriate file naming scheme.)
%    \begin{macrocode}
    \catcode`\-=12\relax
    \catcode`\_=12\relax
    \catcode`\^=12\relax
    \catcode`\~=12\relax
    \catcode`\$=12\relax
    \catcode`\&=12\relax
    \catcode`\"=12\relax
    \catcode`\'=12\relax
    \catcode`\.=12\relax
    \catcode`\/=12\relax
    \catcode`\:=12\relax
    \catcode`\;=12\relax
    \catcode`\%=12\relax
    \everyeof{\noexpand}\relax
    \edef\x{\endgroup\def\noexpand#1{\TeXOSQueryInvoker{#2}}}\x
  \fi
}
%    \end{macrocode}
%\end{macro}
%
%Now for some convenient shortcuts so the user doesn't have to
%remember the command line options. \cs{string} is used in
%case the hyphen character has been made active.
%\begin{macro}{\TeXOSQueryLocale}
%Query the locale and store the result in the control sequence
%provided in the argument.
%    \begin{macrocode}
\def\TeXOSQueryLocale#1{\TeXOSQuery{#1}{\string-l}}
%    \end{macrocode}
%\end{macro}
%
%\begin{macro}{\TeXOSQueryLangTag}
%Query the language tag and store the result in the control sequence
%provided in the argument.
%\changes{1.2}{??}{new}
%    \begin{macrocode}
\def\TeXOSQueryLangTag#1{\TeXOSQuery{#1}{\string-b}}
%    \end{macrocode}
%\end{macro}
%
%\begin{macro}{\TeXOSQueryCwd}
%Query the current working directory.
%    \begin{macrocode}
\def\TeXOSQueryCwd#1{\TeXOSQuery{#1}{\string-c}}
%    \end{macrocode}
%\end{macro}
%
%\begin{macro}{\TeXOSQueryHome}
%Query the user's home directory.
%    \begin{macrocode}
\def\TeXOSQueryHome#1{\TeXOSQuery{#1}{\string-m}}
%    \end{macrocode}
%\end{macro}
%
%\begin{macro}{\TeXOSQueryTmpDir}
%Query the temporary directory.
%    \begin{macrocode}
\def\TeXOSQueryTmpDir#1{\TeXOSQuery{#1}{\string-t}}
%    \end{macrocode}
%\end{macro}
%
%\begin{macro}{\TeXOSQueryVersion}
%Query the operating system version.
%    \begin{macrocode}
\def\TeXOSQueryVersion#1{\TeXOSQuery{#1}{\string-r}}
%    \end{macrocode}
%\end{macro}
%
%\begin{macro}{\TeXOSQueryArch}
%Query the operating system architecture.
%    \begin{macrocode}
\def\TeXOSQueryArch#1{\TeXOSQuery{#1}{\string-a}}
%    \end{macrocode}
%\end{macro}
%
%\begin{macro}{\TeXOSQueryName}
%Query the operating system name.
%    \begin{macrocode}
\def\TeXOSQueryName#1{\TeXOSQuery{#1}{\string-o}}
%    \end{macrocode}
%\end{macro}
%
%\begin{macro}{\TeXOSQueryNow}
%\changes{1.1}{2016-07-14}{changed catcode of D to 12}
%Query the current time stamp.
%    \begin{macrocode}
\def\TeXOSQueryNow#1{%
%    \end{macrocode}
% The \texttt{D} needs category code 12. (Don't need to worry about
% \texttt{Z} as \texttt{texosquery.jar} uses \texttt{+00'00'} for
% UTC+0.) This change can't be done with the other catcode changes
% in \cs{TeXOSQuery}, as this is only appropriate for the PDF dates.
% Save and restore the catcode rather than fiddle around with
% scoping.
%    \begin{macrocode}
  \edef\@texosquery@restore@D{%
    \noexpand\catcode`\noexpand\D=\the\catcode`\D\relax}%
  \catcode`\D=12\relax
  \TeXOSQuery{#1}{\string-n}%
  \@texosquery@restore@D
}
%    \end{macrocode}
%\end{macro}
%
%If the file name is supplied using \cs{jobname} it may have
%double-quotes which will interfere with things.
%\begin{macro}{\texosquerystripquotes}
%    \begin{macrocode}
\def\texosquerystripquotes#1{%
  \@texosquery@stripquotes#1\@mid@texosquery@stripquotes
    "\relax"\relax\@end@texosquery@stripquotes
}
\def\@texosquery@stripquotes#1"#2"{%
  \@@texosquery@stripquotes#1#2%
}
\def\@@texosquery@stripquotes#1\@mid@texosquery@stripquotes#2\@end@texosquery@stripquotes{%
  #1%
}
%    \end{macrocode}
%\end{macro}
%
%\begin{macro}{\TeXOSQueryFileDate}
%\changes{1.1}{2016-07-14}{changed catcode of D to 12}
%Query the time stamp of the file given in the second argument.
%    \begin{macrocode}
\def\TeXOSQueryFileDate#1#2{%
  \edef\@texosquery@restore@D{%
    \noexpand\catcode`\noexpand\D=\the\catcode`\D\relax}%
  \catcode`\D=12\relax
  \TeXOSQuery{#1}{\string-d \string'\texosquerystripquotes{#2}\string'}%
  \@texosquery@restore@D
}
%    \end{macrocode}
%\end{macro}
%
%\begin{macro}{\TeXOSQueryFileSize}
%Query the size of the file given in the second argument.
%    \begin{macrocode}
\def\TeXOSQueryFileSize#1#2{\TeXOSQuery{#1}{\string-s 
 \string'\texosquerystripquotes{#2}\string'}}
%    \end{macrocode}
%\end{macro}
%
%\begin{macro}{\TeXOSQueryFileList}
%List all files in the directory given in the third argument,
%separated by the second argument.
%    \begin{macrocode}
\def\TeXOSQueryFileList#1#2#3{\TeXOSQuery{#1}{%
 \string-i \string'#2\string' 
 \string'\texosquerystripquotes{#3}\string'}}
%    \end{macrocode}
%\end{macro}
%
%\begin{macro}{\TeXOSQueryFilterFileList}
%Filtered list files in the directory given in the fourth argument,
%separated by the second argument. The third argument is the regular
%expression used to filter the list. \emph{Take care of backslashes
%in the regular expression!}
%    \begin{macrocode}
\def\TeXOSQueryFilterFileList#1#2#3#4{\TeXOSQuery{#1}{%
 \string-f \string'#2\string' \string'#3\string' 
 \string'\texosquerystripquotes{#4}\string'}}
%    \end{macrocode}
%\end{macro}
%
%\begin{macro}{\TeXOSQueryFileURI}
%Get the URI of the file given in the second argument.
%    \begin{macrocode}
\def\TeXOSQueryFileURI#1#2{\TeXOSQuery{#1}{\string-u 
 \string'\texosquerystripquotes{#2}\string'}}
%    \end{macrocode}
%\end{macro}
%
%\begin{macro}{\TeXOSQueryFilePath}
%Get the canonical path of the file given in the second argument.
%    \begin{macrocode}
\def\TeXOSQueryFilePath#1#2{\TeXOSQuery{#1}{\string-p 
 \string'\texosquerystripquotes{#2}\string'}}
%    \end{macrocode}
%\end{macro}
%
%\begin{macro}{\TeXOSQueryDirName}
%\changes{1.1}{2016-07-14}{new}
%Get the canonical path of the directory containing the file given 
%in the second argument.
%    \begin{macrocode}
\def\TeXOSQueryDirName#1#2{\TeXOSQuery{#1}{\string-e 
 \string'\texosquerystripquotes{#2}\string'}}
%    \end{macrocode}
%\end{macro}
%
%All done.
%Restore the category code of \texttt{@}:
%    \begin{macrocode}
\@texosquery@restore@at
%    \end{macrocode}
%\iffalse
%    \begin{macrocode}
%</texosquery.tex>
%    \end{macrocode}
%\fi
%
%\subsection{\LaTeX\ Code}
%This is just a simple wrapper for \texttt{texosquery.tex}
%so that it can be loaded using \LaTeX's standard \cs{usepackage}
%method.
%\iffalse
%    \begin{macrocode}
%<*texosquery.sty>
%    \end{macrocode}
%\fi
%Identify package:
%    \begin{macrocode}
\NeedsTeXFormat{LaTeX2e}
\ProvidesPackage{texosquery}[2016/07/14 v1.1 (NLCT)]
%    \end{macrocode}
%Load \texttt{texosquery.tex}:
%    \begin{macrocode}
\input{texosquery}
%    \end{macrocode}
%That's it!
%\iffalse
%    \begin{macrocode}
%</texosquery.sty>
%    \end{macrocode}
%\fi
%
%\Finale
\endinput

%    \end{macrocode}
%That's it!
%\iffalse
%    \begin{macrocode}
%</texosquery.sty>
%    \end{macrocode}
%\fi
%
%\Finale
\endinput

%    \end{macrocode}
%That's it!
%\iffalse
%    \begin{macrocode}
%</texosquery.sty>
%    \end{macrocode}
%\fi
%
%\Finale
\endinput

%\end{verbatim}
%but may prefer the standard package approach:
%\begin{verbatim}
%\usepackage{texosquery}
%\end{verbatim}
%
%\begin{important}
%The commands described below are all fragile.
%\end{important}
%
%The basic command to run \texttt{texosquery} and capture its output
%in a control sequence is:
%\DescribeMacro\TeXOSQuery
%\begin{flushleft}\ttfamily
%\cs{TeXOSQuery}\marg{cs}\marg{args}
%\end{flushleft}
%where \meta{cs} is the control sequence in which to store the
%result and \meta{args} are the command line arguments to pass to 
%\texttt{texosquery}. This first locally changes the category code
%of some problematic characters that may appear in the result.
%The only special characters that aren't changed are the backslash
%\verb|\|, curly braces \verb|{| and \verb|}|, and hash \verb|#|.
%(\texttt{texosquery.jar} will replace \verb|#| with \verb|\#| in places
%where it might possibly occur in the result, but in general it's
%best to avoid these characters in file names.)
%
%There are some short cut commands for convenience, described below.
%If any of these commands cause an error message in the form:
%\begin{verbatim}
%I can't find file `|texosquery'.
%\end{verbatim}
%then check that you have the shell escape on. If the error 
%persists with the shell escape enabled
%(and not restricted), check that \texttt{texosquery} is on 
%your system's path. To do this, open a command prompt or terminal and
%type \texttt{texosquery}. If it isn't installed correctly, there will be a
%message like:
%\begin{verbatim}
%`texosquery' is not recognised
%\end{verbatim}
%or
%\begin{verbatim}
%texosquery: command not found
%\end{verbatim}
%If this happens, check the installation. (Instructions are
%in the accompanying \texttt{README.md} file.)
%
%Dry run mode is determined by the conditional
%\DescribeMacro\ifTeXOSQueryDryRun
%\begin{verbatim}
%\ifTeXOSQueryDryRun
%\end{verbatim}
%If true, the shell escape won't be used and the requested command
%invocation will be printed in the transcript file prefixed with
%\begin{verbatim}
%TeXOSQuery:
%\end{verbatim}
%(the control sequence \meta{cs} will be set to empty).
%This conditional will automatically be switched on unless
%\cs{shellescape} or \cs{pdfshellescape} is 1. (If
%\texttt{texosquery.jar} is later allowed on the restricted list,
%newer versions may change this default.)
%
%If multiple queries are required, it's more efficient to
%perform them all in one go. For example:
%\begin{verbatim}
%\TeXOSQuery{\result}{-l -n -o}
%
%\def\parseresult#1#2#3{%
%  Locale: #1. Now: #2. OS: #3.%
%}
%
%\ifx\result\empty
%  Query failed.
%\else
%  \expandafter\parseresult\result
%\fi
%\end{verbatim}
%(Make sure you have at least v1.1 for this to work correctly.)
%
%The locale (\shortarg{l} or \longarg{locale-lcs}) information can be
%obtained using:
%\DescribeMacro\TeXOSQueryLocale
%\begin{flushleft}\ttfamily
%\cs{TeXOSQueryLocale}\marg{cs}
%\end{flushleft}
%Note that this uses the lower case codeset form, which has a better
%chance of matching the encoding names used by the \texttt{inputenc}
%package. If you want the unprocessed codeset name, you can do:
%\begin{flushleft}\ttfamily
%\cs{TeXOSQuery}\marg{cs}\{\shortarg{L}\}
%\end{flushleft}
%
%The BCP 47 language tag (\shortarg{b} or \longarg{bcp47}) can be obtained using:
%\DescribeMacro\TeXOSQueryLangTag
%\begin{flushleft}\ttfamily
%\cs{TeXOSQueryLangTag}\marg{cs}
%\end{flushleft}
%
%The current working directory (\shortarg{c} or \longarg{cwd}) can
%be obtained using:
%\DescribeMacro\TeXOSQueryCwd
%\begin{flushleft}\ttfamily
%\cs{TeXOSQueryCwd}\marg{cs}
%\end{flushleft}
%
%The home directory (\shortarg{m} or \longarg{userhome}) can
%be obtained using:
%\DescribeMacro\TeXOSQueryHome
%\begin{flushleft}\ttfamily
%\cs{TeXOSQueryHome}\marg{cs}
%\end{flushleft}
%
%The temporary directory (\shortarg{t} or \longarg{tmpdir}) can
%be obtained using:
%\DescribeMacro\TeXOSQueryTmpDir
%\begin{flushleft}\ttfamily
%\cs{TeXOSQueryTmpDir}\marg{cs}
%\end{flushleft}
%
%The OS name (\shortarg{o} or \longarg{osname}) can
%be obtained using:
%\DescribeMacro\TeXOSQueryName
%\begin{flushleft}\ttfamily
%\cs{TeXOSQueryName}\marg{cs}
%\end{flushleft}
%
%The OS version (\shortarg{r} or \longarg{osversion}) can
%be obtained using:
%\DescribeMacro\TeXOSQueryVersion
%\begin{flushleft}\ttfamily
%\cs{TeXOSQueryVersion}\marg{cs}
%\end{flushleft}
%
%The OS architecture (\shortarg{a} or \longarg{osarch}) can
%be obtained using:
%\DescribeMacro\TeXOSQueryArch
%\begin{flushleft}\ttfamily
%\cs{TeXOSQueryArch}\marg{cs}
%\end{flushleft}
%
%The current date-time stamp in PDF format (\shortarg{n} or
%\longarg{pdfnow}) can be obtained using:
%\DescribeMacro\TeXOSQueryNow
%\begin{flushleft}\ttfamily
%\cs{TeXOSQueryNow}\marg{cs}
%\end{flushleft}
%This is provided for the benefit of users who don't have
%\cs{pdfcreationdate} defined by their \TeX\ format (for example,
%\XeTeX). As from v1.1, this ensures that the initial \texttt{D} has
%category code 12 (which won't happen if \cs{TeXOSQuery} is used
%explicitly).
%
%\begin{important}
%The remaining commands all require extra arguments after the
%relevant switch. These are automatically enclosed in single-quotes
%to protect any spaces. If the argument actually contains any single-quote
%characters, make sure you use \verb|\string\'| to prevent
%interference. \emph{However, in general it's a very bad idea to use
%quotes as part of a file name (rather than using them as
%delimiters).}
%Since the file name may need to be obtained from \cs{jobname},
%which sometimes includes double-quotes, the first double-quote pair
%found is stripped in file name arguments. Any other double-quotes
%will need to be protected in the same manner as single-quotes
%(but, again, this shouldn't be an issue if you use a safe
%file naming scheme). Any paths should use a
%forward slash for the directory divider.
%\end{important}
%
%The modification date-time stamp in PDF format for a file
%(\shortarg{d} or \longarg{pdfdate}) can be obtained using:
%\DescribeMacro\TeXOSQueryFileDate
%\begin{flushleft}\ttfamily
%\cs{TeXOSQueryFileDate}\marg{cs}\marg{filename}
%\end{flushleft}
%where \meta{filename} is the name of the file.
%This is provided for the benefit of users who don't have
%\cs{pdffilemoddate} defined by their \TeX\ format. As from v1.1,
%this ensures that the initial \texttt{D} has category code 12 (which 
%won't happen if \cs{TeXOSQuery} is used explicitly).
%
%The size in bytes of a file
%(\shortarg{s} or \longarg{filesize}) can be obtained using:
%\DescribeMacro\TeXOSQueryFileSize
%\begin{flushleft}\ttfamily
%\cs{TeXOSQueryFileSize}\marg{cs}\marg{filename}
%\end{flushleft}
%where \meta{filename} is the name of the file.
%This is provided for the benefit of users who don't have
%\cs{pdffilesize} defined by their \TeX\ format.
%
%The list of files in a given directory
%(\shortarg{i} or \longarg{list}) can be obtained using:
%\DescribeMacro\TeXOSQueryFileList
%\begin{flushleft}\ttfamily
%\cs{TeXOSQueryFileList}\marg{cs}\marg{sep}\marg{dir}
%\end{flushleft}
%where \meta{sep} is the separator and \meta{dir} is the directory name.
%For example:
%\begin{verbatim}
%\TeXOSQueryFileList{\result}{,}{.}
%\end{verbatim}
%will store a comma-separated list of all the files contained in the
%current directory in the control sequence \cs{result}.
%
%A filtered list of files in a given directory
%(\shortarg{f} or \longarg{filterlist}) can be obtained using:
%\DescribeMacro\TeXOSQueryFilterFileList
%\begin{flushleft}\ttfamily
%\cs{TeXOSQueryFilterFileList}\marg{cs}\marg{sep}\marg{regex}\marg{dir}
%\end{flushleft}
%where \meta{regex} is a regular expression. \emph{Take care of any backslashes
%in the regular expression!}
%For example, to list only those files that have an extension:
%\begin{verbatim}
%\TeXOSQueryFilterFileList{\result}{,}{.+\string\..*}{.}
%\end{verbatim}
%Note the use of \verb|\string\.| to ensure that \cs{.} isn't
%interpreted as a command. Another example, list only \texttt{.png}
%and \texttt{.jpg} files in the directory called \texttt{images}:
%\begin{verbatim}
%\TeXOSQueryFilterFileList{\result}{,}{.+\string\.(jpg|png)}{images}
%\end{verbatim}
%
%The URI of a file
%(\shortarg{u} or \longarg{uri}) can be obtained using:
%\DescribeMacro\TeXOSQueryFileURI
%\begin{flushleft}\ttfamily
%\cs{TeXOSQueryFileURI}\marg{cs}\marg{filename}
%\end{flushleft}
%where \meta{filename} is the name of the file.
%(Any percent symbols \verb|%| contained in the URI will have
%their category code set to 12.)
%
%The canonical path of a file
%(\shortarg{p} or \longarg{path}) can be obtained using:
%\DescribeMacro\TeXOSQueryFilePath
%\begin{flushleft}\ttfamily
%\cs{TeXOSQueryFilePath}\marg{cs}\marg{filename}
%\end{flushleft}
%where \meta{filename} is the name of the file.
%
%The canonical path of a file's parent
%(\shortarg{e} or \longarg{dirname}) can be obtained using:
%\DescribeMacro\TeXOSQueryDirName
%\begin{flushleft}\ttfamily
%\cs{TeXOSQueryDirName}\marg{cs}\marg{filename}
%\end{flushleft}
%where \meta{filename} is the name of the file.
%
%\StopEventually{\phantomsection
%\addcontentsline{toc}{section}{Change History}%
%\PrintChanges
%\addcontentsline{toc}{section}{\indexname}%
%\PrintIndex
%}
%
%\section{The Code}
%\changes{1.0}{2016-07-08}{Initial release}
%\subsection{Generic \TeX\ Code}
%\iffalse
%    \begin{macrocode}
%<*texosquery.tex>
%    \end{macrocode}
%\fi
%Change category code of \texttt{@} if necessary.
%    \begin{macrocode}
\ifnum\catcode`\@=11\relax
  \def\@texosquery@restore@at{}%
\else
  \expandafter\edef\csname @texosquery@restore@at\endcsname{%
    \noexpand\catcode`\noexpand\@=\number\catcode`\@\relax
  }%
 \catcode`\@=11\relax
\fi
%    \end{macrocode}
% Check if already loaded.
%    \begin{macrocode}
\ifx\TeXOSQuery\undefined
\else
  \@texosquery@restore@at
  \expandafter\endinput
\fi
%    \end{macrocode}
%\begin{macro}{\TeXOSInvokerName}
%The name of the \texttt{texosquery} application.
%    \begin{macrocode}
\def\TeXOSInvokerName{texosquery}
%    \end{macrocode}
%\end{macro}
%
%\begin{macro}{\TeXOSQueryInvoker}
%If we're using \LaTeX, we'll need to use \cs{@@input} rather
%than \cs{input}.
%    \begin{macrocode}
\ifx\@@input\undefined
  \def\TeXOSQueryInvoker#1{\input|"\TeXOSInvokerName\space#1" }
\else
  \def\TeXOSQueryInvoker#1{\@@input|"\TeXOSInvokerName\space#1" }
\fi
%    \end{macrocode}
%\end{macro}
%
%\begin{macro}{\ifTeXOSQueryDryRun}
%Provide a dry-run mode.
%\changes{1.1}{2016-07-14}{dry run mode only false by default if with
%unrestricted mode}
%    \begin{macrocode}
\newif\ifTeXOSQueryDryRun
\TeXOSQueryDryRuntrue
%    \end{macrocode}
%\end{macro}
%
%If shell escape is unrestricted, automatically switch off dry-run mode.
%    \begin{macrocode}
\ifx\shellescape\undefined
  \ifx\pdfshellescape\undefined
  \else
    \ifnum\pdfshellescape=1\relax
      \TeXOSQueryDryRunfalse
    \fi
  \fi
\else
  \ifnum\shellescape=1\relax
    \TeXOSQueryDryRunfalse
  \fi
\fi
%    \end{macrocode}
%
%\begin{macro}{\TeXOSQuery}
%Use \texttt{texosquery} with the option given in the second
%argument and store the result in control sequence given in the
%first argument.
%    \begin{macrocode}
\def\TeXOSQuery#1#2{%
  \ifTeXOSQueryDryRun
    \begingroup
      \newlinechar=`\^^J
      \message{^^JTeXOSQuery: \TeXOSInvokerName\space#2^^J}%
    \endgroup
    \def#1{}%
  \else
    \begingroup
    \endlinechar=-1\relax
%    \end{macrocode}
%Just in case the result contains any awkward characters that have
%some special meaning to \TeX. (Can't really protect against hash,
%backslash or curly braces, but they're unlikely to occur unless
%the user has a very wacky and inappropriate file naming scheme.)
%    \begin{macrocode}
    \catcode`\-=12\relax
    \catcode`\_=12\relax
    \catcode`\^=12\relax
    \catcode`\~=12\relax
    \catcode`\$=12\relax
    \catcode`\&=12\relax
    \catcode`\"=12\relax
    \catcode`\'=12\relax
    \catcode`\.=12\relax
    \catcode`\/=12\relax
    \catcode`\:=12\relax
    \catcode`\;=12\relax
    \catcode`\%=12\relax
    \everyeof{\noexpand}\relax
    \edef\x{\endgroup\def\noexpand#1{\TeXOSQueryInvoker{#2}}}\x
  \fi
}
%    \end{macrocode}
%\end{macro}
%
%Now for some convenient shortcuts so the user doesn't have to
%remember the command line options. \cs{string} is used in
%case the hyphen character has been made active.
%\begin{macro}{\TeXOSQueryLocale}
%Query the locale and store the result in the control sequence
%provided in the argument.
%    \begin{macrocode}
\def\TeXOSQueryLocale#1{\TeXOSQuery{#1}{\string-l}}
%    \end{macrocode}
%\end{macro}
%
%\begin{macro}{\TeXOSQueryLangTag}
%Query the language tag and store the result in the control sequence
%provided in the argument.
%\changes{1.2}{??}{new}
%    \begin{macrocode}
\def\TeXOSQueryLangTag#1{\TeXOSQuery{#1}{\string-b}}
%    \end{macrocode}
%\end{macro}
%
%\begin{macro}{\TeXOSQueryCwd}
%Query the current working directory.
%    \begin{macrocode}
\def\TeXOSQueryCwd#1{\TeXOSQuery{#1}{\string-c}}
%    \end{macrocode}
%\end{macro}
%
%\begin{macro}{\TeXOSQueryHome}
%Query the user's home directory.
%    \begin{macrocode}
\def\TeXOSQueryHome#1{\TeXOSQuery{#1}{\string-m}}
%    \end{macrocode}
%\end{macro}
%
%\begin{macro}{\TeXOSQueryTmpDir}
%Query the temporary directory.
%    \begin{macrocode}
\def\TeXOSQueryTmpDir#1{\TeXOSQuery{#1}{\string-t}}
%    \end{macrocode}
%\end{macro}
%
%\begin{macro}{\TeXOSQueryVersion}
%Query the operating system version.
%    \begin{macrocode}
\def\TeXOSQueryVersion#1{\TeXOSQuery{#1}{\string-r}}
%    \end{macrocode}
%\end{macro}
%
%\begin{macro}{\TeXOSQueryArch}
%Query the operating system architecture.
%    \begin{macrocode}
\def\TeXOSQueryArch#1{\TeXOSQuery{#1}{\string-a}}
%    \end{macrocode}
%\end{macro}
%
%\begin{macro}{\TeXOSQueryName}
%Query the operating system name.
%    \begin{macrocode}
\def\TeXOSQueryName#1{\TeXOSQuery{#1}{\string-o}}
%    \end{macrocode}
%\end{macro}
%
%\begin{macro}{\TeXOSQueryNow}
%\changes{1.1}{2016-07-14}{changed catcode of D to 12}
%Query the current time stamp.
%    \begin{macrocode}
\def\TeXOSQueryNow#1{%
%    \end{macrocode}
% The \texttt{D} needs category code 12. (Don't need to worry about
% \texttt{Z} as \texttt{texosquery.jar} uses \texttt{+00'00'} for
% UTC+0.) This change can't be done with the other catcode changes
% in \cs{TeXOSQuery}, as this is only appropriate for the PDF dates.
% Save and restore the catcode rather than fiddle around with
% scoping.
%    \begin{macrocode}
  \edef\@texosquery@restore@D{%
    \noexpand\catcode`\noexpand\D=\the\catcode`\D\relax}%
  \catcode`\D=12\relax
  \TeXOSQuery{#1}{\string-n}%
  \@texosquery@restore@D
}
%    \end{macrocode}
%\end{macro}
%
%If the file name is supplied using \cs{jobname} it may have
%double-quotes which will interfere with things.
%\begin{macro}{\texosquerystripquotes}
%    \begin{macrocode}
\def\texosquerystripquotes#1{%
  \@texosquery@stripquotes#1\@mid@texosquery@stripquotes
    "\relax"\relax\@end@texosquery@stripquotes
}
\def\@texosquery@stripquotes#1"#2"{%
  \@@texosquery@stripquotes#1#2%
}
\def\@@texosquery@stripquotes#1\@mid@texosquery@stripquotes#2\@end@texosquery@stripquotes{%
  #1%
}
%    \end{macrocode}
%\end{macro}
%
%\begin{macro}{\TeXOSQueryFileDate}
%\changes{1.1}{2016-07-14}{changed catcode of D to 12}
%Query the time stamp of the file given in the second argument.
%    \begin{macrocode}
\def\TeXOSQueryFileDate#1#2{%
  \edef\@texosquery@restore@D{%
    \noexpand\catcode`\noexpand\D=\the\catcode`\D\relax}%
  \catcode`\D=12\relax
  \TeXOSQuery{#1}{\string-d \string'\texosquerystripquotes{#2}\string'}%
  \@texosquery@restore@D
}
%    \end{macrocode}
%\end{macro}
%
%\begin{macro}{\TeXOSQueryFileSize}
%Query the size of the file given in the second argument.
%    \begin{macrocode}
\def\TeXOSQueryFileSize#1#2{\TeXOSQuery{#1}{\string-s 
 \string'\texosquerystripquotes{#2}\string'}}
%    \end{macrocode}
%\end{macro}
%
%\begin{macro}{\TeXOSQueryFileList}
%List all files in the directory given in the third argument,
%separated by the second argument.
%    \begin{macrocode}
\def\TeXOSQueryFileList#1#2#3{\TeXOSQuery{#1}{%
 \string-i \string'#2\string' 
 \string'\texosquerystripquotes{#3}\string'}}
%    \end{macrocode}
%\end{macro}
%
%\begin{macro}{\TeXOSQueryFilterFileList}
%Filtered list files in the directory given in the fourth argument,
%separated by the second argument. The third argument is the regular
%expression used to filter the list. \emph{Take care of backslashes
%in the regular expression!}
%    \begin{macrocode}
\def\TeXOSQueryFilterFileList#1#2#3#4{\TeXOSQuery{#1}{%
 \string-f \string'#2\string' \string'#3\string' 
 \string'\texosquerystripquotes{#4}\string'}}
%    \end{macrocode}
%\end{macro}
%
%\begin{macro}{\TeXOSQueryFileURI}
%Get the URI of the file given in the second argument.
%    \begin{macrocode}
\def\TeXOSQueryFileURI#1#2{\TeXOSQuery{#1}{\string-u 
 \string'\texosquerystripquotes{#2}\string'}}
%    \end{macrocode}
%\end{macro}
%
%\begin{macro}{\TeXOSQueryFilePath}
%Get the canonical path of the file given in the second argument.
%    \begin{macrocode}
\def\TeXOSQueryFilePath#1#2{\TeXOSQuery{#1}{\string-p 
 \string'\texosquerystripquotes{#2}\string'}}
%    \end{macrocode}
%\end{macro}
%
%\begin{macro}{\TeXOSQueryDirName}
%\changes{1.1}{2016-07-14}{new}
%Get the canonical path of the directory containing the file given 
%in the second argument.
%    \begin{macrocode}
\def\TeXOSQueryDirName#1#2{\TeXOSQuery{#1}{\string-e 
 \string'\texosquerystripquotes{#2}\string'}}
%    \end{macrocode}
%\end{macro}
%
%All done.
%Restore the category code of \texttt{@}:
%    \begin{macrocode}
\@texosquery@restore@at
%    \end{macrocode}
%\iffalse
%    \begin{macrocode}
%</texosquery.tex>
%    \end{macrocode}
%\fi
%
%\subsection{\LaTeX\ Code}
%This is just a simple wrapper for \texttt{texosquery.tex}
%so that it can be loaded using \LaTeX's standard \cs{usepackage}
%method.
%\iffalse
%    \begin{macrocode}
%<*texosquery.sty>
%    \end{macrocode}
%\fi
%Identify package:
%    \begin{macrocode}
\NeedsTeXFormat{LaTeX2e}
\ProvidesPackage{texosquery}[2016/07/14 v1.1 (NLCT)]
%    \end{macrocode}
%Load \texttt{texosquery.tex}:
%    \begin{macrocode}
%\iffalse
%<*package>
%% \CharacterTable
%%  {Upper-case    \A\B\C\D\E\F\G\H\I\J\K\L\M\N\O\P\Q\R\S\T\U\V\W\X\Y\Z
%%   Lower-case    \a\b\c\d\e\f\g\h\i\j\k\l\m\n\o\p\q\r\s\t\u\v\w\x\y\z
%%   Digits        \0\1\2\3\4\5\6\7\8\9
%%   Exclamation   \!     Double quote  \"     Hash (number) \#
%%   Dollar        \$     Percent       \%     Ampersand     \&
%%   Acute accent  \'     Left paren    \(     Right paren   \)
%%   Asterisk      \*     Plus          \+     Comma         \,
%%   Minus         \-     Point         \.     Solidus       \/
%%   Colon         \:     Semicolon     \;     Less than     \<
%%   Equals        \=     Greater than  \>     Question mark \?
%%   Commercial at \@     Left bracket  \[     Backslash     \\
%%   Right bracket \]     Circumflex    \^     Underscore    \_
%%   Grave accent  \`     Left brace    \{     Vertical bar  \|
%%   Right brace   \}     Tilde         \~}
%</package>
%\fi
% \iffalse
% Doc-Source file
% Copyright (C) 2016 Nicola Talbot, all rights reserved.
% \fi
% \iffalse
%<*driver>
\documentclass{ltxdoc}

\usepackage[T1]{fontenc}
\usepackage{tgtermes}
\usepackage{metalogo}
\usepackage[colorlinks,hyperindex=false]{hyperref}

\CheckSum{273}

\RecordChanges
\PageIndex
\CodelineNumbered
\IndexPrologue{}

\renewcommand*{\usage}[1]{\hyperpage{#1}}
\renewcommand*{\main}[1]{\hyperpage{#1}}

\newcommand*{\shortargfmt}[1]{\texttt{-#1}}
\newcommand*{\longargfmt}[1]{\texttt{-{}-#1}}

\newcommand*{\shortarg}[1]{\shortargfmt{#1}%
 \index{texosquery options=\texttt{texosquery} options>#1=\protect\shortargfmt{-#1}|hyperpage}}
\newcommand*{\longarg}[1]{\longargfmt{#1}%
 \index{texosquery options=\texttt{texosquery} options>#1=\protect\longargfmt{#1}|hyperpage}}

\newenvironment{important}
 {\begin{quote}\textbf{Important Note:}}
 {\end{quote}}

\begin{document}
\DocInput{texosquery.dtx}
\end{document}
%</driver>
%\fi
%
%\DeleteShortVerb{|}
%
%\title{texosquery: query OS information from \TeX}
%\author{Nicola L. C. Talbot\\
%\href{http://www.dickimaw-books.com/}{\nolinkurl{dickimaw-books.com}}
%\and Paulo Cereda}
%\date{2016-07-14 (v1.1)}
%\maketitle
%
%\begin{abstract}
%\texttt{texosquery.jar} is a cross-platform Java application
%to query certain OS information designed for use in \TeX's
%shell escape mechanism.
%The accompanying \TeX\ code provides a simple interface to the
%\texttt{texosquery} Java application, which may be skipped
%if you want to explicitly use \cs{input} (but take
%care of any special characters appearing in the result).
%\end{abstract}
%
%\begin{important}
%You will need \TeX's shell escape enabled,
%and you will also need the Java Runtime Environment (JRE) 
%installed to use \texttt{texosquery}.
%\end{important}
%
%If you want to rebuild the application, instructions for
%compiling the source code (including the code for this document)
%are in the accompanying \texttt{README.md} file.
%
%\tableofcontents
%
%\section{texosquery.jar: the Java application}
%The \texttt{texosquery} Java command line application looks up
%certain system information that may be of use in \TeX\ 
%documents. All this information can easily be obtained using 
%native commands, but the Java application allows an 
%OS-independent approach with results that can easily be
%captured by \TeX's shell-escape without having to strip 
%formatting information.
%
%\begin{important}
%\texttt{texosquery} provides read-only actions, and I don't 
%intend adding any actions that modify system settings or files.
%\end{important}
%
%Since the application is designed to work with \TeX, each
%function will display the result on a single line without
%formatting. (For multiple results, each line is grouped 
%from v1.1.) A blank line (or empty group) will be displayed
%if the information isn't available. A forward slash is always 
%used as a directory divider, regardless of the operating 
%system, so the result can be used, for example, in \cs{input} or
%\cs{includegraphics}.
%
%If an input file name is required (for example, with the
%\longargfmt{pdfdate} argument described below) then the file may be in
%the current working directory, relative to the current directory
%(with forward slash \texttt{/} as the directory divider), an absolute path 
%(again with forward slash) or on \TeX's path (in
%which case, \texttt{kpsewhich} is used to locate it).
%
%Command line invocation:
%\begin{flushleft}\ttfamily
%texosquery \meta{action} ...
%\end{flushleft}
%Available actions (at least one required):
%\begin{description}
%\item[\shortarg{L} or \longarg{locale}] Display the locale
%information in the form
%\begin{flushleft}\ttfamily
%\meta{lang}-\meta{region}.\meta{codeset}@\meta{modifier}
%\end{flushleft}
%where \meta{lang} is the ISO code for the language (e.g.\ 
%\texttt{en}),
%\meta{region} is the ISO code for the region (e.g.\ \texttt{GB}),
%\meta{codeset} is the default code set (e.g.\ \texttt{UTF-8})
% and \meta{modifier} is the modifier. Elements may be omitted
% if unavailable. For example, \texttt{en-GB.UTF-8} has the
% \meta{modifier} omitted, and \texttt{en} has all but the language
% omitted.
%\item[\shortarg{l} or \longarg{locale-lcs}] As the above but the
%codeset (if present) is converted to lower case and any hyphens are
%stripped. For example, if \longargfmt{locale} returns
%\texttt{en-GB.UTF-8}, then \longargfmt{locale-lcs} would return
%\texttt{en-GB.utf8}.
%\item[\shortarg{b} or \longarg{bcp47}] (New to version 1.2.)
%Displays the BCP~47 language tag.
%
%\item[\shortarg{c} or \longarg{cwd}] Displays the current working
%directory.
%\item[\shortarg{m} or \longarg{userhome}] Displays the user's home
%directory.
%\item[\shortarg{t} or \longarg{tmpdir}] Displays the temporary
%directory.
%\item[\shortarg{o} or \longarg{osname}] Displays the operating
%system name.
%\item[\shortarg{r} or \longarg{osversion}] Displays the operating
%system version.
%\item[\shortarg{a} or \longarg{osarch}] Displays the operating
%system architecture.
%\item[\shortarg{n} or \longarg{pdfnow}] Displays the current
% date and time in PDF format. For example 
%\begin{verbatim}
%D:20160704131006+01'00'
%\end{verbatim}
%Note that some, but not all, \TeX\ formats provide
%\cs{pdfcreationdate}, which is more efficient than using the shell
%escape, but this can be used
%as a fallback method for those that don't (for example, \XeTeX).
%\item[\shortarg{d} \meta{file} or \longarg{pdfdate} \meta{file}]
% Displays the last modified time stamp of the given file in PDF format or a blank line
% if the file doesn't exist or the file permissions prohibit this
% action.
%Again some, but not all, \TeX\ formats provide
%\cs{pdffilemoddate}\marg{file}, which is more efficient than using
%the shell escape.
%\item[\shortarg{s} \meta{file} or \longarg{filesize} \meta{file}]
% Displays the size in bytes of the given file or a blank line
% if the file doesn't exist or the file permissions prohibit this
% action.
%Some, but not all, \TeX\ formats provide
%\cs{pdffilesize}\marg{file}, which is more efficient than using
%the shell escape.
%\item[\shortarg{i} \meta{sep} \meta{dir} or \longarg{list}
%\meta{sep} \meta{dir}]
%List all files in the given directory with the output on a single
%line using \meta{sep} as the separator between entries. Note that
%the list doesn't include the full path, just the file names.
%\item[\shortarg{f} \meta{sep} \meta{regex} \meta{dir} or
%\longarg{filterlist} \meta{sep} \meta{regex} \meta{dir}]
%Like the above but only lists those files whose name matches the
%regular expression given in \meta{regex}. Note that this tests for a
%complete match on the file name (not including path). For example,
%if \meta{regexp} is \texttt{foo.*}, it will only match files whose
%name starts with \texttt{foo} (for example, \texttt{foobar} will
%match but \texttt{barfoo} won't).
%\item[\shortarg{u} \meta{file} or \longarg{uri} \meta{file}]
%Displays the URI of the given file or a blank line if the file
%doesn't exist or the file permissions prohibit this action.
%Note that this may include in a percent character in the result. The \TeX\ 
%command \cs{TeXOSQuery} protects against this by changing the
%category code, but if you explicitly call \texttt{texosquery}
%using the shell escape, you'll need to be careful of this.
%\item[\shortarg{p} \meta{file} or \longarg{path} \meta{file}]
%Displays the canonical path of the given file or a blank line if
%the file doesn't exist or the file permissions prohibit this
%action.
%\item[\shortarg{e} \meta{file} or \longarg{dirname} \meta{file}]
%(New to v1.1.)
%Displays the canonical path of the given file's parent (that is, the
%directory containing \meta{file}) or a blank line if
%the file doesn't exist or the file permissions prohibit this
%action. Note that this is different to the Unix-like 
%\texttt{dirname} command, which will return a relative path if
%\meta{file} isn't an absolute path.
%\item[\shortarg{h} or \longarg{help}] Displays help message and
%exits.
%\item[\shortarg{v} or \longarg{version}] Displays version
%information and exits.
%\end{description}
%
%If multiple options are given, they will be processed in the
%order specified in the command line invocation. Each result will
%be displayed on a separate line. As from v1.1, if there are
%multiple actions, each result will be grouped. This makes it easier
%to process the results in \TeX. For example:
%\begin{verbatim}
%texosquery -l
%\end{verbatim}
%This just produces (for me):
%\begin{verbatim}
%en-GB.utf8
%\end{verbatim}
%whereas
%\begin{verbatim}
%texosquery -l -n
%\end{verbatim}
%produces:
%\begin{verbatim}
%{en-GB.utf8}
%{D:20160714112732+01'00'}
%\end{verbatim}
%Note that unavailable information will produce an empty group.
%For example (assuming \texttt{nofile} doesn't exist):
%\begin{verbatim}
%texosquery -l -d nofile -n
%\end{verbatim}
%produces:
%\begin{verbatim}
%{en-GB.utf8}
%{}
%{D:20160714112732+01'00'}
%\end{verbatim}
%whereas 
%\begin{verbatim}
%texosquery -d nofile
%\end{verbatim}
%just displays an empty line.
%
%\section{texosquery.tex: generic \TeX\ code}
%You can run \texttt{texosquery} directly from \TeX's shell escape.
%For example:
%\begin{verbatim}
%\input|"texosquery --locale"
%\end{verbatim}
%However, \texttt{texosquery.tex} provides generic \TeX\ code to do
%this for you and store the result in a control sequence.
%
%Plain \TeX\ users can input this file through the usual \cs{input}
%method:
%\begin{verbatim}
%\input texosquery
%\end{verbatim}
%\LaTeX\ users may also simply input this file:
%\begin{verbatim}
%%\iffalse
%<*package>
%% \CharacterTable
%%  {Upper-case    \A\B\C\D\E\F\G\H\I\J\K\L\M\N\O\P\Q\R\S\T\U\V\W\X\Y\Z
%%   Lower-case    \a\b\c\d\e\f\g\h\i\j\k\l\m\n\o\p\q\r\s\t\u\v\w\x\y\z
%%   Digits        \0\1\2\3\4\5\6\7\8\9
%%   Exclamation   \!     Double quote  \"     Hash (number) \#
%%   Dollar        \$     Percent       \%     Ampersand     \&
%%   Acute accent  \'     Left paren    \(     Right paren   \)
%%   Asterisk      \*     Plus          \+     Comma         \,
%%   Minus         \-     Point         \.     Solidus       \/
%%   Colon         \:     Semicolon     \;     Less than     \<
%%   Equals        \=     Greater than  \>     Question mark \?
%%   Commercial at \@     Left bracket  \[     Backslash     \\
%%   Right bracket \]     Circumflex    \^     Underscore    \_
%%   Grave accent  \`     Left brace    \{     Vertical bar  \|
%%   Right brace   \}     Tilde         \~}
%</package>
%\fi
% \iffalse
% Doc-Source file
% Copyright (C) 2016 Nicola Talbot, all rights reserved.
% \fi
% \iffalse
%<*driver>
\documentclass{ltxdoc}

\usepackage[T1]{fontenc}
\usepackage{tgtermes}
\usepackage{metalogo}
\usepackage[colorlinks,hyperindex=false]{hyperref}

\CheckSum{273}

\RecordChanges
\PageIndex
\CodelineNumbered
\IndexPrologue{}

\renewcommand*{\usage}[1]{\hyperpage{#1}}
\renewcommand*{\main}[1]{\hyperpage{#1}}

\newcommand*{\shortargfmt}[1]{\texttt{-#1}}
\newcommand*{\longargfmt}[1]{\texttt{-{}-#1}}

\newcommand*{\shortarg}[1]{\shortargfmt{#1}%
 \index{texosquery options=\texttt{texosquery} options>#1=\protect\shortargfmt{-#1}|hyperpage}}
\newcommand*{\longarg}[1]{\longargfmt{#1}%
 \index{texosquery options=\texttt{texosquery} options>#1=\protect\longargfmt{#1}|hyperpage}}

\newenvironment{important}
 {\begin{quote}\textbf{Important Note:}}
 {\end{quote}}

\begin{document}
\DocInput{texosquery.dtx}
\end{document}
%</driver>
%\fi
%
%\DeleteShortVerb{|}
%
%\title{texosquery: query OS information from \TeX}
%\author{Nicola L. C. Talbot\\
%\href{http://www.dickimaw-books.com/}{\nolinkurl{dickimaw-books.com}}
%\and Paulo Cereda}
%\date{2016-07-14 (v1.1)}
%\maketitle
%
%\begin{abstract}
%\texttt{texosquery.jar} is a cross-platform Java application
%to query certain OS information designed for use in \TeX's
%shell escape mechanism.
%The accompanying \TeX\ code provides a simple interface to the
%\texttt{texosquery} Java application, which may be skipped
%if you want to explicitly use \cs{input} (but take
%care of any special characters appearing in the result).
%\end{abstract}
%
%\begin{important}
%You will need \TeX's shell escape enabled,
%and you will also need the Java Runtime Environment (JRE) 
%installed to use \texttt{texosquery}.
%\end{important}
%
%If you want to rebuild the application, instructions for
%compiling the source code (including the code for this document)
%are in the accompanying \texttt{README.md} file.
%
%\tableofcontents
%
%\section{texosquery.jar: the Java application}
%The \texttt{texosquery} Java command line application looks up
%certain system information that may be of use in \TeX\ 
%documents. All this information can easily be obtained using 
%native commands, but the Java application allows an 
%OS-independent approach with results that can easily be
%captured by \TeX's shell-escape without having to strip 
%formatting information.
%
%\begin{important}
%\texttt{texosquery} provides read-only actions, and I don't 
%intend adding any actions that modify system settings or files.
%\end{important}
%
%Since the application is designed to work with \TeX, each
%function will display the result on a single line without
%formatting. (For multiple results, each line is grouped 
%from v1.1.) A blank line (or empty group) will be displayed
%if the information isn't available. A forward slash is always 
%used as a directory divider, regardless of the operating 
%system, so the result can be used, for example, in \cs{input} or
%\cs{includegraphics}.
%
%If an input file name is required (for example, with the
%\longargfmt{pdfdate} argument described below) then the file may be in
%the current working directory, relative to the current directory
%(with forward slash \texttt{/} as the directory divider), an absolute path 
%(again with forward slash) or on \TeX's path (in
%which case, \texttt{kpsewhich} is used to locate it).
%
%Command line invocation:
%\begin{flushleft}\ttfamily
%texosquery \meta{action} ...
%\end{flushleft}
%Available actions (at least one required):
%\begin{description}
%\item[\shortarg{L} or \longarg{locale}] Display the locale
%information in the form
%\begin{flushleft}\ttfamily
%\meta{lang}-\meta{region}.\meta{codeset}@\meta{modifier}
%\end{flushleft}
%where \meta{lang} is the ISO code for the language (e.g.\ 
%\texttt{en}),
%\meta{region} is the ISO code for the region (e.g.\ \texttt{GB}),
%\meta{codeset} is the default code set (e.g.\ \texttt{UTF-8})
% and \meta{modifier} is the modifier. Elements may be omitted
% if unavailable. For example, \texttt{en-GB.UTF-8} has the
% \meta{modifier} omitted, and \texttt{en} has all but the language
% omitted.
%\item[\shortarg{l} or \longarg{locale-lcs}] As the above but the
%codeset (if present) is converted to lower case and any hyphens are
%stripped. For example, if \longargfmt{locale} returns
%\texttt{en-GB.UTF-8}, then \longargfmt{locale-lcs} would return
%\texttt{en-GB.utf8}.
%\item[\shortarg{b} or \longarg{bcp47}] (New to version 1.2.)
%Displays the BCP~47 language tag.
%
%\item[\shortarg{c} or \longarg{cwd}] Displays the current working
%directory.
%\item[\shortarg{m} or \longarg{userhome}] Displays the user's home
%directory.
%\item[\shortarg{t} or \longarg{tmpdir}] Displays the temporary
%directory.
%\item[\shortarg{o} or \longarg{osname}] Displays the operating
%system name.
%\item[\shortarg{r} or \longarg{osversion}] Displays the operating
%system version.
%\item[\shortarg{a} or \longarg{osarch}] Displays the operating
%system architecture.
%\item[\shortarg{n} or \longarg{pdfnow}] Displays the current
% date and time in PDF format. For example 
%\begin{verbatim}
%D:20160704131006+01'00'
%\end{verbatim}
%Note that some, but not all, \TeX\ formats provide
%\cs{pdfcreationdate}, which is more efficient than using the shell
%escape, but this can be used
%as a fallback method for those that don't (for example, \XeTeX).
%\item[\shortarg{d} \meta{file} or \longarg{pdfdate} \meta{file}]
% Displays the last modified time stamp of the given file in PDF format or a blank line
% if the file doesn't exist or the file permissions prohibit this
% action.
%Again some, but not all, \TeX\ formats provide
%\cs{pdffilemoddate}\marg{file}, which is more efficient than using
%the shell escape.
%\item[\shortarg{s} \meta{file} or \longarg{filesize} \meta{file}]
% Displays the size in bytes of the given file or a blank line
% if the file doesn't exist or the file permissions prohibit this
% action.
%Some, but not all, \TeX\ formats provide
%\cs{pdffilesize}\marg{file}, which is more efficient than using
%the shell escape.
%\item[\shortarg{i} \meta{sep} \meta{dir} or \longarg{list}
%\meta{sep} \meta{dir}]
%List all files in the given directory with the output on a single
%line using \meta{sep} as the separator between entries. Note that
%the list doesn't include the full path, just the file names.
%\item[\shortarg{f} \meta{sep} \meta{regex} \meta{dir} or
%\longarg{filterlist} \meta{sep} \meta{regex} \meta{dir}]
%Like the above but only lists those files whose name matches the
%regular expression given in \meta{regex}. Note that this tests for a
%complete match on the file name (not including path). For example,
%if \meta{regexp} is \texttt{foo.*}, it will only match files whose
%name starts with \texttt{foo} (for example, \texttt{foobar} will
%match but \texttt{barfoo} won't).
%\item[\shortarg{u} \meta{file} or \longarg{uri} \meta{file}]
%Displays the URI of the given file or a blank line if the file
%doesn't exist or the file permissions prohibit this action.
%Note that this may include in a percent character in the result. The \TeX\ 
%command \cs{TeXOSQuery} protects against this by changing the
%category code, but if you explicitly call \texttt{texosquery}
%using the shell escape, you'll need to be careful of this.
%\item[\shortarg{p} \meta{file} or \longarg{path} \meta{file}]
%Displays the canonical path of the given file or a blank line if
%the file doesn't exist or the file permissions prohibit this
%action.
%\item[\shortarg{e} \meta{file} or \longarg{dirname} \meta{file}]
%(New to v1.1.)
%Displays the canonical path of the given file's parent (that is, the
%directory containing \meta{file}) or a blank line if
%the file doesn't exist or the file permissions prohibit this
%action. Note that this is different to the Unix-like 
%\texttt{dirname} command, which will return a relative path if
%\meta{file} isn't an absolute path.
%\item[\shortarg{h} or \longarg{help}] Displays help message and
%exits.
%\item[\shortarg{v} or \longarg{version}] Displays version
%information and exits.
%\end{description}
%
%If multiple options are given, they will be processed in the
%order specified in the command line invocation. Each result will
%be displayed on a separate line. As from v1.1, if there are
%multiple actions, each result will be grouped. This makes it easier
%to process the results in \TeX. For example:
%\begin{verbatim}
%texosquery -l
%\end{verbatim}
%This just produces (for me):
%\begin{verbatim}
%en-GB.utf8
%\end{verbatim}
%whereas
%\begin{verbatim}
%texosquery -l -n
%\end{verbatim}
%produces:
%\begin{verbatim}
%{en-GB.utf8}
%{D:20160714112732+01'00'}
%\end{verbatim}
%Note that unavailable information will produce an empty group.
%For example (assuming \texttt{nofile} doesn't exist):
%\begin{verbatim}
%texosquery -l -d nofile -n
%\end{verbatim}
%produces:
%\begin{verbatim}
%{en-GB.utf8}
%{}
%{D:20160714112732+01'00'}
%\end{verbatim}
%whereas 
%\begin{verbatim}
%texosquery -d nofile
%\end{verbatim}
%just displays an empty line.
%
%\section{texosquery.tex: generic \TeX\ code}
%You can run \texttt{texosquery} directly from \TeX's shell escape.
%For example:
%\begin{verbatim}
%\input|"texosquery --locale"
%\end{verbatim}
%However, \texttt{texosquery.tex} provides generic \TeX\ code to do
%this for you and store the result in a control sequence.
%
%Plain \TeX\ users can input this file through the usual \cs{input}
%method:
%\begin{verbatim}
%\input texosquery
%\end{verbatim}
%\LaTeX\ users may also simply input this file:
%\begin{verbatim}
%%\iffalse
%<*package>
%% \CharacterTable
%%  {Upper-case    \A\B\C\D\E\F\G\H\I\J\K\L\M\N\O\P\Q\R\S\T\U\V\W\X\Y\Z
%%   Lower-case    \a\b\c\d\e\f\g\h\i\j\k\l\m\n\o\p\q\r\s\t\u\v\w\x\y\z
%%   Digits        \0\1\2\3\4\5\6\7\8\9
%%   Exclamation   \!     Double quote  \"     Hash (number) \#
%%   Dollar        \$     Percent       \%     Ampersand     \&
%%   Acute accent  \'     Left paren    \(     Right paren   \)
%%   Asterisk      \*     Plus          \+     Comma         \,
%%   Minus         \-     Point         \.     Solidus       \/
%%   Colon         \:     Semicolon     \;     Less than     \<
%%   Equals        \=     Greater than  \>     Question mark \?
%%   Commercial at \@     Left bracket  \[     Backslash     \\
%%   Right bracket \]     Circumflex    \^     Underscore    \_
%%   Grave accent  \`     Left brace    \{     Vertical bar  \|
%%   Right brace   \}     Tilde         \~}
%</package>
%\fi
% \iffalse
% Doc-Source file
% Copyright (C) 2016 Nicola Talbot, all rights reserved.
% \fi
% \iffalse
%<*driver>
\documentclass{ltxdoc}

\usepackage[T1]{fontenc}
\usepackage{tgtermes}
\usepackage{metalogo}
\usepackage[colorlinks,hyperindex=false]{hyperref}

\CheckSum{273}

\RecordChanges
\PageIndex
\CodelineNumbered
\IndexPrologue{}

\renewcommand*{\usage}[1]{\hyperpage{#1}}
\renewcommand*{\main}[1]{\hyperpage{#1}}

\newcommand*{\shortargfmt}[1]{\texttt{-#1}}
\newcommand*{\longargfmt}[1]{\texttt{-{}-#1}}

\newcommand*{\shortarg}[1]{\shortargfmt{#1}%
 \index{texosquery options=\texttt{texosquery} options>#1=\protect\shortargfmt{-#1}|hyperpage}}
\newcommand*{\longarg}[1]{\longargfmt{#1}%
 \index{texosquery options=\texttt{texosquery} options>#1=\protect\longargfmt{#1}|hyperpage}}

\newenvironment{important}
 {\begin{quote}\textbf{Important Note:}}
 {\end{quote}}

\begin{document}
\DocInput{texosquery.dtx}
\end{document}
%</driver>
%\fi
%
%\DeleteShortVerb{|}
%
%\title{texosquery: query OS information from \TeX}
%\author{Nicola L. C. Talbot\\
%\href{http://www.dickimaw-books.com/}{\nolinkurl{dickimaw-books.com}}
%\and Paulo Cereda}
%\date{2016-07-14 (v1.1)}
%\maketitle
%
%\begin{abstract}
%\texttt{texosquery.jar} is a cross-platform Java application
%to query certain OS information designed for use in \TeX's
%shell escape mechanism.
%The accompanying \TeX\ code provides a simple interface to the
%\texttt{texosquery} Java application, which may be skipped
%if you want to explicitly use \cs{input} (but take
%care of any special characters appearing in the result).
%\end{abstract}
%
%\begin{important}
%You will need \TeX's shell escape enabled,
%and you will also need the Java Runtime Environment (JRE) 
%installed to use \texttt{texosquery}.
%\end{important}
%
%If you want to rebuild the application, instructions for
%compiling the source code (including the code for this document)
%are in the accompanying \texttt{README.md} file.
%
%\tableofcontents
%
%\section{texosquery.jar: the Java application}
%The \texttt{texosquery} Java command line application looks up
%certain system information that may be of use in \TeX\ 
%documents. All this information can easily be obtained using 
%native commands, but the Java application allows an 
%OS-independent approach with results that can easily be
%captured by \TeX's shell-escape without having to strip 
%formatting information.
%
%\begin{important}
%\texttt{texosquery} provides read-only actions, and I don't 
%intend adding any actions that modify system settings or files.
%\end{important}
%
%Since the application is designed to work with \TeX, each
%function will display the result on a single line without
%formatting. (For multiple results, each line is grouped 
%from v1.1.) A blank line (or empty group) will be displayed
%if the information isn't available. A forward slash is always 
%used as a directory divider, regardless of the operating 
%system, so the result can be used, for example, in \cs{input} or
%\cs{includegraphics}.
%
%If an input file name is required (for example, with the
%\longargfmt{pdfdate} argument described below) then the file may be in
%the current working directory, relative to the current directory
%(with forward slash \texttt{/} as the directory divider), an absolute path 
%(again with forward slash) or on \TeX's path (in
%which case, \texttt{kpsewhich} is used to locate it).
%
%Command line invocation:
%\begin{flushleft}\ttfamily
%texosquery \meta{action} ...
%\end{flushleft}
%Available actions (at least one required):
%\begin{description}
%\item[\shortarg{L} or \longarg{locale}] Display the locale
%information in the form
%\begin{flushleft}\ttfamily
%\meta{lang}-\meta{region}.\meta{codeset}@\meta{modifier}
%\end{flushleft}
%where \meta{lang} is the ISO code for the language (e.g.\ 
%\texttt{en}),
%\meta{region} is the ISO code for the region (e.g.\ \texttt{GB}),
%\meta{codeset} is the default code set (e.g.\ \texttt{UTF-8})
% and \meta{modifier} is the modifier. Elements may be omitted
% if unavailable. For example, \texttt{en-GB.UTF-8} has the
% \meta{modifier} omitted, and \texttt{en} has all but the language
% omitted.
%\item[\shortarg{l} or \longarg{locale-lcs}] As the above but the
%codeset (if present) is converted to lower case and any hyphens are
%stripped. For example, if \longargfmt{locale} returns
%\texttt{en-GB.UTF-8}, then \longargfmt{locale-lcs} would return
%\texttt{en-GB.utf8}.
%\item[\shortarg{b} or \longarg{bcp47}] (New to version 1.2.)
%Displays the BCP~47 language tag.
%
%\item[\shortarg{c} or \longarg{cwd}] Displays the current working
%directory.
%\item[\shortarg{m} or \longarg{userhome}] Displays the user's home
%directory.
%\item[\shortarg{t} or \longarg{tmpdir}] Displays the temporary
%directory.
%\item[\shortarg{o} or \longarg{osname}] Displays the operating
%system name.
%\item[\shortarg{r} or \longarg{osversion}] Displays the operating
%system version.
%\item[\shortarg{a} or \longarg{osarch}] Displays the operating
%system architecture.
%\item[\shortarg{n} or \longarg{pdfnow}] Displays the current
% date and time in PDF format. For example 
%\begin{verbatim}
%D:20160704131006+01'00'
%\end{verbatim}
%Note that some, but not all, \TeX\ formats provide
%\cs{pdfcreationdate}, which is more efficient than using the shell
%escape, but this can be used
%as a fallback method for those that don't (for example, \XeTeX).
%\item[\shortarg{d} \meta{file} or \longarg{pdfdate} \meta{file}]
% Displays the last modified time stamp of the given file in PDF format or a blank line
% if the file doesn't exist or the file permissions prohibit this
% action.
%Again some, but not all, \TeX\ formats provide
%\cs{pdffilemoddate}\marg{file}, which is more efficient than using
%the shell escape.
%\item[\shortarg{s} \meta{file} or \longarg{filesize} \meta{file}]
% Displays the size in bytes of the given file or a blank line
% if the file doesn't exist or the file permissions prohibit this
% action.
%Some, but not all, \TeX\ formats provide
%\cs{pdffilesize}\marg{file}, which is more efficient than using
%the shell escape.
%\item[\shortarg{i} \meta{sep} \meta{dir} or \longarg{list}
%\meta{sep} \meta{dir}]
%List all files in the given directory with the output on a single
%line using \meta{sep} as the separator between entries. Note that
%the list doesn't include the full path, just the file names.
%\item[\shortarg{f} \meta{sep} \meta{regex} \meta{dir} or
%\longarg{filterlist} \meta{sep} \meta{regex} \meta{dir}]
%Like the above but only lists those files whose name matches the
%regular expression given in \meta{regex}. Note that this tests for a
%complete match on the file name (not including path). For example,
%if \meta{regexp} is \texttt{foo.*}, it will only match files whose
%name starts with \texttt{foo} (for example, \texttt{foobar} will
%match but \texttt{barfoo} won't).
%\item[\shortarg{u} \meta{file} or \longarg{uri} \meta{file}]
%Displays the URI of the given file or a blank line if the file
%doesn't exist or the file permissions prohibit this action.
%Note that this may include in a percent character in the result. The \TeX\ 
%command \cs{TeXOSQuery} protects against this by changing the
%category code, but if you explicitly call \texttt{texosquery}
%using the shell escape, you'll need to be careful of this.
%\item[\shortarg{p} \meta{file} or \longarg{path} \meta{file}]
%Displays the canonical path of the given file or a blank line if
%the file doesn't exist or the file permissions prohibit this
%action.
%\item[\shortarg{e} \meta{file} or \longarg{dirname} \meta{file}]
%(New to v1.1.)
%Displays the canonical path of the given file's parent (that is, the
%directory containing \meta{file}) or a blank line if
%the file doesn't exist or the file permissions prohibit this
%action. Note that this is different to the Unix-like 
%\texttt{dirname} command, which will return a relative path if
%\meta{file} isn't an absolute path.
%\item[\shortarg{h} or \longarg{help}] Displays help message and
%exits.
%\item[\shortarg{v} or \longarg{version}] Displays version
%information and exits.
%\end{description}
%
%If multiple options are given, they will be processed in the
%order specified in the command line invocation. Each result will
%be displayed on a separate line. As from v1.1, if there are
%multiple actions, each result will be grouped. This makes it easier
%to process the results in \TeX. For example:
%\begin{verbatim}
%texosquery -l
%\end{verbatim}
%This just produces (for me):
%\begin{verbatim}
%en-GB.utf8
%\end{verbatim}
%whereas
%\begin{verbatim}
%texosquery -l -n
%\end{verbatim}
%produces:
%\begin{verbatim}
%{en-GB.utf8}
%{D:20160714112732+01'00'}
%\end{verbatim}
%Note that unavailable information will produce an empty group.
%For example (assuming \texttt{nofile} doesn't exist):
%\begin{verbatim}
%texosquery -l -d nofile -n
%\end{verbatim}
%produces:
%\begin{verbatim}
%{en-GB.utf8}
%{}
%{D:20160714112732+01'00'}
%\end{verbatim}
%whereas 
%\begin{verbatim}
%texosquery -d nofile
%\end{verbatim}
%just displays an empty line.
%
%\section{texosquery.tex: generic \TeX\ code}
%You can run \texttt{texosquery} directly from \TeX's shell escape.
%For example:
%\begin{verbatim}
%\input|"texosquery --locale"
%\end{verbatim}
%However, \texttt{texosquery.tex} provides generic \TeX\ code to do
%this for you and store the result in a control sequence.
%
%Plain \TeX\ users can input this file through the usual \cs{input}
%method:
%\begin{verbatim}
%\input texosquery
%\end{verbatim}
%\LaTeX\ users may also simply input this file:
%\begin{verbatim}
%\input{texosquery}
%\end{verbatim}
%but may prefer the standard package approach:
%\begin{verbatim}
%\usepackage{texosquery}
%\end{verbatim}
%
%\begin{important}
%The commands described below are all fragile.
%\end{important}
%
%The basic command to run \texttt{texosquery} and capture its output
%in a control sequence is:
%\DescribeMacro\TeXOSQuery
%\begin{flushleft}\ttfamily
%\cs{TeXOSQuery}\marg{cs}\marg{args}
%\end{flushleft}
%where \meta{cs} is the control sequence in which to store the
%result and \meta{args} are the command line arguments to pass to 
%\texttt{texosquery}. This first locally changes the category code
%of some problematic characters that may appear in the result.
%The only special characters that aren't changed are the backslash
%\verb|\|, curly braces \verb|{| and \verb|}|, and hash \verb|#|.
%(\texttt{texosquery.jar} will replace \verb|#| with \verb|\#| in places
%where it might possibly occur in the result, but in general it's
%best to avoid these characters in file names.)
%
%There are some short cut commands for convenience, described below.
%If any of these commands cause an error message in the form:
%\begin{verbatim}
%I can't find file `|texosquery'.
%\end{verbatim}
%then check that you have the shell escape on. If the error 
%persists with the shell escape enabled
%(and not restricted), check that \texttt{texosquery} is on 
%your system's path. To do this, open a command prompt or terminal and
%type \texttt{texosquery}. If it isn't installed correctly, there will be a
%message like:
%\begin{verbatim}
%`texosquery' is not recognised
%\end{verbatim}
%or
%\begin{verbatim}
%texosquery: command not found
%\end{verbatim}
%If this happens, check the installation. (Instructions are
%in the accompanying \texttt{README.md} file.)
%
%Dry run mode is determined by the conditional
%\DescribeMacro\ifTeXOSQueryDryRun
%\begin{verbatim}
%\ifTeXOSQueryDryRun
%\end{verbatim}
%If true, the shell escape won't be used and the requested command
%invocation will be printed in the transcript file prefixed with
%\begin{verbatim}
%TeXOSQuery:
%\end{verbatim}
%(the control sequence \meta{cs} will be set to empty).
%This conditional will automatically be switched on unless
%\cs{shellescape} or \cs{pdfshellescape} is 1. (If
%\texttt{texosquery.jar} is later allowed on the restricted list,
%newer versions may change this default.)
%
%If multiple queries are required, it's more efficient to
%perform them all in one go. For example:
%\begin{verbatim}
%\TeXOSQuery{\result}{-l -n -o}
%
%\def\parseresult#1#2#3{%
%  Locale: #1. Now: #2. OS: #3.%
%}
%
%\ifx\result\empty
%  Query failed.
%\else
%  \expandafter\parseresult\result
%\fi
%\end{verbatim}
%(Make sure you have at least v1.1 for this to work correctly.)
%
%The locale (\shortarg{l} or \longarg{locale-lcs}) information can be
%obtained using:
%\DescribeMacro\TeXOSQueryLocale
%\begin{flushleft}\ttfamily
%\cs{TeXOSQueryLocale}\marg{cs}
%\end{flushleft}
%Note that this uses the lower case codeset form, which has a better
%chance of matching the encoding names used by the \texttt{inputenc}
%package. If you want the unprocessed codeset name, you can do:
%\begin{flushleft}\ttfamily
%\cs{TeXOSQuery}\marg{cs}\{\shortarg{L}\}
%\end{flushleft}
%
%The BCP 47 language tag (\shortarg{b} or \longarg{bcp47}) can be obtained using:
%\DescribeMacro\TeXOSQueryLangTag
%\begin{flushleft}\ttfamily
%\cs{TeXOSQueryLangTag}\marg{cs}
%\end{flushleft}
%
%The current working directory (\shortarg{c} or \longarg{cwd}) can
%be obtained using:
%\DescribeMacro\TeXOSQueryCwd
%\begin{flushleft}\ttfamily
%\cs{TeXOSQueryCwd}\marg{cs}
%\end{flushleft}
%
%The home directory (\shortarg{m} or \longarg{userhome}) can
%be obtained using:
%\DescribeMacro\TeXOSQueryHome
%\begin{flushleft}\ttfamily
%\cs{TeXOSQueryHome}\marg{cs}
%\end{flushleft}
%
%The temporary directory (\shortarg{t} or \longarg{tmpdir}) can
%be obtained using:
%\DescribeMacro\TeXOSQueryTmpDir
%\begin{flushleft}\ttfamily
%\cs{TeXOSQueryTmpDir}\marg{cs}
%\end{flushleft}
%
%The OS name (\shortarg{o} or \longarg{osname}) can
%be obtained using:
%\DescribeMacro\TeXOSQueryName
%\begin{flushleft}\ttfamily
%\cs{TeXOSQueryName}\marg{cs}
%\end{flushleft}
%
%The OS version (\shortarg{r} or \longarg{osversion}) can
%be obtained using:
%\DescribeMacro\TeXOSQueryVersion
%\begin{flushleft}\ttfamily
%\cs{TeXOSQueryVersion}\marg{cs}
%\end{flushleft}
%
%The OS architecture (\shortarg{a} or \longarg{osarch}) can
%be obtained using:
%\DescribeMacro\TeXOSQueryArch
%\begin{flushleft}\ttfamily
%\cs{TeXOSQueryArch}\marg{cs}
%\end{flushleft}
%
%The current date-time stamp in PDF format (\shortarg{n} or
%\longarg{pdfnow}) can be obtained using:
%\DescribeMacro\TeXOSQueryNow
%\begin{flushleft}\ttfamily
%\cs{TeXOSQueryNow}\marg{cs}
%\end{flushleft}
%This is provided for the benefit of users who don't have
%\cs{pdfcreationdate} defined by their \TeX\ format (for example,
%\XeTeX). As from v1.1, this ensures that the initial \texttt{D} has
%category code 12 (which won't happen if \cs{TeXOSQuery} is used
%explicitly).
%
%\begin{important}
%The remaining commands all require extra arguments after the
%relevant switch. These are automatically enclosed in single-quotes
%to protect any spaces. If the argument actually contains any single-quote
%characters, make sure you use \verb|\string\'| to prevent
%interference. \emph{However, in general it's a very bad idea to use
%quotes as part of a file name (rather than using them as
%delimiters).}
%Since the file name may need to be obtained from \cs{jobname},
%which sometimes includes double-quotes, the first double-quote pair
%found is stripped in file name arguments. Any other double-quotes
%will need to be protected in the same manner as single-quotes
%(but, again, this shouldn't be an issue if you use a safe
%file naming scheme). Any paths should use a
%forward slash for the directory divider.
%\end{important}
%
%The modification date-time stamp in PDF format for a file
%(\shortarg{d} or \longarg{pdfdate}) can be obtained using:
%\DescribeMacro\TeXOSQueryFileDate
%\begin{flushleft}\ttfamily
%\cs{TeXOSQueryFileDate}\marg{cs}\marg{filename}
%\end{flushleft}
%where \meta{filename} is the name of the file.
%This is provided for the benefit of users who don't have
%\cs{pdffilemoddate} defined by their \TeX\ format. As from v1.1,
%this ensures that the initial \texttt{D} has category code 12 (which 
%won't happen if \cs{TeXOSQuery} is used explicitly).
%
%The size in bytes of a file
%(\shortarg{s} or \longarg{filesize}) can be obtained using:
%\DescribeMacro\TeXOSQueryFileSize
%\begin{flushleft}\ttfamily
%\cs{TeXOSQueryFileSize}\marg{cs}\marg{filename}
%\end{flushleft}
%where \meta{filename} is the name of the file.
%This is provided for the benefit of users who don't have
%\cs{pdffilesize} defined by their \TeX\ format.
%
%The list of files in a given directory
%(\shortarg{i} or \longarg{list}) can be obtained using:
%\DescribeMacro\TeXOSQueryFileList
%\begin{flushleft}\ttfamily
%\cs{TeXOSQueryFileList}\marg{cs}\marg{sep}\marg{dir}
%\end{flushleft}
%where \meta{sep} is the separator and \meta{dir} is the directory name.
%For example:
%\begin{verbatim}
%\TeXOSQueryFileList{\result}{,}{.}
%\end{verbatim}
%will store a comma-separated list of all the files contained in the
%current directory in the control sequence \cs{result}.
%
%A filtered list of files in a given directory
%(\shortarg{f} or \longarg{filterlist}) can be obtained using:
%\DescribeMacro\TeXOSQueryFilterFileList
%\begin{flushleft}\ttfamily
%\cs{TeXOSQueryFilterFileList}\marg{cs}\marg{sep}\marg{regex}\marg{dir}
%\end{flushleft}
%where \meta{regex} is a regular expression. \emph{Take care of any backslashes
%in the regular expression!}
%For example, to list only those files that have an extension:
%\begin{verbatim}
%\TeXOSQueryFilterFileList{\result}{,}{.+\string\..*}{.}
%\end{verbatim}
%Note the use of \verb|\string\.| to ensure that \cs{.} isn't
%interpreted as a command. Another example, list only \texttt{.png}
%and \texttt{.jpg} files in the directory called \texttt{images}:
%\begin{verbatim}
%\TeXOSQueryFilterFileList{\result}{,}{.+\string\.(jpg|png)}{images}
%\end{verbatim}
%
%The URI of a file
%(\shortarg{u} or \longarg{uri}) can be obtained using:
%\DescribeMacro\TeXOSQueryFileURI
%\begin{flushleft}\ttfamily
%\cs{TeXOSQueryFileURI}\marg{cs}\marg{filename}
%\end{flushleft}
%where \meta{filename} is the name of the file.
%(Any percent symbols \verb|%| contained in the URI will have
%their category code set to 12.)
%
%The canonical path of a file
%(\shortarg{p} or \longarg{path}) can be obtained using:
%\DescribeMacro\TeXOSQueryFilePath
%\begin{flushleft}\ttfamily
%\cs{TeXOSQueryFilePath}\marg{cs}\marg{filename}
%\end{flushleft}
%where \meta{filename} is the name of the file.
%
%The canonical path of a file's parent
%(\shortarg{e} or \longarg{dirname}) can be obtained using:
%\DescribeMacro\TeXOSQueryDirName
%\begin{flushleft}\ttfamily
%\cs{TeXOSQueryDirName}\marg{cs}\marg{filename}
%\end{flushleft}
%where \meta{filename} is the name of the file.
%
%\StopEventually{\phantomsection
%\addcontentsline{toc}{section}{Change History}%
%\PrintChanges
%\addcontentsline{toc}{section}{\indexname}%
%\PrintIndex
%}
%
%\section{The Code}
%\changes{1.0}{2016-07-08}{Initial release}
%\subsection{Generic \TeX\ Code}
%\iffalse
%    \begin{macrocode}
%<*texosquery.tex>
%    \end{macrocode}
%\fi
%Change category code of \texttt{@} if necessary.
%    \begin{macrocode}
\ifnum\catcode`\@=11\relax
  \def\@texosquery@restore@at{}%
\else
  \expandafter\edef\csname @texosquery@restore@at\endcsname{%
    \noexpand\catcode`\noexpand\@=\number\catcode`\@\relax
  }%
 \catcode`\@=11\relax
\fi
%    \end{macrocode}
% Check if already loaded.
%    \begin{macrocode}
\ifx\TeXOSQuery\undefined
\else
  \@texosquery@restore@at
  \expandafter\endinput
\fi
%    \end{macrocode}
%\begin{macro}{\TeXOSInvokerName}
%The name of the \texttt{texosquery} application.
%    \begin{macrocode}
\def\TeXOSInvokerName{texosquery}
%    \end{macrocode}
%\end{macro}
%
%\begin{macro}{\TeXOSQueryInvoker}
%If we're using \LaTeX, we'll need to use \cs{@@input} rather
%than \cs{input}.
%    \begin{macrocode}
\ifx\@@input\undefined
  \def\TeXOSQueryInvoker#1{\input|"\TeXOSInvokerName\space#1" }
\else
  \def\TeXOSQueryInvoker#1{\@@input|"\TeXOSInvokerName\space#1" }
\fi
%    \end{macrocode}
%\end{macro}
%
%\begin{macro}{\ifTeXOSQueryDryRun}
%Provide a dry-run mode.
%\changes{1.1}{2016-07-14}{dry run mode only false by default if with
%unrestricted mode}
%    \begin{macrocode}
\newif\ifTeXOSQueryDryRun
\TeXOSQueryDryRuntrue
%    \end{macrocode}
%\end{macro}
%
%If shell escape is unrestricted, automatically switch off dry-run mode.
%    \begin{macrocode}
\ifx\shellescape\undefined
  \ifx\pdfshellescape\undefined
  \else
    \ifnum\pdfshellescape=1\relax
      \TeXOSQueryDryRunfalse
    \fi
  \fi
\else
  \ifnum\shellescape=1\relax
    \TeXOSQueryDryRunfalse
  \fi
\fi
%    \end{macrocode}
%
%\begin{macro}{\TeXOSQuery}
%Use \texttt{texosquery} with the option given in the second
%argument and store the result in control sequence given in the
%first argument.
%    \begin{macrocode}
\def\TeXOSQuery#1#2{%
  \ifTeXOSQueryDryRun
    \begingroup
      \newlinechar=`\^^J
      \message{^^JTeXOSQuery: \TeXOSInvokerName\space#2^^J}%
    \endgroup
    \def#1{}%
  \else
    \begingroup
    \endlinechar=-1\relax
%    \end{macrocode}
%Just in case the result contains any awkward characters that have
%some special meaning to \TeX. (Can't really protect against hash,
%backslash or curly braces, but they're unlikely to occur unless
%the user has a very wacky and inappropriate file naming scheme.)
%    \begin{macrocode}
    \catcode`\-=12\relax
    \catcode`\_=12\relax
    \catcode`\^=12\relax
    \catcode`\~=12\relax
    \catcode`\$=12\relax
    \catcode`\&=12\relax
    \catcode`\"=12\relax
    \catcode`\'=12\relax
    \catcode`\.=12\relax
    \catcode`\/=12\relax
    \catcode`\:=12\relax
    \catcode`\;=12\relax
    \catcode`\%=12\relax
    \everyeof{\noexpand}\relax
    \edef\x{\endgroup\def\noexpand#1{\TeXOSQueryInvoker{#2}}}\x
  \fi
}
%    \end{macrocode}
%\end{macro}
%
%Now for some convenient shortcuts so the user doesn't have to
%remember the command line options. \cs{string} is used in
%case the hyphen character has been made active.
%\begin{macro}{\TeXOSQueryLocale}
%Query the locale and store the result in the control sequence
%provided in the argument.
%    \begin{macrocode}
\def\TeXOSQueryLocale#1{\TeXOSQuery{#1}{\string-l}}
%    \end{macrocode}
%\end{macro}
%
%\begin{macro}{\TeXOSQueryLangTag}
%Query the language tag and store the result in the control sequence
%provided in the argument.
%\changes{1.2}{??}{new}
%    \begin{macrocode}
\def\TeXOSQueryLangTag#1{\TeXOSQuery{#1}{\string-b}}
%    \end{macrocode}
%\end{macro}
%
%\begin{macro}{\TeXOSQueryCwd}
%Query the current working directory.
%    \begin{macrocode}
\def\TeXOSQueryCwd#1{\TeXOSQuery{#1}{\string-c}}
%    \end{macrocode}
%\end{macro}
%
%\begin{macro}{\TeXOSQueryHome}
%Query the user's home directory.
%    \begin{macrocode}
\def\TeXOSQueryHome#1{\TeXOSQuery{#1}{\string-m}}
%    \end{macrocode}
%\end{macro}
%
%\begin{macro}{\TeXOSQueryTmpDir}
%Query the temporary directory.
%    \begin{macrocode}
\def\TeXOSQueryTmpDir#1{\TeXOSQuery{#1}{\string-t}}
%    \end{macrocode}
%\end{macro}
%
%\begin{macro}{\TeXOSQueryVersion}
%Query the operating system version.
%    \begin{macrocode}
\def\TeXOSQueryVersion#1{\TeXOSQuery{#1}{\string-r}}
%    \end{macrocode}
%\end{macro}
%
%\begin{macro}{\TeXOSQueryArch}
%Query the operating system architecture.
%    \begin{macrocode}
\def\TeXOSQueryArch#1{\TeXOSQuery{#1}{\string-a}}
%    \end{macrocode}
%\end{macro}
%
%\begin{macro}{\TeXOSQueryName}
%Query the operating system name.
%    \begin{macrocode}
\def\TeXOSQueryName#1{\TeXOSQuery{#1}{\string-o}}
%    \end{macrocode}
%\end{macro}
%
%\begin{macro}{\TeXOSQueryNow}
%\changes{1.1}{2016-07-14}{changed catcode of D to 12}
%Query the current time stamp.
%    \begin{macrocode}
\def\TeXOSQueryNow#1{%
%    \end{macrocode}
% The \texttt{D} needs category code 12. (Don't need to worry about
% \texttt{Z} as \texttt{texosquery.jar} uses \texttt{+00'00'} for
% UTC+0.) This change can't be done with the other catcode changes
% in \cs{TeXOSQuery}, as this is only appropriate for the PDF dates.
% Save and restore the catcode rather than fiddle around with
% scoping.
%    \begin{macrocode}
  \edef\@texosquery@restore@D{%
    \noexpand\catcode`\noexpand\D=\the\catcode`\D\relax}%
  \catcode`\D=12\relax
  \TeXOSQuery{#1}{\string-n}%
  \@texosquery@restore@D
}
%    \end{macrocode}
%\end{macro}
%
%If the file name is supplied using \cs{jobname} it may have
%double-quotes which will interfere with things.
%\begin{macro}{\texosquerystripquotes}
%    \begin{macrocode}
\def\texosquerystripquotes#1{%
  \@texosquery@stripquotes#1\@mid@texosquery@stripquotes
    "\relax"\relax\@end@texosquery@stripquotes
}
\def\@texosquery@stripquotes#1"#2"{%
  \@@texosquery@stripquotes#1#2%
}
\def\@@texosquery@stripquotes#1\@mid@texosquery@stripquotes#2\@end@texosquery@stripquotes{%
  #1%
}
%    \end{macrocode}
%\end{macro}
%
%\begin{macro}{\TeXOSQueryFileDate}
%\changes{1.1}{2016-07-14}{changed catcode of D to 12}
%Query the time stamp of the file given in the second argument.
%    \begin{macrocode}
\def\TeXOSQueryFileDate#1#2{%
  \edef\@texosquery@restore@D{%
    \noexpand\catcode`\noexpand\D=\the\catcode`\D\relax}%
  \catcode`\D=12\relax
  \TeXOSQuery{#1}{\string-d \string'\texosquerystripquotes{#2}\string'}%
  \@texosquery@restore@D
}
%    \end{macrocode}
%\end{macro}
%
%\begin{macro}{\TeXOSQueryFileSize}
%Query the size of the file given in the second argument.
%    \begin{macrocode}
\def\TeXOSQueryFileSize#1#2{\TeXOSQuery{#1}{\string-s 
 \string'\texosquerystripquotes{#2}\string'}}
%    \end{macrocode}
%\end{macro}
%
%\begin{macro}{\TeXOSQueryFileList}
%List all files in the directory given in the third argument,
%separated by the second argument.
%    \begin{macrocode}
\def\TeXOSQueryFileList#1#2#3{\TeXOSQuery{#1}{%
 \string-i \string'#2\string' 
 \string'\texosquerystripquotes{#3}\string'}}
%    \end{macrocode}
%\end{macro}
%
%\begin{macro}{\TeXOSQueryFilterFileList}
%Filtered list files in the directory given in the fourth argument,
%separated by the second argument. The third argument is the regular
%expression used to filter the list. \emph{Take care of backslashes
%in the regular expression!}
%    \begin{macrocode}
\def\TeXOSQueryFilterFileList#1#2#3#4{\TeXOSQuery{#1}{%
 \string-f \string'#2\string' \string'#3\string' 
 \string'\texosquerystripquotes{#4}\string'}}
%    \end{macrocode}
%\end{macro}
%
%\begin{macro}{\TeXOSQueryFileURI}
%Get the URI of the file given in the second argument.
%    \begin{macrocode}
\def\TeXOSQueryFileURI#1#2{\TeXOSQuery{#1}{\string-u 
 \string'\texosquerystripquotes{#2}\string'}}
%    \end{macrocode}
%\end{macro}
%
%\begin{macro}{\TeXOSQueryFilePath}
%Get the canonical path of the file given in the second argument.
%    \begin{macrocode}
\def\TeXOSQueryFilePath#1#2{\TeXOSQuery{#1}{\string-p 
 \string'\texosquerystripquotes{#2}\string'}}
%    \end{macrocode}
%\end{macro}
%
%\begin{macro}{\TeXOSQueryDirName}
%\changes{1.1}{2016-07-14}{new}
%Get the canonical path of the directory containing the file given 
%in the second argument.
%    \begin{macrocode}
\def\TeXOSQueryDirName#1#2{\TeXOSQuery{#1}{\string-e 
 \string'\texosquerystripquotes{#2}\string'}}
%    \end{macrocode}
%\end{macro}
%
%All done.
%Restore the category code of \texttt{@}:
%    \begin{macrocode}
\@texosquery@restore@at
%    \end{macrocode}
%\iffalse
%    \begin{macrocode}
%</texosquery.tex>
%    \end{macrocode}
%\fi
%
%\subsection{\LaTeX\ Code}
%This is just a simple wrapper for \texttt{texosquery.tex}
%so that it can be loaded using \LaTeX's standard \cs{usepackage}
%method.
%\iffalse
%    \begin{macrocode}
%<*texosquery.sty>
%    \end{macrocode}
%\fi
%Identify package:
%    \begin{macrocode}
\NeedsTeXFormat{LaTeX2e}
\ProvidesPackage{texosquery}[2016/07/14 v1.1 (NLCT)]
%    \end{macrocode}
%Load \texttt{texosquery.tex}:
%    \begin{macrocode}
\input{texosquery}
%    \end{macrocode}
%That's it!
%\iffalse
%    \begin{macrocode}
%</texosquery.sty>
%    \end{macrocode}
%\fi
%
%\Finale
\endinput

%\end{verbatim}
%but may prefer the standard package approach:
%\begin{verbatim}
%\usepackage{texosquery}
%\end{verbatim}
%
%\begin{important}
%The commands described below are all fragile.
%\end{important}
%
%The basic command to run \texttt{texosquery} and capture its output
%in a control sequence is:
%\DescribeMacro\TeXOSQuery
%\begin{flushleft}\ttfamily
%\cs{TeXOSQuery}\marg{cs}\marg{args}
%\end{flushleft}
%where \meta{cs} is the control sequence in which to store the
%result and \meta{args} are the command line arguments to pass to 
%\texttt{texosquery}. This first locally changes the category code
%of some problematic characters that may appear in the result.
%The only special characters that aren't changed are the backslash
%\verb|\|, curly braces \verb|{| and \verb|}|, and hash \verb|#|.
%(\texttt{texosquery.jar} will replace \verb|#| with \verb|\#| in places
%where it might possibly occur in the result, but in general it's
%best to avoid these characters in file names.)
%
%There are some short cut commands for convenience, described below.
%If any of these commands cause an error message in the form:
%\begin{verbatim}
%I can't find file `|texosquery'.
%\end{verbatim}
%then check that you have the shell escape on. If the error 
%persists with the shell escape enabled
%(and not restricted), check that \texttt{texosquery} is on 
%your system's path. To do this, open a command prompt or terminal and
%type \texttt{texosquery}. If it isn't installed correctly, there will be a
%message like:
%\begin{verbatim}
%`texosquery' is not recognised
%\end{verbatim}
%or
%\begin{verbatim}
%texosquery: command not found
%\end{verbatim}
%If this happens, check the installation. (Instructions are
%in the accompanying \texttt{README.md} file.)
%
%Dry run mode is determined by the conditional
%\DescribeMacro\ifTeXOSQueryDryRun
%\begin{verbatim}
%\ifTeXOSQueryDryRun
%\end{verbatim}
%If true, the shell escape won't be used and the requested command
%invocation will be printed in the transcript file prefixed with
%\begin{verbatim}
%TeXOSQuery:
%\end{verbatim}
%(the control sequence \meta{cs} will be set to empty).
%This conditional will automatically be switched on unless
%\cs{shellescape} or \cs{pdfshellescape} is 1. (If
%\texttt{texosquery.jar} is later allowed on the restricted list,
%newer versions may change this default.)
%
%If multiple queries are required, it's more efficient to
%perform them all in one go. For example:
%\begin{verbatim}
%\TeXOSQuery{\result}{-l -n -o}
%
%\def\parseresult#1#2#3{%
%  Locale: #1. Now: #2. OS: #3.%
%}
%
%\ifx\result\empty
%  Query failed.
%\else
%  \expandafter\parseresult\result
%\fi
%\end{verbatim}
%(Make sure you have at least v1.1 for this to work correctly.)
%
%The locale (\shortarg{l} or \longarg{locale-lcs}) information can be
%obtained using:
%\DescribeMacro\TeXOSQueryLocale
%\begin{flushleft}\ttfamily
%\cs{TeXOSQueryLocale}\marg{cs}
%\end{flushleft}
%Note that this uses the lower case codeset form, which has a better
%chance of matching the encoding names used by the \texttt{inputenc}
%package. If you want the unprocessed codeset name, you can do:
%\begin{flushleft}\ttfamily
%\cs{TeXOSQuery}\marg{cs}\{\shortarg{L}\}
%\end{flushleft}
%
%The BCP 47 language tag (\shortarg{b} or \longarg{bcp47}) can be obtained using:
%\DescribeMacro\TeXOSQueryLangTag
%\begin{flushleft}\ttfamily
%\cs{TeXOSQueryLangTag}\marg{cs}
%\end{flushleft}
%
%The current working directory (\shortarg{c} or \longarg{cwd}) can
%be obtained using:
%\DescribeMacro\TeXOSQueryCwd
%\begin{flushleft}\ttfamily
%\cs{TeXOSQueryCwd}\marg{cs}
%\end{flushleft}
%
%The home directory (\shortarg{m} or \longarg{userhome}) can
%be obtained using:
%\DescribeMacro\TeXOSQueryHome
%\begin{flushleft}\ttfamily
%\cs{TeXOSQueryHome}\marg{cs}
%\end{flushleft}
%
%The temporary directory (\shortarg{t} or \longarg{tmpdir}) can
%be obtained using:
%\DescribeMacro\TeXOSQueryTmpDir
%\begin{flushleft}\ttfamily
%\cs{TeXOSQueryTmpDir}\marg{cs}
%\end{flushleft}
%
%The OS name (\shortarg{o} or \longarg{osname}) can
%be obtained using:
%\DescribeMacro\TeXOSQueryName
%\begin{flushleft}\ttfamily
%\cs{TeXOSQueryName}\marg{cs}
%\end{flushleft}
%
%The OS version (\shortarg{r} or \longarg{osversion}) can
%be obtained using:
%\DescribeMacro\TeXOSQueryVersion
%\begin{flushleft}\ttfamily
%\cs{TeXOSQueryVersion}\marg{cs}
%\end{flushleft}
%
%The OS architecture (\shortarg{a} or \longarg{osarch}) can
%be obtained using:
%\DescribeMacro\TeXOSQueryArch
%\begin{flushleft}\ttfamily
%\cs{TeXOSQueryArch}\marg{cs}
%\end{flushleft}
%
%The current date-time stamp in PDF format (\shortarg{n} or
%\longarg{pdfnow}) can be obtained using:
%\DescribeMacro\TeXOSQueryNow
%\begin{flushleft}\ttfamily
%\cs{TeXOSQueryNow}\marg{cs}
%\end{flushleft}
%This is provided for the benefit of users who don't have
%\cs{pdfcreationdate} defined by their \TeX\ format (for example,
%\XeTeX). As from v1.1, this ensures that the initial \texttt{D} has
%category code 12 (which won't happen if \cs{TeXOSQuery} is used
%explicitly).
%
%\begin{important}
%The remaining commands all require extra arguments after the
%relevant switch. These are automatically enclosed in single-quotes
%to protect any spaces. If the argument actually contains any single-quote
%characters, make sure you use \verb|\string\'| to prevent
%interference. \emph{However, in general it's a very bad idea to use
%quotes as part of a file name (rather than using them as
%delimiters).}
%Since the file name may need to be obtained from \cs{jobname},
%which sometimes includes double-quotes, the first double-quote pair
%found is stripped in file name arguments. Any other double-quotes
%will need to be protected in the same manner as single-quotes
%(but, again, this shouldn't be an issue if you use a safe
%file naming scheme). Any paths should use a
%forward slash for the directory divider.
%\end{important}
%
%The modification date-time stamp in PDF format for a file
%(\shortarg{d} or \longarg{pdfdate}) can be obtained using:
%\DescribeMacro\TeXOSQueryFileDate
%\begin{flushleft}\ttfamily
%\cs{TeXOSQueryFileDate}\marg{cs}\marg{filename}
%\end{flushleft}
%where \meta{filename} is the name of the file.
%This is provided for the benefit of users who don't have
%\cs{pdffilemoddate} defined by their \TeX\ format. As from v1.1,
%this ensures that the initial \texttt{D} has category code 12 (which 
%won't happen if \cs{TeXOSQuery} is used explicitly).
%
%The size in bytes of a file
%(\shortarg{s} or \longarg{filesize}) can be obtained using:
%\DescribeMacro\TeXOSQueryFileSize
%\begin{flushleft}\ttfamily
%\cs{TeXOSQueryFileSize}\marg{cs}\marg{filename}
%\end{flushleft}
%where \meta{filename} is the name of the file.
%This is provided for the benefit of users who don't have
%\cs{pdffilesize} defined by their \TeX\ format.
%
%The list of files in a given directory
%(\shortarg{i} or \longarg{list}) can be obtained using:
%\DescribeMacro\TeXOSQueryFileList
%\begin{flushleft}\ttfamily
%\cs{TeXOSQueryFileList}\marg{cs}\marg{sep}\marg{dir}
%\end{flushleft}
%where \meta{sep} is the separator and \meta{dir} is the directory name.
%For example:
%\begin{verbatim}
%\TeXOSQueryFileList{\result}{,}{.}
%\end{verbatim}
%will store a comma-separated list of all the files contained in the
%current directory in the control sequence \cs{result}.
%
%A filtered list of files in a given directory
%(\shortarg{f} or \longarg{filterlist}) can be obtained using:
%\DescribeMacro\TeXOSQueryFilterFileList
%\begin{flushleft}\ttfamily
%\cs{TeXOSQueryFilterFileList}\marg{cs}\marg{sep}\marg{regex}\marg{dir}
%\end{flushleft}
%where \meta{regex} is a regular expression. \emph{Take care of any backslashes
%in the regular expression!}
%For example, to list only those files that have an extension:
%\begin{verbatim}
%\TeXOSQueryFilterFileList{\result}{,}{.+\string\..*}{.}
%\end{verbatim}
%Note the use of \verb|\string\.| to ensure that \cs{.} isn't
%interpreted as a command. Another example, list only \texttt{.png}
%and \texttt{.jpg} files in the directory called \texttt{images}:
%\begin{verbatim}
%\TeXOSQueryFilterFileList{\result}{,}{.+\string\.(jpg|png)}{images}
%\end{verbatim}
%
%The URI of a file
%(\shortarg{u} or \longarg{uri}) can be obtained using:
%\DescribeMacro\TeXOSQueryFileURI
%\begin{flushleft}\ttfamily
%\cs{TeXOSQueryFileURI}\marg{cs}\marg{filename}
%\end{flushleft}
%where \meta{filename} is the name of the file.
%(Any percent symbols \verb|%| contained in the URI will have
%their category code set to 12.)
%
%The canonical path of a file
%(\shortarg{p} or \longarg{path}) can be obtained using:
%\DescribeMacro\TeXOSQueryFilePath
%\begin{flushleft}\ttfamily
%\cs{TeXOSQueryFilePath}\marg{cs}\marg{filename}
%\end{flushleft}
%where \meta{filename} is the name of the file.
%
%The canonical path of a file's parent
%(\shortarg{e} or \longarg{dirname}) can be obtained using:
%\DescribeMacro\TeXOSQueryDirName
%\begin{flushleft}\ttfamily
%\cs{TeXOSQueryDirName}\marg{cs}\marg{filename}
%\end{flushleft}
%where \meta{filename} is the name of the file.
%
%\StopEventually{\phantomsection
%\addcontentsline{toc}{section}{Change History}%
%\PrintChanges
%\addcontentsline{toc}{section}{\indexname}%
%\PrintIndex
%}
%
%\section{The Code}
%\changes{1.0}{2016-07-08}{Initial release}
%\subsection{Generic \TeX\ Code}
%\iffalse
%    \begin{macrocode}
%<*texosquery.tex>
%    \end{macrocode}
%\fi
%Change category code of \texttt{@} if necessary.
%    \begin{macrocode}
\ifnum\catcode`\@=11\relax
  \def\@texosquery@restore@at{}%
\else
  \expandafter\edef\csname @texosquery@restore@at\endcsname{%
    \noexpand\catcode`\noexpand\@=\number\catcode`\@\relax
  }%
 \catcode`\@=11\relax
\fi
%    \end{macrocode}
% Check if already loaded.
%    \begin{macrocode}
\ifx\TeXOSQuery\undefined
\else
  \@texosquery@restore@at
  \expandafter\endinput
\fi
%    \end{macrocode}
%\begin{macro}{\TeXOSInvokerName}
%The name of the \texttt{texosquery} application.
%    \begin{macrocode}
\def\TeXOSInvokerName{texosquery}
%    \end{macrocode}
%\end{macro}
%
%\begin{macro}{\TeXOSQueryInvoker}
%If we're using \LaTeX, we'll need to use \cs{@@input} rather
%than \cs{input}.
%    \begin{macrocode}
\ifx\@@input\undefined
  \def\TeXOSQueryInvoker#1{\input|"\TeXOSInvokerName\space#1" }
\else
  \def\TeXOSQueryInvoker#1{\@@input|"\TeXOSInvokerName\space#1" }
\fi
%    \end{macrocode}
%\end{macro}
%
%\begin{macro}{\ifTeXOSQueryDryRun}
%Provide a dry-run mode.
%\changes{1.1}{2016-07-14}{dry run mode only false by default if with
%unrestricted mode}
%    \begin{macrocode}
\newif\ifTeXOSQueryDryRun
\TeXOSQueryDryRuntrue
%    \end{macrocode}
%\end{macro}
%
%If shell escape is unrestricted, automatically switch off dry-run mode.
%    \begin{macrocode}
\ifx\shellescape\undefined
  \ifx\pdfshellescape\undefined
  \else
    \ifnum\pdfshellescape=1\relax
      \TeXOSQueryDryRunfalse
    \fi
  \fi
\else
  \ifnum\shellescape=1\relax
    \TeXOSQueryDryRunfalse
  \fi
\fi
%    \end{macrocode}
%
%\begin{macro}{\TeXOSQuery}
%Use \texttt{texosquery} with the option given in the second
%argument and store the result in control sequence given in the
%first argument.
%    \begin{macrocode}
\def\TeXOSQuery#1#2{%
  \ifTeXOSQueryDryRun
    \begingroup
      \newlinechar=`\^^J
      \message{^^JTeXOSQuery: \TeXOSInvokerName\space#2^^J}%
    \endgroup
    \def#1{}%
  \else
    \begingroup
    \endlinechar=-1\relax
%    \end{macrocode}
%Just in case the result contains any awkward characters that have
%some special meaning to \TeX. (Can't really protect against hash,
%backslash or curly braces, but they're unlikely to occur unless
%the user has a very wacky and inappropriate file naming scheme.)
%    \begin{macrocode}
    \catcode`\-=12\relax
    \catcode`\_=12\relax
    \catcode`\^=12\relax
    \catcode`\~=12\relax
    \catcode`\$=12\relax
    \catcode`\&=12\relax
    \catcode`\"=12\relax
    \catcode`\'=12\relax
    \catcode`\.=12\relax
    \catcode`\/=12\relax
    \catcode`\:=12\relax
    \catcode`\;=12\relax
    \catcode`\%=12\relax
    \everyeof{\noexpand}\relax
    \edef\x{\endgroup\def\noexpand#1{\TeXOSQueryInvoker{#2}}}\x
  \fi
}
%    \end{macrocode}
%\end{macro}
%
%Now for some convenient shortcuts so the user doesn't have to
%remember the command line options. \cs{string} is used in
%case the hyphen character has been made active.
%\begin{macro}{\TeXOSQueryLocale}
%Query the locale and store the result in the control sequence
%provided in the argument.
%    \begin{macrocode}
\def\TeXOSQueryLocale#1{\TeXOSQuery{#1}{\string-l}}
%    \end{macrocode}
%\end{macro}
%
%\begin{macro}{\TeXOSQueryLangTag}
%Query the language tag and store the result in the control sequence
%provided in the argument.
%\changes{1.2}{??}{new}
%    \begin{macrocode}
\def\TeXOSQueryLangTag#1{\TeXOSQuery{#1}{\string-b}}
%    \end{macrocode}
%\end{macro}
%
%\begin{macro}{\TeXOSQueryCwd}
%Query the current working directory.
%    \begin{macrocode}
\def\TeXOSQueryCwd#1{\TeXOSQuery{#1}{\string-c}}
%    \end{macrocode}
%\end{macro}
%
%\begin{macro}{\TeXOSQueryHome}
%Query the user's home directory.
%    \begin{macrocode}
\def\TeXOSQueryHome#1{\TeXOSQuery{#1}{\string-m}}
%    \end{macrocode}
%\end{macro}
%
%\begin{macro}{\TeXOSQueryTmpDir}
%Query the temporary directory.
%    \begin{macrocode}
\def\TeXOSQueryTmpDir#1{\TeXOSQuery{#1}{\string-t}}
%    \end{macrocode}
%\end{macro}
%
%\begin{macro}{\TeXOSQueryVersion}
%Query the operating system version.
%    \begin{macrocode}
\def\TeXOSQueryVersion#1{\TeXOSQuery{#1}{\string-r}}
%    \end{macrocode}
%\end{macro}
%
%\begin{macro}{\TeXOSQueryArch}
%Query the operating system architecture.
%    \begin{macrocode}
\def\TeXOSQueryArch#1{\TeXOSQuery{#1}{\string-a}}
%    \end{macrocode}
%\end{macro}
%
%\begin{macro}{\TeXOSQueryName}
%Query the operating system name.
%    \begin{macrocode}
\def\TeXOSQueryName#1{\TeXOSQuery{#1}{\string-o}}
%    \end{macrocode}
%\end{macro}
%
%\begin{macro}{\TeXOSQueryNow}
%\changes{1.1}{2016-07-14}{changed catcode of D to 12}
%Query the current time stamp.
%    \begin{macrocode}
\def\TeXOSQueryNow#1{%
%    \end{macrocode}
% The \texttt{D} needs category code 12. (Don't need to worry about
% \texttt{Z} as \texttt{texosquery.jar} uses \texttt{+00'00'} for
% UTC+0.) This change can't be done with the other catcode changes
% in \cs{TeXOSQuery}, as this is only appropriate for the PDF dates.
% Save and restore the catcode rather than fiddle around with
% scoping.
%    \begin{macrocode}
  \edef\@texosquery@restore@D{%
    \noexpand\catcode`\noexpand\D=\the\catcode`\D\relax}%
  \catcode`\D=12\relax
  \TeXOSQuery{#1}{\string-n}%
  \@texosquery@restore@D
}
%    \end{macrocode}
%\end{macro}
%
%If the file name is supplied using \cs{jobname} it may have
%double-quotes which will interfere with things.
%\begin{macro}{\texosquerystripquotes}
%    \begin{macrocode}
\def\texosquerystripquotes#1{%
  \@texosquery@stripquotes#1\@mid@texosquery@stripquotes
    "\relax"\relax\@end@texosquery@stripquotes
}
\def\@texosquery@stripquotes#1"#2"{%
  \@@texosquery@stripquotes#1#2%
}
\def\@@texosquery@stripquotes#1\@mid@texosquery@stripquotes#2\@end@texosquery@stripquotes{%
  #1%
}
%    \end{macrocode}
%\end{macro}
%
%\begin{macro}{\TeXOSQueryFileDate}
%\changes{1.1}{2016-07-14}{changed catcode of D to 12}
%Query the time stamp of the file given in the second argument.
%    \begin{macrocode}
\def\TeXOSQueryFileDate#1#2{%
  \edef\@texosquery@restore@D{%
    \noexpand\catcode`\noexpand\D=\the\catcode`\D\relax}%
  \catcode`\D=12\relax
  \TeXOSQuery{#1}{\string-d \string'\texosquerystripquotes{#2}\string'}%
  \@texosquery@restore@D
}
%    \end{macrocode}
%\end{macro}
%
%\begin{macro}{\TeXOSQueryFileSize}
%Query the size of the file given in the second argument.
%    \begin{macrocode}
\def\TeXOSQueryFileSize#1#2{\TeXOSQuery{#1}{\string-s 
 \string'\texosquerystripquotes{#2}\string'}}
%    \end{macrocode}
%\end{macro}
%
%\begin{macro}{\TeXOSQueryFileList}
%List all files in the directory given in the third argument,
%separated by the second argument.
%    \begin{macrocode}
\def\TeXOSQueryFileList#1#2#3{\TeXOSQuery{#1}{%
 \string-i \string'#2\string' 
 \string'\texosquerystripquotes{#3}\string'}}
%    \end{macrocode}
%\end{macro}
%
%\begin{macro}{\TeXOSQueryFilterFileList}
%Filtered list files in the directory given in the fourth argument,
%separated by the second argument. The third argument is the regular
%expression used to filter the list. \emph{Take care of backslashes
%in the regular expression!}
%    \begin{macrocode}
\def\TeXOSQueryFilterFileList#1#2#3#4{\TeXOSQuery{#1}{%
 \string-f \string'#2\string' \string'#3\string' 
 \string'\texosquerystripquotes{#4}\string'}}
%    \end{macrocode}
%\end{macro}
%
%\begin{macro}{\TeXOSQueryFileURI}
%Get the URI of the file given in the second argument.
%    \begin{macrocode}
\def\TeXOSQueryFileURI#1#2{\TeXOSQuery{#1}{\string-u 
 \string'\texosquerystripquotes{#2}\string'}}
%    \end{macrocode}
%\end{macro}
%
%\begin{macro}{\TeXOSQueryFilePath}
%Get the canonical path of the file given in the second argument.
%    \begin{macrocode}
\def\TeXOSQueryFilePath#1#2{\TeXOSQuery{#1}{\string-p 
 \string'\texosquerystripquotes{#2}\string'}}
%    \end{macrocode}
%\end{macro}
%
%\begin{macro}{\TeXOSQueryDirName}
%\changes{1.1}{2016-07-14}{new}
%Get the canonical path of the directory containing the file given 
%in the second argument.
%    \begin{macrocode}
\def\TeXOSQueryDirName#1#2{\TeXOSQuery{#1}{\string-e 
 \string'\texosquerystripquotes{#2}\string'}}
%    \end{macrocode}
%\end{macro}
%
%All done.
%Restore the category code of \texttt{@}:
%    \begin{macrocode}
\@texosquery@restore@at
%    \end{macrocode}
%\iffalse
%    \begin{macrocode}
%</texosquery.tex>
%    \end{macrocode}
%\fi
%
%\subsection{\LaTeX\ Code}
%This is just a simple wrapper for \texttt{texosquery.tex}
%so that it can be loaded using \LaTeX's standard \cs{usepackage}
%method.
%\iffalse
%    \begin{macrocode}
%<*texosquery.sty>
%    \end{macrocode}
%\fi
%Identify package:
%    \begin{macrocode}
\NeedsTeXFormat{LaTeX2e}
\ProvidesPackage{texosquery}[2016/07/14 v1.1 (NLCT)]
%    \end{macrocode}
%Load \texttt{texosquery.tex}:
%    \begin{macrocode}
%\iffalse
%<*package>
%% \CharacterTable
%%  {Upper-case    \A\B\C\D\E\F\G\H\I\J\K\L\M\N\O\P\Q\R\S\T\U\V\W\X\Y\Z
%%   Lower-case    \a\b\c\d\e\f\g\h\i\j\k\l\m\n\o\p\q\r\s\t\u\v\w\x\y\z
%%   Digits        \0\1\2\3\4\5\6\7\8\9
%%   Exclamation   \!     Double quote  \"     Hash (number) \#
%%   Dollar        \$     Percent       \%     Ampersand     \&
%%   Acute accent  \'     Left paren    \(     Right paren   \)
%%   Asterisk      \*     Plus          \+     Comma         \,
%%   Minus         \-     Point         \.     Solidus       \/
%%   Colon         \:     Semicolon     \;     Less than     \<
%%   Equals        \=     Greater than  \>     Question mark \?
%%   Commercial at \@     Left bracket  \[     Backslash     \\
%%   Right bracket \]     Circumflex    \^     Underscore    \_
%%   Grave accent  \`     Left brace    \{     Vertical bar  \|
%%   Right brace   \}     Tilde         \~}
%</package>
%\fi
% \iffalse
% Doc-Source file
% Copyright (C) 2016 Nicola Talbot, all rights reserved.
% \fi
% \iffalse
%<*driver>
\documentclass{ltxdoc}

\usepackage[T1]{fontenc}
\usepackage{tgtermes}
\usepackage{metalogo}
\usepackage[colorlinks,hyperindex=false]{hyperref}

\CheckSum{273}

\RecordChanges
\PageIndex
\CodelineNumbered
\IndexPrologue{}

\renewcommand*{\usage}[1]{\hyperpage{#1}}
\renewcommand*{\main}[1]{\hyperpage{#1}}

\newcommand*{\shortargfmt}[1]{\texttt{-#1}}
\newcommand*{\longargfmt}[1]{\texttt{-{}-#1}}

\newcommand*{\shortarg}[1]{\shortargfmt{#1}%
 \index{texosquery options=\texttt{texosquery} options>#1=\protect\shortargfmt{-#1}|hyperpage}}
\newcommand*{\longarg}[1]{\longargfmt{#1}%
 \index{texosquery options=\texttt{texosquery} options>#1=\protect\longargfmt{#1}|hyperpage}}

\newenvironment{important}
 {\begin{quote}\textbf{Important Note:}}
 {\end{quote}}

\begin{document}
\DocInput{texosquery.dtx}
\end{document}
%</driver>
%\fi
%
%\DeleteShortVerb{|}
%
%\title{texosquery: query OS information from \TeX}
%\author{Nicola L. C. Talbot\\
%\href{http://www.dickimaw-books.com/}{\nolinkurl{dickimaw-books.com}}
%\and Paulo Cereda}
%\date{2016-07-14 (v1.1)}
%\maketitle
%
%\begin{abstract}
%\texttt{texosquery.jar} is a cross-platform Java application
%to query certain OS information designed for use in \TeX's
%shell escape mechanism.
%The accompanying \TeX\ code provides a simple interface to the
%\texttt{texosquery} Java application, which may be skipped
%if you want to explicitly use \cs{input} (but take
%care of any special characters appearing in the result).
%\end{abstract}
%
%\begin{important}
%You will need \TeX's shell escape enabled,
%and you will also need the Java Runtime Environment (JRE) 
%installed to use \texttt{texosquery}.
%\end{important}
%
%If you want to rebuild the application, instructions for
%compiling the source code (including the code for this document)
%are in the accompanying \texttt{README.md} file.
%
%\tableofcontents
%
%\section{texosquery.jar: the Java application}
%The \texttt{texosquery} Java command line application looks up
%certain system information that may be of use in \TeX\ 
%documents. All this information can easily be obtained using 
%native commands, but the Java application allows an 
%OS-independent approach with results that can easily be
%captured by \TeX's shell-escape without having to strip 
%formatting information.
%
%\begin{important}
%\texttt{texosquery} provides read-only actions, and I don't 
%intend adding any actions that modify system settings or files.
%\end{important}
%
%Since the application is designed to work with \TeX, each
%function will display the result on a single line without
%formatting. (For multiple results, each line is grouped 
%from v1.1.) A blank line (or empty group) will be displayed
%if the information isn't available. A forward slash is always 
%used as a directory divider, regardless of the operating 
%system, so the result can be used, for example, in \cs{input} or
%\cs{includegraphics}.
%
%If an input file name is required (for example, with the
%\longargfmt{pdfdate} argument described below) then the file may be in
%the current working directory, relative to the current directory
%(with forward slash \texttt{/} as the directory divider), an absolute path 
%(again with forward slash) or on \TeX's path (in
%which case, \texttt{kpsewhich} is used to locate it).
%
%Command line invocation:
%\begin{flushleft}\ttfamily
%texosquery \meta{action} ...
%\end{flushleft}
%Available actions (at least one required):
%\begin{description}
%\item[\shortarg{L} or \longarg{locale}] Display the locale
%information in the form
%\begin{flushleft}\ttfamily
%\meta{lang}-\meta{region}.\meta{codeset}@\meta{modifier}
%\end{flushleft}
%where \meta{lang} is the ISO code for the language (e.g.\ 
%\texttt{en}),
%\meta{region} is the ISO code for the region (e.g.\ \texttt{GB}),
%\meta{codeset} is the default code set (e.g.\ \texttt{UTF-8})
% and \meta{modifier} is the modifier. Elements may be omitted
% if unavailable. For example, \texttt{en-GB.UTF-8} has the
% \meta{modifier} omitted, and \texttt{en} has all but the language
% omitted.
%\item[\shortarg{l} or \longarg{locale-lcs}] As the above but the
%codeset (if present) is converted to lower case and any hyphens are
%stripped. For example, if \longargfmt{locale} returns
%\texttt{en-GB.UTF-8}, then \longargfmt{locale-lcs} would return
%\texttt{en-GB.utf8}.
%\item[\shortarg{b} or \longarg{bcp47}] (New to version 1.2.)
%Displays the BCP~47 language tag.
%
%\item[\shortarg{c} or \longarg{cwd}] Displays the current working
%directory.
%\item[\shortarg{m} or \longarg{userhome}] Displays the user's home
%directory.
%\item[\shortarg{t} or \longarg{tmpdir}] Displays the temporary
%directory.
%\item[\shortarg{o} or \longarg{osname}] Displays the operating
%system name.
%\item[\shortarg{r} or \longarg{osversion}] Displays the operating
%system version.
%\item[\shortarg{a} or \longarg{osarch}] Displays the operating
%system architecture.
%\item[\shortarg{n} or \longarg{pdfnow}] Displays the current
% date and time in PDF format. For example 
%\begin{verbatim}
%D:20160704131006+01'00'
%\end{verbatim}
%Note that some, but not all, \TeX\ formats provide
%\cs{pdfcreationdate}, which is more efficient than using the shell
%escape, but this can be used
%as a fallback method for those that don't (for example, \XeTeX).
%\item[\shortarg{d} \meta{file} or \longarg{pdfdate} \meta{file}]
% Displays the last modified time stamp of the given file in PDF format or a blank line
% if the file doesn't exist or the file permissions prohibit this
% action.
%Again some, but not all, \TeX\ formats provide
%\cs{pdffilemoddate}\marg{file}, which is more efficient than using
%the shell escape.
%\item[\shortarg{s} \meta{file} or \longarg{filesize} \meta{file}]
% Displays the size in bytes of the given file or a blank line
% if the file doesn't exist or the file permissions prohibit this
% action.
%Some, but not all, \TeX\ formats provide
%\cs{pdffilesize}\marg{file}, which is more efficient than using
%the shell escape.
%\item[\shortarg{i} \meta{sep} \meta{dir} or \longarg{list}
%\meta{sep} \meta{dir}]
%List all files in the given directory with the output on a single
%line using \meta{sep} as the separator between entries. Note that
%the list doesn't include the full path, just the file names.
%\item[\shortarg{f} \meta{sep} \meta{regex} \meta{dir} or
%\longarg{filterlist} \meta{sep} \meta{regex} \meta{dir}]
%Like the above but only lists those files whose name matches the
%regular expression given in \meta{regex}. Note that this tests for a
%complete match on the file name (not including path). For example,
%if \meta{regexp} is \texttt{foo.*}, it will only match files whose
%name starts with \texttt{foo} (for example, \texttt{foobar} will
%match but \texttt{barfoo} won't).
%\item[\shortarg{u} \meta{file} or \longarg{uri} \meta{file}]
%Displays the URI of the given file or a blank line if the file
%doesn't exist or the file permissions prohibit this action.
%Note that this may include in a percent character in the result. The \TeX\ 
%command \cs{TeXOSQuery} protects against this by changing the
%category code, but if you explicitly call \texttt{texosquery}
%using the shell escape, you'll need to be careful of this.
%\item[\shortarg{p} \meta{file} or \longarg{path} \meta{file}]
%Displays the canonical path of the given file or a blank line if
%the file doesn't exist or the file permissions prohibit this
%action.
%\item[\shortarg{e} \meta{file} or \longarg{dirname} \meta{file}]
%(New to v1.1.)
%Displays the canonical path of the given file's parent (that is, the
%directory containing \meta{file}) or a blank line if
%the file doesn't exist or the file permissions prohibit this
%action. Note that this is different to the Unix-like 
%\texttt{dirname} command, which will return a relative path if
%\meta{file} isn't an absolute path.
%\item[\shortarg{h} or \longarg{help}] Displays help message and
%exits.
%\item[\shortarg{v} or \longarg{version}] Displays version
%information and exits.
%\end{description}
%
%If multiple options are given, they will be processed in the
%order specified in the command line invocation. Each result will
%be displayed on a separate line. As from v1.1, if there are
%multiple actions, each result will be grouped. This makes it easier
%to process the results in \TeX. For example:
%\begin{verbatim}
%texosquery -l
%\end{verbatim}
%This just produces (for me):
%\begin{verbatim}
%en-GB.utf8
%\end{verbatim}
%whereas
%\begin{verbatim}
%texosquery -l -n
%\end{verbatim}
%produces:
%\begin{verbatim}
%{en-GB.utf8}
%{D:20160714112732+01'00'}
%\end{verbatim}
%Note that unavailable information will produce an empty group.
%For example (assuming \texttt{nofile} doesn't exist):
%\begin{verbatim}
%texosquery -l -d nofile -n
%\end{verbatim}
%produces:
%\begin{verbatim}
%{en-GB.utf8}
%{}
%{D:20160714112732+01'00'}
%\end{verbatim}
%whereas 
%\begin{verbatim}
%texosquery -d nofile
%\end{verbatim}
%just displays an empty line.
%
%\section{texosquery.tex: generic \TeX\ code}
%You can run \texttt{texosquery} directly from \TeX's shell escape.
%For example:
%\begin{verbatim}
%\input|"texosquery --locale"
%\end{verbatim}
%However, \texttt{texosquery.tex} provides generic \TeX\ code to do
%this for you and store the result in a control sequence.
%
%Plain \TeX\ users can input this file through the usual \cs{input}
%method:
%\begin{verbatim}
%\input texosquery
%\end{verbatim}
%\LaTeX\ users may also simply input this file:
%\begin{verbatim}
%\input{texosquery}
%\end{verbatim}
%but may prefer the standard package approach:
%\begin{verbatim}
%\usepackage{texosquery}
%\end{verbatim}
%
%\begin{important}
%The commands described below are all fragile.
%\end{important}
%
%The basic command to run \texttt{texosquery} and capture its output
%in a control sequence is:
%\DescribeMacro\TeXOSQuery
%\begin{flushleft}\ttfamily
%\cs{TeXOSQuery}\marg{cs}\marg{args}
%\end{flushleft}
%where \meta{cs} is the control sequence in which to store the
%result and \meta{args} are the command line arguments to pass to 
%\texttt{texosquery}. This first locally changes the category code
%of some problematic characters that may appear in the result.
%The only special characters that aren't changed are the backslash
%\verb|\|, curly braces \verb|{| and \verb|}|, and hash \verb|#|.
%(\texttt{texosquery.jar} will replace \verb|#| with \verb|\#| in places
%where it might possibly occur in the result, but in general it's
%best to avoid these characters in file names.)
%
%There are some short cut commands for convenience, described below.
%If any of these commands cause an error message in the form:
%\begin{verbatim}
%I can't find file `|texosquery'.
%\end{verbatim}
%then check that you have the shell escape on. If the error 
%persists with the shell escape enabled
%(and not restricted), check that \texttt{texosquery} is on 
%your system's path. To do this, open a command prompt or terminal and
%type \texttt{texosquery}. If it isn't installed correctly, there will be a
%message like:
%\begin{verbatim}
%`texosquery' is not recognised
%\end{verbatim}
%or
%\begin{verbatim}
%texosquery: command not found
%\end{verbatim}
%If this happens, check the installation. (Instructions are
%in the accompanying \texttt{README.md} file.)
%
%Dry run mode is determined by the conditional
%\DescribeMacro\ifTeXOSQueryDryRun
%\begin{verbatim}
%\ifTeXOSQueryDryRun
%\end{verbatim}
%If true, the shell escape won't be used and the requested command
%invocation will be printed in the transcript file prefixed with
%\begin{verbatim}
%TeXOSQuery:
%\end{verbatim}
%(the control sequence \meta{cs} will be set to empty).
%This conditional will automatically be switched on unless
%\cs{shellescape} or \cs{pdfshellescape} is 1. (If
%\texttt{texosquery.jar} is later allowed on the restricted list,
%newer versions may change this default.)
%
%If multiple queries are required, it's more efficient to
%perform them all in one go. For example:
%\begin{verbatim}
%\TeXOSQuery{\result}{-l -n -o}
%
%\def\parseresult#1#2#3{%
%  Locale: #1. Now: #2. OS: #3.%
%}
%
%\ifx\result\empty
%  Query failed.
%\else
%  \expandafter\parseresult\result
%\fi
%\end{verbatim}
%(Make sure you have at least v1.1 for this to work correctly.)
%
%The locale (\shortarg{l} or \longarg{locale-lcs}) information can be
%obtained using:
%\DescribeMacro\TeXOSQueryLocale
%\begin{flushleft}\ttfamily
%\cs{TeXOSQueryLocale}\marg{cs}
%\end{flushleft}
%Note that this uses the lower case codeset form, which has a better
%chance of matching the encoding names used by the \texttt{inputenc}
%package. If you want the unprocessed codeset name, you can do:
%\begin{flushleft}\ttfamily
%\cs{TeXOSQuery}\marg{cs}\{\shortarg{L}\}
%\end{flushleft}
%
%The BCP 47 language tag (\shortarg{b} or \longarg{bcp47}) can be obtained using:
%\DescribeMacro\TeXOSQueryLangTag
%\begin{flushleft}\ttfamily
%\cs{TeXOSQueryLangTag}\marg{cs}
%\end{flushleft}
%
%The current working directory (\shortarg{c} or \longarg{cwd}) can
%be obtained using:
%\DescribeMacro\TeXOSQueryCwd
%\begin{flushleft}\ttfamily
%\cs{TeXOSQueryCwd}\marg{cs}
%\end{flushleft}
%
%The home directory (\shortarg{m} or \longarg{userhome}) can
%be obtained using:
%\DescribeMacro\TeXOSQueryHome
%\begin{flushleft}\ttfamily
%\cs{TeXOSQueryHome}\marg{cs}
%\end{flushleft}
%
%The temporary directory (\shortarg{t} or \longarg{tmpdir}) can
%be obtained using:
%\DescribeMacro\TeXOSQueryTmpDir
%\begin{flushleft}\ttfamily
%\cs{TeXOSQueryTmpDir}\marg{cs}
%\end{flushleft}
%
%The OS name (\shortarg{o} or \longarg{osname}) can
%be obtained using:
%\DescribeMacro\TeXOSQueryName
%\begin{flushleft}\ttfamily
%\cs{TeXOSQueryName}\marg{cs}
%\end{flushleft}
%
%The OS version (\shortarg{r} or \longarg{osversion}) can
%be obtained using:
%\DescribeMacro\TeXOSQueryVersion
%\begin{flushleft}\ttfamily
%\cs{TeXOSQueryVersion}\marg{cs}
%\end{flushleft}
%
%The OS architecture (\shortarg{a} or \longarg{osarch}) can
%be obtained using:
%\DescribeMacro\TeXOSQueryArch
%\begin{flushleft}\ttfamily
%\cs{TeXOSQueryArch}\marg{cs}
%\end{flushleft}
%
%The current date-time stamp in PDF format (\shortarg{n} or
%\longarg{pdfnow}) can be obtained using:
%\DescribeMacro\TeXOSQueryNow
%\begin{flushleft}\ttfamily
%\cs{TeXOSQueryNow}\marg{cs}
%\end{flushleft}
%This is provided for the benefit of users who don't have
%\cs{pdfcreationdate} defined by their \TeX\ format (for example,
%\XeTeX). As from v1.1, this ensures that the initial \texttt{D} has
%category code 12 (which won't happen if \cs{TeXOSQuery} is used
%explicitly).
%
%\begin{important}
%The remaining commands all require extra arguments after the
%relevant switch. These are automatically enclosed in single-quotes
%to protect any spaces. If the argument actually contains any single-quote
%characters, make sure you use \verb|\string\'| to prevent
%interference. \emph{However, in general it's a very bad idea to use
%quotes as part of a file name (rather than using them as
%delimiters).}
%Since the file name may need to be obtained from \cs{jobname},
%which sometimes includes double-quotes, the first double-quote pair
%found is stripped in file name arguments. Any other double-quotes
%will need to be protected in the same manner as single-quotes
%(but, again, this shouldn't be an issue if you use a safe
%file naming scheme). Any paths should use a
%forward slash for the directory divider.
%\end{important}
%
%The modification date-time stamp in PDF format for a file
%(\shortarg{d} or \longarg{pdfdate}) can be obtained using:
%\DescribeMacro\TeXOSQueryFileDate
%\begin{flushleft}\ttfamily
%\cs{TeXOSQueryFileDate}\marg{cs}\marg{filename}
%\end{flushleft}
%where \meta{filename} is the name of the file.
%This is provided for the benefit of users who don't have
%\cs{pdffilemoddate} defined by their \TeX\ format. As from v1.1,
%this ensures that the initial \texttt{D} has category code 12 (which 
%won't happen if \cs{TeXOSQuery} is used explicitly).
%
%The size in bytes of a file
%(\shortarg{s} or \longarg{filesize}) can be obtained using:
%\DescribeMacro\TeXOSQueryFileSize
%\begin{flushleft}\ttfamily
%\cs{TeXOSQueryFileSize}\marg{cs}\marg{filename}
%\end{flushleft}
%where \meta{filename} is the name of the file.
%This is provided for the benefit of users who don't have
%\cs{pdffilesize} defined by their \TeX\ format.
%
%The list of files in a given directory
%(\shortarg{i} or \longarg{list}) can be obtained using:
%\DescribeMacro\TeXOSQueryFileList
%\begin{flushleft}\ttfamily
%\cs{TeXOSQueryFileList}\marg{cs}\marg{sep}\marg{dir}
%\end{flushleft}
%where \meta{sep} is the separator and \meta{dir} is the directory name.
%For example:
%\begin{verbatim}
%\TeXOSQueryFileList{\result}{,}{.}
%\end{verbatim}
%will store a comma-separated list of all the files contained in the
%current directory in the control sequence \cs{result}.
%
%A filtered list of files in a given directory
%(\shortarg{f} or \longarg{filterlist}) can be obtained using:
%\DescribeMacro\TeXOSQueryFilterFileList
%\begin{flushleft}\ttfamily
%\cs{TeXOSQueryFilterFileList}\marg{cs}\marg{sep}\marg{regex}\marg{dir}
%\end{flushleft}
%where \meta{regex} is a regular expression. \emph{Take care of any backslashes
%in the regular expression!}
%For example, to list only those files that have an extension:
%\begin{verbatim}
%\TeXOSQueryFilterFileList{\result}{,}{.+\string\..*}{.}
%\end{verbatim}
%Note the use of \verb|\string\.| to ensure that \cs{.} isn't
%interpreted as a command. Another example, list only \texttt{.png}
%and \texttt{.jpg} files in the directory called \texttt{images}:
%\begin{verbatim}
%\TeXOSQueryFilterFileList{\result}{,}{.+\string\.(jpg|png)}{images}
%\end{verbatim}
%
%The URI of a file
%(\shortarg{u} or \longarg{uri}) can be obtained using:
%\DescribeMacro\TeXOSQueryFileURI
%\begin{flushleft}\ttfamily
%\cs{TeXOSQueryFileURI}\marg{cs}\marg{filename}
%\end{flushleft}
%where \meta{filename} is the name of the file.
%(Any percent symbols \verb|%| contained in the URI will have
%their category code set to 12.)
%
%The canonical path of a file
%(\shortarg{p} or \longarg{path}) can be obtained using:
%\DescribeMacro\TeXOSQueryFilePath
%\begin{flushleft}\ttfamily
%\cs{TeXOSQueryFilePath}\marg{cs}\marg{filename}
%\end{flushleft}
%where \meta{filename} is the name of the file.
%
%The canonical path of a file's parent
%(\shortarg{e} or \longarg{dirname}) can be obtained using:
%\DescribeMacro\TeXOSQueryDirName
%\begin{flushleft}\ttfamily
%\cs{TeXOSQueryDirName}\marg{cs}\marg{filename}
%\end{flushleft}
%where \meta{filename} is the name of the file.
%
%\StopEventually{\phantomsection
%\addcontentsline{toc}{section}{Change History}%
%\PrintChanges
%\addcontentsline{toc}{section}{\indexname}%
%\PrintIndex
%}
%
%\section{The Code}
%\changes{1.0}{2016-07-08}{Initial release}
%\subsection{Generic \TeX\ Code}
%\iffalse
%    \begin{macrocode}
%<*texosquery.tex>
%    \end{macrocode}
%\fi
%Change category code of \texttt{@} if necessary.
%    \begin{macrocode}
\ifnum\catcode`\@=11\relax
  \def\@texosquery@restore@at{}%
\else
  \expandafter\edef\csname @texosquery@restore@at\endcsname{%
    \noexpand\catcode`\noexpand\@=\number\catcode`\@\relax
  }%
 \catcode`\@=11\relax
\fi
%    \end{macrocode}
% Check if already loaded.
%    \begin{macrocode}
\ifx\TeXOSQuery\undefined
\else
  \@texosquery@restore@at
  \expandafter\endinput
\fi
%    \end{macrocode}
%\begin{macro}{\TeXOSInvokerName}
%The name of the \texttt{texosquery} application.
%    \begin{macrocode}
\def\TeXOSInvokerName{texosquery}
%    \end{macrocode}
%\end{macro}
%
%\begin{macro}{\TeXOSQueryInvoker}
%If we're using \LaTeX, we'll need to use \cs{@@input} rather
%than \cs{input}.
%    \begin{macrocode}
\ifx\@@input\undefined
  \def\TeXOSQueryInvoker#1{\input|"\TeXOSInvokerName\space#1" }
\else
  \def\TeXOSQueryInvoker#1{\@@input|"\TeXOSInvokerName\space#1" }
\fi
%    \end{macrocode}
%\end{macro}
%
%\begin{macro}{\ifTeXOSQueryDryRun}
%Provide a dry-run mode.
%\changes{1.1}{2016-07-14}{dry run mode only false by default if with
%unrestricted mode}
%    \begin{macrocode}
\newif\ifTeXOSQueryDryRun
\TeXOSQueryDryRuntrue
%    \end{macrocode}
%\end{macro}
%
%If shell escape is unrestricted, automatically switch off dry-run mode.
%    \begin{macrocode}
\ifx\shellescape\undefined
  \ifx\pdfshellescape\undefined
  \else
    \ifnum\pdfshellescape=1\relax
      \TeXOSQueryDryRunfalse
    \fi
  \fi
\else
  \ifnum\shellescape=1\relax
    \TeXOSQueryDryRunfalse
  \fi
\fi
%    \end{macrocode}
%
%\begin{macro}{\TeXOSQuery}
%Use \texttt{texosquery} with the option given in the second
%argument and store the result in control sequence given in the
%first argument.
%    \begin{macrocode}
\def\TeXOSQuery#1#2{%
  \ifTeXOSQueryDryRun
    \begingroup
      \newlinechar=`\^^J
      \message{^^JTeXOSQuery: \TeXOSInvokerName\space#2^^J}%
    \endgroup
    \def#1{}%
  \else
    \begingroup
    \endlinechar=-1\relax
%    \end{macrocode}
%Just in case the result contains any awkward characters that have
%some special meaning to \TeX. (Can't really protect against hash,
%backslash or curly braces, but they're unlikely to occur unless
%the user has a very wacky and inappropriate file naming scheme.)
%    \begin{macrocode}
    \catcode`\-=12\relax
    \catcode`\_=12\relax
    \catcode`\^=12\relax
    \catcode`\~=12\relax
    \catcode`\$=12\relax
    \catcode`\&=12\relax
    \catcode`\"=12\relax
    \catcode`\'=12\relax
    \catcode`\.=12\relax
    \catcode`\/=12\relax
    \catcode`\:=12\relax
    \catcode`\;=12\relax
    \catcode`\%=12\relax
    \everyeof{\noexpand}\relax
    \edef\x{\endgroup\def\noexpand#1{\TeXOSQueryInvoker{#2}}}\x
  \fi
}
%    \end{macrocode}
%\end{macro}
%
%Now for some convenient shortcuts so the user doesn't have to
%remember the command line options. \cs{string} is used in
%case the hyphen character has been made active.
%\begin{macro}{\TeXOSQueryLocale}
%Query the locale and store the result in the control sequence
%provided in the argument.
%    \begin{macrocode}
\def\TeXOSQueryLocale#1{\TeXOSQuery{#1}{\string-l}}
%    \end{macrocode}
%\end{macro}
%
%\begin{macro}{\TeXOSQueryLangTag}
%Query the language tag and store the result in the control sequence
%provided in the argument.
%\changes{1.2}{??}{new}
%    \begin{macrocode}
\def\TeXOSQueryLangTag#1{\TeXOSQuery{#1}{\string-b}}
%    \end{macrocode}
%\end{macro}
%
%\begin{macro}{\TeXOSQueryCwd}
%Query the current working directory.
%    \begin{macrocode}
\def\TeXOSQueryCwd#1{\TeXOSQuery{#1}{\string-c}}
%    \end{macrocode}
%\end{macro}
%
%\begin{macro}{\TeXOSQueryHome}
%Query the user's home directory.
%    \begin{macrocode}
\def\TeXOSQueryHome#1{\TeXOSQuery{#1}{\string-m}}
%    \end{macrocode}
%\end{macro}
%
%\begin{macro}{\TeXOSQueryTmpDir}
%Query the temporary directory.
%    \begin{macrocode}
\def\TeXOSQueryTmpDir#1{\TeXOSQuery{#1}{\string-t}}
%    \end{macrocode}
%\end{macro}
%
%\begin{macro}{\TeXOSQueryVersion}
%Query the operating system version.
%    \begin{macrocode}
\def\TeXOSQueryVersion#1{\TeXOSQuery{#1}{\string-r}}
%    \end{macrocode}
%\end{macro}
%
%\begin{macro}{\TeXOSQueryArch}
%Query the operating system architecture.
%    \begin{macrocode}
\def\TeXOSQueryArch#1{\TeXOSQuery{#1}{\string-a}}
%    \end{macrocode}
%\end{macro}
%
%\begin{macro}{\TeXOSQueryName}
%Query the operating system name.
%    \begin{macrocode}
\def\TeXOSQueryName#1{\TeXOSQuery{#1}{\string-o}}
%    \end{macrocode}
%\end{macro}
%
%\begin{macro}{\TeXOSQueryNow}
%\changes{1.1}{2016-07-14}{changed catcode of D to 12}
%Query the current time stamp.
%    \begin{macrocode}
\def\TeXOSQueryNow#1{%
%    \end{macrocode}
% The \texttt{D} needs category code 12. (Don't need to worry about
% \texttt{Z} as \texttt{texosquery.jar} uses \texttt{+00'00'} for
% UTC+0.) This change can't be done with the other catcode changes
% in \cs{TeXOSQuery}, as this is only appropriate for the PDF dates.
% Save and restore the catcode rather than fiddle around with
% scoping.
%    \begin{macrocode}
  \edef\@texosquery@restore@D{%
    \noexpand\catcode`\noexpand\D=\the\catcode`\D\relax}%
  \catcode`\D=12\relax
  \TeXOSQuery{#1}{\string-n}%
  \@texosquery@restore@D
}
%    \end{macrocode}
%\end{macro}
%
%If the file name is supplied using \cs{jobname} it may have
%double-quotes which will interfere with things.
%\begin{macro}{\texosquerystripquotes}
%    \begin{macrocode}
\def\texosquerystripquotes#1{%
  \@texosquery@stripquotes#1\@mid@texosquery@stripquotes
    "\relax"\relax\@end@texosquery@stripquotes
}
\def\@texosquery@stripquotes#1"#2"{%
  \@@texosquery@stripquotes#1#2%
}
\def\@@texosquery@stripquotes#1\@mid@texosquery@stripquotes#2\@end@texosquery@stripquotes{%
  #1%
}
%    \end{macrocode}
%\end{macro}
%
%\begin{macro}{\TeXOSQueryFileDate}
%\changes{1.1}{2016-07-14}{changed catcode of D to 12}
%Query the time stamp of the file given in the second argument.
%    \begin{macrocode}
\def\TeXOSQueryFileDate#1#2{%
  \edef\@texosquery@restore@D{%
    \noexpand\catcode`\noexpand\D=\the\catcode`\D\relax}%
  \catcode`\D=12\relax
  \TeXOSQuery{#1}{\string-d \string'\texosquerystripquotes{#2}\string'}%
  \@texosquery@restore@D
}
%    \end{macrocode}
%\end{macro}
%
%\begin{macro}{\TeXOSQueryFileSize}
%Query the size of the file given in the second argument.
%    \begin{macrocode}
\def\TeXOSQueryFileSize#1#2{\TeXOSQuery{#1}{\string-s 
 \string'\texosquerystripquotes{#2}\string'}}
%    \end{macrocode}
%\end{macro}
%
%\begin{macro}{\TeXOSQueryFileList}
%List all files in the directory given in the third argument,
%separated by the second argument.
%    \begin{macrocode}
\def\TeXOSQueryFileList#1#2#3{\TeXOSQuery{#1}{%
 \string-i \string'#2\string' 
 \string'\texosquerystripquotes{#3}\string'}}
%    \end{macrocode}
%\end{macro}
%
%\begin{macro}{\TeXOSQueryFilterFileList}
%Filtered list files in the directory given in the fourth argument,
%separated by the second argument. The third argument is the regular
%expression used to filter the list. \emph{Take care of backslashes
%in the regular expression!}
%    \begin{macrocode}
\def\TeXOSQueryFilterFileList#1#2#3#4{\TeXOSQuery{#1}{%
 \string-f \string'#2\string' \string'#3\string' 
 \string'\texosquerystripquotes{#4}\string'}}
%    \end{macrocode}
%\end{macro}
%
%\begin{macro}{\TeXOSQueryFileURI}
%Get the URI of the file given in the second argument.
%    \begin{macrocode}
\def\TeXOSQueryFileURI#1#2{\TeXOSQuery{#1}{\string-u 
 \string'\texosquerystripquotes{#2}\string'}}
%    \end{macrocode}
%\end{macro}
%
%\begin{macro}{\TeXOSQueryFilePath}
%Get the canonical path of the file given in the second argument.
%    \begin{macrocode}
\def\TeXOSQueryFilePath#1#2{\TeXOSQuery{#1}{\string-p 
 \string'\texosquerystripquotes{#2}\string'}}
%    \end{macrocode}
%\end{macro}
%
%\begin{macro}{\TeXOSQueryDirName}
%\changes{1.1}{2016-07-14}{new}
%Get the canonical path of the directory containing the file given 
%in the second argument.
%    \begin{macrocode}
\def\TeXOSQueryDirName#1#2{\TeXOSQuery{#1}{\string-e 
 \string'\texosquerystripquotes{#2}\string'}}
%    \end{macrocode}
%\end{macro}
%
%All done.
%Restore the category code of \texttt{@}:
%    \begin{macrocode}
\@texosquery@restore@at
%    \end{macrocode}
%\iffalse
%    \begin{macrocode}
%</texosquery.tex>
%    \end{macrocode}
%\fi
%
%\subsection{\LaTeX\ Code}
%This is just a simple wrapper for \texttt{texosquery.tex}
%so that it can be loaded using \LaTeX's standard \cs{usepackage}
%method.
%\iffalse
%    \begin{macrocode}
%<*texosquery.sty>
%    \end{macrocode}
%\fi
%Identify package:
%    \begin{macrocode}
\NeedsTeXFormat{LaTeX2e}
\ProvidesPackage{texosquery}[2016/07/14 v1.1 (NLCT)]
%    \end{macrocode}
%Load \texttt{texosquery.tex}:
%    \begin{macrocode}
\input{texosquery}
%    \end{macrocode}
%That's it!
%\iffalse
%    \begin{macrocode}
%</texosquery.sty>
%    \end{macrocode}
%\fi
%
%\Finale
\endinput

%    \end{macrocode}
%That's it!
%\iffalse
%    \begin{macrocode}
%</texosquery.sty>
%    \end{macrocode}
%\fi
%
%\Finale
\endinput

%\end{verbatim}
%but may prefer the standard package approach:
%\begin{verbatim}
%\usepackage{texosquery}
%\end{verbatim}
%
%\begin{important}
%The commands described below are all fragile.
%\end{important}
%
%The basic command to run \texttt{texosquery} and capture its output
%in a control sequence is:
%\DescribeMacro\TeXOSQuery
%\begin{flushleft}\ttfamily
%\cs{TeXOSQuery}\marg{cs}\marg{args}
%\end{flushleft}
%where \meta{cs} is the control sequence in which to store the
%result and \meta{args} are the command line arguments to pass to 
%\texttt{texosquery}. This first locally changes the category code
%of some problematic characters that may appear in the result.
%The only special characters that aren't changed are the backslash
%\verb|\|, curly braces \verb|{| and \verb|}|, and hash \verb|#|.
%(\texttt{texosquery.jar} will replace \verb|#| with \verb|\#| in places
%where it might possibly occur in the result, but in general it's
%best to avoid these characters in file names.)
%
%There are some short cut commands for convenience, described below.
%If any of these commands cause an error message in the form:
%\begin{verbatim}
%I can't find file `|texosquery'.
%\end{verbatim}
%then check that you have the shell escape on. If the error 
%persists with the shell escape enabled
%(and not restricted), check that \texttt{texosquery} is on 
%your system's path. To do this, open a command prompt or terminal and
%type \texttt{texosquery}. If it isn't installed correctly, there will be a
%message like:
%\begin{verbatim}
%`texosquery' is not recognised
%\end{verbatim}
%or
%\begin{verbatim}
%texosquery: command not found
%\end{verbatim}
%If this happens, check the installation. (Instructions are
%in the accompanying \texttt{README.md} file.)
%
%Dry run mode is determined by the conditional
%\DescribeMacro\ifTeXOSQueryDryRun
%\begin{verbatim}
%\ifTeXOSQueryDryRun
%\end{verbatim}
%If true, the shell escape won't be used and the requested command
%invocation will be printed in the transcript file prefixed with
%\begin{verbatim}
%TeXOSQuery:
%\end{verbatim}
%(the control sequence \meta{cs} will be set to empty).
%This conditional will automatically be switched on unless
%\cs{shellescape} or \cs{pdfshellescape} is 1. (If
%\texttt{texosquery.jar} is later allowed on the restricted list,
%newer versions may change this default.)
%
%If multiple queries are required, it's more efficient to
%perform them all in one go. For example:
%\begin{verbatim}
%\TeXOSQuery{\result}{-l -n -o}
%
%\def\parseresult#1#2#3{%
%  Locale: #1. Now: #2. OS: #3.%
%}
%
%\ifx\result\empty
%  Query failed.
%\else
%  \expandafter\parseresult\result
%\fi
%\end{verbatim}
%(Make sure you have at least v1.1 for this to work correctly.)
%
%The locale (\shortarg{l} or \longarg{locale-lcs}) information can be
%obtained using:
%\DescribeMacro\TeXOSQueryLocale
%\begin{flushleft}\ttfamily
%\cs{TeXOSQueryLocale}\marg{cs}
%\end{flushleft}
%Note that this uses the lower case codeset form, which has a better
%chance of matching the encoding names used by the \texttt{inputenc}
%package. If you want the unprocessed codeset name, you can do:
%\begin{flushleft}\ttfamily
%\cs{TeXOSQuery}\marg{cs}\{\shortarg{L}\}
%\end{flushleft}
%
%The BCP 47 language tag (\shortarg{b} or \longarg{bcp47}) can be obtained using:
%\DescribeMacro\TeXOSQueryLangTag
%\begin{flushleft}\ttfamily
%\cs{TeXOSQueryLangTag}\marg{cs}
%\end{flushleft}
%
%The current working directory (\shortarg{c} or \longarg{cwd}) can
%be obtained using:
%\DescribeMacro\TeXOSQueryCwd
%\begin{flushleft}\ttfamily
%\cs{TeXOSQueryCwd}\marg{cs}
%\end{flushleft}
%
%The home directory (\shortarg{m} or \longarg{userhome}) can
%be obtained using:
%\DescribeMacro\TeXOSQueryHome
%\begin{flushleft}\ttfamily
%\cs{TeXOSQueryHome}\marg{cs}
%\end{flushleft}
%
%The temporary directory (\shortarg{t} or \longarg{tmpdir}) can
%be obtained using:
%\DescribeMacro\TeXOSQueryTmpDir
%\begin{flushleft}\ttfamily
%\cs{TeXOSQueryTmpDir}\marg{cs}
%\end{flushleft}
%
%The OS name (\shortarg{o} or \longarg{osname}) can
%be obtained using:
%\DescribeMacro\TeXOSQueryName
%\begin{flushleft}\ttfamily
%\cs{TeXOSQueryName}\marg{cs}
%\end{flushleft}
%
%The OS version (\shortarg{r} or \longarg{osversion}) can
%be obtained using:
%\DescribeMacro\TeXOSQueryVersion
%\begin{flushleft}\ttfamily
%\cs{TeXOSQueryVersion}\marg{cs}
%\end{flushleft}
%
%The OS architecture (\shortarg{a} or \longarg{osarch}) can
%be obtained using:
%\DescribeMacro\TeXOSQueryArch
%\begin{flushleft}\ttfamily
%\cs{TeXOSQueryArch}\marg{cs}
%\end{flushleft}
%
%The current date-time stamp in PDF format (\shortarg{n} or
%\longarg{pdfnow}) can be obtained using:
%\DescribeMacro\TeXOSQueryNow
%\begin{flushleft}\ttfamily
%\cs{TeXOSQueryNow}\marg{cs}
%\end{flushleft}
%This is provided for the benefit of users who don't have
%\cs{pdfcreationdate} defined by their \TeX\ format (for example,
%\XeTeX). As from v1.1, this ensures that the initial \texttt{D} has
%category code 12 (which won't happen if \cs{TeXOSQuery} is used
%explicitly).
%
%\begin{important}
%The remaining commands all require extra arguments after the
%relevant switch. These are automatically enclosed in single-quotes
%to protect any spaces. If the argument actually contains any single-quote
%characters, make sure you use \verb|\string\'| to prevent
%interference. \emph{However, in general it's a very bad idea to use
%quotes as part of a file name (rather than using them as
%delimiters).}
%Since the file name may need to be obtained from \cs{jobname},
%which sometimes includes double-quotes, the first double-quote pair
%found is stripped in file name arguments. Any other double-quotes
%will need to be protected in the same manner as single-quotes
%(but, again, this shouldn't be an issue if you use a safe
%file naming scheme). Any paths should use a
%forward slash for the directory divider.
%\end{important}
%
%The modification date-time stamp in PDF format for a file
%(\shortarg{d} or \longarg{pdfdate}) can be obtained using:
%\DescribeMacro\TeXOSQueryFileDate
%\begin{flushleft}\ttfamily
%\cs{TeXOSQueryFileDate}\marg{cs}\marg{filename}
%\end{flushleft}
%where \meta{filename} is the name of the file.
%This is provided for the benefit of users who don't have
%\cs{pdffilemoddate} defined by their \TeX\ format. As from v1.1,
%this ensures that the initial \texttt{D} has category code 12 (which 
%won't happen if \cs{TeXOSQuery} is used explicitly).
%
%The size in bytes of a file
%(\shortarg{s} or \longarg{filesize}) can be obtained using:
%\DescribeMacro\TeXOSQueryFileSize
%\begin{flushleft}\ttfamily
%\cs{TeXOSQueryFileSize}\marg{cs}\marg{filename}
%\end{flushleft}
%where \meta{filename} is the name of the file.
%This is provided for the benefit of users who don't have
%\cs{pdffilesize} defined by their \TeX\ format.
%
%The list of files in a given directory
%(\shortarg{i} or \longarg{list}) can be obtained using:
%\DescribeMacro\TeXOSQueryFileList
%\begin{flushleft}\ttfamily
%\cs{TeXOSQueryFileList}\marg{cs}\marg{sep}\marg{dir}
%\end{flushleft}
%where \meta{sep} is the separator and \meta{dir} is the directory name.
%For example:
%\begin{verbatim}
%\TeXOSQueryFileList{\result}{,}{.}
%\end{verbatim}
%will store a comma-separated list of all the files contained in the
%current directory in the control sequence \cs{result}.
%
%A filtered list of files in a given directory
%(\shortarg{f} or \longarg{filterlist}) can be obtained using:
%\DescribeMacro\TeXOSQueryFilterFileList
%\begin{flushleft}\ttfamily
%\cs{TeXOSQueryFilterFileList}\marg{cs}\marg{sep}\marg{regex}\marg{dir}
%\end{flushleft}
%where \meta{regex} is a regular expression. \emph{Take care of any backslashes
%in the regular expression!}
%For example, to list only those files that have an extension:
%\begin{verbatim}
%\TeXOSQueryFilterFileList{\result}{,}{.+\string\..*}{.}
%\end{verbatim}
%Note the use of \verb|\string\.| to ensure that \cs{.} isn't
%interpreted as a command. Another example, list only \texttt{.png}
%and \texttt{.jpg} files in the directory called \texttt{images}:
%\begin{verbatim}
%\TeXOSQueryFilterFileList{\result}{,}{.+\string\.(jpg|png)}{images}
%\end{verbatim}
%
%The URI of a file
%(\shortarg{u} or \longarg{uri}) can be obtained using:
%\DescribeMacro\TeXOSQueryFileURI
%\begin{flushleft}\ttfamily
%\cs{TeXOSQueryFileURI}\marg{cs}\marg{filename}
%\end{flushleft}
%where \meta{filename} is the name of the file.
%(Any percent symbols \verb|%| contained in the URI will have
%their category code set to 12.)
%
%The canonical path of a file
%(\shortarg{p} or \longarg{path}) can be obtained using:
%\DescribeMacro\TeXOSQueryFilePath
%\begin{flushleft}\ttfamily
%\cs{TeXOSQueryFilePath}\marg{cs}\marg{filename}
%\end{flushleft}
%where \meta{filename} is the name of the file.
%
%The canonical path of a file's parent
%(\shortarg{e} or \longarg{dirname}) can be obtained using:
%\DescribeMacro\TeXOSQueryDirName
%\begin{flushleft}\ttfamily
%\cs{TeXOSQueryDirName}\marg{cs}\marg{filename}
%\end{flushleft}
%where \meta{filename} is the name of the file.
%
%\StopEventually{\phantomsection
%\addcontentsline{toc}{section}{Change History}%
%\PrintChanges
%\addcontentsline{toc}{section}{\indexname}%
%\PrintIndex
%}
%
%\section{The Code}
%\changes{1.0}{2016-07-08}{Initial release}
%\subsection{Generic \TeX\ Code}
%\iffalse
%    \begin{macrocode}
%<*texosquery.tex>
%    \end{macrocode}
%\fi
%Change category code of \texttt{@} if necessary.
%    \begin{macrocode}
\ifnum\catcode`\@=11\relax
  \def\@texosquery@restore@at{}%
\else
  \expandafter\edef\csname @texosquery@restore@at\endcsname{%
    \noexpand\catcode`\noexpand\@=\number\catcode`\@\relax
  }%
 \catcode`\@=11\relax
\fi
%    \end{macrocode}
% Check if already loaded.
%    \begin{macrocode}
\ifx\TeXOSQuery\undefined
\else
  \@texosquery@restore@at
  \expandafter\endinput
\fi
%    \end{macrocode}
%\begin{macro}{\TeXOSInvokerName}
%The name of the \texttt{texosquery} application.
%    \begin{macrocode}
\def\TeXOSInvokerName{texosquery}
%    \end{macrocode}
%\end{macro}
%
%\begin{macro}{\TeXOSQueryInvoker}
%If we're using \LaTeX, we'll need to use \cs{@@input} rather
%than \cs{input}.
%    \begin{macrocode}
\ifx\@@input\undefined
  \def\TeXOSQueryInvoker#1{\input|"\TeXOSInvokerName\space#1" }
\else
  \def\TeXOSQueryInvoker#1{\@@input|"\TeXOSInvokerName\space#1" }
\fi
%    \end{macrocode}
%\end{macro}
%
%\begin{macro}{\ifTeXOSQueryDryRun}
%Provide a dry-run mode.
%\changes{1.1}{2016-07-14}{dry run mode only false by default if with
%unrestricted mode}
%    \begin{macrocode}
\newif\ifTeXOSQueryDryRun
\TeXOSQueryDryRuntrue
%    \end{macrocode}
%\end{macro}
%
%If shell escape is unrestricted, automatically switch off dry-run mode.
%    \begin{macrocode}
\ifx\shellescape\undefined
  \ifx\pdfshellescape\undefined
  \else
    \ifnum\pdfshellescape=1\relax
      \TeXOSQueryDryRunfalse
    \fi
  \fi
\else
  \ifnum\shellescape=1\relax
    \TeXOSQueryDryRunfalse
  \fi
\fi
%    \end{macrocode}
%
%\begin{macro}{\TeXOSQuery}
%Use \texttt{texosquery} with the option given in the second
%argument and store the result in control sequence given in the
%first argument.
%    \begin{macrocode}
\def\TeXOSQuery#1#2{%
  \ifTeXOSQueryDryRun
    \begingroup
      \newlinechar=`\^^J
      \message{^^JTeXOSQuery: \TeXOSInvokerName\space#2^^J}%
    \endgroup
    \def#1{}%
  \else
    \begingroup
    \endlinechar=-1\relax
%    \end{macrocode}
%Just in case the result contains any awkward characters that have
%some special meaning to \TeX. (Can't really protect against hash,
%backslash or curly braces, but they're unlikely to occur unless
%the user has a very wacky and inappropriate file naming scheme.)
%    \begin{macrocode}
    \catcode`\-=12\relax
    \catcode`\_=12\relax
    \catcode`\^=12\relax
    \catcode`\~=12\relax
    \catcode`\$=12\relax
    \catcode`\&=12\relax
    \catcode`\"=12\relax
    \catcode`\'=12\relax
    \catcode`\.=12\relax
    \catcode`\/=12\relax
    \catcode`\:=12\relax
    \catcode`\;=12\relax
    \catcode`\%=12\relax
    \everyeof{\noexpand}\relax
    \edef\x{\endgroup\def\noexpand#1{\TeXOSQueryInvoker{#2}}}\x
  \fi
}
%    \end{macrocode}
%\end{macro}
%
%Now for some convenient shortcuts so the user doesn't have to
%remember the command line options. \cs{string} is used in
%case the hyphen character has been made active.
%\begin{macro}{\TeXOSQueryLocale}
%Query the locale and store the result in the control sequence
%provided in the argument.
%    \begin{macrocode}
\def\TeXOSQueryLocale#1{\TeXOSQuery{#1}{\string-l}}
%    \end{macrocode}
%\end{macro}
%
%\begin{macro}{\TeXOSQueryLangTag}
%Query the language tag and store the result in the control sequence
%provided in the argument.
%\changes{1.2}{??}{new}
%    \begin{macrocode}
\def\TeXOSQueryLangTag#1{\TeXOSQuery{#1}{\string-b}}
%    \end{macrocode}
%\end{macro}
%
%\begin{macro}{\TeXOSQueryCwd}
%Query the current working directory.
%    \begin{macrocode}
\def\TeXOSQueryCwd#1{\TeXOSQuery{#1}{\string-c}}
%    \end{macrocode}
%\end{macro}
%
%\begin{macro}{\TeXOSQueryHome}
%Query the user's home directory.
%    \begin{macrocode}
\def\TeXOSQueryHome#1{\TeXOSQuery{#1}{\string-m}}
%    \end{macrocode}
%\end{macro}
%
%\begin{macro}{\TeXOSQueryTmpDir}
%Query the temporary directory.
%    \begin{macrocode}
\def\TeXOSQueryTmpDir#1{\TeXOSQuery{#1}{\string-t}}
%    \end{macrocode}
%\end{macro}
%
%\begin{macro}{\TeXOSQueryVersion}
%Query the operating system version.
%    \begin{macrocode}
\def\TeXOSQueryVersion#1{\TeXOSQuery{#1}{\string-r}}
%    \end{macrocode}
%\end{macro}
%
%\begin{macro}{\TeXOSQueryArch}
%Query the operating system architecture.
%    \begin{macrocode}
\def\TeXOSQueryArch#1{\TeXOSQuery{#1}{\string-a}}
%    \end{macrocode}
%\end{macro}
%
%\begin{macro}{\TeXOSQueryName}
%Query the operating system name.
%    \begin{macrocode}
\def\TeXOSQueryName#1{\TeXOSQuery{#1}{\string-o}}
%    \end{macrocode}
%\end{macro}
%
%\begin{macro}{\TeXOSQueryNow}
%\changes{1.1}{2016-07-14}{changed catcode of D to 12}
%Query the current time stamp.
%    \begin{macrocode}
\def\TeXOSQueryNow#1{%
%    \end{macrocode}
% The \texttt{D} needs category code 12. (Don't need to worry about
% \texttt{Z} as \texttt{texosquery.jar} uses \texttt{+00'00'} for
% UTC+0.) This change can't be done with the other catcode changes
% in \cs{TeXOSQuery}, as this is only appropriate for the PDF dates.
% Save and restore the catcode rather than fiddle around with
% scoping.
%    \begin{macrocode}
  \edef\@texosquery@restore@D{%
    \noexpand\catcode`\noexpand\D=\the\catcode`\D\relax}%
  \catcode`\D=12\relax
  \TeXOSQuery{#1}{\string-n}%
  \@texosquery@restore@D
}
%    \end{macrocode}
%\end{macro}
%
%If the file name is supplied using \cs{jobname} it may have
%double-quotes which will interfere with things.
%\begin{macro}{\texosquerystripquotes}
%    \begin{macrocode}
\def\texosquerystripquotes#1{%
  \@texosquery@stripquotes#1\@mid@texosquery@stripquotes
    "\relax"\relax\@end@texosquery@stripquotes
}
\def\@texosquery@stripquotes#1"#2"{%
  \@@texosquery@stripquotes#1#2%
}
\def\@@texosquery@stripquotes#1\@mid@texosquery@stripquotes#2\@end@texosquery@stripquotes{%
  #1%
}
%    \end{macrocode}
%\end{macro}
%
%\begin{macro}{\TeXOSQueryFileDate}
%\changes{1.1}{2016-07-14}{changed catcode of D to 12}
%Query the time stamp of the file given in the second argument.
%    \begin{macrocode}
\def\TeXOSQueryFileDate#1#2{%
  \edef\@texosquery@restore@D{%
    \noexpand\catcode`\noexpand\D=\the\catcode`\D\relax}%
  \catcode`\D=12\relax
  \TeXOSQuery{#1}{\string-d \string'\texosquerystripquotes{#2}\string'}%
  \@texosquery@restore@D
}
%    \end{macrocode}
%\end{macro}
%
%\begin{macro}{\TeXOSQueryFileSize}
%Query the size of the file given in the second argument.
%    \begin{macrocode}
\def\TeXOSQueryFileSize#1#2{\TeXOSQuery{#1}{\string-s 
 \string'\texosquerystripquotes{#2}\string'}}
%    \end{macrocode}
%\end{macro}
%
%\begin{macro}{\TeXOSQueryFileList}
%List all files in the directory given in the third argument,
%separated by the second argument.
%    \begin{macrocode}
\def\TeXOSQueryFileList#1#2#3{\TeXOSQuery{#1}{%
 \string-i \string'#2\string' 
 \string'\texosquerystripquotes{#3}\string'}}
%    \end{macrocode}
%\end{macro}
%
%\begin{macro}{\TeXOSQueryFilterFileList}
%Filtered list files in the directory given in the fourth argument,
%separated by the second argument. The third argument is the regular
%expression used to filter the list. \emph{Take care of backslashes
%in the regular expression!}
%    \begin{macrocode}
\def\TeXOSQueryFilterFileList#1#2#3#4{\TeXOSQuery{#1}{%
 \string-f \string'#2\string' \string'#3\string' 
 \string'\texosquerystripquotes{#4}\string'}}
%    \end{macrocode}
%\end{macro}
%
%\begin{macro}{\TeXOSQueryFileURI}
%Get the URI of the file given in the second argument.
%    \begin{macrocode}
\def\TeXOSQueryFileURI#1#2{\TeXOSQuery{#1}{\string-u 
 \string'\texosquerystripquotes{#2}\string'}}
%    \end{macrocode}
%\end{macro}
%
%\begin{macro}{\TeXOSQueryFilePath}
%Get the canonical path of the file given in the second argument.
%    \begin{macrocode}
\def\TeXOSQueryFilePath#1#2{\TeXOSQuery{#1}{\string-p 
 \string'\texosquerystripquotes{#2}\string'}}
%    \end{macrocode}
%\end{macro}
%
%\begin{macro}{\TeXOSQueryDirName}
%\changes{1.1}{2016-07-14}{new}
%Get the canonical path of the directory containing the file given 
%in the second argument.
%    \begin{macrocode}
\def\TeXOSQueryDirName#1#2{\TeXOSQuery{#1}{\string-e 
 \string'\texosquerystripquotes{#2}\string'}}
%    \end{macrocode}
%\end{macro}
%
%All done.
%Restore the category code of \texttt{@}:
%    \begin{macrocode}
\@texosquery@restore@at
%    \end{macrocode}
%\iffalse
%    \begin{macrocode}
%</texosquery.tex>
%    \end{macrocode}
%\fi
%
%\subsection{\LaTeX\ Code}
%This is just a simple wrapper for \texttt{texosquery.tex}
%so that it can be loaded using \LaTeX's standard \cs{usepackage}
%method.
%\iffalse
%    \begin{macrocode}
%<*texosquery.sty>
%    \end{macrocode}
%\fi
%Identify package:
%    \begin{macrocode}
\NeedsTeXFormat{LaTeX2e}
\ProvidesPackage{texosquery}[2016/07/14 v1.1 (NLCT)]
%    \end{macrocode}
%Load \texttt{texosquery.tex}:
%    \begin{macrocode}
%\iffalse
%<*package>
%% \CharacterTable
%%  {Upper-case    \A\B\C\D\E\F\G\H\I\J\K\L\M\N\O\P\Q\R\S\T\U\V\W\X\Y\Z
%%   Lower-case    \a\b\c\d\e\f\g\h\i\j\k\l\m\n\o\p\q\r\s\t\u\v\w\x\y\z
%%   Digits        \0\1\2\3\4\5\6\7\8\9
%%   Exclamation   \!     Double quote  \"     Hash (number) \#
%%   Dollar        \$     Percent       \%     Ampersand     \&
%%   Acute accent  \'     Left paren    \(     Right paren   \)
%%   Asterisk      \*     Plus          \+     Comma         \,
%%   Minus         \-     Point         \.     Solidus       \/
%%   Colon         \:     Semicolon     \;     Less than     \<
%%   Equals        \=     Greater than  \>     Question mark \?
%%   Commercial at \@     Left bracket  \[     Backslash     \\
%%   Right bracket \]     Circumflex    \^     Underscore    \_
%%   Grave accent  \`     Left brace    \{     Vertical bar  \|
%%   Right brace   \}     Tilde         \~}
%</package>
%\fi
% \iffalse
% Doc-Source file
% Copyright (C) 2016 Nicola Talbot, all rights reserved.
% \fi
% \iffalse
%<*driver>
\documentclass{ltxdoc}

\usepackage[T1]{fontenc}
\usepackage{tgtermes}
\usepackage{metalogo}
\usepackage[colorlinks,hyperindex=false]{hyperref}

\CheckSum{273}

\RecordChanges
\PageIndex
\CodelineNumbered
\IndexPrologue{}

\renewcommand*{\usage}[1]{\hyperpage{#1}}
\renewcommand*{\main}[1]{\hyperpage{#1}}

\newcommand*{\shortargfmt}[1]{\texttt{-#1}}
\newcommand*{\longargfmt}[1]{\texttt{-{}-#1}}

\newcommand*{\shortarg}[1]{\shortargfmt{#1}%
 \index{texosquery options=\texttt{texosquery} options>#1=\protect\shortargfmt{-#1}|hyperpage}}
\newcommand*{\longarg}[1]{\longargfmt{#1}%
 \index{texosquery options=\texttt{texosquery} options>#1=\protect\longargfmt{#1}|hyperpage}}

\newenvironment{important}
 {\begin{quote}\textbf{Important Note:}}
 {\end{quote}}

\begin{document}
\DocInput{texosquery.dtx}
\end{document}
%</driver>
%\fi
%
%\DeleteShortVerb{|}
%
%\title{texosquery: query OS information from \TeX}
%\author{Nicola L. C. Talbot\\
%\href{http://www.dickimaw-books.com/}{\nolinkurl{dickimaw-books.com}}
%\and Paulo Cereda}
%\date{2016-07-14 (v1.1)}
%\maketitle
%
%\begin{abstract}
%\texttt{texosquery.jar} is a cross-platform Java application
%to query certain OS information designed for use in \TeX's
%shell escape mechanism.
%The accompanying \TeX\ code provides a simple interface to the
%\texttt{texosquery} Java application, which may be skipped
%if you want to explicitly use \cs{input} (but take
%care of any special characters appearing in the result).
%\end{abstract}
%
%\begin{important}
%You will need \TeX's shell escape enabled,
%and you will also need the Java Runtime Environment (JRE) 
%installed to use \texttt{texosquery}.
%\end{important}
%
%If you want to rebuild the application, instructions for
%compiling the source code (including the code for this document)
%are in the accompanying \texttt{README.md} file.
%
%\tableofcontents
%
%\section{texosquery.jar: the Java application}
%The \texttt{texosquery} Java command line application looks up
%certain system information that may be of use in \TeX\ 
%documents. All this information can easily be obtained using 
%native commands, but the Java application allows an 
%OS-independent approach with results that can easily be
%captured by \TeX's shell-escape without having to strip 
%formatting information.
%
%\begin{important}
%\texttt{texosquery} provides read-only actions, and I don't 
%intend adding any actions that modify system settings or files.
%\end{important}
%
%Since the application is designed to work with \TeX, each
%function will display the result on a single line without
%formatting. (For multiple results, each line is grouped 
%from v1.1.) A blank line (or empty group) will be displayed
%if the information isn't available. A forward slash is always 
%used as a directory divider, regardless of the operating 
%system, so the result can be used, for example, in \cs{input} or
%\cs{includegraphics}.
%
%If an input file name is required (for example, with the
%\longargfmt{pdfdate} argument described below) then the file may be in
%the current working directory, relative to the current directory
%(with forward slash \texttt{/} as the directory divider), an absolute path 
%(again with forward slash) or on \TeX's path (in
%which case, \texttt{kpsewhich} is used to locate it).
%
%Command line invocation:
%\begin{flushleft}\ttfamily
%texosquery \meta{action} ...
%\end{flushleft}
%Available actions (at least one required):
%\begin{description}
%\item[\shortarg{L} or \longarg{locale}] Display the locale
%information in the form
%\begin{flushleft}\ttfamily
%\meta{lang}-\meta{region}.\meta{codeset}@\meta{modifier}
%\end{flushleft}
%where \meta{lang} is the ISO code for the language (e.g.\ 
%\texttt{en}),
%\meta{region} is the ISO code for the region (e.g.\ \texttt{GB}),
%\meta{codeset} is the default code set (e.g.\ \texttt{UTF-8})
% and \meta{modifier} is the modifier. Elements may be omitted
% if unavailable. For example, \texttt{en-GB.UTF-8} has the
% \meta{modifier} omitted, and \texttt{en} has all but the language
% omitted.
%\item[\shortarg{l} or \longarg{locale-lcs}] As the above but the
%codeset (if present) is converted to lower case and any hyphens are
%stripped. For example, if \longargfmt{locale} returns
%\texttt{en-GB.UTF-8}, then \longargfmt{locale-lcs} would return
%\texttt{en-GB.utf8}.
%\item[\shortarg{b} or \longarg{bcp47}] (New to version 1.2.)
%Displays the BCP~47 language tag.
%
%\item[\shortarg{c} or \longarg{cwd}] Displays the current working
%directory.
%\item[\shortarg{m} or \longarg{userhome}] Displays the user's home
%directory.
%\item[\shortarg{t} or \longarg{tmpdir}] Displays the temporary
%directory.
%\item[\shortarg{o} or \longarg{osname}] Displays the operating
%system name.
%\item[\shortarg{r} or \longarg{osversion}] Displays the operating
%system version.
%\item[\shortarg{a} or \longarg{osarch}] Displays the operating
%system architecture.
%\item[\shortarg{n} or \longarg{pdfnow}] Displays the current
% date and time in PDF format. For example 
%\begin{verbatim}
%D:20160704131006+01'00'
%\end{verbatim}
%Note that some, but not all, \TeX\ formats provide
%\cs{pdfcreationdate}, which is more efficient than using the shell
%escape, but this can be used
%as a fallback method for those that don't (for example, \XeTeX).
%\item[\shortarg{d} \meta{file} or \longarg{pdfdate} \meta{file}]
% Displays the last modified time stamp of the given file in PDF format or a blank line
% if the file doesn't exist or the file permissions prohibit this
% action.
%Again some, but not all, \TeX\ formats provide
%\cs{pdffilemoddate}\marg{file}, which is more efficient than using
%the shell escape.
%\item[\shortarg{s} \meta{file} or \longarg{filesize} \meta{file}]
% Displays the size in bytes of the given file or a blank line
% if the file doesn't exist or the file permissions prohibit this
% action.
%Some, but not all, \TeX\ formats provide
%\cs{pdffilesize}\marg{file}, which is more efficient than using
%the shell escape.
%\item[\shortarg{i} \meta{sep} \meta{dir} or \longarg{list}
%\meta{sep} \meta{dir}]
%List all files in the given directory with the output on a single
%line using \meta{sep} as the separator between entries. Note that
%the list doesn't include the full path, just the file names.
%\item[\shortarg{f} \meta{sep} \meta{regex} \meta{dir} or
%\longarg{filterlist} \meta{sep} \meta{regex} \meta{dir}]
%Like the above but only lists those files whose name matches the
%regular expression given in \meta{regex}. Note that this tests for a
%complete match on the file name (not including path). For example,
%if \meta{regexp} is \texttt{foo.*}, it will only match files whose
%name starts with \texttt{foo} (for example, \texttt{foobar} will
%match but \texttt{barfoo} won't).
%\item[\shortarg{u} \meta{file} or \longarg{uri} \meta{file}]
%Displays the URI of the given file or a blank line if the file
%doesn't exist or the file permissions prohibit this action.
%Note that this may include in a percent character in the result. The \TeX\ 
%command \cs{TeXOSQuery} protects against this by changing the
%category code, but if you explicitly call \texttt{texosquery}
%using the shell escape, you'll need to be careful of this.
%\item[\shortarg{p} \meta{file} or \longarg{path} \meta{file}]
%Displays the canonical path of the given file or a blank line if
%the file doesn't exist or the file permissions prohibit this
%action.
%\item[\shortarg{e} \meta{file} or \longarg{dirname} \meta{file}]
%(New to v1.1.)
%Displays the canonical path of the given file's parent (that is, the
%directory containing \meta{file}) or a blank line if
%the file doesn't exist or the file permissions prohibit this
%action. Note that this is different to the Unix-like 
%\texttt{dirname} command, which will return a relative path if
%\meta{file} isn't an absolute path.
%\item[\shortarg{h} or \longarg{help}] Displays help message and
%exits.
%\item[\shortarg{v} or \longarg{version}] Displays version
%information and exits.
%\end{description}
%
%If multiple options are given, they will be processed in the
%order specified in the command line invocation. Each result will
%be displayed on a separate line. As from v1.1, if there are
%multiple actions, each result will be grouped. This makes it easier
%to process the results in \TeX. For example:
%\begin{verbatim}
%texosquery -l
%\end{verbatim}
%This just produces (for me):
%\begin{verbatim}
%en-GB.utf8
%\end{verbatim}
%whereas
%\begin{verbatim}
%texosquery -l -n
%\end{verbatim}
%produces:
%\begin{verbatim}
%{en-GB.utf8}
%{D:20160714112732+01'00'}
%\end{verbatim}
%Note that unavailable information will produce an empty group.
%For example (assuming \texttt{nofile} doesn't exist):
%\begin{verbatim}
%texosquery -l -d nofile -n
%\end{verbatim}
%produces:
%\begin{verbatim}
%{en-GB.utf8}
%{}
%{D:20160714112732+01'00'}
%\end{verbatim}
%whereas 
%\begin{verbatim}
%texosquery -d nofile
%\end{verbatim}
%just displays an empty line.
%
%\section{texosquery.tex: generic \TeX\ code}
%You can run \texttt{texosquery} directly from \TeX's shell escape.
%For example:
%\begin{verbatim}
%\input|"texosquery --locale"
%\end{verbatim}
%However, \texttt{texosquery.tex} provides generic \TeX\ code to do
%this for you and store the result in a control sequence.
%
%Plain \TeX\ users can input this file through the usual \cs{input}
%method:
%\begin{verbatim}
%\input texosquery
%\end{verbatim}
%\LaTeX\ users may also simply input this file:
%\begin{verbatim}
%%\iffalse
%<*package>
%% \CharacterTable
%%  {Upper-case    \A\B\C\D\E\F\G\H\I\J\K\L\M\N\O\P\Q\R\S\T\U\V\W\X\Y\Z
%%   Lower-case    \a\b\c\d\e\f\g\h\i\j\k\l\m\n\o\p\q\r\s\t\u\v\w\x\y\z
%%   Digits        \0\1\2\3\4\5\6\7\8\9
%%   Exclamation   \!     Double quote  \"     Hash (number) \#
%%   Dollar        \$     Percent       \%     Ampersand     \&
%%   Acute accent  \'     Left paren    \(     Right paren   \)
%%   Asterisk      \*     Plus          \+     Comma         \,
%%   Minus         \-     Point         \.     Solidus       \/
%%   Colon         \:     Semicolon     \;     Less than     \<
%%   Equals        \=     Greater than  \>     Question mark \?
%%   Commercial at \@     Left bracket  \[     Backslash     \\
%%   Right bracket \]     Circumflex    \^     Underscore    \_
%%   Grave accent  \`     Left brace    \{     Vertical bar  \|
%%   Right brace   \}     Tilde         \~}
%</package>
%\fi
% \iffalse
% Doc-Source file
% Copyright (C) 2016 Nicola Talbot, all rights reserved.
% \fi
% \iffalse
%<*driver>
\documentclass{ltxdoc}

\usepackage[T1]{fontenc}
\usepackage{tgtermes}
\usepackage{metalogo}
\usepackage[colorlinks,hyperindex=false]{hyperref}

\CheckSum{273}

\RecordChanges
\PageIndex
\CodelineNumbered
\IndexPrologue{}

\renewcommand*{\usage}[1]{\hyperpage{#1}}
\renewcommand*{\main}[1]{\hyperpage{#1}}

\newcommand*{\shortargfmt}[1]{\texttt{-#1}}
\newcommand*{\longargfmt}[1]{\texttt{-{}-#1}}

\newcommand*{\shortarg}[1]{\shortargfmt{#1}%
 \index{texosquery options=\texttt{texosquery} options>#1=\protect\shortargfmt{-#1}|hyperpage}}
\newcommand*{\longarg}[1]{\longargfmt{#1}%
 \index{texosquery options=\texttt{texosquery} options>#1=\protect\longargfmt{#1}|hyperpage}}

\newenvironment{important}
 {\begin{quote}\textbf{Important Note:}}
 {\end{quote}}

\begin{document}
\DocInput{texosquery.dtx}
\end{document}
%</driver>
%\fi
%
%\DeleteShortVerb{|}
%
%\title{texosquery: query OS information from \TeX}
%\author{Nicola L. C. Talbot\\
%\href{http://www.dickimaw-books.com/}{\nolinkurl{dickimaw-books.com}}
%\and Paulo Cereda}
%\date{2016-07-14 (v1.1)}
%\maketitle
%
%\begin{abstract}
%\texttt{texosquery.jar} is a cross-platform Java application
%to query certain OS information designed for use in \TeX's
%shell escape mechanism.
%The accompanying \TeX\ code provides a simple interface to the
%\texttt{texosquery} Java application, which may be skipped
%if you want to explicitly use \cs{input} (but take
%care of any special characters appearing in the result).
%\end{abstract}
%
%\begin{important}
%You will need \TeX's shell escape enabled,
%and you will also need the Java Runtime Environment (JRE) 
%installed to use \texttt{texosquery}.
%\end{important}
%
%If you want to rebuild the application, instructions for
%compiling the source code (including the code for this document)
%are in the accompanying \texttt{README.md} file.
%
%\tableofcontents
%
%\section{texosquery.jar: the Java application}
%The \texttt{texosquery} Java command line application looks up
%certain system information that may be of use in \TeX\ 
%documents. All this information can easily be obtained using 
%native commands, but the Java application allows an 
%OS-independent approach with results that can easily be
%captured by \TeX's shell-escape without having to strip 
%formatting information.
%
%\begin{important}
%\texttt{texosquery} provides read-only actions, and I don't 
%intend adding any actions that modify system settings or files.
%\end{important}
%
%Since the application is designed to work with \TeX, each
%function will display the result on a single line without
%formatting. (For multiple results, each line is grouped 
%from v1.1.) A blank line (or empty group) will be displayed
%if the information isn't available. A forward slash is always 
%used as a directory divider, regardless of the operating 
%system, so the result can be used, for example, in \cs{input} or
%\cs{includegraphics}.
%
%If an input file name is required (for example, with the
%\longargfmt{pdfdate} argument described below) then the file may be in
%the current working directory, relative to the current directory
%(with forward slash \texttt{/} as the directory divider), an absolute path 
%(again with forward slash) or on \TeX's path (in
%which case, \texttt{kpsewhich} is used to locate it).
%
%Command line invocation:
%\begin{flushleft}\ttfamily
%texosquery \meta{action} ...
%\end{flushleft}
%Available actions (at least one required):
%\begin{description}
%\item[\shortarg{L} or \longarg{locale}] Display the locale
%information in the form
%\begin{flushleft}\ttfamily
%\meta{lang}-\meta{region}.\meta{codeset}@\meta{modifier}
%\end{flushleft}
%where \meta{lang} is the ISO code for the language (e.g.\ 
%\texttt{en}),
%\meta{region} is the ISO code for the region (e.g.\ \texttt{GB}),
%\meta{codeset} is the default code set (e.g.\ \texttt{UTF-8})
% and \meta{modifier} is the modifier. Elements may be omitted
% if unavailable. For example, \texttt{en-GB.UTF-8} has the
% \meta{modifier} omitted, and \texttt{en} has all but the language
% omitted.
%\item[\shortarg{l} or \longarg{locale-lcs}] As the above but the
%codeset (if present) is converted to lower case and any hyphens are
%stripped. For example, if \longargfmt{locale} returns
%\texttt{en-GB.UTF-8}, then \longargfmt{locale-lcs} would return
%\texttt{en-GB.utf8}.
%\item[\shortarg{b} or \longarg{bcp47}] (New to version 1.2.)
%Displays the BCP~47 language tag.
%
%\item[\shortarg{c} or \longarg{cwd}] Displays the current working
%directory.
%\item[\shortarg{m} or \longarg{userhome}] Displays the user's home
%directory.
%\item[\shortarg{t} or \longarg{tmpdir}] Displays the temporary
%directory.
%\item[\shortarg{o} or \longarg{osname}] Displays the operating
%system name.
%\item[\shortarg{r} or \longarg{osversion}] Displays the operating
%system version.
%\item[\shortarg{a} or \longarg{osarch}] Displays the operating
%system architecture.
%\item[\shortarg{n} or \longarg{pdfnow}] Displays the current
% date and time in PDF format. For example 
%\begin{verbatim}
%D:20160704131006+01'00'
%\end{verbatim}
%Note that some, but not all, \TeX\ formats provide
%\cs{pdfcreationdate}, which is more efficient than using the shell
%escape, but this can be used
%as a fallback method for those that don't (for example, \XeTeX).
%\item[\shortarg{d} \meta{file} or \longarg{pdfdate} \meta{file}]
% Displays the last modified time stamp of the given file in PDF format or a blank line
% if the file doesn't exist or the file permissions prohibit this
% action.
%Again some, but not all, \TeX\ formats provide
%\cs{pdffilemoddate}\marg{file}, which is more efficient than using
%the shell escape.
%\item[\shortarg{s} \meta{file} or \longarg{filesize} \meta{file}]
% Displays the size in bytes of the given file or a blank line
% if the file doesn't exist or the file permissions prohibit this
% action.
%Some, but not all, \TeX\ formats provide
%\cs{pdffilesize}\marg{file}, which is more efficient than using
%the shell escape.
%\item[\shortarg{i} \meta{sep} \meta{dir} or \longarg{list}
%\meta{sep} \meta{dir}]
%List all files in the given directory with the output on a single
%line using \meta{sep} as the separator between entries. Note that
%the list doesn't include the full path, just the file names.
%\item[\shortarg{f} \meta{sep} \meta{regex} \meta{dir} or
%\longarg{filterlist} \meta{sep} \meta{regex} \meta{dir}]
%Like the above but only lists those files whose name matches the
%regular expression given in \meta{regex}. Note that this tests for a
%complete match on the file name (not including path). For example,
%if \meta{regexp} is \texttt{foo.*}, it will only match files whose
%name starts with \texttt{foo} (for example, \texttt{foobar} will
%match but \texttt{barfoo} won't).
%\item[\shortarg{u} \meta{file} or \longarg{uri} \meta{file}]
%Displays the URI of the given file or a blank line if the file
%doesn't exist or the file permissions prohibit this action.
%Note that this may include in a percent character in the result. The \TeX\ 
%command \cs{TeXOSQuery} protects against this by changing the
%category code, but if you explicitly call \texttt{texosquery}
%using the shell escape, you'll need to be careful of this.
%\item[\shortarg{p} \meta{file} or \longarg{path} \meta{file}]
%Displays the canonical path of the given file or a blank line if
%the file doesn't exist or the file permissions prohibit this
%action.
%\item[\shortarg{e} \meta{file} or \longarg{dirname} \meta{file}]
%(New to v1.1.)
%Displays the canonical path of the given file's parent (that is, the
%directory containing \meta{file}) or a blank line if
%the file doesn't exist or the file permissions prohibit this
%action. Note that this is different to the Unix-like 
%\texttt{dirname} command, which will return a relative path if
%\meta{file} isn't an absolute path.
%\item[\shortarg{h} or \longarg{help}] Displays help message and
%exits.
%\item[\shortarg{v} or \longarg{version}] Displays version
%information and exits.
%\end{description}
%
%If multiple options are given, they will be processed in the
%order specified in the command line invocation. Each result will
%be displayed on a separate line. As from v1.1, if there are
%multiple actions, each result will be grouped. This makes it easier
%to process the results in \TeX. For example:
%\begin{verbatim}
%texosquery -l
%\end{verbatim}
%This just produces (for me):
%\begin{verbatim}
%en-GB.utf8
%\end{verbatim}
%whereas
%\begin{verbatim}
%texosquery -l -n
%\end{verbatim}
%produces:
%\begin{verbatim}
%{en-GB.utf8}
%{D:20160714112732+01'00'}
%\end{verbatim}
%Note that unavailable information will produce an empty group.
%For example (assuming \texttt{nofile} doesn't exist):
%\begin{verbatim}
%texosquery -l -d nofile -n
%\end{verbatim}
%produces:
%\begin{verbatim}
%{en-GB.utf8}
%{}
%{D:20160714112732+01'00'}
%\end{verbatim}
%whereas 
%\begin{verbatim}
%texosquery -d nofile
%\end{verbatim}
%just displays an empty line.
%
%\section{texosquery.tex: generic \TeX\ code}
%You can run \texttt{texosquery} directly from \TeX's shell escape.
%For example:
%\begin{verbatim}
%\input|"texosquery --locale"
%\end{verbatim}
%However, \texttt{texosquery.tex} provides generic \TeX\ code to do
%this for you and store the result in a control sequence.
%
%Plain \TeX\ users can input this file through the usual \cs{input}
%method:
%\begin{verbatim}
%\input texosquery
%\end{verbatim}
%\LaTeX\ users may also simply input this file:
%\begin{verbatim}
%\input{texosquery}
%\end{verbatim}
%but may prefer the standard package approach:
%\begin{verbatim}
%\usepackage{texosquery}
%\end{verbatim}
%
%\begin{important}
%The commands described below are all fragile.
%\end{important}
%
%The basic command to run \texttt{texosquery} and capture its output
%in a control sequence is:
%\DescribeMacro\TeXOSQuery
%\begin{flushleft}\ttfamily
%\cs{TeXOSQuery}\marg{cs}\marg{args}
%\end{flushleft}
%where \meta{cs} is the control sequence in which to store the
%result and \meta{args} are the command line arguments to pass to 
%\texttt{texosquery}. This first locally changes the category code
%of some problematic characters that may appear in the result.
%The only special characters that aren't changed are the backslash
%\verb|\|, curly braces \verb|{| and \verb|}|, and hash \verb|#|.
%(\texttt{texosquery.jar} will replace \verb|#| with \verb|\#| in places
%where it might possibly occur in the result, but in general it's
%best to avoid these characters in file names.)
%
%There are some short cut commands for convenience, described below.
%If any of these commands cause an error message in the form:
%\begin{verbatim}
%I can't find file `|texosquery'.
%\end{verbatim}
%then check that you have the shell escape on. If the error 
%persists with the shell escape enabled
%(and not restricted), check that \texttt{texosquery} is on 
%your system's path. To do this, open a command prompt or terminal and
%type \texttt{texosquery}. If it isn't installed correctly, there will be a
%message like:
%\begin{verbatim}
%`texosquery' is not recognised
%\end{verbatim}
%or
%\begin{verbatim}
%texosquery: command not found
%\end{verbatim}
%If this happens, check the installation. (Instructions are
%in the accompanying \texttt{README.md} file.)
%
%Dry run mode is determined by the conditional
%\DescribeMacro\ifTeXOSQueryDryRun
%\begin{verbatim}
%\ifTeXOSQueryDryRun
%\end{verbatim}
%If true, the shell escape won't be used and the requested command
%invocation will be printed in the transcript file prefixed with
%\begin{verbatim}
%TeXOSQuery:
%\end{verbatim}
%(the control sequence \meta{cs} will be set to empty).
%This conditional will automatically be switched on unless
%\cs{shellescape} or \cs{pdfshellescape} is 1. (If
%\texttt{texosquery.jar} is later allowed on the restricted list,
%newer versions may change this default.)
%
%If multiple queries are required, it's more efficient to
%perform them all in one go. For example:
%\begin{verbatim}
%\TeXOSQuery{\result}{-l -n -o}
%
%\def\parseresult#1#2#3{%
%  Locale: #1. Now: #2. OS: #3.%
%}
%
%\ifx\result\empty
%  Query failed.
%\else
%  \expandafter\parseresult\result
%\fi
%\end{verbatim}
%(Make sure you have at least v1.1 for this to work correctly.)
%
%The locale (\shortarg{l} or \longarg{locale-lcs}) information can be
%obtained using:
%\DescribeMacro\TeXOSQueryLocale
%\begin{flushleft}\ttfamily
%\cs{TeXOSQueryLocale}\marg{cs}
%\end{flushleft}
%Note that this uses the lower case codeset form, which has a better
%chance of matching the encoding names used by the \texttt{inputenc}
%package. If you want the unprocessed codeset name, you can do:
%\begin{flushleft}\ttfamily
%\cs{TeXOSQuery}\marg{cs}\{\shortarg{L}\}
%\end{flushleft}
%
%The BCP 47 language tag (\shortarg{b} or \longarg{bcp47}) can be obtained using:
%\DescribeMacro\TeXOSQueryLangTag
%\begin{flushleft}\ttfamily
%\cs{TeXOSQueryLangTag}\marg{cs}
%\end{flushleft}
%
%The current working directory (\shortarg{c} or \longarg{cwd}) can
%be obtained using:
%\DescribeMacro\TeXOSQueryCwd
%\begin{flushleft}\ttfamily
%\cs{TeXOSQueryCwd}\marg{cs}
%\end{flushleft}
%
%The home directory (\shortarg{m} or \longarg{userhome}) can
%be obtained using:
%\DescribeMacro\TeXOSQueryHome
%\begin{flushleft}\ttfamily
%\cs{TeXOSQueryHome}\marg{cs}
%\end{flushleft}
%
%The temporary directory (\shortarg{t} or \longarg{tmpdir}) can
%be obtained using:
%\DescribeMacro\TeXOSQueryTmpDir
%\begin{flushleft}\ttfamily
%\cs{TeXOSQueryTmpDir}\marg{cs}
%\end{flushleft}
%
%The OS name (\shortarg{o} or \longarg{osname}) can
%be obtained using:
%\DescribeMacro\TeXOSQueryName
%\begin{flushleft}\ttfamily
%\cs{TeXOSQueryName}\marg{cs}
%\end{flushleft}
%
%The OS version (\shortarg{r} or \longarg{osversion}) can
%be obtained using:
%\DescribeMacro\TeXOSQueryVersion
%\begin{flushleft}\ttfamily
%\cs{TeXOSQueryVersion}\marg{cs}
%\end{flushleft}
%
%The OS architecture (\shortarg{a} or \longarg{osarch}) can
%be obtained using:
%\DescribeMacro\TeXOSQueryArch
%\begin{flushleft}\ttfamily
%\cs{TeXOSQueryArch}\marg{cs}
%\end{flushleft}
%
%The current date-time stamp in PDF format (\shortarg{n} or
%\longarg{pdfnow}) can be obtained using:
%\DescribeMacro\TeXOSQueryNow
%\begin{flushleft}\ttfamily
%\cs{TeXOSQueryNow}\marg{cs}
%\end{flushleft}
%This is provided for the benefit of users who don't have
%\cs{pdfcreationdate} defined by their \TeX\ format (for example,
%\XeTeX). As from v1.1, this ensures that the initial \texttt{D} has
%category code 12 (which won't happen if \cs{TeXOSQuery} is used
%explicitly).
%
%\begin{important}
%The remaining commands all require extra arguments after the
%relevant switch. These are automatically enclosed in single-quotes
%to protect any spaces. If the argument actually contains any single-quote
%characters, make sure you use \verb|\string\'| to prevent
%interference. \emph{However, in general it's a very bad idea to use
%quotes as part of a file name (rather than using them as
%delimiters).}
%Since the file name may need to be obtained from \cs{jobname},
%which sometimes includes double-quotes, the first double-quote pair
%found is stripped in file name arguments. Any other double-quotes
%will need to be protected in the same manner as single-quotes
%(but, again, this shouldn't be an issue if you use a safe
%file naming scheme). Any paths should use a
%forward slash for the directory divider.
%\end{important}
%
%The modification date-time stamp in PDF format for a file
%(\shortarg{d} or \longarg{pdfdate}) can be obtained using:
%\DescribeMacro\TeXOSQueryFileDate
%\begin{flushleft}\ttfamily
%\cs{TeXOSQueryFileDate}\marg{cs}\marg{filename}
%\end{flushleft}
%where \meta{filename} is the name of the file.
%This is provided for the benefit of users who don't have
%\cs{pdffilemoddate} defined by their \TeX\ format. As from v1.1,
%this ensures that the initial \texttt{D} has category code 12 (which 
%won't happen if \cs{TeXOSQuery} is used explicitly).
%
%The size in bytes of a file
%(\shortarg{s} or \longarg{filesize}) can be obtained using:
%\DescribeMacro\TeXOSQueryFileSize
%\begin{flushleft}\ttfamily
%\cs{TeXOSQueryFileSize}\marg{cs}\marg{filename}
%\end{flushleft}
%where \meta{filename} is the name of the file.
%This is provided for the benefit of users who don't have
%\cs{pdffilesize} defined by their \TeX\ format.
%
%The list of files in a given directory
%(\shortarg{i} or \longarg{list}) can be obtained using:
%\DescribeMacro\TeXOSQueryFileList
%\begin{flushleft}\ttfamily
%\cs{TeXOSQueryFileList}\marg{cs}\marg{sep}\marg{dir}
%\end{flushleft}
%where \meta{sep} is the separator and \meta{dir} is the directory name.
%For example:
%\begin{verbatim}
%\TeXOSQueryFileList{\result}{,}{.}
%\end{verbatim}
%will store a comma-separated list of all the files contained in the
%current directory in the control sequence \cs{result}.
%
%A filtered list of files in a given directory
%(\shortarg{f} or \longarg{filterlist}) can be obtained using:
%\DescribeMacro\TeXOSQueryFilterFileList
%\begin{flushleft}\ttfamily
%\cs{TeXOSQueryFilterFileList}\marg{cs}\marg{sep}\marg{regex}\marg{dir}
%\end{flushleft}
%where \meta{regex} is a regular expression. \emph{Take care of any backslashes
%in the regular expression!}
%For example, to list only those files that have an extension:
%\begin{verbatim}
%\TeXOSQueryFilterFileList{\result}{,}{.+\string\..*}{.}
%\end{verbatim}
%Note the use of \verb|\string\.| to ensure that \cs{.} isn't
%interpreted as a command. Another example, list only \texttt{.png}
%and \texttt{.jpg} files in the directory called \texttt{images}:
%\begin{verbatim}
%\TeXOSQueryFilterFileList{\result}{,}{.+\string\.(jpg|png)}{images}
%\end{verbatim}
%
%The URI of a file
%(\shortarg{u} or \longarg{uri}) can be obtained using:
%\DescribeMacro\TeXOSQueryFileURI
%\begin{flushleft}\ttfamily
%\cs{TeXOSQueryFileURI}\marg{cs}\marg{filename}
%\end{flushleft}
%where \meta{filename} is the name of the file.
%(Any percent symbols \verb|%| contained in the URI will have
%their category code set to 12.)
%
%The canonical path of a file
%(\shortarg{p} or \longarg{path}) can be obtained using:
%\DescribeMacro\TeXOSQueryFilePath
%\begin{flushleft}\ttfamily
%\cs{TeXOSQueryFilePath}\marg{cs}\marg{filename}
%\end{flushleft}
%where \meta{filename} is the name of the file.
%
%The canonical path of a file's parent
%(\shortarg{e} or \longarg{dirname}) can be obtained using:
%\DescribeMacro\TeXOSQueryDirName
%\begin{flushleft}\ttfamily
%\cs{TeXOSQueryDirName}\marg{cs}\marg{filename}
%\end{flushleft}
%where \meta{filename} is the name of the file.
%
%\StopEventually{\phantomsection
%\addcontentsline{toc}{section}{Change History}%
%\PrintChanges
%\addcontentsline{toc}{section}{\indexname}%
%\PrintIndex
%}
%
%\section{The Code}
%\changes{1.0}{2016-07-08}{Initial release}
%\subsection{Generic \TeX\ Code}
%\iffalse
%    \begin{macrocode}
%<*texosquery.tex>
%    \end{macrocode}
%\fi
%Change category code of \texttt{@} if necessary.
%    \begin{macrocode}
\ifnum\catcode`\@=11\relax
  \def\@texosquery@restore@at{}%
\else
  \expandafter\edef\csname @texosquery@restore@at\endcsname{%
    \noexpand\catcode`\noexpand\@=\number\catcode`\@\relax
  }%
 \catcode`\@=11\relax
\fi
%    \end{macrocode}
% Check if already loaded.
%    \begin{macrocode}
\ifx\TeXOSQuery\undefined
\else
  \@texosquery@restore@at
  \expandafter\endinput
\fi
%    \end{macrocode}
%\begin{macro}{\TeXOSInvokerName}
%The name of the \texttt{texosquery} application.
%    \begin{macrocode}
\def\TeXOSInvokerName{texosquery}
%    \end{macrocode}
%\end{macro}
%
%\begin{macro}{\TeXOSQueryInvoker}
%If we're using \LaTeX, we'll need to use \cs{@@input} rather
%than \cs{input}.
%    \begin{macrocode}
\ifx\@@input\undefined
  \def\TeXOSQueryInvoker#1{\input|"\TeXOSInvokerName\space#1" }
\else
  \def\TeXOSQueryInvoker#1{\@@input|"\TeXOSInvokerName\space#1" }
\fi
%    \end{macrocode}
%\end{macro}
%
%\begin{macro}{\ifTeXOSQueryDryRun}
%Provide a dry-run mode.
%\changes{1.1}{2016-07-14}{dry run mode only false by default if with
%unrestricted mode}
%    \begin{macrocode}
\newif\ifTeXOSQueryDryRun
\TeXOSQueryDryRuntrue
%    \end{macrocode}
%\end{macro}
%
%If shell escape is unrestricted, automatically switch off dry-run mode.
%    \begin{macrocode}
\ifx\shellescape\undefined
  \ifx\pdfshellescape\undefined
  \else
    \ifnum\pdfshellescape=1\relax
      \TeXOSQueryDryRunfalse
    \fi
  \fi
\else
  \ifnum\shellescape=1\relax
    \TeXOSQueryDryRunfalse
  \fi
\fi
%    \end{macrocode}
%
%\begin{macro}{\TeXOSQuery}
%Use \texttt{texosquery} with the option given in the second
%argument and store the result in control sequence given in the
%first argument.
%    \begin{macrocode}
\def\TeXOSQuery#1#2{%
  \ifTeXOSQueryDryRun
    \begingroup
      \newlinechar=`\^^J
      \message{^^JTeXOSQuery: \TeXOSInvokerName\space#2^^J}%
    \endgroup
    \def#1{}%
  \else
    \begingroup
    \endlinechar=-1\relax
%    \end{macrocode}
%Just in case the result contains any awkward characters that have
%some special meaning to \TeX. (Can't really protect against hash,
%backslash or curly braces, but they're unlikely to occur unless
%the user has a very wacky and inappropriate file naming scheme.)
%    \begin{macrocode}
    \catcode`\-=12\relax
    \catcode`\_=12\relax
    \catcode`\^=12\relax
    \catcode`\~=12\relax
    \catcode`\$=12\relax
    \catcode`\&=12\relax
    \catcode`\"=12\relax
    \catcode`\'=12\relax
    \catcode`\.=12\relax
    \catcode`\/=12\relax
    \catcode`\:=12\relax
    \catcode`\;=12\relax
    \catcode`\%=12\relax
    \everyeof{\noexpand}\relax
    \edef\x{\endgroup\def\noexpand#1{\TeXOSQueryInvoker{#2}}}\x
  \fi
}
%    \end{macrocode}
%\end{macro}
%
%Now for some convenient shortcuts so the user doesn't have to
%remember the command line options. \cs{string} is used in
%case the hyphen character has been made active.
%\begin{macro}{\TeXOSQueryLocale}
%Query the locale and store the result in the control sequence
%provided in the argument.
%    \begin{macrocode}
\def\TeXOSQueryLocale#1{\TeXOSQuery{#1}{\string-l}}
%    \end{macrocode}
%\end{macro}
%
%\begin{macro}{\TeXOSQueryLangTag}
%Query the language tag and store the result in the control sequence
%provided in the argument.
%\changes{1.2}{??}{new}
%    \begin{macrocode}
\def\TeXOSQueryLangTag#1{\TeXOSQuery{#1}{\string-b}}
%    \end{macrocode}
%\end{macro}
%
%\begin{macro}{\TeXOSQueryCwd}
%Query the current working directory.
%    \begin{macrocode}
\def\TeXOSQueryCwd#1{\TeXOSQuery{#1}{\string-c}}
%    \end{macrocode}
%\end{macro}
%
%\begin{macro}{\TeXOSQueryHome}
%Query the user's home directory.
%    \begin{macrocode}
\def\TeXOSQueryHome#1{\TeXOSQuery{#1}{\string-m}}
%    \end{macrocode}
%\end{macro}
%
%\begin{macro}{\TeXOSQueryTmpDir}
%Query the temporary directory.
%    \begin{macrocode}
\def\TeXOSQueryTmpDir#1{\TeXOSQuery{#1}{\string-t}}
%    \end{macrocode}
%\end{macro}
%
%\begin{macro}{\TeXOSQueryVersion}
%Query the operating system version.
%    \begin{macrocode}
\def\TeXOSQueryVersion#1{\TeXOSQuery{#1}{\string-r}}
%    \end{macrocode}
%\end{macro}
%
%\begin{macro}{\TeXOSQueryArch}
%Query the operating system architecture.
%    \begin{macrocode}
\def\TeXOSQueryArch#1{\TeXOSQuery{#1}{\string-a}}
%    \end{macrocode}
%\end{macro}
%
%\begin{macro}{\TeXOSQueryName}
%Query the operating system name.
%    \begin{macrocode}
\def\TeXOSQueryName#1{\TeXOSQuery{#1}{\string-o}}
%    \end{macrocode}
%\end{macro}
%
%\begin{macro}{\TeXOSQueryNow}
%\changes{1.1}{2016-07-14}{changed catcode of D to 12}
%Query the current time stamp.
%    \begin{macrocode}
\def\TeXOSQueryNow#1{%
%    \end{macrocode}
% The \texttt{D} needs category code 12. (Don't need to worry about
% \texttt{Z} as \texttt{texosquery.jar} uses \texttt{+00'00'} for
% UTC+0.) This change can't be done with the other catcode changes
% in \cs{TeXOSQuery}, as this is only appropriate for the PDF dates.
% Save and restore the catcode rather than fiddle around with
% scoping.
%    \begin{macrocode}
  \edef\@texosquery@restore@D{%
    \noexpand\catcode`\noexpand\D=\the\catcode`\D\relax}%
  \catcode`\D=12\relax
  \TeXOSQuery{#1}{\string-n}%
  \@texosquery@restore@D
}
%    \end{macrocode}
%\end{macro}
%
%If the file name is supplied using \cs{jobname} it may have
%double-quotes which will interfere with things.
%\begin{macro}{\texosquerystripquotes}
%    \begin{macrocode}
\def\texosquerystripquotes#1{%
  \@texosquery@stripquotes#1\@mid@texosquery@stripquotes
    "\relax"\relax\@end@texosquery@stripquotes
}
\def\@texosquery@stripquotes#1"#2"{%
  \@@texosquery@stripquotes#1#2%
}
\def\@@texosquery@stripquotes#1\@mid@texosquery@stripquotes#2\@end@texosquery@stripquotes{%
  #1%
}
%    \end{macrocode}
%\end{macro}
%
%\begin{macro}{\TeXOSQueryFileDate}
%\changes{1.1}{2016-07-14}{changed catcode of D to 12}
%Query the time stamp of the file given in the second argument.
%    \begin{macrocode}
\def\TeXOSQueryFileDate#1#2{%
  \edef\@texosquery@restore@D{%
    \noexpand\catcode`\noexpand\D=\the\catcode`\D\relax}%
  \catcode`\D=12\relax
  \TeXOSQuery{#1}{\string-d \string'\texosquerystripquotes{#2}\string'}%
  \@texosquery@restore@D
}
%    \end{macrocode}
%\end{macro}
%
%\begin{macro}{\TeXOSQueryFileSize}
%Query the size of the file given in the second argument.
%    \begin{macrocode}
\def\TeXOSQueryFileSize#1#2{\TeXOSQuery{#1}{\string-s 
 \string'\texosquerystripquotes{#2}\string'}}
%    \end{macrocode}
%\end{macro}
%
%\begin{macro}{\TeXOSQueryFileList}
%List all files in the directory given in the third argument,
%separated by the second argument.
%    \begin{macrocode}
\def\TeXOSQueryFileList#1#2#3{\TeXOSQuery{#1}{%
 \string-i \string'#2\string' 
 \string'\texosquerystripquotes{#3}\string'}}
%    \end{macrocode}
%\end{macro}
%
%\begin{macro}{\TeXOSQueryFilterFileList}
%Filtered list files in the directory given in the fourth argument,
%separated by the second argument. The third argument is the regular
%expression used to filter the list. \emph{Take care of backslashes
%in the regular expression!}
%    \begin{macrocode}
\def\TeXOSQueryFilterFileList#1#2#3#4{\TeXOSQuery{#1}{%
 \string-f \string'#2\string' \string'#3\string' 
 \string'\texosquerystripquotes{#4}\string'}}
%    \end{macrocode}
%\end{macro}
%
%\begin{macro}{\TeXOSQueryFileURI}
%Get the URI of the file given in the second argument.
%    \begin{macrocode}
\def\TeXOSQueryFileURI#1#2{\TeXOSQuery{#1}{\string-u 
 \string'\texosquerystripquotes{#2}\string'}}
%    \end{macrocode}
%\end{macro}
%
%\begin{macro}{\TeXOSQueryFilePath}
%Get the canonical path of the file given in the second argument.
%    \begin{macrocode}
\def\TeXOSQueryFilePath#1#2{\TeXOSQuery{#1}{\string-p 
 \string'\texosquerystripquotes{#2}\string'}}
%    \end{macrocode}
%\end{macro}
%
%\begin{macro}{\TeXOSQueryDirName}
%\changes{1.1}{2016-07-14}{new}
%Get the canonical path of the directory containing the file given 
%in the second argument.
%    \begin{macrocode}
\def\TeXOSQueryDirName#1#2{\TeXOSQuery{#1}{\string-e 
 \string'\texosquerystripquotes{#2}\string'}}
%    \end{macrocode}
%\end{macro}
%
%All done.
%Restore the category code of \texttt{@}:
%    \begin{macrocode}
\@texosquery@restore@at
%    \end{macrocode}
%\iffalse
%    \begin{macrocode}
%</texosquery.tex>
%    \end{macrocode}
%\fi
%
%\subsection{\LaTeX\ Code}
%This is just a simple wrapper for \texttt{texosquery.tex}
%so that it can be loaded using \LaTeX's standard \cs{usepackage}
%method.
%\iffalse
%    \begin{macrocode}
%<*texosquery.sty>
%    \end{macrocode}
%\fi
%Identify package:
%    \begin{macrocode}
\NeedsTeXFormat{LaTeX2e}
\ProvidesPackage{texosquery}[2016/07/14 v1.1 (NLCT)]
%    \end{macrocode}
%Load \texttt{texosquery.tex}:
%    \begin{macrocode}
\input{texosquery}
%    \end{macrocode}
%That's it!
%\iffalse
%    \begin{macrocode}
%</texosquery.sty>
%    \end{macrocode}
%\fi
%
%\Finale
\endinput

%\end{verbatim}
%but may prefer the standard package approach:
%\begin{verbatim}
%\usepackage{texosquery}
%\end{verbatim}
%
%\begin{important}
%The commands described below are all fragile.
%\end{important}
%
%The basic command to run \texttt{texosquery} and capture its output
%in a control sequence is:
%\DescribeMacro\TeXOSQuery
%\begin{flushleft}\ttfamily
%\cs{TeXOSQuery}\marg{cs}\marg{args}
%\end{flushleft}
%where \meta{cs} is the control sequence in which to store the
%result and \meta{args} are the command line arguments to pass to 
%\texttt{texosquery}. This first locally changes the category code
%of some problematic characters that may appear in the result.
%The only special characters that aren't changed are the backslash
%\verb|\|, curly braces \verb|{| and \verb|}|, and hash \verb|#|.
%(\texttt{texosquery.jar} will replace \verb|#| with \verb|\#| in places
%where it might possibly occur in the result, but in general it's
%best to avoid these characters in file names.)
%
%There are some short cut commands for convenience, described below.
%If any of these commands cause an error message in the form:
%\begin{verbatim}
%I can't find file `|texosquery'.
%\end{verbatim}
%then check that you have the shell escape on. If the error 
%persists with the shell escape enabled
%(and not restricted), check that \texttt{texosquery} is on 
%your system's path. To do this, open a command prompt or terminal and
%type \texttt{texosquery}. If it isn't installed correctly, there will be a
%message like:
%\begin{verbatim}
%`texosquery' is not recognised
%\end{verbatim}
%or
%\begin{verbatim}
%texosquery: command not found
%\end{verbatim}
%If this happens, check the installation. (Instructions are
%in the accompanying \texttt{README.md} file.)
%
%Dry run mode is determined by the conditional
%\DescribeMacro\ifTeXOSQueryDryRun
%\begin{verbatim}
%\ifTeXOSQueryDryRun
%\end{verbatim}
%If true, the shell escape won't be used and the requested command
%invocation will be printed in the transcript file prefixed with
%\begin{verbatim}
%TeXOSQuery:
%\end{verbatim}
%(the control sequence \meta{cs} will be set to empty).
%This conditional will automatically be switched on unless
%\cs{shellescape} or \cs{pdfshellescape} is 1. (If
%\texttt{texosquery.jar} is later allowed on the restricted list,
%newer versions may change this default.)
%
%If multiple queries are required, it's more efficient to
%perform them all in one go. For example:
%\begin{verbatim}
%\TeXOSQuery{\result}{-l -n -o}
%
%\def\parseresult#1#2#3{%
%  Locale: #1. Now: #2. OS: #3.%
%}
%
%\ifx\result\empty
%  Query failed.
%\else
%  \expandafter\parseresult\result
%\fi
%\end{verbatim}
%(Make sure you have at least v1.1 for this to work correctly.)
%
%The locale (\shortarg{l} or \longarg{locale-lcs}) information can be
%obtained using:
%\DescribeMacro\TeXOSQueryLocale
%\begin{flushleft}\ttfamily
%\cs{TeXOSQueryLocale}\marg{cs}
%\end{flushleft}
%Note that this uses the lower case codeset form, which has a better
%chance of matching the encoding names used by the \texttt{inputenc}
%package. If you want the unprocessed codeset name, you can do:
%\begin{flushleft}\ttfamily
%\cs{TeXOSQuery}\marg{cs}\{\shortarg{L}\}
%\end{flushleft}
%
%The BCP 47 language tag (\shortarg{b} or \longarg{bcp47}) can be obtained using:
%\DescribeMacro\TeXOSQueryLangTag
%\begin{flushleft}\ttfamily
%\cs{TeXOSQueryLangTag}\marg{cs}
%\end{flushleft}
%
%The current working directory (\shortarg{c} or \longarg{cwd}) can
%be obtained using:
%\DescribeMacro\TeXOSQueryCwd
%\begin{flushleft}\ttfamily
%\cs{TeXOSQueryCwd}\marg{cs}
%\end{flushleft}
%
%The home directory (\shortarg{m} or \longarg{userhome}) can
%be obtained using:
%\DescribeMacro\TeXOSQueryHome
%\begin{flushleft}\ttfamily
%\cs{TeXOSQueryHome}\marg{cs}
%\end{flushleft}
%
%The temporary directory (\shortarg{t} or \longarg{tmpdir}) can
%be obtained using:
%\DescribeMacro\TeXOSQueryTmpDir
%\begin{flushleft}\ttfamily
%\cs{TeXOSQueryTmpDir}\marg{cs}
%\end{flushleft}
%
%The OS name (\shortarg{o} or \longarg{osname}) can
%be obtained using:
%\DescribeMacro\TeXOSQueryName
%\begin{flushleft}\ttfamily
%\cs{TeXOSQueryName}\marg{cs}
%\end{flushleft}
%
%The OS version (\shortarg{r} or \longarg{osversion}) can
%be obtained using:
%\DescribeMacro\TeXOSQueryVersion
%\begin{flushleft}\ttfamily
%\cs{TeXOSQueryVersion}\marg{cs}
%\end{flushleft}
%
%The OS architecture (\shortarg{a} or \longarg{osarch}) can
%be obtained using:
%\DescribeMacro\TeXOSQueryArch
%\begin{flushleft}\ttfamily
%\cs{TeXOSQueryArch}\marg{cs}
%\end{flushleft}
%
%The current date-time stamp in PDF format (\shortarg{n} or
%\longarg{pdfnow}) can be obtained using:
%\DescribeMacro\TeXOSQueryNow
%\begin{flushleft}\ttfamily
%\cs{TeXOSQueryNow}\marg{cs}
%\end{flushleft}
%This is provided for the benefit of users who don't have
%\cs{pdfcreationdate} defined by their \TeX\ format (for example,
%\XeTeX). As from v1.1, this ensures that the initial \texttt{D} has
%category code 12 (which won't happen if \cs{TeXOSQuery} is used
%explicitly).
%
%\begin{important}
%The remaining commands all require extra arguments after the
%relevant switch. These are automatically enclosed in single-quotes
%to protect any spaces. If the argument actually contains any single-quote
%characters, make sure you use \verb|\string\'| to prevent
%interference. \emph{However, in general it's a very bad idea to use
%quotes as part of a file name (rather than using them as
%delimiters).}
%Since the file name may need to be obtained from \cs{jobname},
%which sometimes includes double-quotes, the first double-quote pair
%found is stripped in file name arguments. Any other double-quotes
%will need to be protected in the same manner as single-quotes
%(but, again, this shouldn't be an issue if you use a safe
%file naming scheme). Any paths should use a
%forward slash for the directory divider.
%\end{important}
%
%The modification date-time stamp in PDF format for a file
%(\shortarg{d} or \longarg{pdfdate}) can be obtained using:
%\DescribeMacro\TeXOSQueryFileDate
%\begin{flushleft}\ttfamily
%\cs{TeXOSQueryFileDate}\marg{cs}\marg{filename}
%\end{flushleft}
%where \meta{filename} is the name of the file.
%This is provided for the benefit of users who don't have
%\cs{pdffilemoddate} defined by their \TeX\ format. As from v1.1,
%this ensures that the initial \texttt{D} has category code 12 (which 
%won't happen if \cs{TeXOSQuery} is used explicitly).
%
%The size in bytes of a file
%(\shortarg{s} or \longarg{filesize}) can be obtained using:
%\DescribeMacro\TeXOSQueryFileSize
%\begin{flushleft}\ttfamily
%\cs{TeXOSQueryFileSize}\marg{cs}\marg{filename}
%\end{flushleft}
%where \meta{filename} is the name of the file.
%This is provided for the benefit of users who don't have
%\cs{pdffilesize} defined by their \TeX\ format.
%
%The list of files in a given directory
%(\shortarg{i} or \longarg{list}) can be obtained using:
%\DescribeMacro\TeXOSQueryFileList
%\begin{flushleft}\ttfamily
%\cs{TeXOSQueryFileList}\marg{cs}\marg{sep}\marg{dir}
%\end{flushleft}
%where \meta{sep} is the separator and \meta{dir} is the directory name.
%For example:
%\begin{verbatim}
%\TeXOSQueryFileList{\result}{,}{.}
%\end{verbatim}
%will store a comma-separated list of all the files contained in the
%current directory in the control sequence \cs{result}.
%
%A filtered list of files in a given directory
%(\shortarg{f} or \longarg{filterlist}) can be obtained using:
%\DescribeMacro\TeXOSQueryFilterFileList
%\begin{flushleft}\ttfamily
%\cs{TeXOSQueryFilterFileList}\marg{cs}\marg{sep}\marg{regex}\marg{dir}
%\end{flushleft}
%where \meta{regex} is a regular expression. \emph{Take care of any backslashes
%in the regular expression!}
%For example, to list only those files that have an extension:
%\begin{verbatim}
%\TeXOSQueryFilterFileList{\result}{,}{.+\string\..*}{.}
%\end{verbatim}
%Note the use of \verb|\string\.| to ensure that \cs{.} isn't
%interpreted as a command. Another example, list only \texttt{.png}
%and \texttt{.jpg} files in the directory called \texttt{images}:
%\begin{verbatim}
%\TeXOSQueryFilterFileList{\result}{,}{.+\string\.(jpg|png)}{images}
%\end{verbatim}
%
%The URI of a file
%(\shortarg{u} or \longarg{uri}) can be obtained using:
%\DescribeMacro\TeXOSQueryFileURI
%\begin{flushleft}\ttfamily
%\cs{TeXOSQueryFileURI}\marg{cs}\marg{filename}
%\end{flushleft}
%where \meta{filename} is the name of the file.
%(Any percent symbols \verb|%| contained in the URI will have
%their category code set to 12.)
%
%The canonical path of a file
%(\shortarg{p} or \longarg{path}) can be obtained using:
%\DescribeMacro\TeXOSQueryFilePath
%\begin{flushleft}\ttfamily
%\cs{TeXOSQueryFilePath}\marg{cs}\marg{filename}
%\end{flushleft}
%where \meta{filename} is the name of the file.
%
%The canonical path of a file's parent
%(\shortarg{e} or \longarg{dirname}) can be obtained using:
%\DescribeMacro\TeXOSQueryDirName
%\begin{flushleft}\ttfamily
%\cs{TeXOSQueryDirName}\marg{cs}\marg{filename}
%\end{flushleft}
%where \meta{filename} is the name of the file.
%
%\StopEventually{\phantomsection
%\addcontentsline{toc}{section}{Change History}%
%\PrintChanges
%\addcontentsline{toc}{section}{\indexname}%
%\PrintIndex
%}
%
%\section{The Code}
%\changes{1.0}{2016-07-08}{Initial release}
%\subsection{Generic \TeX\ Code}
%\iffalse
%    \begin{macrocode}
%<*texosquery.tex>
%    \end{macrocode}
%\fi
%Change category code of \texttt{@} if necessary.
%    \begin{macrocode}
\ifnum\catcode`\@=11\relax
  \def\@texosquery@restore@at{}%
\else
  \expandafter\edef\csname @texosquery@restore@at\endcsname{%
    \noexpand\catcode`\noexpand\@=\number\catcode`\@\relax
  }%
 \catcode`\@=11\relax
\fi
%    \end{macrocode}
% Check if already loaded.
%    \begin{macrocode}
\ifx\TeXOSQuery\undefined
\else
  \@texosquery@restore@at
  \expandafter\endinput
\fi
%    \end{macrocode}
%\begin{macro}{\TeXOSInvokerName}
%The name of the \texttt{texosquery} application.
%    \begin{macrocode}
\def\TeXOSInvokerName{texosquery}
%    \end{macrocode}
%\end{macro}
%
%\begin{macro}{\TeXOSQueryInvoker}
%If we're using \LaTeX, we'll need to use \cs{@@input} rather
%than \cs{input}.
%    \begin{macrocode}
\ifx\@@input\undefined
  \def\TeXOSQueryInvoker#1{\input|"\TeXOSInvokerName\space#1" }
\else
  \def\TeXOSQueryInvoker#1{\@@input|"\TeXOSInvokerName\space#1" }
\fi
%    \end{macrocode}
%\end{macro}
%
%\begin{macro}{\ifTeXOSQueryDryRun}
%Provide a dry-run mode.
%\changes{1.1}{2016-07-14}{dry run mode only false by default if with
%unrestricted mode}
%    \begin{macrocode}
\newif\ifTeXOSQueryDryRun
\TeXOSQueryDryRuntrue
%    \end{macrocode}
%\end{macro}
%
%If shell escape is unrestricted, automatically switch off dry-run mode.
%    \begin{macrocode}
\ifx\shellescape\undefined
  \ifx\pdfshellescape\undefined
  \else
    \ifnum\pdfshellescape=1\relax
      \TeXOSQueryDryRunfalse
    \fi
  \fi
\else
  \ifnum\shellescape=1\relax
    \TeXOSQueryDryRunfalse
  \fi
\fi
%    \end{macrocode}
%
%\begin{macro}{\TeXOSQuery}
%Use \texttt{texosquery} with the option given in the second
%argument and store the result in control sequence given in the
%first argument.
%    \begin{macrocode}
\def\TeXOSQuery#1#2{%
  \ifTeXOSQueryDryRun
    \begingroup
      \newlinechar=`\^^J
      \message{^^JTeXOSQuery: \TeXOSInvokerName\space#2^^J}%
    \endgroup
    \def#1{}%
  \else
    \begingroup
    \endlinechar=-1\relax
%    \end{macrocode}
%Just in case the result contains any awkward characters that have
%some special meaning to \TeX. (Can't really protect against hash,
%backslash or curly braces, but they're unlikely to occur unless
%the user has a very wacky and inappropriate file naming scheme.)
%    \begin{macrocode}
    \catcode`\-=12\relax
    \catcode`\_=12\relax
    \catcode`\^=12\relax
    \catcode`\~=12\relax
    \catcode`\$=12\relax
    \catcode`\&=12\relax
    \catcode`\"=12\relax
    \catcode`\'=12\relax
    \catcode`\.=12\relax
    \catcode`\/=12\relax
    \catcode`\:=12\relax
    \catcode`\;=12\relax
    \catcode`\%=12\relax
    \everyeof{\noexpand}\relax
    \edef\x{\endgroup\def\noexpand#1{\TeXOSQueryInvoker{#2}}}\x
  \fi
}
%    \end{macrocode}
%\end{macro}
%
%Now for some convenient shortcuts so the user doesn't have to
%remember the command line options. \cs{string} is used in
%case the hyphen character has been made active.
%\begin{macro}{\TeXOSQueryLocale}
%Query the locale and store the result in the control sequence
%provided in the argument.
%    \begin{macrocode}
\def\TeXOSQueryLocale#1{\TeXOSQuery{#1}{\string-l}}
%    \end{macrocode}
%\end{macro}
%
%\begin{macro}{\TeXOSQueryLangTag}
%Query the language tag and store the result in the control sequence
%provided in the argument.
%\changes{1.2}{??}{new}
%    \begin{macrocode}
\def\TeXOSQueryLangTag#1{\TeXOSQuery{#1}{\string-b}}
%    \end{macrocode}
%\end{macro}
%
%\begin{macro}{\TeXOSQueryCwd}
%Query the current working directory.
%    \begin{macrocode}
\def\TeXOSQueryCwd#1{\TeXOSQuery{#1}{\string-c}}
%    \end{macrocode}
%\end{macro}
%
%\begin{macro}{\TeXOSQueryHome}
%Query the user's home directory.
%    \begin{macrocode}
\def\TeXOSQueryHome#1{\TeXOSQuery{#1}{\string-m}}
%    \end{macrocode}
%\end{macro}
%
%\begin{macro}{\TeXOSQueryTmpDir}
%Query the temporary directory.
%    \begin{macrocode}
\def\TeXOSQueryTmpDir#1{\TeXOSQuery{#1}{\string-t}}
%    \end{macrocode}
%\end{macro}
%
%\begin{macro}{\TeXOSQueryVersion}
%Query the operating system version.
%    \begin{macrocode}
\def\TeXOSQueryVersion#1{\TeXOSQuery{#1}{\string-r}}
%    \end{macrocode}
%\end{macro}
%
%\begin{macro}{\TeXOSQueryArch}
%Query the operating system architecture.
%    \begin{macrocode}
\def\TeXOSQueryArch#1{\TeXOSQuery{#1}{\string-a}}
%    \end{macrocode}
%\end{macro}
%
%\begin{macro}{\TeXOSQueryName}
%Query the operating system name.
%    \begin{macrocode}
\def\TeXOSQueryName#1{\TeXOSQuery{#1}{\string-o}}
%    \end{macrocode}
%\end{macro}
%
%\begin{macro}{\TeXOSQueryNow}
%\changes{1.1}{2016-07-14}{changed catcode of D to 12}
%Query the current time stamp.
%    \begin{macrocode}
\def\TeXOSQueryNow#1{%
%    \end{macrocode}
% The \texttt{D} needs category code 12. (Don't need to worry about
% \texttt{Z} as \texttt{texosquery.jar} uses \texttt{+00'00'} for
% UTC+0.) This change can't be done with the other catcode changes
% in \cs{TeXOSQuery}, as this is only appropriate for the PDF dates.
% Save and restore the catcode rather than fiddle around with
% scoping.
%    \begin{macrocode}
  \edef\@texosquery@restore@D{%
    \noexpand\catcode`\noexpand\D=\the\catcode`\D\relax}%
  \catcode`\D=12\relax
  \TeXOSQuery{#1}{\string-n}%
  \@texosquery@restore@D
}
%    \end{macrocode}
%\end{macro}
%
%If the file name is supplied using \cs{jobname} it may have
%double-quotes which will interfere with things.
%\begin{macro}{\texosquerystripquotes}
%    \begin{macrocode}
\def\texosquerystripquotes#1{%
  \@texosquery@stripquotes#1\@mid@texosquery@stripquotes
    "\relax"\relax\@end@texosquery@stripquotes
}
\def\@texosquery@stripquotes#1"#2"{%
  \@@texosquery@stripquotes#1#2%
}
\def\@@texosquery@stripquotes#1\@mid@texosquery@stripquotes#2\@end@texosquery@stripquotes{%
  #1%
}
%    \end{macrocode}
%\end{macro}
%
%\begin{macro}{\TeXOSQueryFileDate}
%\changes{1.1}{2016-07-14}{changed catcode of D to 12}
%Query the time stamp of the file given in the second argument.
%    \begin{macrocode}
\def\TeXOSQueryFileDate#1#2{%
  \edef\@texosquery@restore@D{%
    \noexpand\catcode`\noexpand\D=\the\catcode`\D\relax}%
  \catcode`\D=12\relax
  \TeXOSQuery{#1}{\string-d \string'\texosquerystripquotes{#2}\string'}%
  \@texosquery@restore@D
}
%    \end{macrocode}
%\end{macro}
%
%\begin{macro}{\TeXOSQueryFileSize}
%Query the size of the file given in the second argument.
%    \begin{macrocode}
\def\TeXOSQueryFileSize#1#2{\TeXOSQuery{#1}{\string-s 
 \string'\texosquerystripquotes{#2}\string'}}
%    \end{macrocode}
%\end{macro}
%
%\begin{macro}{\TeXOSQueryFileList}
%List all files in the directory given in the third argument,
%separated by the second argument.
%    \begin{macrocode}
\def\TeXOSQueryFileList#1#2#3{\TeXOSQuery{#1}{%
 \string-i \string'#2\string' 
 \string'\texosquerystripquotes{#3}\string'}}
%    \end{macrocode}
%\end{macro}
%
%\begin{macro}{\TeXOSQueryFilterFileList}
%Filtered list files in the directory given in the fourth argument,
%separated by the second argument. The third argument is the regular
%expression used to filter the list. \emph{Take care of backslashes
%in the regular expression!}
%    \begin{macrocode}
\def\TeXOSQueryFilterFileList#1#2#3#4{\TeXOSQuery{#1}{%
 \string-f \string'#2\string' \string'#3\string' 
 \string'\texosquerystripquotes{#4}\string'}}
%    \end{macrocode}
%\end{macro}
%
%\begin{macro}{\TeXOSQueryFileURI}
%Get the URI of the file given in the second argument.
%    \begin{macrocode}
\def\TeXOSQueryFileURI#1#2{\TeXOSQuery{#1}{\string-u 
 \string'\texosquerystripquotes{#2}\string'}}
%    \end{macrocode}
%\end{macro}
%
%\begin{macro}{\TeXOSQueryFilePath}
%Get the canonical path of the file given in the second argument.
%    \begin{macrocode}
\def\TeXOSQueryFilePath#1#2{\TeXOSQuery{#1}{\string-p 
 \string'\texosquerystripquotes{#2}\string'}}
%    \end{macrocode}
%\end{macro}
%
%\begin{macro}{\TeXOSQueryDirName}
%\changes{1.1}{2016-07-14}{new}
%Get the canonical path of the directory containing the file given 
%in the second argument.
%    \begin{macrocode}
\def\TeXOSQueryDirName#1#2{\TeXOSQuery{#1}{\string-e 
 \string'\texosquerystripquotes{#2}\string'}}
%    \end{macrocode}
%\end{macro}
%
%All done.
%Restore the category code of \texttt{@}:
%    \begin{macrocode}
\@texosquery@restore@at
%    \end{macrocode}
%\iffalse
%    \begin{macrocode}
%</texosquery.tex>
%    \end{macrocode}
%\fi
%
%\subsection{\LaTeX\ Code}
%This is just a simple wrapper for \texttt{texosquery.tex}
%so that it can be loaded using \LaTeX's standard \cs{usepackage}
%method.
%\iffalse
%    \begin{macrocode}
%<*texosquery.sty>
%    \end{macrocode}
%\fi
%Identify package:
%    \begin{macrocode}
\NeedsTeXFormat{LaTeX2e}
\ProvidesPackage{texosquery}[2016/07/14 v1.1 (NLCT)]
%    \end{macrocode}
%Load \texttt{texosquery.tex}:
%    \begin{macrocode}
%\iffalse
%<*package>
%% \CharacterTable
%%  {Upper-case    \A\B\C\D\E\F\G\H\I\J\K\L\M\N\O\P\Q\R\S\T\U\V\W\X\Y\Z
%%   Lower-case    \a\b\c\d\e\f\g\h\i\j\k\l\m\n\o\p\q\r\s\t\u\v\w\x\y\z
%%   Digits        \0\1\2\3\4\5\6\7\8\9
%%   Exclamation   \!     Double quote  \"     Hash (number) \#
%%   Dollar        \$     Percent       \%     Ampersand     \&
%%   Acute accent  \'     Left paren    \(     Right paren   \)
%%   Asterisk      \*     Plus          \+     Comma         \,
%%   Minus         \-     Point         \.     Solidus       \/
%%   Colon         \:     Semicolon     \;     Less than     \<
%%   Equals        \=     Greater than  \>     Question mark \?
%%   Commercial at \@     Left bracket  \[     Backslash     \\
%%   Right bracket \]     Circumflex    \^     Underscore    \_
%%   Grave accent  \`     Left brace    \{     Vertical bar  \|
%%   Right brace   \}     Tilde         \~}
%</package>
%\fi
% \iffalse
% Doc-Source file
% Copyright (C) 2016 Nicola Talbot, all rights reserved.
% \fi
% \iffalse
%<*driver>
\documentclass{ltxdoc}

\usepackage[T1]{fontenc}
\usepackage{tgtermes}
\usepackage{metalogo}
\usepackage[colorlinks,hyperindex=false]{hyperref}

\CheckSum{273}

\RecordChanges
\PageIndex
\CodelineNumbered
\IndexPrologue{}

\renewcommand*{\usage}[1]{\hyperpage{#1}}
\renewcommand*{\main}[1]{\hyperpage{#1}}

\newcommand*{\shortargfmt}[1]{\texttt{-#1}}
\newcommand*{\longargfmt}[1]{\texttt{-{}-#1}}

\newcommand*{\shortarg}[1]{\shortargfmt{#1}%
 \index{texosquery options=\texttt{texosquery} options>#1=\protect\shortargfmt{-#1}|hyperpage}}
\newcommand*{\longarg}[1]{\longargfmt{#1}%
 \index{texosquery options=\texttt{texosquery} options>#1=\protect\longargfmt{#1}|hyperpage}}

\newenvironment{important}
 {\begin{quote}\textbf{Important Note:}}
 {\end{quote}}

\begin{document}
\DocInput{texosquery.dtx}
\end{document}
%</driver>
%\fi
%
%\DeleteShortVerb{|}
%
%\title{texosquery: query OS information from \TeX}
%\author{Nicola L. C. Talbot\\
%\href{http://www.dickimaw-books.com/}{\nolinkurl{dickimaw-books.com}}
%\and Paulo Cereda}
%\date{2016-07-14 (v1.1)}
%\maketitle
%
%\begin{abstract}
%\texttt{texosquery.jar} is a cross-platform Java application
%to query certain OS information designed for use in \TeX's
%shell escape mechanism.
%The accompanying \TeX\ code provides a simple interface to the
%\texttt{texosquery} Java application, which may be skipped
%if you want to explicitly use \cs{input} (but take
%care of any special characters appearing in the result).
%\end{abstract}
%
%\begin{important}
%You will need \TeX's shell escape enabled,
%and you will also need the Java Runtime Environment (JRE) 
%installed to use \texttt{texosquery}.
%\end{important}
%
%If you want to rebuild the application, instructions for
%compiling the source code (including the code for this document)
%are in the accompanying \texttt{README.md} file.
%
%\tableofcontents
%
%\section{texosquery.jar: the Java application}
%The \texttt{texosquery} Java command line application looks up
%certain system information that may be of use in \TeX\ 
%documents. All this information can easily be obtained using 
%native commands, but the Java application allows an 
%OS-independent approach with results that can easily be
%captured by \TeX's shell-escape without having to strip 
%formatting information.
%
%\begin{important}
%\texttt{texosquery} provides read-only actions, and I don't 
%intend adding any actions that modify system settings or files.
%\end{important}
%
%Since the application is designed to work with \TeX, each
%function will display the result on a single line without
%formatting. (For multiple results, each line is grouped 
%from v1.1.) A blank line (or empty group) will be displayed
%if the information isn't available. A forward slash is always 
%used as a directory divider, regardless of the operating 
%system, so the result can be used, for example, in \cs{input} or
%\cs{includegraphics}.
%
%If an input file name is required (for example, with the
%\longargfmt{pdfdate} argument described below) then the file may be in
%the current working directory, relative to the current directory
%(with forward slash \texttt{/} as the directory divider), an absolute path 
%(again with forward slash) or on \TeX's path (in
%which case, \texttt{kpsewhich} is used to locate it).
%
%Command line invocation:
%\begin{flushleft}\ttfamily
%texosquery \meta{action} ...
%\end{flushleft}
%Available actions (at least one required):
%\begin{description}
%\item[\shortarg{L} or \longarg{locale}] Display the locale
%information in the form
%\begin{flushleft}\ttfamily
%\meta{lang}-\meta{region}.\meta{codeset}@\meta{modifier}
%\end{flushleft}
%where \meta{lang} is the ISO code for the language (e.g.\ 
%\texttt{en}),
%\meta{region} is the ISO code for the region (e.g.\ \texttt{GB}),
%\meta{codeset} is the default code set (e.g.\ \texttt{UTF-8})
% and \meta{modifier} is the modifier. Elements may be omitted
% if unavailable. For example, \texttt{en-GB.UTF-8} has the
% \meta{modifier} omitted, and \texttt{en} has all but the language
% omitted.
%\item[\shortarg{l} or \longarg{locale-lcs}] As the above but the
%codeset (if present) is converted to lower case and any hyphens are
%stripped. For example, if \longargfmt{locale} returns
%\texttt{en-GB.UTF-8}, then \longargfmt{locale-lcs} would return
%\texttt{en-GB.utf8}.
%\item[\shortarg{b} or \longarg{bcp47}] (New to version 1.2.)
%Displays the BCP~47 language tag.
%
%\item[\shortarg{c} or \longarg{cwd}] Displays the current working
%directory.
%\item[\shortarg{m} or \longarg{userhome}] Displays the user's home
%directory.
%\item[\shortarg{t} or \longarg{tmpdir}] Displays the temporary
%directory.
%\item[\shortarg{o} or \longarg{osname}] Displays the operating
%system name.
%\item[\shortarg{r} or \longarg{osversion}] Displays the operating
%system version.
%\item[\shortarg{a} or \longarg{osarch}] Displays the operating
%system architecture.
%\item[\shortarg{n} or \longarg{pdfnow}] Displays the current
% date and time in PDF format. For example 
%\begin{verbatim}
%D:20160704131006+01'00'
%\end{verbatim}
%Note that some, but not all, \TeX\ formats provide
%\cs{pdfcreationdate}, which is more efficient than using the shell
%escape, but this can be used
%as a fallback method for those that don't (for example, \XeTeX).
%\item[\shortarg{d} \meta{file} or \longarg{pdfdate} \meta{file}]
% Displays the last modified time stamp of the given file in PDF format or a blank line
% if the file doesn't exist or the file permissions prohibit this
% action.
%Again some, but not all, \TeX\ formats provide
%\cs{pdffilemoddate}\marg{file}, which is more efficient than using
%the shell escape.
%\item[\shortarg{s} \meta{file} or \longarg{filesize} \meta{file}]
% Displays the size in bytes of the given file or a blank line
% if the file doesn't exist or the file permissions prohibit this
% action.
%Some, but not all, \TeX\ formats provide
%\cs{pdffilesize}\marg{file}, which is more efficient than using
%the shell escape.
%\item[\shortarg{i} \meta{sep} \meta{dir} or \longarg{list}
%\meta{sep} \meta{dir}]
%List all files in the given directory with the output on a single
%line using \meta{sep} as the separator between entries. Note that
%the list doesn't include the full path, just the file names.
%\item[\shortarg{f} \meta{sep} \meta{regex} \meta{dir} or
%\longarg{filterlist} \meta{sep} \meta{regex} \meta{dir}]
%Like the above but only lists those files whose name matches the
%regular expression given in \meta{regex}. Note that this tests for a
%complete match on the file name (not including path). For example,
%if \meta{regexp} is \texttt{foo.*}, it will only match files whose
%name starts with \texttt{foo} (for example, \texttt{foobar} will
%match but \texttt{barfoo} won't).
%\item[\shortarg{u} \meta{file} or \longarg{uri} \meta{file}]
%Displays the URI of the given file or a blank line if the file
%doesn't exist or the file permissions prohibit this action.
%Note that this may include in a percent character in the result. The \TeX\ 
%command \cs{TeXOSQuery} protects against this by changing the
%category code, but if you explicitly call \texttt{texosquery}
%using the shell escape, you'll need to be careful of this.
%\item[\shortarg{p} \meta{file} or \longarg{path} \meta{file}]
%Displays the canonical path of the given file or a blank line if
%the file doesn't exist or the file permissions prohibit this
%action.
%\item[\shortarg{e} \meta{file} or \longarg{dirname} \meta{file}]
%(New to v1.1.)
%Displays the canonical path of the given file's parent (that is, the
%directory containing \meta{file}) or a blank line if
%the file doesn't exist or the file permissions prohibit this
%action. Note that this is different to the Unix-like 
%\texttt{dirname} command, which will return a relative path if
%\meta{file} isn't an absolute path.
%\item[\shortarg{h} or \longarg{help}] Displays help message and
%exits.
%\item[\shortarg{v} or \longarg{version}] Displays version
%information and exits.
%\end{description}
%
%If multiple options are given, they will be processed in the
%order specified in the command line invocation. Each result will
%be displayed on a separate line. As from v1.1, if there are
%multiple actions, each result will be grouped. This makes it easier
%to process the results in \TeX. For example:
%\begin{verbatim}
%texosquery -l
%\end{verbatim}
%This just produces (for me):
%\begin{verbatim}
%en-GB.utf8
%\end{verbatim}
%whereas
%\begin{verbatim}
%texosquery -l -n
%\end{verbatim}
%produces:
%\begin{verbatim}
%{en-GB.utf8}
%{D:20160714112732+01'00'}
%\end{verbatim}
%Note that unavailable information will produce an empty group.
%For example (assuming \texttt{nofile} doesn't exist):
%\begin{verbatim}
%texosquery -l -d nofile -n
%\end{verbatim}
%produces:
%\begin{verbatim}
%{en-GB.utf8}
%{}
%{D:20160714112732+01'00'}
%\end{verbatim}
%whereas 
%\begin{verbatim}
%texosquery -d nofile
%\end{verbatim}
%just displays an empty line.
%
%\section{texosquery.tex: generic \TeX\ code}
%You can run \texttt{texosquery} directly from \TeX's shell escape.
%For example:
%\begin{verbatim}
%\input|"texosquery --locale"
%\end{verbatim}
%However, \texttt{texosquery.tex} provides generic \TeX\ code to do
%this for you and store the result in a control sequence.
%
%Plain \TeX\ users can input this file through the usual \cs{input}
%method:
%\begin{verbatim}
%\input texosquery
%\end{verbatim}
%\LaTeX\ users may also simply input this file:
%\begin{verbatim}
%\input{texosquery}
%\end{verbatim}
%but may prefer the standard package approach:
%\begin{verbatim}
%\usepackage{texosquery}
%\end{verbatim}
%
%\begin{important}
%The commands described below are all fragile.
%\end{important}
%
%The basic command to run \texttt{texosquery} and capture its output
%in a control sequence is:
%\DescribeMacro\TeXOSQuery
%\begin{flushleft}\ttfamily
%\cs{TeXOSQuery}\marg{cs}\marg{args}
%\end{flushleft}
%where \meta{cs} is the control sequence in which to store the
%result and \meta{args} are the command line arguments to pass to 
%\texttt{texosquery}. This first locally changes the category code
%of some problematic characters that may appear in the result.
%The only special characters that aren't changed are the backslash
%\verb|\|, curly braces \verb|{| and \verb|}|, and hash \verb|#|.
%(\texttt{texosquery.jar} will replace \verb|#| with \verb|\#| in places
%where it might possibly occur in the result, but in general it's
%best to avoid these characters in file names.)
%
%There are some short cut commands for convenience, described below.
%If any of these commands cause an error message in the form:
%\begin{verbatim}
%I can't find file `|texosquery'.
%\end{verbatim}
%then check that you have the shell escape on. If the error 
%persists with the shell escape enabled
%(and not restricted), check that \texttt{texosquery} is on 
%your system's path. To do this, open a command prompt or terminal and
%type \texttt{texosquery}. If it isn't installed correctly, there will be a
%message like:
%\begin{verbatim}
%`texosquery' is not recognised
%\end{verbatim}
%or
%\begin{verbatim}
%texosquery: command not found
%\end{verbatim}
%If this happens, check the installation. (Instructions are
%in the accompanying \texttt{README.md} file.)
%
%Dry run mode is determined by the conditional
%\DescribeMacro\ifTeXOSQueryDryRun
%\begin{verbatim}
%\ifTeXOSQueryDryRun
%\end{verbatim}
%If true, the shell escape won't be used and the requested command
%invocation will be printed in the transcript file prefixed with
%\begin{verbatim}
%TeXOSQuery:
%\end{verbatim}
%(the control sequence \meta{cs} will be set to empty).
%This conditional will automatically be switched on unless
%\cs{shellescape} or \cs{pdfshellescape} is 1. (If
%\texttt{texosquery.jar} is later allowed on the restricted list,
%newer versions may change this default.)
%
%If multiple queries are required, it's more efficient to
%perform them all in one go. For example:
%\begin{verbatim}
%\TeXOSQuery{\result}{-l -n -o}
%
%\def\parseresult#1#2#3{%
%  Locale: #1. Now: #2. OS: #3.%
%}
%
%\ifx\result\empty
%  Query failed.
%\else
%  \expandafter\parseresult\result
%\fi
%\end{verbatim}
%(Make sure you have at least v1.1 for this to work correctly.)
%
%The locale (\shortarg{l} or \longarg{locale-lcs}) information can be
%obtained using:
%\DescribeMacro\TeXOSQueryLocale
%\begin{flushleft}\ttfamily
%\cs{TeXOSQueryLocale}\marg{cs}
%\end{flushleft}
%Note that this uses the lower case codeset form, which has a better
%chance of matching the encoding names used by the \texttt{inputenc}
%package. If you want the unprocessed codeset name, you can do:
%\begin{flushleft}\ttfamily
%\cs{TeXOSQuery}\marg{cs}\{\shortarg{L}\}
%\end{flushleft}
%
%The BCP 47 language tag (\shortarg{b} or \longarg{bcp47}) can be obtained using:
%\DescribeMacro\TeXOSQueryLangTag
%\begin{flushleft}\ttfamily
%\cs{TeXOSQueryLangTag}\marg{cs}
%\end{flushleft}
%
%The current working directory (\shortarg{c} or \longarg{cwd}) can
%be obtained using:
%\DescribeMacro\TeXOSQueryCwd
%\begin{flushleft}\ttfamily
%\cs{TeXOSQueryCwd}\marg{cs}
%\end{flushleft}
%
%The home directory (\shortarg{m} or \longarg{userhome}) can
%be obtained using:
%\DescribeMacro\TeXOSQueryHome
%\begin{flushleft}\ttfamily
%\cs{TeXOSQueryHome}\marg{cs}
%\end{flushleft}
%
%The temporary directory (\shortarg{t} or \longarg{tmpdir}) can
%be obtained using:
%\DescribeMacro\TeXOSQueryTmpDir
%\begin{flushleft}\ttfamily
%\cs{TeXOSQueryTmpDir}\marg{cs}
%\end{flushleft}
%
%The OS name (\shortarg{o} or \longarg{osname}) can
%be obtained using:
%\DescribeMacro\TeXOSQueryName
%\begin{flushleft}\ttfamily
%\cs{TeXOSQueryName}\marg{cs}
%\end{flushleft}
%
%The OS version (\shortarg{r} or \longarg{osversion}) can
%be obtained using:
%\DescribeMacro\TeXOSQueryVersion
%\begin{flushleft}\ttfamily
%\cs{TeXOSQueryVersion}\marg{cs}
%\end{flushleft}
%
%The OS architecture (\shortarg{a} or \longarg{osarch}) can
%be obtained using:
%\DescribeMacro\TeXOSQueryArch
%\begin{flushleft}\ttfamily
%\cs{TeXOSQueryArch}\marg{cs}
%\end{flushleft}
%
%The current date-time stamp in PDF format (\shortarg{n} or
%\longarg{pdfnow}) can be obtained using:
%\DescribeMacro\TeXOSQueryNow
%\begin{flushleft}\ttfamily
%\cs{TeXOSQueryNow}\marg{cs}
%\end{flushleft}
%This is provided for the benefit of users who don't have
%\cs{pdfcreationdate} defined by their \TeX\ format (for example,
%\XeTeX). As from v1.1, this ensures that the initial \texttt{D} has
%category code 12 (which won't happen if \cs{TeXOSQuery} is used
%explicitly).
%
%\begin{important}
%The remaining commands all require extra arguments after the
%relevant switch. These are automatically enclosed in single-quotes
%to protect any spaces. If the argument actually contains any single-quote
%characters, make sure you use \verb|\string\'| to prevent
%interference. \emph{However, in general it's a very bad idea to use
%quotes as part of a file name (rather than using them as
%delimiters).}
%Since the file name may need to be obtained from \cs{jobname},
%which sometimes includes double-quotes, the first double-quote pair
%found is stripped in file name arguments. Any other double-quotes
%will need to be protected in the same manner as single-quotes
%(but, again, this shouldn't be an issue if you use a safe
%file naming scheme). Any paths should use a
%forward slash for the directory divider.
%\end{important}
%
%The modification date-time stamp in PDF format for a file
%(\shortarg{d} or \longarg{pdfdate}) can be obtained using:
%\DescribeMacro\TeXOSQueryFileDate
%\begin{flushleft}\ttfamily
%\cs{TeXOSQueryFileDate}\marg{cs}\marg{filename}
%\end{flushleft}
%where \meta{filename} is the name of the file.
%This is provided for the benefit of users who don't have
%\cs{pdffilemoddate} defined by their \TeX\ format. As from v1.1,
%this ensures that the initial \texttt{D} has category code 12 (which 
%won't happen if \cs{TeXOSQuery} is used explicitly).
%
%The size in bytes of a file
%(\shortarg{s} or \longarg{filesize}) can be obtained using:
%\DescribeMacro\TeXOSQueryFileSize
%\begin{flushleft}\ttfamily
%\cs{TeXOSQueryFileSize}\marg{cs}\marg{filename}
%\end{flushleft}
%where \meta{filename} is the name of the file.
%This is provided for the benefit of users who don't have
%\cs{pdffilesize} defined by their \TeX\ format.
%
%The list of files in a given directory
%(\shortarg{i} or \longarg{list}) can be obtained using:
%\DescribeMacro\TeXOSQueryFileList
%\begin{flushleft}\ttfamily
%\cs{TeXOSQueryFileList}\marg{cs}\marg{sep}\marg{dir}
%\end{flushleft}
%where \meta{sep} is the separator and \meta{dir} is the directory name.
%For example:
%\begin{verbatim}
%\TeXOSQueryFileList{\result}{,}{.}
%\end{verbatim}
%will store a comma-separated list of all the files contained in the
%current directory in the control sequence \cs{result}.
%
%A filtered list of files in a given directory
%(\shortarg{f} or \longarg{filterlist}) can be obtained using:
%\DescribeMacro\TeXOSQueryFilterFileList
%\begin{flushleft}\ttfamily
%\cs{TeXOSQueryFilterFileList}\marg{cs}\marg{sep}\marg{regex}\marg{dir}
%\end{flushleft}
%where \meta{regex} is a regular expression. \emph{Take care of any backslashes
%in the regular expression!}
%For example, to list only those files that have an extension:
%\begin{verbatim}
%\TeXOSQueryFilterFileList{\result}{,}{.+\string\..*}{.}
%\end{verbatim}
%Note the use of \verb|\string\.| to ensure that \cs{.} isn't
%interpreted as a command. Another example, list only \texttt{.png}
%and \texttt{.jpg} files in the directory called \texttt{images}:
%\begin{verbatim}
%\TeXOSQueryFilterFileList{\result}{,}{.+\string\.(jpg|png)}{images}
%\end{verbatim}
%
%The URI of a file
%(\shortarg{u} or \longarg{uri}) can be obtained using:
%\DescribeMacro\TeXOSQueryFileURI
%\begin{flushleft}\ttfamily
%\cs{TeXOSQueryFileURI}\marg{cs}\marg{filename}
%\end{flushleft}
%where \meta{filename} is the name of the file.
%(Any percent symbols \verb|%| contained in the URI will have
%their category code set to 12.)
%
%The canonical path of a file
%(\shortarg{p} or \longarg{path}) can be obtained using:
%\DescribeMacro\TeXOSQueryFilePath
%\begin{flushleft}\ttfamily
%\cs{TeXOSQueryFilePath}\marg{cs}\marg{filename}
%\end{flushleft}
%where \meta{filename} is the name of the file.
%
%The canonical path of a file's parent
%(\shortarg{e} or \longarg{dirname}) can be obtained using:
%\DescribeMacro\TeXOSQueryDirName
%\begin{flushleft}\ttfamily
%\cs{TeXOSQueryDirName}\marg{cs}\marg{filename}
%\end{flushleft}
%where \meta{filename} is the name of the file.
%
%\StopEventually{\phantomsection
%\addcontentsline{toc}{section}{Change History}%
%\PrintChanges
%\addcontentsline{toc}{section}{\indexname}%
%\PrintIndex
%}
%
%\section{The Code}
%\changes{1.0}{2016-07-08}{Initial release}
%\subsection{Generic \TeX\ Code}
%\iffalse
%    \begin{macrocode}
%<*texosquery.tex>
%    \end{macrocode}
%\fi
%Change category code of \texttt{@} if necessary.
%    \begin{macrocode}
\ifnum\catcode`\@=11\relax
  \def\@texosquery@restore@at{}%
\else
  \expandafter\edef\csname @texosquery@restore@at\endcsname{%
    \noexpand\catcode`\noexpand\@=\number\catcode`\@\relax
  }%
 \catcode`\@=11\relax
\fi
%    \end{macrocode}
% Check if already loaded.
%    \begin{macrocode}
\ifx\TeXOSQuery\undefined
\else
  \@texosquery@restore@at
  \expandafter\endinput
\fi
%    \end{macrocode}
%\begin{macro}{\TeXOSInvokerName}
%The name of the \texttt{texosquery} application.
%    \begin{macrocode}
\def\TeXOSInvokerName{texosquery}
%    \end{macrocode}
%\end{macro}
%
%\begin{macro}{\TeXOSQueryInvoker}
%If we're using \LaTeX, we'll need to use \cs{@@input} rather
%than \cs{input}.
%    \begin{macrocode}
\ifx\@@input\undefined
  \def\TeXOSQueryInvoker#1{\input|"\TeXOSInvokerName\space#1" }
\else
  \def\TeXOSQueryInvoker#1{\@@input|"\TeXOSInvokerName\space#1" }
\fi
%    \end{macrocode}
%\end{macro}
%
%\begin{macro}{\ifTeXOSQueryDryRun}
%Provide a dry-run mode.
%\changes{1.1}{2016-07-14}{dry run mode only false by default if with
%unrestricted mode}
%    \begin{macrocode}
\newif\ifTeXOSQueryDryRun
\TeXOSQueryDryRuntrue
%    \end{macrocode}
%\end{macro}
%
%If shell escape is unrestricted, automatically switch off dry-run mode.
%    \begin{macrocode}
\ifx\shellescape\undefined
  \ifx\pdfshellescape\undefined
  \else
    \ifnum\pdfshellescape=1\relax
      \TeXOSQueryDryRunfalse
    \fi
  \fi
\else
  \ifnum\shellescape=1\relax
    \TeXOSQueryDryRunfalse
  \fi
\fi
%    \end{macrocode}
%
%\begin{macro}{\TeXOSQuery}
%Use \texttt{texosquery} with the option given in the second
%argument and store the result in control sequence given in the
%first argument.
%    \begin{macrocode}
\def\TeXOSQuery#1#2{%
  \ifTeXOSQueryDryRun
    \begingroup
      \newlinechar=`\^^J
      \message{^^JTeXOSQuery: \TeXOSInvokerName\space#2^^J}%
    \endgroup
    \def#1{}%
  \else
    \begingroup
    \endlinechar=-1\relax
%    \end{macrocode}
%Just in case the result contains any awkward characters that have
%some special meaning to \TeX. (Can't really protect against hash,
%backslash or curly braces, but they're unlikely to occur unless
%the user has a very wacky and inappropriate file naming scheme.)
%    \begin{macrocode}
    \catcode`\-=12\relax
    \catcode`\_=12\relax
    \catcode`\^=12\relax
    \catcode`\~=12\relax
    \catcode`\$=12\relax
    \catcode`\&=12\relax
    \catcode`\"=12\relax
    \catcode`\'=12\relax
    \catcode`\.=12\relax
    \catcode`\/=12\relax
    \catcode`\:=12\relax
    \catcode`\;=12\relax
    \catcode`\%=12\relax
    \everyeof{\noexpand}\relax
    \edef\x{\endgroup\def\noexpand#1{\TeXOSQueryInvoker{#2}}}\x
  \fi
}
%    \end{macrocode}
%\end{macro}
%
%Now for some convenient shortcuts so the user doesn't have to
%remember the command line options. \cs{string} is used in
%case the hyphen character has been made active.
%\begin{macro}{\TeXOSQueryLocale}
%Query the locale and store the result in the control sequence
%provided in the argument.
%    \begin{macrocode}
\def\TeXOSQueryLocale#1{\TeXOSQuery{#1}{\string-l}}
%    \end{macrocode}
%\end{macro}
%
%\begin{macro}{\TeXOSQueryLangTag}
%Query the language tag and store the result in the control sequence
%provided in the argument.
%\changes{1.2}{??}{new}
%    \begin{macrocode}
\def\TeXOSQueryLangTag#1{\TeXOSQuery{#1}{\string-b}}
%    \end{macrocode}
%\end{macro}
%
%\begin{macro}{\TeXOSQueryCwd}
%Query the current working directory.
%    \begin{macrocode}
\def\TeXOSQueryCwd#1{\TeXOSQuery{#1}{\string-c}}
%    \end{macrocode}
%\end{macro}
%
%\begin{macro}{\TeXOSQueryHome}
%Query the user's home directory.
%    \begin{macrocode}
\def\TeXOSQueryHome#1{\TeXOSQuery{#1}{\string-m}}
%    \end{macrocode}
%\end{macro}
%
%\begin{macro}{\TeXOSQueryTmpDir}
%Query the temporary directory.
%    \begin{macrocode}
\def\TeXOSQueryTmpDir#1{\TeXOSQuery{#1}{\string-t}}
%    \end{macrocode}
%\end{macro}
%
%\begin{macro}{\TeXOSQueryVersion}
%Query the operating system version.
%    \begin{macrocode}
\def\TeXOSQueryVersion#1{\TeXOSQuery{#1}{\string-r}}
%    \end{macrocode}
%\end{macro}
%
%\begin{macro}{\TeXOSQueryArch}
%Query the operating system architecture.
%    \begin{macrocode}
\def\TeXOSQueryArch#1{\TeXOSQuery{#1}{\string-a}}
%    \end{macrocode}
%\end{macro}
%
%\begin{macro}{\TeXOSQueryName}
%Query the operating system name.
%    \begin{macrocode}
\def\TeXOSQueryName#1{\TeXOSQuery{#1}{\string-o}}
%    \end{macrocode}
%\end{macro}
%
%\begin{macro}{\TeXOSQueryNow}
%\changes{1.1}{2016-07-14}{changed catcode of D to 12}
%Query the current time stamp.
%    \begin{macrocode}
\def\TeXOSQueryNow#1{%
%    \end{macrocode}
% The \texttt{D} needs category code 12. (Don't need to worry about
% \texttt{Z} as \texttt{texosquery.jar} uses \texttt{+00'00'} for
% UTC+0.) This change can't be done with the other catcode changes
% in \cs{TeXOSQuery}, as this is only appropriate for the PDF dates.
% Save and restore the catcode rather than fiddle around with
% scoping.
%    \begin{macrocode}
  \edef\@texosquery@restore@D{%
    \noexpand\catcode`\noexpand\D=\the\catcode`\D\relax}%
  \catcode`\D=12\relax
  \TeXOSQuery{#1}{\string-n}%
  \@texosquery@restore@D
}
%    \end{macrocode}
%\end{macro}
%
%If the file name is supplied using \cs{jobname} it may have
%double-quotes which will interfere with things.
%\begin{macro}{\texosquerystripquotes}
%    \begin{macrocode}
\def\texosquerystripquotes#1{%
  \@texosquery@stripquotes#1\@mid@texosquery@stripquotes
    "\relax"\relax\@end@texosquery@stripquotes
}
\def\@texosquery@stripquotes#1"#2"{%
  \@@texosquery@stripquotes#1#2%
}
\def\@@texosquery@stripquotes#1\@mid@texosquery@stripquotes#2\@end@texosquery@stripquotes{%
  #1%
}
%    \end{macrocode}
%\end{macro}
%
%\begin{macro}{\TeXOSQueryFileDate}
%\changes{1.1}{2016-07-14}{changed catcode of D to 12}
%Query the time stamp of the file given in the second argument.
%    \begin{macrocode}
\def\TeXOSQueryFileDate#1#2{%
  \edef\@texosquery@restore@D{%
    \noexpand\catcode`\noexpand\D=\the\catcode`\D\relax}%
  \catcode`\D=12\relax
  \TeXOSQuery{#1}{\string-d \string'\texosquerystripquotes{#2}\string'}%
  \@texosquery@restore@D
}
%    \end{macrocode}
%\end{macro}
%
%\begin{macro}{\TeXOSQueryFileSize}
%Query the size of the file given in the second argument.
%    \begin{macrocode}
\def\TeXOSQueryFileSize#1#2{\TeXOSQuery{#1}{\string-s 
 \string'\texosquerystripquotes{#2}\string'}}
%    \end{macrocode}
%\end{macro}
%
%\begin{macro}{\TeXOSQueryFileList}
%List all files in the directory given in the third argument,
%separated by the second argument.
%    \begin{macrocode}
\def\TeXOSQueryFileList#1#2#3{\TeXOSQuery{#1}{%
 \string-i \string'#2\string' 
 \string'\texosquerystripquotes{#3}\string'}}
%    \end{macrocode}
%\end{macro}
%
%\begin{macro}{\TeXOSQueryFilterFileList}
%Filtered list files in the directory given in the fourth argument,
%separated by the second argument. The third argument is the regular
%expression used to filter the list. \emph{Take care of backslashes
%in the regular expression!}
%    \begin{macrocode}
\def\TeXOSQueryFilterFileList#1#2#3#4{\TeXOSQuery{#1}{%
 \string-f \string'#2\string' \string'#3\string' 
 \string'\texosquerystripquotes{#4}\string'}}
%    \end{macrocode}
%\end{macro}
%
%\begin{macro}{\TeXOSQueryFileURI}
%Get the URI of the file given in the second argument.
%    \begin{macrocode}
\def\TeXOSQueryFileURI#1#2{\TeXOSQuery{#1}{\string-u 
 \string'\texosquerystripquotes{#2}\string'}}
%    \end{macrocode}
%\end{macro}
%
%\begin{macro}{\TeXOSQueryFilePath}
%Get the canonical path of the file given in the second argument.
%    \begin{macrocode}
\def\TeXOSQueryFilePath#1#2{\TeXOSQuery{#1}{\string-p 
 \string'\texosquerystripquotes{#2}\string'}}
%    \end{macrocode}
%\end{macro}
%
%\begin{macro}{\TeXOSQueryDirName}
%\changes{1.1}{2016-07-14}{new}
%Get the canonical path of the directory containing the file given 
%in the second argument.
%    \begin{macrocode}
\def\TeXOSQueryDirName#1#2{\TeXOSQuery{#1}{\string-e 
 \string'\texosquerystripquotes{#2}\string'}}
%    \end{macrocode}
%\end{macro}
%
%All done.
%Restore the category code of \texttt{@}:
%    \begin{macrocode}
\@texosquery@restore@at
%    \end{macrocode}
%\iffalse
%    \begin{macrocode}
%</texosquery.tex>
%    \end{macrocode}
%\fi
%
%\subsection{\LaTeX\ Code}
%This is just a simple wrapper for \texttt{texosquery.tex}
%so that it can be loaded using \LaTeX's standard \cs{usepackage}
%method.
%\iffalse
%    \begin{macrocode}
%<*texosquery.sty>
%    \end{macrocode}
%\fi
%Identify package:
%    \begin{macrocode}
\NeedsTeXFormat{LaTeX2e}
\ProvidesPackage{texosquery}[2016/07/14 v1.1 (NLCT)]
%    \end{macrocode}
%Load \texttt{texosquery.tex}:
%    \begin{macrocode}
\input{texosquery}
%    \end{macrocode}
%That's it!
%\iffalse
%    \begin{macrocode}
%</texosquery.sty>
%    \end{macrocode}
%\fi
%
%\Finale
\endinput

%    \end{macrocode}
%That's it!
%\iffalse
%    \begin{macrocode}
%</texosquery.sty>
%    \end{macrocode}
%\fi
%
%\Finale
\endinput

%    \end{macrocode}
%That's it!
%\iffalse
%    \begin{macrocode}
%</texosquery.sty>
%    \end{macrocode}
%\fi
%
%\Finale
\endinput

%    \end{macrocode}
%That's it!
%\iffalse
%    \begin{macrocode}
%</texosquery.sty>
%    \end{macrocode}
%\fi
%
%\Finale
\endinput
